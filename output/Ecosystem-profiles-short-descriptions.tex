% Options for packages loaded elsewhere
\PassOptionsToPackage{unicode}{hyperref}
\PassOptionsToPackage{hyphens}{url}
\PassOptionsToPackage{dvipsnames,svgnames,x11names}{xcolor}
%
\documentclass[
  letterpaper,
  DIV=11,
  numbers=noendperiod]{scrartcl}

\usepackage{amsmath,amssymb}
\usepackage{iftex}
\ifPDFTeX
  \usepackage[T1]{fontenc}
  \usepackage[utf8]{inputenc}
  \usepackage{textcomp} % provide euro and other symbols
\else % if luatex or xetex
  \usepackage{unicode-math}
  \defaultfontfeatures{Scale=MatchLowercase}
  \defaultfontfeatures[\rmfamily]{Ligatures=TeX,Scale=1}
\fi
\usepackage{lmodern}
\ifPDFTeX\else  
    % xetex/luatex font selection
\fi
% Use upquote if available, for straight quotes in verbatim environments
\IfFileExists{upquote.sty}{\usepackage{upquote}}{}
\IfFileExists{microtype.sty}{% use microtype if available
  \usepackage[]{microtype}
  \UseMicrotypeSet[protrusion]{basicmath} % disable protrusion for tt fonts
}{}
\makeatletter
\@ifundefined{KOMAClassName}{% if non-KOMA class
  \IfFileExists{parskip.sty}{%
    \usepackage{parskip}
  }{% else
    \setlength{\parindent}{0pt}
    \setlength{\parskip}{6pt plus 2pt minus 1pt}}
}{% if KOMA class
  \KOMAoptions{parskip=half}}
\makeatother
\usepackage{xcolor}
\setlength{\emergencystretch}{3em} % prevent overfull lines
\setcounter{secnumdepth}{-\maxdimen} % remove section numbering
% Make \paragraph and \subparagraph free-standing
\makeatletter
\ifx\paragraph\undefined\else
  \let\oldparagraph\paragraph
  \renewcommand{\paragraph}{
    \@ifstar
      \xxxParagraphStar
      \xxxParagraphNoStar
  }
  \newcommand{\xxxParagraphStar}[1]{\oldparagraph*{#1}\mbox{}}
  \newcommand{\xxxParagraphNoStar}[1]{\oldparagraph{#1}\mbox{}}
\fi
\ifx\subparagraph\undefined\else
  \let\oldsubparagraph\subparagraph
  \renewcommand{\subparagraph}{
    \@ifstar
      \xxxSubParagraphStar
      \xxxSubParagraphNoStar
  }
  \newcommand{\xxxSubParagraphStar}[1]{\oldsubparagraph*{#1}\mbox{}}
  \newcommand{\xxxSubParagraphNoStar}[1]{\oldsubparagraph{#1}\mbox{}}
\fi
\makeatother


\providecommand{\tightlist}{%
  \setlength{\itemsep}{0pt}\setlength{\parskip}{0pt}}\usepackage{longtable,booktabs,array}
\usepackage{calc} % for calculating minipage widths
% Correct order of tables after \paragraph or \subparagraph
\usepackage{etoolbox}
\makeatletter
\patchcmd\longtable{\par}{\if@noskipsec\mbox{}\fi\par}{}{}
\makeatother
% Allow footnotes in longtable head/foot
\IfFileExists{footnotehyper.sty}{\usepackage{footnotehyper}}{\usepackage{footnote}}
\makesavenoteenv{longtable}
\usepackage{graphicx}
\makeatletter
\def\maxwidth{\ifdim\Gin@nat@width>\linewidth\linewidth\else\Gin@nat@width\fi}
\def\maxheight{\ifdim\Gin@nat@height>\textheight\textheight\else\Gin@nat@height\fi}
\makeatother
% Scale images if necessary, so that they will not overflow the page
% margins by default, and it is still possible to overwrite the defaults
% using explicit options in \includegraphics[width, height, ...]{}
\setkeys{Gin}{width=\maxwidth,height=\maxheight,keepaspectratio}
% Set default figure placement to htbp
\makeatletter
\def\fps@figure{htbp}
\makeatother

\KOMAoption{captions}{tableheading}
\makeatletter
\@ifpackageloaded{caption}{}{\usepackage{caption}}
\AtBeginDocument{%
\ifdefined\contentsname
  \renewcommand*\contentsname{Table of contents}
\else
  \newcommand\contentsname{Table of contents}
\fi
\ifdefined\listfigurename
  \renewcommand*\listfigurename{List of Figures}
\else
  \newcommand\listfigurename{List of Figures}
\fi
\ifdefined\listtablename
  \renewcommand*\listtablename{List of Tables}
\else
  \newcommand\listtablename{List of Tables}
\fi
\ifdefined\figurename
  \renewcommand*\figurename{Figure}
\else
  \newcommand\figurename{Figure}
\fi
\ifdefined\tablename
  \renewcommand*\tablename{Table}
\else
  \newcommand\tablename{Table}
\fi
}
\@ifpackageloaded{float}{}{\usepackage{float}}
\floatstyle{ruled}
\@ifundefined{c@chapter}{\newfloat{codelisting}{h}{lop}}{\newfloat{codelisting}{h}{lop}[chapter]}
\floatname{codelisting}{Listing}
\newcommand*\listoflistings{\listof{codelisting}{List of Listings}}
\makeatother
\makeatletter
\makeatother
\makeatletter
\@ifpackageloaded{caption}{}{\usepackage{caption}}
\@ifpackageloaded{subcaption}{}{\usepackage{subcaption}}
\makeatother

\ifLuaTeX
  \usepackage{selnolig}  % disable illegal ligatures
\fi
\usepackage{bookmark}

\IfFileExists{xurl.sty}{\usepackage{xurl}}{} % add URL line breaks if available
\urlstyle{same} % disable monospaced font for URLs
\hypersetup{
  colorlinks=true,
  linkcolor={blue},
  filecolor={Maroon},
  citecolor={Blue},
  urlcolor={Blue},
  pdfcreator={LaTeX via pandoc}}


\author{}
\date{}

\begin{document}


\section{F1.1 Permanent upland
streams}\label{f1.1-permanent-upland-streams}

Belongs to biome F1. Rivers and streams biome, part of the Freshwater
realm.

\subsection{Short description}\label{short-description}

These small rivers or streams in mountainous or hilly areas are
characterised by steep gradients and fast flow. They flow all year,
increasing in wet periods, in humid tropical and temperate zones. Stones
are common along their rapids and pools, turning over and oxygenating
the water. Dependent organisms are specialised for these high
flow-velocity environments, with resources for food webs derived mainly
from the stream and inputs from adjacent and upstream vegetation.

\subsection{Key Features}\label{key-features}

High-medium velocity, low-medium volume perennial flows with abundant
benthic filter feeders, algal biofilms \& small fish.

\subsection{Ecological traits}\label{ecological-traits}

These 1st-3rd order streams generally have steep gradients, fast flows,
coarse substrates, often with a riffle-pool (shallow and fast vs deeper
and slow) sequence of habitats, and periodic (usually seasonal)
high-flow events. Many organisms have specialised morphological and
behavioural adaptations to high flow-velocity environments. Riparian
trees produce copious leaf fall that provide allochthonous subsidies,
and support somewhat separate foodwebs to those based on in situ primary
production by bryophytes and biofilms. Tree shade conversely
light-limits productivity, a trade-off that relaxes seasonally where
deciduous trees dominate. Microbes and detritivores (e.g.~invertebrate
shredders) break down leaf fall and other organic matter. Microbial
biofilms comprising algae, fungi and bacteria establish on rocks and
process dissolved organic matter. Invertebrates include shredders
(consuming coarse particles), grazers (consuming biofilm), collectors
and filter feeders (consuming benthic and suspended fine particles,
respectively), and predators. Many benthic macroinvertebrates, mostly
insects, have aquatic larvae and terrestrial adults. Filter feeders have
traits adapted to swift flows, allowing them to hold fast to substrates
while capturing resources, while benthic bryophytes provide shelter for
other organisms. Fish are typically small predators of aquatic
invertebrates and insects on the water surface. Birds typically have
specialised foraging behaviours (e.g.~dippers and kingfishers). Trophic
cascades involving rapid algal growth, invertebrate grazers and fish are
common.

\subsection{Key Ecological Drivers}\label{key-ecological-drivers}

Upland streams have flash flow regimes with high velocity and relatively
low, but variable perennial volume. Turbulence sustains highly
oxygenation. Groundwater-delivered subsidies support streamflow, with up
to 50\% of summer flow and 100\% of winter flow originating as
groundwater. This modulates stream temperatures, keeping temperatures
lower in summer and higher in winter; and deliver nutrients, especially
if there are N-fixing plants, along the groundwater flow path. They flow
down moderate to steep slopes causing considerable erosion and sediment
transport. These factors drive nutrient and organic matter transport
downstream. Flow volume and variability, including periodic flood
regimes, depend on rainfall seasonality, snowmelt from cold-climate
catchments, as well as catchment size. Peat-rich catchments feed dark
dystrophic waters to the streams.

\subsection{Distribution}\label{distribution}

High proportion of global stream length. In steep to moderate terrain
throughout the humid tropical and temperate zones, rarely extending to
boreal latitudes.

\section{F1.2 Permanent lowland
rivers}\label{f1.2-permanent-lowland-rivers}

Belongs to biome F1. Rivers and streams biome, part of the Freshwater
realm.

\subsection{Short description}\label{short-description-1}

Lowland rivers with slow continuous flows up to 10,000m3/s are common at
low elevations throughout tropical and temperate parts of the world.
These are productive ecosystems with major energy and fine sediment
inputs from floodplains and upper catchments. Zooplankton can be
abundant, along with aquatic plants and diverse communities of fish able
to tolerate a range of temperatures and oxygen concentrations, as well
as reptiles, birds, and mammals that depend wholly or partly on lowland
lotic aquatic habitats.

\subsection{Key Features}\label{key-features-1}

Low-medium velocity, high volume, perennial flows with abundant
zooplankton, fish, macrophytes, macroinvertebrates \& piscivores.

\subsection{Ecological traits}\label{ecological-traits-1}

Small-medium lowland rivers (stream orders 4-9) are productive
depositional ecosystems with trophic webs that are less diverse than
large lowland rivers (F1.7). Macrophytes rooted in benthos or along the
river margins contribute most primary production, but allochthonous
inputs from floodplains and upper catchments generally dominate energy
flow in the system. The biota tolerates a range of temperatures, which
vary with catchment climate. Aquatic biota have physiological,
morphological and even behavioural adaptations to lower oxygen
concentrations, which may vary seasonally and diurnally. Zooplankton can
be abundant in slower deeper rivers. Sessile (e.g.~mussels) and
scavenging (e.g.~crayfish) macroinvertebrates are associated with the
hyporheic zone and structurally complex microhabitats in moderate flow
environments, including fine sediment and woody debris. Fish communities
are diverse and may contribute to complex trophic networks. They include
large predatory fish (e.g.~sturgeons), smaller predators of
invertebrates, herbivores, and detritivores. The feeding activities and
movement of piscivorous birds (e.g.~cormorants), diadromous fish
(seawater-freshwater migrants), mammals (e.g.~otters), and reptiles
(e.g.~turtles) extend trophic network beyond instream waters. Riparian
zones vary in complexity from forested banks to shallow areas where
emergent, floating and submerged macrophyte vegetation grows.
Intermittently connected oxbow lakes or billabongs increase the
complexity of associated habitats, providing more lentic waters for a
range of aquatic fauna and flora.

\subsection{Key Ecological Drivers}\label{key-ecological-drivers-1}

These rivers are distinguished by shallow gradients, low turbulence, low
to moderate flow velocity and moderate flow volumes
(\textless10,000m3/s). Flows are continuous but may vary seasonally
depending on catchment precipitation. This combination of features is
most common at low altitudes below 200 m and rarely occurs above 1,500
m. River channels are tens to a few hundred metres wide and up to tens
of metres deep with mostly soft sediment substrates. They are dominated
by depositional processes. Surface water and groundwater mix in the
alluvium in the hyporheic zone, which plays an important role in
nutrient cycling. Overbank flows increase turbulence and turbidity.
Locally or temporally important erosional processes redistribute
sediment and produce geomorphically dynamic depositional features
(e.g.~braided channels and point bars). Nutrient levels depend on
riparian/floodplain inputs and vary with catchment geochemistry. Oxygen
and temperatures also vary with climate and catchment features. For
catchments with extensive peatlands, waters may be tannin-rich, poorly
oxygenated, acidic and dark, thus reducing productivity and diversity.

\subsection{Distribution}\label{distribution-1}

Distributed throughout tropical and temperate lowlands but very uncommon
in arid zones. They are absent from boreal zones, where they are
replaced by F1.3.

\section{F1.3 Freeze-thaw rivers and
streams}\label{f1.3-freeze-thaw-rivers-and-streams}

Belongs to biome F1. Rivers and streams biome, part of the Freshwater
realm.

\subsection{Short description}\label{short-description-2}

In cold climates at high latitudes or altitudes, the surfaces of both
small streams and large rivers freeze in winter. In winter, the layer of
surface ice reduces nutrient inputs and light penetration, limiting the
productivity of these ecosystems and the diversity of their biota. In
spring, meltwaters transport increased organic matter and nutrients,
producing seasonal peaks in abundance of algae and phytoplankton.
Animals, such as fish and beavers, tolerate near-freezing water
temperatures, while a range of invertebrates and other vertebrates come
to forage from spring to autumn.

\subsection{Key Features}\label{key-features-2}

Cold-climate streams with seasonally frozen surface water and variable
melt flows and aquatic biota with cold-resistance and/or seasonal
dormancy.

\subsection{Ecological traits}\label{ecological-traits-2}

In seasonally cold montane and boreal environments, the surfaces of both
small streams and large rivers freeze in winter. These systems have
relatively simple trophic networks with low functional and taxonomic
diversity, but the biota may include local endemics. In small, shallow
streams, substrate algae are the principal autotrophs, while
phytoplankton occur in larger rivers and benthic macrophytes are rare.
All are seasonally inactive or curtailed when temperatures are cold and
surface ice reduces light penetration through the water. Bottom-up
regulatory processes dominate. Subsidies of dissolved organic carbon and
nutrients from spring meltwaters and riparian vegetation along smaller
streams are crucial to maintaining the detritivores that dominate the
trophic network. Overall decomposition rates of coarse particles are
low, but can exceed rates per degree day in warmer climates as the fauna
are adapted to cold temperatures. Microbial decomposers often dominate
small streams, but in larger rivers, the massive increase in fine
organic particles in spring meltwaters can support abundant filter
feeders which consume huge quantities of suspended particles and
redeposit them within the river bed. Resident invertebrates survive cold
temperatures through dormant life stages, extended life cycles and
physiological adaptations. Vertebrate habitat specialists (e.g.~dippers,
small fish, beavers, and otters) tolerate low temperatures with traits
such as subcuticular fat, thick hydrophobic, and/or aerated fur or
feathers. Many fish disperse from frozen habitat to deeper water refuges
during the winter (e.g.~deep pools) before foraging in the meltwater
streams from spring to autumn. In the larger rivers, fish, and
particularly migratory salmonids returning to their natal streams and
rivers for breeding, are a food source for itinerant terrestrial
predators such as bears. When they die after reproduction, their
decomposition in turn provides huge inputs of energy and nutrients to
the system.

\subsection{Key Ecological Drivers}\label{key-ecological-drivers-2}

These rivers experience low winter temperatures and seasonal freeze-thaw
regimes. Winter freezing is generally limited to the surface but can
extend to the substrate forming `anchor ice'. Flows may continue below
the ice or may be intermittent in smaller streams or dry climates.
Freezing reduces resource availability by reducing nutrient inputs,
allochthonous organic matter and light penetration through the water.
Light may also be attenuated at high latitudes and by high turbidity in
erosional streams. Meltwaters drive increased flow and flooding in
spring and summer. Carbon and nutrient concentrations are greatest
during spring floods, and pH tends to decrease with flow during spring
and autumn. When catchments include extensive peatlands, waters may be
tannin-rich, acidic and dark, thereby reducing light penetration and
productivity.

\subsection{Distribution}\label{distribution-2}

Restricted to boreal, subarctic, alpine and subalpine regions, with
limited examples in the subantarctic and Antarctic.

\section{F1.4 Seasonal upland
streams}\label{f1.4-seasonal-upland-streams}

Belongs to biome F1. Rivers and streams biome, part of the Freshwater
realm.

\subsection{Short description}\label{short-description-3}

Seasonal rainfall patterns in large parts of the tropics and temperate
regions generate flows that are hugely variable in narrow and steep
upland streams. Globally, these streams account for the greatest stream
length of any flowing ecosystem. During the dry season, flows in some
streams are reduced to very levels, while in others flow ceases
altogether and water persists only in isolated stagnant pools. Algae and
leaf fall support moderate productivity, with seasonal floods sending
organic matter downstream. The diversity of organisms fluctuates
seasonally, with many localised (endemic) species, and specialised
adaptations that enable animals to survive both flooding and dry
conditions.

\subsection{Key Features}\label{key-features-3}

High-medium velocity, low-medium volume, highly seasonal flows with
abundant benthic filter feeders, algal biofilms \& small fish.

\subsection{Ecological traits}\label{ecological-traits-3}

Upland streams (orders 1-4) with highly seasonal flows generally have
low to moderate productivity and a simpler trophic structure than
lowland rivers. They tend to be shallow, hence benthic algae are major
contributors to in-stream food webs and productivity, but riparian zones
and catchments both contribute allochthonous energy and organic carbon
through leaf fall, which may include an annual deciduous component.
Primary production also varies with light availability and flow.
Taxonomic diversity varies between streams, but can be lower than
permanent streams and relatively high in endemism. Traits that enable
biota to persist in narrow and shallow channels with large seasonal
variations in flow velocity, episodes of torrential flow, and seasonal
desiccation include small body sizes (especially in resident fish),
dormant life phases and/or burrowing (crustaceans), omnivorous diets and
high dispersal ability, including seasonal migration. Compared to
lowland rivers, the trophic structure has a higher representation of
algal and omnivorous feeders and low numbers of larger predators. Birds
show specialist feeding strategies (e.g.~dippers). Diversity and
abundance of invertebrates and their predators (e.g.~birds) fluctuate in
response to seasonal flood regimes.

\subsection{Key Ecological Drivers}\label{key-ecological-drivers-3}

Flow and flood regimes in these rivers are highly variable between
marked wet and dry seasons, with associated changes in water quality as
solute concentration varies with volume. They may be perennial, with
flows much-reduced in the dry season, or seasonally intermittent with
flows ceasing and water persisting in isolated stagnant pools. Channels
are narrow with steep to moderate gradients, seasonally high velocity
and sometimes large volumes of water, producing overbank flows. This
results in considerable turbulence, turbidity, and erosion during the
wet season and coarse substrates (cobbles and boulders). Seasonal floods
are critical to allochtonous subsidies and downstream exports of organic
matter and nutrients.

\subsection{Distribution}\label{distribution-3}

Elevated regions in seasonal tropical, subtropical and temperate
climates worldwide.

\section{F1.5 Seasonal lowland
rivers}\label{f1.5-seasonal-lowland-rivers}

Belongs to biome F1. Rivers and streams biome, part of the Freshwater
realm.

\subsection{Short description}\label{short-description-4}

These medium to large rivers in tropical, subtropical and temperate
lowlands have markedly seasonal flows due to seasonal water supply in
the catchments. Their single or multi-channelled forms link to
floodplain wetlands, and transport large floods during wet seasons:
summer in the tropics or winter-spring in temperate latitudes.
Productivity is high, both within channels and on connected floodplains,
with algae and aquatic plants supporting complex food webs, and
providing seasonal nurseries for breeding animals.

\subsection{Key Features}\label{key-features-4}

Highly productive large rivers with seasonal hydrology large floodplain
subsidies. Short food chains support large mobile predaors.

\subsection{Ecological traits}\label{ecological-traits-4}

These large riverine systems (stream orders 5-9) can be highly
productive with trophic structures and processes shaped by seasonal
hydrology and linkages to floodplain wetlands. In combination with
biophysical heterogeneity, this temporal variability promotes functional
diversity in the biota. Although trophic networks are complex due to the
diversity of food sources and the extent of omnivory amongst consumers,
food chains tend to be short and large mobile predators such as otters,
large piscivorous waterbirds, sharks, dolphins, and crocodilians (in the
tropics) can have a major impact on the food webs. Benthic algae are key
contributors to primary productivity, although macrophytes become more
important during the peak and late wet season when they also provide
substrate for epiphytic algae. Rivers receive very significant resource
subsidies from both algae and macrophytes on adjacent floodplains when
they are connected by flows. Enhanced longitudinal hydrological
connectivity during the wet season enables fish and other large aquatic
consumers to function as mobile links, extending floodplain and
estuarine resource subsidies upstream. Life cycle processess including
reproduction, recruitment, and dispersal in most biota are tightly cued
to seasonally high flow periods, often with floodplain nursery areas for
river fish, amphibians and larger invertebrates.

\subsection{Key Ecological Drivers}\label{key-ecological-drivers-4}

These rivers are driven by cyclical, seasonal flow regimes. High-volume
flows and floods occur during summer in the tropics or winter-spring at
temperate latitudes, with two peaks in some areas. A decline of flows
and reduced flood residence times during the transition to the dry
season is followed by low and disconnected flows during the dry season.
Turbidity, light availability, erosion, sedimentation, lateral and
longitudinal connectivity, biological activity, dissolved oxygen and
solute concentrations all vary with this seasonal cycle. The
inter-annual variability of this pattern depends on the catchment
precipitation and sources of inflow that offset or mute the influences
of rainfall seasonality (e.g.~snow melt in South Asia). Streams may be
single, multi-channelled or complex anabranching systems.

\subsection{Distribution}\label{distribution-4}

Tropical, subtropical and temperate lowlands with seasonal inflow
patterns worldwide.

\section{F1.6 Episodic arid rivers}\label{f1.6-episodic-arid-rivers}

Belongs to biome F1. Rivers and streams biome, part of the Freshwater
realm.

\subsection{Short description}\label{short-description-5}

These desert rivers occur mostly in flat areas of arid and semi-arid
mid-latitudes. Channels are typically broad, flat, and often branching,
with soft sandy sediments. They are dry most of the time, but punctuated
by high-volume, short duration flows that transport nutrients and
stimulate high productivity by algae and zooplankton. Plants and animals
can either tolerate or avoid long, dry periods and then exploit short
pulses of abundant resources, producing hotspots of biodiversity and
ecological activity in arid landscapes.

\subsection{Key Features}\label{key-features-5}

Rivers with high temporal flow variability which determines periods of
high and low productivity, supporting high levels of biodiversity and
complex trophic networks during floods and simple trophic networks
during dry periods.

\subsection{Ecological traits}\label{ecological-traits-5}

Episodic rivers have high temporal variability in flows and resource
availability, shaping a low-diversity biota with periodically high
abundance of some organisms. Productivity is episodically high and
punctuated by longer periods of low productivity (i.e.~boom-bust
dynamics). The trophic structure can be complex and dominated by
autochthonous primary production. Even though riparian vegetation is
sparse, allochthonous inputs from connected floodplains may be
important. Top-down control of ecosystem structure is evident in some
desert streams. Episodic rivers are hotspots of biodiversity and
ecological activity in arid landscapes, acting as both evolutionary and
ecological refuges. Most biota have ruderal life cycles, dormancy
phases, or high mobility enabling them to tolerate or avoid long, dry
periods and to exploit short pulses of high resource availability during
flooding. During dry periods, many organisms survive as dormant life
phases (e.g.~eggs or seeds), by reducing metabolism, or by persisting in
perennial refugia (e.g.~waterholes, shallow aquifers). They may rapidly
recolonise the channel network during flow (networkers). Waterbirds
survive dry phases by moving elsewhere, returning to breed during flows.
The abundance of water, nutrients and food during flows and floods
initiates rapid primary production (especially by algae), breeding and
recruitment. Zooplankton are abundant in slower reaches during periods
of flow. Macroinvertebrates such as sessile filter-feeders
(e.g.~mussels) and scavengers (e.g.~crayfish) may occur in moderate flow
environments with complex microhabitats in fine sediment and amongst
woody debris. Assemblages of fish and amphibians are dominated by small
body sizes. Most fish species use inundated floodplains in larval,
juvenile and mature life stages, and produce massive biomass after large
floods. Organisms generally tolerate wide ranges of temperature,
salinity, and oxygen.

\subsection{Key Ecological Drivers}\label{key-ecological-drivers-5}

These mostly lowland systems are distinguished by highly episodic flows
and flood regimes that vary with catchment size and precipitation.
High-volume, short duration flows (days to weeks, rarely months)
punctuate long dry periods fill channels and flood wetlands. Low
elevational gradients and shallow channels result in low turbulence and
low to moderate flow velocity. Lowland stream channels are broad, flat,
and often anastomising, with mostly soft sandy sediments. Groundwater is
usually within rooting zones of perennial plants, which may establish in
channels after flow events. Sediment loads drive periodically high
turbidity. Locally or temporally important erosional processes have
roles in geomorphic dynamism redistributing sediment in depositional
features (e.g.~braided channels and point bars). Upland streams are
prone to erosive flash floods. High nutrient levels are due to large
catchments and riparian inputs but depend on catchment geochemistry.
These rivers often flow over naturally saline soils. Salinity can thus
be high and increases in drying phases.

\subsection{Distribution}\label{distribution-5}

Arid and semi-arid mid-latitudes, in lowlands, and some uplands, but
rarely above 1,500 m elevation.

\section{F1.7 Large lowland rivers}\label{f1.7-large-lowland-rivers}

Belongs to biome F1. Rivers and streams biome, part of the Freshwater
realm.

\subsection{Short description}\label{short-description-6}

These very large rivers transport massive volumes of freshwater
(\textgreater10,000m3/s) through flat lowlands, mostly in tropical or
subtropical regions. Their very large flow volumes, diverse habitats and
slow to moderate flows make them highly productive. High nutrient levels
come from upstream catchments and floodplains, with additional
productivity contributed by in-channel algae and aquatic plants. Their
food webs are complex, with a high diversity of plants and animals,
including large-bodied fish, reptiles and mammals.

\subsection{Key Features}\label{key-features-6}

Large highly productive rivers with megaflow rates and complex food
webs, reflecting the extent of habitat, connections with floodplains and
available niches for plants, invertebrates and large vertebrates
including aquatic mammals..

\subsection{Ecological traits}\label{ecological-traits-6}

Large lowland rivers (typically stream orders 8-12) are highly
productive environments with complex trophic webs which are supported by
very large flow volumes. Primary production is mostly from autochthonous
phytoplankton and riparian macrophytes, with allochthonous inputs from
floodplains and upper catchments generally dominating energy flow in the
system. The fauna includes a significant diversity of pelagic organisms.
Zooplankton are abundant, while sessile (e.g.~mussels), burrowing
(e.g.~annelids) and scavenging (e.g.~crustaceans) macroinvertebrates
occur in the fine sediment and amongst woody debris. Fish communities
are diverse and contribute to complex trophic networks. They include
large predatory fish (e.g.~freshwater sawfish, Pirhana, Alligator Gar)
and in some rivers endemic River Dolphins, smaller predators of
invertebrates (benthic and pelagic feeders), phytoplankton herbivores,
and detritivores. The feeding activities and movement of semi-aquatic
piscivorous birds (e.g.~cormorants), mammals (e.g.~otters), and reptiles
(e.g.~turtles, crocodilians) connect the trophic network to other
ecosystems beyond instream waters. Riparian and large floodplain zones
vary in complexity from forested banks, to productive lentic oxbow lakes
and extensive and complex flooded areas where emergent and floodplain
vegetation grows (e.g.~reeds and macrophytes, shrubs, trees). Riparian
zones can be complex but have less direct influence on large rivers than
on smaller river ecosystems.

\subsection{Key Ecological Drivers}\label{key-ecological-drivers-6}

These rivers have shallow gradients with low turbulence, low to moderate
flow velocity and very high flow volumes (\textgreater10,000m3/s), which
are continuous but may vary seasonally depending on catchment area and
precipitation (e.g.~Congo up to 41,000 m3/s, Amazon up to 175,000 m3/s).
River channels are wide (e.g.~Amazon River; 11 km in dry season, up to
25km when flooded at its widest point) and deep (e.g.~Congo up to 200m;
Mississippi up to 60m) with mostly soft sediment substrates. They are
dominated by depositional processes so turbidity may be high. Overbank
flows increase turbulence and turbidity. Locally or temporally important
erosional processes redistribute sediment and produce geomorphically
dynamic depositional features (e.g.~braided channels, islands and point
bars). Nutrient levels are high due to large catchments and
riparian/floodplain inputs but vary with catchment geochemistry.
Moderate water temperatures are buffered due to large catchments.

\subsection{Distribution}\label{distribution-6}

Tropical and subtropical lowlands, with a few extending to temperate
zones. They are absent from arid regions, and in boreal zones are
replaced by F1.3.

\section{F2.1 Large permanent freshwater
lakes}\label{f2.1-large-permanent-freshwater-lakes}

Belongs to biome F2. Lakes biome, part of the Freshwater realm.

\subsection{Short description}\label{short-description-7}

Large volumes of permanent water in these lakes buffers water
temperatures and effects of nutrient input on water quality. The spatial
extent and range of habitats support very large numbers of species in
some groups such as fish, some of which are unique to a single lake,
often composed of closely related species, endemic to a lake. The high
primary productivity from algae and aquatic plants lake supports diverse
foodwebs. High numbers of plankton support large numbers of waterbugs,
fish, frogs, reptiles, waterbirds, and mammals. Bacteria play key roles
in cycling organic matter.

\subsection{Key Features}\label{key-features-7}

Large (usually \textgreater100km2) permanent freshwater lakes connected
to rivers, with high spatial and bathymetric niche diversity supporting
complex trophic networks supported by planktonic algae, high diversity
and endemism.

\subsection{Ecological traits}\label{ecological-traits-7}

Large permanent freshwater lakes, generally exceeding 100 km2, are
prominent landscape features connected to one or more rivers either
terminally or as flow-through systems. Shoreline complexity, depth,
bathymetric stratification, and benthic topography promote niche
diversity and zonation. High niche diversity and large volumes of
permanent water (extensive, stable, connected habitat) support complex
trophic webs with high diversity and abundance. High primary
productivity may vary seasonally, driving succession, depending on
climate, light availability, and nutrient regimes. Autochthonous energy
from abundant pelagic algae (mainly diatoms and cyanobacteria) and from
benthic macrophytes and algal biofilms (in shallow areas) is
supplemented by allochthonous inflows that depend on catchment
characteristics, climate, season, and hydrological connectivity.
Zooplankton, invertebrate consumers, and herbivorous fish sustain high
planktonic turnover and support upper trophic levels with abundant and
diverse predatory fish, amphibians, reptiles, waterbirds, and mammals.
This bottom-up web is coupled to a microbial loop, which returns
dissolved organic matter to the web (rapidly in warm temperatures) via
heterotrophic bacteria. Obligate freshwater biota in large lakes,
including aquatic macrophytes and macroinvertebrates (e.g.~crustaceans)
and fish, often display high catchment-level endemism, in part due to
long histories of environmental variability in isolation. Marked niche
differentiation in life history and behavioural feeding and reproductive
traits enables sympatric speciation and characterises the most diverse
assemblages of macroinvertebrates and fish (e.g.~\textasciitilde500
cichlid fish species in Lake Victoria). Large predators are critical in
top-down regulation of lower trophic levels. Large lake volume buffers
against nutrient-mediated change from oligotrophic to eutrophic states.
Recruitment of many organisms is strongly influenced by physical
processes such as large inflow events. Mobile birds and terrestrial
mammals use the lakes as breeding sites and/or sources of drinking water
and play key roles in the inter-catchment transfer of nutrients and
organic matter and the dispersal of biota.

\subsection{Key Ecological Drivers}\label{key-ecological-drivers-7}

Large water volumes influence resource availability, environmental
stability (through thermal buffering), and niche diversity. Water is
from catchment inflows, which may vary seasonally with climate. Large
lakes influence regional climate through evaporation, cooling, and
convection feedbacks. These processes also influence nutrient
availability, along with catchment and lake substrates and vertical
mixing. Mixing may be monomictic (i.e.~annual) or meromictic
(i.e.~seldom), especially in large tropical lakes, depending on inflow,
depth, wind regimes, and seasonal temperature variation. Light varies
with lake depth, turbidity, cloud cover, and latitude.

\subsection{Distribution}\label{distribution-7}

Humid temperate and tropical regions on large land masses.

\section{F2.10 Subglacial lakes}\label{f2.10-subglacial-lakes}

Belongs to biome F2. Lakes biome, part of the Freshwater realm.

\subsection{Short description}\label{short-description-8}

These hidden lakes exist beneath permanent ice sheets, sometimes tens to
thousands of metres below, mostly in Antarctica and Greenland. Bacteria
and other microbes are the only forms of life, but there is a surprising
diversity of them. Productivity is very low in the dark and freezing
conditions, with species relying on metabolism of chemicals such as
methane, iron and sulphur to support the simple foodweb.

\subsection{Key Features}\label{key-features-8}

Lakes beneath permanent ice sheets with a truncated microbial food web,
including chemoautotrophic and heterotrophic of bacteria and archaea.

\subsection{Ecological traits}\label{ecological-traits-8}

Remarkable lacustrine ecosystems occur beneath permanent ice sheets.
They are placed within the Lakes biome (biome F2) due to their
relationships with some Freeze-thaw lakes (F2.4), but they share several
key features with the Subterranean freshwater biome (biome SF1).
Evidence of their existence first emerged in 1973 from airborne
radar-echo sounding imagery, which penetrates the ice cover and shows
lakes as uniformly flat structures with high basal reflectivity. The
biota of these ecosystems is very poorly known due to technological
limitations on access and concerns about the risk of contamination from
coring. Only a few shallow lakes up to 1 km beneath ice have been
surveyed (e.g.~Lake Whillams in West Antarctica and Grímsvötn Lake in
Iceland). The exclusively microbial trophic web is truncated, with no
photoautotrophs and apparently few multi-cellular predators, but
taxonomic diversity is high across bacteria and archaea, with some
eukaryotes also represented. Chemosynthesis form the base of the trophic
web, chemolithoautotrophic species using reduced N, Fe and S and methane
in energy-generating metabolic pathways. The abundance of
micro-organisms is comparable to that in groundwater (SF1.2) (104 -- 105
cells.ml-1), with diverse morphotypes represented including long and
short filaments, thin and thick rods, spirals, vibrio, cocci and
diplococci. Subglacial lakes share several biotic traits with
extremophiles within ice (T6.1), subterranean waters (SF1.1, SF1.2) and
deep oceans (e.g.~M2.3, M2.4, M3.3), including very low productivity,
slow growth rates, large cell sizes and aphotic energy synthesis.
Although microbes of the few surveyed subglacial lakes, and from
accreted ice which has refrozen from lake water, have DNA profiles
similar to those of other contemporary microbes, the biota in deeper
disconnected lake waters and associated lake-floor sediments, could be
highly relictual if it evolved in stable isolation over millions of
years under extreme selection pressures.

\subsection{Key Ecological Drivers}\label{key-ecological-drivers-8}

Subglacial lakes vary in size from less than 1 km2 to
\textasciitilde10,000 km2, and most are 10-20 m deep, but Lake Vostok
(Antarctica) is at least 1,000 m deep. The environment is characterised
by high isostatic pressure (up to \textasciitilde350 atmospheres),
constant cold temperatures marginally below 0°C, low-nutrient levels,
and an absence of sunlight. Oxygen concentrations can be high due to
equilibration with gas hydrates from the melting ice sheet base ice, but
declines with depth in amictic lakes due to limited mixing, depending on
convection gradients generated by cold meltwater from the ice ceiling
and geothermal heating from below. Chemical weathering of basal debris
is the main source of nutrients supplemented by ice melt.

\subsection{Distribution}\label{distribution-8}

Some \textasciitilde400 subglacial lakes in Antarctica,
\textasciitilde60 in Greenland and a few in Iceland and Canada have been
identified from radar remote sensing and modelling.

\section{F2.2 Small permanent freshwater
lakes}\label{f2.2-small-permanent-freshwater-lakes}

Belongs to biome F2. Lakes biome, part of the Freshwater realm.

\subsection{Short description}\label{short-description-9}

With a surface area of up to 100 km2, the diversity of small permanent
lakes, ponds and pools depends on their size, depth and connectivity.
Littoral vegetation and benthic energy pathways are critical to
productivity and food web complexity. Deep lakes have plankton,
supporting fish, birds and frogs, in different habitats of the lake.
Shallow lakes are often more productive, providing breeding habitat for
birds, frogs and reptiles, but limited buffering against nutrient inputs
may result in regime shifts between alternative stable states dominated
either by large aquatic plants or phytoplankton.

\subsection{Key Features}\label{key-features-9}

Small permanent freshwater lakes or ponds with niche diversity strongly
related to size and depth, and resource subsidies from catchments.
Littoral zones and benthic macrophytes are important contributors to
productivity.

\subsection{Ecological traits}\label{ecological-traits-9}

Small permanent freshwater lakes, pools or ponds are lentic environments
with relatively high perimeter-to-surface area and
surface-area-to-volume ratios. Most are \textless1 km2 in area, but this
functional group includes lakes of transitional sizes up to 100 km2,
while the largest lakes (\textgreater100 km2) are classified in F2.1.
Niche diversity increases with lake size. Although less diverse than
larger lakes, these lakes may support phytoplankton, zooplankton,
shallow-water macrophytes, invertebrates, sedentary and migratory fish,
reptiles, waterbirds, and mammals. Primary productivity, dominated by
cyanobacteria, algae, and macrophytes, arises from allochthonous and
autochthonous energy sources, which vary with lake and catchment
features, climate, and hydrological connectivity. Productivity can be
highly seasonal, depending on climate, light, and nutrients. Permanent
water and connectivity are critical to obligate freshwater biota, such
as fish, invertebrates, and aquatic macrophytes. Trophic structure and
complexity depend on lake size, depth, location, and connectivity.
Littoral zones and benthic pathways are integral to overall production
and trophic interactions. Shallow lakes tend to be more productive (by
volume and area) than deep lakes because light penetrates to the bottom,
establishing competition between benthic macrophytes and phytoplankton,
more complex trophic networks and stronger top-down regulation leading
to alternative stable states and possible regime shifts between them.
Clear lakes in macrophyte-dominated states support higher biodiversity
than phytoplankton-dominated eutrophic lakes. Deep lakes are more
dependent on planktonic primary production, which supports zooplankton,
benthic microbial and invertebrate detritivores. Herbivorous fish and
zooplankton regulate the main primary producers (biofilms and
phytoplankton). The main predators are fish, macroinvertebrates,
amphibians and birds, many of which have specialised feeding traits tied
to different habitat niches (e.g.~benthic or pelagic), but there are few
filter-feeders. In many regions, shallow lakes provide critical breeding
habitat for waterbirds, amphibians, and reptiles, while visiting mammals
transfer nutrients, organic matter, and biota.

\subsection{Key Ecological Drivers}\label{key-ecological-drivers-9}

These lakes may be hydrologically isolated, groundwater-dependent or
connected to rivers as terminal or flow-through systems. Nutrients
depend on catchment size and substrates. Some lakes (e.g.~on leached
coastal sandplains or peaty landscapes) have dystrophic waters. The
seasonality and amount of inflow, size, depth (mixing regime and light
penetration), pH, nutrients, salinity, and tanins shape lake ecology and
biota. Seasonal cycles of temperature, inflow and wind (which drives
vertical mixing) may generate monomictic or dimictic temperature
stratification regimes in deeper lakes, while shallow lakes are
polymicitic, sometimes with short periods of multiple stratification.
Seasonal factors such as light, increases in temperature, and flows into
lakes can induce breeding and recruitment.

\subsection{Distribution}\label{distribution-9}

Mainly in humid temperate and tropical regions, rarely semi-arid or arid
zones.

\section{F2.3 Seasonal freshwater
lakes}\label{f2.3-seasonal-freshwater-lakes}

Belongs to biome F2. Lakes biome, part of the Freshwater realm.

\subsection{Short description}\label{short-description-10}

Small seasonal lakes, pools and rock holes have plants and animals
specialised to seasonally changing wet and dry conditions in temperate
and wet-dry tropical regions. Their energy comes mostly from algae and
plants. To survive the annual wet/dry cycles, the animals and plants
have dormant life stages, such as eggs or seeds, within the lake
sediments, or they shelter in damp burrows or other refuges. Plants and
animals can build up high abundances during wet seasons, supporting
plankton and waterbugs, frogs, birds and mammals but, in most cases, no
fish.

\subsection{Key Features}\label{key-features-10}

Mostly small and shallow well mixed freshwater lakes with seasonal
patterns of filling and seasonally variable abundance and composition of
aquatic biota, including species with dormant life phases and some that
retreat to refuges in dry seasons.

\subsection{Ecological traits}\label{ecological-traits-10}

These small (mostly \textless5 km2 in area) and shallow (\textless2 m
deep) seasonal freshwater lakes, vernal pools, turloughs, or gnammas
(panholes, rock pools), have a seasonal aquatic biota. Hydrological
isolation promotes biotic insularity and local endemism, which occurs in
some Mediterranean climate regions. Autochthonous energy sources are
supplemented by limited allochthonous inputs from small catchments and
groundwater. Seasonal variation in biota and productivity outweighs
inter-annual variation, unlike in ephemeral lakes (F2.5 and F2.7).
Filling induces microbial activity, the germination of seeds and algal
spores, hatching and emergence of invertebrates, and growth and
reproduction by specialists and opportunistic colonists. Wind-induced
mixing oxygenates the water, but eutrophic or unmixed waters may become
anoxic and dominated by air-breathers as peak productivity and biomass
fuel high biological oxygen demand. Anoxia may be abated diurnally by
photosynthetic activity. Resident biota persists through seasonal drying
on lake margins or in sediments as desiccation-resistant dormant or
quiescent life stages, e.g.~crayfish may retreat to burrows that extend
to the water table, turtles may aestivate in sediments or fringing
vegetation, amphibious perennial plants may persist on lake margins or
in seedbanks. Trophic networks and niche diversity are driven by
bottom-up processes, especially submerged and emergent macrophytes, and
depend on productivity and lake size. Cyanobacteria, algae, and
macrophytes are the major primary producers, while annual grasses may
colonise dry lake beds. The most diverse lakes exhibit zonation and
support phytoplankton, zooplankton, macrophytes, macroinvertebrate
consumers, and seasonally resident amphibians (especially juvenile
aquatic phases), waterbirds, and mammals. Rock pools have simple trophic
structure, based primarily on epilithic algae or macrophytes, and
invertebrates, but no fish. Invertebrates and amphibians may reach high
diversity and abundance in the absence of fish.

\subsection{Key Ecological Drivers}\label{key-ecological-drivers-10}

Seasonal rainfall, surface flows, groundwater fluctuation and seasonally
high evapo-transpiration drive annual filling and drying. These lakes
are polymicitc, mixing continuously when filled. Impermeable substrates
(e.g.~clay or bedrock) impede infiltration in some lakes; in others
groundwater percolates up through sand, peat or fissures in karstic
limestone (turloughs). Small catchments, low-relief terrain, high
area-to-volume ratios, and hydrological isolation promote seasonal
fluctuation. Most lakes are hydrologically isolated, but some become
connected seasonally by sheet flows or drainage lines. These
hydrogeomorphic features also limit nutrient supply, in turn limiting pH
buffering. Water fluctuations drive high rates of organic decomposition,
denitrification, and sediment retention. High alkalinity reflects high
anaerobic respiration. Groundwater flows may ameliorate hydrological
isolation. Seasonal filling and drying induce spatio-temporal
variability in temperature, depth, pH, dissolved oxygen, salinity, and
nutrients, resulting in zonation within lakes and high variability among
them.

\subsection{Distribution}\label{distribution-10}

Subhumid temperate and wet-dry tropical regions in monsoonal and
Mediterranean-type climates but usually not semi-arid or arid regions.

\section{F2.4 Freeze-thaw freshwater
lakes}\label{f2.4-freeze-thaw-freshwater-lakes}

Belongs to biome F2. Lakes biome, part of the Freshwater realm.

\subsection{Short description}\label{short-description-11}

Many plants and animals survive surface freezing of freshwater lakes, in
dormant life stages, by reducing activity beneath the ice, or by moving.
Such freshwater lakes vary enormously in size and distribution,
providing a wide range of habitats for many organisms, which undergo a
succession of emergence during lake thaw. The annual thaw triggers
highly productive plant and animal activity, beginning with diatom algae
and then zooplankton. Habitat diversity increases with lake size,
increasing the variety of plankton, aquatic plants, waterbugs, birds,
and sometimes fish.

\subsection{Key Features}\label{key-features-11}

Waterbodies with frozen surfaces for at least one month of the year,
with spring thaw initiating trophic successional dynamics beginning with
a flush of diatom productivity. Deeper lakes may be cold stratified and
fish tolerate oxygen depletion in winter.

\subsection{Ecological traits}\label{ecological-traits-11}

The majority of the surface of these lakes is frozen for at least a
month in most years. Their varied origins (tectonic, riverine,
fluvioglacial), size and depth affect composition and function.
Allochthonous and autochthonous energy sources vary with lake and
catchment features. Productivity is highly seasonal, sustained in winter
largely by the metabolism of microbial photoautotrophs, chemautotrophs
and zooplankton that remain active under low light, nutrients, and
temperatures. Spring thaw initiates a seasonal succession, increasing
productivity and re-establishing complex trophic networks, depending on
lake area, depth, connectivity, and nutrient availability. Diatoms are
usually first to become photosynthetically active, followed by small and
motile zooplankton, which respond to increased food availability, and
cyanobacteria later in summer when grazing pressure is high. Large lakes
with high habitat complexity (e.g.~Lake Baikal) support phytoplankton,
zooplankton, macrophytes (in shallow waters), invertebrate consumers,
migratory fish (in connected lakes), waterbirds, and mammals. Their
upper trophic levels are more abundant, diverse, and endemic than in
smaller lakes. Herbivorous fish and zooplankton are significant top-down
regulators of the main primary producers (i.e.~biofilms and
phytoplankton). These, in turn, are regulated by predatory fish, which
may be limited by prey availability and competition. The biota is
spatially structured by seasonally dynamic gradients in cold
stratification, light, nutrient levels, and turbulence. Traits such as
resting stages, dormancy, freeze-cued spore production in phytoplankton,
and the ability of fish to access low oxygen exchange enable persistence
through cold winters under the ice and through seasonal patterns of
nutrient availability.

\subsection{Key Ecological Drivers}\label{key-ecological-drivers-11}

Seasonal freeze-thaw cycles typically generate dimictic temperature
stratification regimes (i.e.~mixing twice per year), where cold water
lies above warm water in winter and vice versa in summer. Shallow lakes
may mix continuously (polymicitic) during the summer and may freeze
completely during winter. Mixing occurs in autumn and spring. Freezing
reduces light penetration and turbulence, subduing summer depth
gradients in temperature, oxygen, and nutrients. Ice also limits
atmospheric inputs, including gas exchange. Very low temperatures reduce
the growth rates, diversity, and abundance of fish. Many lakes are
stream sources. Lake sizes vary from \textless1 ha to more than 30,000
km2, profoundly affecting niche diversity and trophic complexity.
Freezing varies with the area and depth of lakes. Thawing is often
accompanied by flooding in spring, ameliorating light and temperature
gradients, and increasing mixing. Dark-water inflows from peatlands in
catchments influence water chemistry, light penetration, and
productivity.

\subsection{Distribution}\label{distribution-11}

Predominantly across the high latitudes of the Northern Hemisphere and
high altitudes of South America, New Zealand and Tasmania.

\section{F2.5 Ephemeral freshwater
lakes}\label{f2.5-ephemeral-freshwater-lakes}

Belongs to biome F2. Lakes biome, part of the Freshwater realm.

\subsection{Short description}\label{short-description-12}

These are shallow lakes that are mostly dry, and then fill for weeks or
months, before drying again. During dry periods, many animals or plants
survive as eggs, seeds or other dormant forms, while other species
disperse. Floods, bring water from surrounding catchments and
floodplains with organic matter, nutrients and fine sediments, and
trigger movement of birds and mammals. Floods activate simple foodwebs,
comprising abundant algae, zooplankton, waterbugs and crustaceans, which
have rapid life-cycles able to exploit windows of productivity. This
produces food for frogs and visiting waterbirds.

\subsection{Key Features}\label{key-features-12}

Shallow temporary lakes, depressions or pans with long dry periods of
low productivity, punctuated by episodes of inflow that bring large
resource subsidies from catchments, resulting in high productivity,
population turnover and trophic connectivity.

\subsection{Ecological traits}\label{ecological-traits-12}

Shallow ephemeral freshwater bodies are also known as depressions,
playas, clay pans, or pans. Long periods of low productivity during dry
phases are punctuated by episodes of high production after filling.
Trophic structure is relatively simple with mostly benthic, filamentous,
and planktonic algae, detritivorous and predatory zooplankton
(e.g.~rotifers and Daphnia), crustaceans, insects, and in some lakes,
molluscs. The often high invertebrate biomass provides food for
amphibians and itinerant waterbirds. Terrestrial mammals use the lakes
to drink and bathe and may transfer nutrients, organic matter, and
'hitch-hiking' biota. Diversity may be high in boom phases but there are
only a few local endemics (e.g.~narrow-ranged charophytes). Specialised
and opportunistic biota exploit boom-bust resource availability through
life-cycle traits that confer tolerance to desiccation
(e.g.~desiccation-resistant eggs in crustaceans) and/or enable rapid
hatching, development, breeding, and recruitment when water arrives.
Much of the biota (e.g.~opportunistic insects) have widely dispersing
adult phases enabling rapid colonisation and re-colonisation. Filling
events initiate succession with spikes of primary production, allowing
short temporal windows for consumers to grow and reproduce, and for
itinerant predators to aggregate. Drying initiates senescence,
dispersal, and dormancy until the next filling event.

\subsection{Key Ecological Drivers}\label{key-ecological-drivers-12}

Arid climates have highly variable hydrology. Episodic inundation after
rain is relatively short (days to months) due to high evaporation rates
and infiltration. Drainage systems are closed or nearly so, with
channels or sheet inflow from flat, sparsely vegetated catchments.
Inflows bring allochthonous organic matter and nutrients and are
typically turbid with fine particles. Clay-textured lake bottoms hold
water by limiting percolation but may include sand particles. Bottom
sediments release nutrients rapidly after filling. Lakes are shallow,
flat-bottomed and polymicitic when filled with small volumes, so light
and oxygen are generally not limiting. Persistent turbidity may limit
light but oxygen production by macrophytes and flocculation
(i.e.~clumping) from increasing salinity during drying reduce turbidity
over time. Shallow depth promotes high daytime water temperatures (when
filling in summer) and high diurnal temperature variability.

\subsection{Distribution}\label{distribution-12}

Semi-arid and arid regions at mid-latitudes of the Americas, Africa,
Asia, and Australia.

\section{F2.6 Permanent salt and soda
lakes}\label{f2.6-permanent-salt-and-soda-lakes}

Belongs to biome F2. Lakes biome, part of the Freshwater realm.

\subsection{Short description}\label{short-description-13}

These lakes are usually large and shallow in semi-arid regions, with
high concentrations of salts, mediated by inflows of water. Their
productivity from growth of algae and plants can support large numbers,
but low diversity of organisms equipped with tolerance to high salinity
and other solutes. They have relatively simple foodwebs, with high
numbers of microbes and plankton, crustaceans, insect larvae, fish and
specialised waterbirds such as flamingos.

\subsection{Key Features}\label{key-features-13}

Permanent waterbodies with high inorganic solute concentrations
(particularly sodium), supporting simple trophic networks, including
cyanobacteria and algae, invertebrates and specialist birds.

\subsection{Ecological traits}\label{ecological-traits-13}

Permanent salt lakes have waters with periodically or permanently high
sodium chloride concentrations. This group includes lakes with high
concentrations of other ions (e.g.~carbonate in soda lakes). Unlike in
hypersaline lakes, productivity is not suppressed and autotrophs may be
abundant, including phytoplankton, cyanobacteria, green algae, and
submerged and emergent macrophytes. These, supplemented by allochthonous
energy and C inputs from lake catchments, support relatively simple
trophic networks characterised by few species in high abundance and some
regional endemism. The high biomass of archaeal and bacterial
decomposers and phytoplankton in turn supports abundant consumers
including brine shrimps, copepods, insects and other invertebrates,
fish, and waterbirds (e.g.~flamingos). Predators and herbivores that
become dominant at low salinity exert top-down control on algae and
low-order consumers. Species niches are structured by spatial and
temporal salinity gradients. Species in the most saline conditions tend
to have broader ranges of salinity tolerance. Increasing salinity
generally reduces diversity and the importance of top-down trophic
regulation but not necessarily the abundance of organisms, except at
hypersaline levels. Many organisms tolerate high salinity through
osmotic regulation (at a high metabolic cost), limiting productivity and
competitive ability.

\subsection{Key Ecological Drivers}\label{key-ecological-drivers-13}

Permanent salt lakes tend to be large and restricted to semi-arid
climates with high evaporation but with reliable inflow sources
(e.g.~snowmelt). They may be thousands of hectares in size and several
metres deep. A few are much larger and deeper (e.g.~Caspian Sea), while
some volcanic lakes are small and deep. Endorheic drainage promotes salt
accumulation, but lake volume and reliable water inflows are critical to
buffering salinity below extreme levels. Salinity varies temporally from
0.3\% to rarely more than 10\% depending on lake size, temperature, and
the balance between freshwater inflows, precipitation, and evaporation.
Inflow is critical to ecosystem dynamics, partly by driving the indirect
effects of salinity on trophic or engineering processes. Within lakes,
salt concentrations may be vertically stratified (i.e.~meromictic) due
to the higher density of saltwater compared to freshwater inflow and
slow mixing. Dissolved oxygen is inversely related to salinity, hence
anoxia is common at depth in meromictic lakes. Ionic composition and
concentration varies greatly among lakes due to differences in substrate
and inflow, with carbonate, sulphate, sulphide, ammonia, and/or
phosphorus sometimes reaching high levels, and pH varying from 3 to 11.

\subsection{Distribution}\label{distribution-13}

Mostly in semi-arid regions of Africa, southern Australia, Eurasia, and
western parts of North and South America.

\section{F2.7 Ephemeral salt lakes}\label{f2.7-ephemeral-salt-lakes}

Belongs to biome F2. Lakes biome, part of the Freshwater realm.

\subsection{Short description}\label{short-description-14}

Ephemeral salt lakes in semi-arid and arid regions are shallow, with
extreme variation in salinity during wet-dry cycles that limits life to
a low diversity of specialised salt-tolerant species. The lakes are dry
and salt-encrusted most of the time, but episodic inundation dilutes
salt, allowing high growth of algae and larger plants which support
crustaceans, insect larvae, fish and specialist waterbirds. These
species use dormant life stages to survive drying, or disperse rapidly
to other habitats when the lake dries.

\subsection{Key Features}\label{key-features-14}

Salt lakes with salt crusts in long dry phases and short productive wet
phases. Trophic networks are simple but high productivity is driven by
bacteria and phytoplankton, supporting specialist birds.

\subsection{Ecological traits}\label{ecological-traits-14}

Ephemeral salt lakes or playas have relatively short-lived wet phases
and long dry periods of years to decades. During filling phases, inflow
dilutes salinity to moderate levels, and allochthonous energy and carbon
inputs from lake catchments supplement autochthonous energy produced by
abundant phytoplankton, cyanobacteria, diatoms, green algae, submerged
and emergent macrophytes, and fringing halophytes. In drying phases,
increasing salinity generally reduces diversity and top-down trophic
regulation, but not necessarily the abundance of organisms -- except at
hypersaline levels, which suppress productivity. Trophic networks are
simple and characterised by few species that are often highly abundant
during wet phases. The high biomass of archaeal and bacterial
decomposers and phytoplankton in turn support abundant consumers,
including crustaceans (e.g.~brine shrimps and copepods), insects and
other invertebrates, fish, and specialist waterbirds (e.g.~banded
stilts, flamingos). Predators and herbivores that dominate at low
salinity levels exert top-down control on algae and low-order consumers.
Species niches are strongly structured by spatial and temporal salinity
gradients and endorheic drainage promotes regional endemism. Species
that persist in the most saline conditions tend to have broad salinity
tolerance. Many organisms regulate salinity osmotically at a high
metabolic cost, limiting productivity and competitive ability. Many
specialised opportunists are able to exploit boom-bust resource cycles
through life-cycle traits that promote persistence during dry periods
(e.g.~desiccation-resistant eggs in crustaceans and/or rapid hatching,
development, breeding, and recruitment). Much of the biota (e.g.~insects
and birds) have widely dispersed adult phases enabling rapid
colonisation. Filling events drive specialised succession, with short
windows of opportunity to grow and reproduce reset by drying until the
next filling event.

\subsection{Key Ecological Drivers}\label{key-ecological-drivers-14}

Ephemeral salt lakes are up to 10,000 km2 in area and usually less than
a few metres deep. They may be weakly vertically stratified
(i.e.~meromictic) due to the slow mixing of freshwater inflow with
higher density saltwater. Endorheic drainage promotes salt accumulation.
Salinity varies temporally from 0.3\% to over 26\% depending on lake
size, depth temperature, and the balance between freshwater inflows,
precipitation, and evaporation. Inflow is critical to ecosystem
dynamics, mediates wet-dry phases, and drives the indirect effects of
salinity on trophic and ecosystem processes. Dissolved oxygen is
inversely related to salinity, hence anoxia is common in hypersaline
lake states. Ionic composition varies, with carbonate, sulphate,
sulphide, ammonia, and/or phosphorus sometimes at high levels, and pH
varying from 3 to 11.

\subsection{Distribution}\label{distribution-14}

Mostly in arid and semi-arid Africa, Eurasia, Australia, and North and
South America.

\section{F2.8 Artesian springs and
oases}\label{f2.8-artesian-springs-and-oases}

Belongs to biome F2. Lakes biome, part of the Freshwater realm.

\subsection{Short description}\label{short-description-15}

Surface waterbodies fed by (often warm) groundwaters rising to the
surface are scattered in dry landscapes of Africa, the Middle East,
Eurasia, North America and Australia, but also occur in humid
landscapes. Algae, floating plants and leaf fall support waterbugs,
crustaceans and small fish making simple foodwebs with some locally
restricted species found nowhere else. These ecosystems are sometimes
important waterholes for birds and mammals, in otherwise dry landscapes.

\subsection{Key Features}\label{key-features-15}

Groundwater dependent ecosystems from artesian waters discharged to the
surface, maintaining relatively stable water levels. Often insular
systems with high endemism.

\subsection{Ecological traits}\label{ecological-traits-15}

These groundwater-dependent systems are fed by artesian waters that
discharge to the surface. They are.surrounded by dry landscapes and
receive little surface inflow, being predominantly disconnected from
surface-stream networks. Insularity from the broader landscape results
in high levels of endemism in sedentary aquatic biota, which are likely
descendants of relic species from a wetter past. Springs may be
spatially clustered due to their association with geological features
such as faults or outcropping aquifers. Even springs in close proximity
may have distinct physical and biological differences. Some springs have
outflow streams, which may support different assemblages of plants and
invertebrates to those in the spring orifice. Artesian springs and oases
tend to have simple trophic structures. Autotrophs include aquatic algae
and floating vascular plants, with emergent amphibious plants in shallow
waters. Terrestrial plants around the perimeter contribute subsidies of
organic matter and nutrients through litter fall. Consumers and
predators include crustaceans, molluscs, arachnids, insects, and
small-bodied fish. Most biota are poorly dispersed and have continuous
life cycles and other traits specialised for persistence in
hydrologically stable, warm, or hot mineral-rich water. Springs and
oases are reliable watering points for wide-ranging birds and mammals,
which function as mobile links for resources and promote the dispersal
of other biota between isolated wetlands in the dryland matrix.

\subsection{Key Ecological Drivers}\label{key-ecological-drivers-15}

Flow of artesian water to the surface is critical to these wetlands,
which receive little input from precipitation or runoff. Hydrological
variability is low compared to other wetland types, but hydrological
connections with deep regional aquifers, basin-fill sediments and local
watershed recharge drive lagged flow dynamics. Flows vary over
geological timeframes, with evidence of cyclic growth, waning, and
extinction. Discharge waters tend to have elevated temperatures, are
polymicitic and enriched in minerals that reflect their geological
origins. The precipitation of dissolved minerals (e.g.~carbonates) and
deposition by wind and water form characteristic cones or mounds known
as ``mound springs''. Perennial flows and hydrological isolation from
other spatially and temporally restricted surface waters make these
wetlands important ecological refuges in arid landscapes.

\subsection{Distribution}\label{distribution-15}

Scattered throughout arid regions in southern Africa, the Sahara, the
Middle East, central Eurasia, southwest of North America, and
Australia's Great Artesian Basin.

\section{F2.9 Geothermal pools and
wetlands}\label{f2.9-geothermal-pools-and-wetlands}

Belongs to biome F2. Lakes biome, part of the Freshwater realm.

\subsection{Short description}\label{short-description-16}

Geothermal pools and associated wetlands are fed by deeply circulating
groundwater that mixes with magma and hot rocks in volcanically active
regions. Mineral concentrations are therefore high and produce
chemically precipitated substrates as waters cool. The extreme
temperatures and water chemistry limit life to a low diversity of
specialised bacteria, extensive algal mats and insect larvae which can
live in warm acid or alkaline water with high mineral content. Away from
their hottest waters, aquatic plants, crustaceans, frogs, fish, snakes
and birds can all occur.

\subsection{Key Features}\label{key-features-16}

Hot springs, geysers and mud pots dependent on groundwater interactions
with magma and hot rocks, supporting highly specialised low diversity
biota tolerate of high temperatures and high concentrations of inorganic
salts.

\subsection{Ecological traits}\label{ecological-traits-16}

These hot springs, geysers, mud pots and associated wetlands result from
interactions of deeply circulating groundwater with magma and hot rocks
that produce chemically precipitated substrates. They support a
specialised but low-diversity biota structured by extreme thermal and
geochemical gradients. Energy is almost entirely autochthonous,
productivity is low, and trophic networks are very simple. Primary
producers include chemoautotrophic bacteria and archaea, as well as
photoautotrophic cyanobacteria, diatoms, algae, and macrophytes.
Thermophilic and metallophilic microbes dominate the most extreme
environments in vent pools, while mat-forming green algae and
animal-protists occur in warm acidic waters. Thermophilic blue-green
algae reach optimum growth above 45°C. Diatoms occur in less acidic warm
waters. Aquatic macrophytes occur on sinter aprons and wetlands with
temperatures below 35°C. Herbivores are scarce, allowing thick algal
mats to develop. These are inhabited by invertebrate detritivores,
notably dipterans and coleopterans, which may tolerate temperatures up
to 55°C. Molluscs and crustaceans occupy less extreme microhabitats
(notably in hard water hot springs), as do vertebrates such as
amphibians, fish, snakes and visiting birds. Microinvertebrates such as
rotifers and ostracods are common. Invertebrates, snakes and fish
exhibit some endemism due to habitat insularity. Specialised
physiological traits enabling metabolic function in extreme temperatures
include thermophilic proteins with short amino-acid lengths, chaperone
molecules that assist protein folding, branched chain fatty acids and
polyamines for membrane stabilisation, DNA repair systems, and
upregulated glycolysis providing energy to regulate heat stress. Three
mechanisms enable metabolic function in extremely acidic (pH\textless3)
geothermal waters: proton efflux via active transport pumps that counter
proton influx, decreased permeability of cell membranes to suppress
proton entry into the cytoplasm, and strong protein and DNA repair
systems. Similar mechanisms enable metabolic function in waters with
high concentrations of metal toxins. A succession of animal and plant
communities occur with distance from the spring source as temperatures
cool and minerals precipitate.

\subsection{Key Ecological Drivers}\label{key-ecological-drivers-16}

Continual flows of geothermal groundwater sustain these polymicitic
water bodies. Permanent surface waters may be clear or highly turbid
with suspended solids as in `mud volcanoes'. Water temperatures vary
from hot (\textgreater44°C) to extreme (\textgreater80°C) on local
gradients (e.g.~vent pools, geysers, mounds, sinter aprons, terraces,
and outflow streams). The pH is either extremely acid (2--4) or
neutral-alkaline (7--11). Mineral salts are concentrated, but
composition varies greatly among sites with properties of the underlying
bedrock. Dissolved and precipitated minerals include very high
concentrations of silicon, calcium or iron, but also arsenic, antimony,
copper, zinc, cadmium, lead, polonium or mercury, usually as oxides,
sulphides, or sulphates, but nutrients such as nitrogen and phosphorus
may be scarce.

\subsection{Distribution}\label{distribution-16}

Tectonically or volcanically active areas from tropical to subpolar
latitudes. Notable examples in Yellowstone (USA), Iceland, New Zealand,
Atacama (Chile), Japan and east Africa.

\section{F3.1 Large reservoirs}\label{f3.1-large-reservoirs}

Belongs to biome F3. Artificial wetlands biome, part of the Freshwater
realm.

\subsection{Short description}\label{short-description-17}

Large dams or reservoirs occur in humid, populated areas of the world.
Their biological productivity and diversity is generally limited due to
their depth and frequent and large changes in water level. Shallow zones
have the highest diversity, with simple food webs of algae, waterbugs,
birds, frogs and aquatic plants, often supporting introduced fish
species. Plankton live at the surface, but life is scarce in the depths.
Algal blooms may be common if there are high nutrient inputs from
rivers.

\subsection{Key Features}\label{key-features-17}

Large, usually deep stratifed waterbodies impounded by walls across
outflow channels. Productivity and biotic diversity are lower than
unregulated lakes of simila rsize and complexity. Trophic networks are
simple.

\subsection{Ecological traits}\label{ecological-traits-17}

Rivers are impounded by the construction of dam walls, creating large
freshwater reservoirs, mostly 15--250 m deep. Primary productivity is
low to moderate and restricted to the euphotic zone (limnetic and
littoral zones), varying with turbidity and associated light
penetration, nutrient availability, and water temperature. Trophic
networks are simple with low species diversity and endemism. Shallow
littoral zones have the highest species diversity including benthic
algae, macroinvertebrates, fish, waterbirds, aquatic reptiles, aquatic
macrophytes, and terrestrial or amphibious vertebrates. Phytoplankton
and zooplankton occur through the littoral and limnetic zones. The
profundal zone lacks primary producers and, if oxygenated, is dominated
by benthic detritivores and microbial decomposers. Fish communities
inhabit the limnetic and littoral zones and may be dominated by managed
species and opportunists. Reservoirs may undergo eutrophic succession
due to inflow from catchments with sustained fertiliser application or
other nutrient inputs.

\subsection{Key Ecological Drivers}\label{key-ecological-drivers-17}

Reservoirs receive water from the rivers they impound. Managed release
or diversion of water alters natural variability. Large variations in
water level produce wide margins that are intermittently inundated or
dry, limiting productivity and the number of species able to persist
there. Inflow volumes may be regulated. Inflows may contain high
concentrations of phosphorus and/or nitrogen (e.g.~from sewerage
treatment effluents or fertilised farmland), leading to eutrophication.
Reservoirs in upper catchments generally receive less nutrients and
cooler water (due to altitude) than those located downstream.
Geomorphology, substrate, and land use of the river basin influence the
amount of inflowing suspended sediment, and hence turbidity, light
penetration, and the productivity of planktonic and benthic algae, as
well as rates of sediment build-up on the reservoir floor. Depth
gradients in light and oxygen, as well as thermal stratification,
strongly influence the structure of biotic communities and trophic
interactions, as do human introductions of fish, aquatic plants, and
other alien species.

\subsection{Distribution}\label{distribution-17}

Large reservoirs are scattered across all continents with the greatest
concentrations in Asia, Europe, and North America. Globally, there are
more than 3000 reservoirs with a surface area ≥ 50km2. Spatial data are
incomplete for some countiries.

\section{F3.2 Constructed lacustrine
wetlands}\label{f3.2-constructed-lacustrine-wetlands}

Belongs to biome F3. Artificial wetlands biome, part of the Freshwater
realm.

\subsection{Short description}\label{short-description-18}

Small farm dams, wastewater ponds and mine pits generally form lake-like
environments, common in humid and semi-arid climates world-wide.
Nutrient inputs vary greatly depending on purpose and surrounding land
uses. They are often warm and shallow, and biological productivity and
diversity vary widely depending on the cover and state of fringing
vegetation, with the most diverse examples rivalling equivalent natural
wetlands. Aquatic plants, plankton, algae and waterbugs may dominate in
the shallows, supporting amphibians, turtles, fish, and waterbirds.

\subsection{Key Features}\label{key-features-18}

Small, shallow open waterbodies with high or low productivity depending
on nutrient subsidies and complexity of littoral zones and benthos
Relatively simple trophic networks with algae, macrophytes, zooplankton,
aquatic invertebrates and amphibians.

\subsection{Ecological traits}\label{ecological-traits-18}

Shallow, open water bodies have been constructed in diverse landscapes
and climates. They may be fringed by amphibious vegetation, or else
bedrock or bare soil maintained by earthworks or livestock trampling.
Emergents rarely extend throughout the water body, but submerged
macrophytes are often present. Productivity ranges from very high in
wastewater ponds to low in mining and excavation pits, depending on
depth, shape, history and management. Taxonomic and functional diversity
range from levels comparable to natural lakes to much less, depending on
productivity, complexity of aquatic or fringing vegetation, water
quality, management and proximity to other waterbodies or vegetation.
Trophic structure includes phytoplankton and microbial detritivores,
with planktonic and invertebrate predators dominating limnetic zones.
Macrophytes may occur in shallow littoral zones or submerged habitats,
and some artificial water bodies include higher trophic levels including
macroinvertebrates, amphibians, turtles, fish, and waterbirds. Fish may
be introduced by people or arrive by flows connected to source
populations, where these exist. Endemism is generally low, but these
waterbodies may be important refuges for some species now highly
depleted in their natural habitats. Life histories often reflect those
found in natural waterbodies nearby, but widely dispersed opportunists
dominate where water quality is poor. Intermittent water bodies support
biota with drought resistance or avoidance traits, while permanently
inundated systems provide habitat for mobile species such as waterbirds.

\subsection{Key Ecological Drivers}\label{key-ecological-drivers-18}

Water bodies are constructed for agriculture, mining, stormwater,
ornamentation, wastewater, or other uses, or fill depressions left by
earthworks, obstructing surface flow or headwater channels. Humans may
directly or indirectly regulate inputs of water and chemicals
(e.g.~fertilisers, flocculants, herbicides), as well as water drawdown.
Climate and weather also affect hydrology. Shallow depth and lack of
shade may expose open water to rapid solar heating and hence diurnally
warm temperatures. Substrates include silt, clay, sand, gravel, cobbles
or bedrock, and fine sediments of organic material may build up over
time. Nutrient levels are highest in wastewater or with run-off from
fertilised agricultural land or urban surfaces. Some water bodies
(e.g.~mines and industrial wastewaters) have concentrated chemical
toxins, extremes of pH or high salinities. Humans may actively introduce
and remove the biota of various trophic levels (e.g.~bacteria, algae,
fish, and macrophytes) for water quality management or human
consumption.

\subsection{Distribution}\label{distribution-18}

Scattered across most regions of the world occupied by humans. Farm dams
covered an estimated 77,000km2 globally in 2006.

\section{F3.3 Rice paddies}\label{f3.3-rice-paddies}

Belongs to biome F3. Artificial wetlands biome, part of the Freshwater
realm.

\subsection{Short description}\label{short-description-19}

Rice paddies cover more than a million square kilometres mostly in
tropical to warm temperate climates, especially Southeast Asia. They are
filled by rainfall or river water diversions. Their levees and channels
retain shallow water areas, with nutrients inputs from inflows and
fertilisers. Planting and harvest establishes a regular cycle of
disturbance, with many paddies also supporting production of fish and
crustaceans, Their simple foodwebs are adapted to temporary flooding and
the harvest cycle, including algae and plankton, waterbugs, frogs and
waterbirds.

\subsection{Key Features}\label{key-features-19}

Artificial wetlands with limited horizontal and vertical heterogeneity,
filled seasonally with water from rivers or rainfall and frequently
disturbed by planting and harvest of rice. Simple trophic networks with
colonists from rivers and wetlands that may also include managed fish
populations.

\subsection{Ecological traits}\label{ecological-traits-19}

Rice paddies are artificial wetlands with low horizontal and vertical
heterogeneity fed by rain or irrigation water diverted from rivers. They
are predominantly temporary wetlands, regularly filled and dried,
although some are permanently inundated, functioning as simplified
marshes. Allochthonous inputs come from water inflow but also include
the introduction of rice, other production organisms (e.g.~fish and
crustaceans), and fertilisers that promote rice growth. Simplified
trophic networks are sustained by highly seasonal, deterministic
flooding and drying regimes and the agricultural management of harvest
crops, weeds, and pests. Cultivated macrophytes dominate primary
production, but other autotrophs including archaea, cyanobacteria,
phytoplankton, and benthic or epiphytic algae also contribute. During
flooded periods, microbial changes produce anoxic soil conditions and
emissions by methanogenic archaea. Opportunistic colonists include
consumers such as invertebrates, zooplankton, insects, fish, frogs, and
waterbirds, as well as other aquatic plants. Often they come from nearby
natural wetlands or rivers and may breed within rice paddies. During dry
phases, obligate aquatic organisms are confined to wet refugia away from
rice paddies. These species possess traits that promote tolerance to low
water quality and predator avoidance. Others organisms, including many
invertebrates and plants, have rapid life cycles and dormancy traits
allowing persistence as eggs or seeds during dry phases.

\subsection{Key Ecological Drivers}\label{key-ecological-drivers-19}

Engineering of levees and channels enables the retention of standing
water a few centimetres above the soil surface and rapid drying at
harvest time. This requires reliable water supply either through summer
rains in the seasonal tropics or irrigation in warm-temperate or
semi-arid climates. The water has high oxygen content and usually warm
temperatures. Deterministic water regimes and shallow depths limit niche
diversity and have major influences on the physical, chemical, and
biological properties of soils, which contain high nutrient levels. Rice
paddies are often established on former floodplains but may also be
created on terraced hillsides. Other human interventions include
cultivation and harvest, aquaculture, and the addition of fertilisers,
herbicides, and pesticides.

\subsection{Distribution}\label{distribution-19}

More than a million square kilometres, mostly in tropical and
subtropical Southeast Asia, with small areas in Africa, Europe, South
America, North America, and Australia.

\section{F3.4 Freshwater aquafarms}\label{f3.4-freshwater-aquafarms}

Belongs to biome F3. Artificial wetlands biome, part of the Freshwater
realm.

\subsection{Short description}\label{short-description-20}

Freshwater aquafarms are ponds constructed from earthworks or cages
built within freshwater lakes, rivers and reservoirs. They are most
common in Asia. and used to produce species. Their commercial production
of fish and crustacean involves intensive interventions, including
focussed inputs of food and nutrients, and control of competitors,
predators and diseases that may limit production of target species.
Consequently habitat diversity and primary production are low, and
non-target biota is limited to opportunistic colonisers from adjacent
water sources, including insects, fish, frogs, waterbirds and some
aquatic plants.

\subsection{Key Features}\label{key-features-20}

Artificial mostly permanent waterbodies managed for production of fish
or crustaceans with managed inputs of nutrients and energy Simple
trophic networks of opportunistic colonists supported mainly be algal
productivity.

\subsection{Ecological traits}\label{ecological-traits-20}

Freshwater aquaculture systems are mostly permanent water bodies in
either purpose-built ponds, tanks, or enclosed cages within artificial
reservoirs (F3.1), canals (F3.5), freshwater lakes (F2.1 and F2.2), or
lowland rivers (F1.2). These systems are shaped by large allochthonous
inputs of energy and nutrients to promote secondary productivity by one
or a few target consumer species (mainly fish or crustaceans), which are
harvested as adults and restocked as juveniles on a regular basis. Fish
are sometimes raised in mixed production systems within rice paddies
(F3.3), but aquaculture ponds may also be co-located with rice paddies,
which are centrally located and elevated above the level of the ponds.
The enclosed structures exclude predators of the target species, while
intensive anthropogenic management of hydrology, oxygenation, toxins,
competitors, and pathogens maintains a simplified trophic structure and
near-optimal survival and growth conditions for the target species.
Intensive management and low niche diversity within the enclosures limit
the functional diversity of biota within the system. However, biofilms
and phytoplankton contribute low levels of primary production,
sustaining zooplankton and other herbivores, while microbial and
invertebrate detritivores break down particulate organic matter. Most of
these organisms are opportunistic colonists, as are insects, fish,
frogs, and waterbirds, as well as aquatic macrophytes. Often these
disperse from nearby natural wetlands, rivers, and host waterbodies.

\subsection{Key Ecological Drivers}\label{key-ecological-drivers-20}

Aquafarms are small artificial water bodies with low horizontal and
vertical heterogeneity. Water regimes are mostly perennial but may be
seasonal (e.g.~when integrated with rice production). Engineering of
tanks, channels, and cages enables the intensive management of water,
nutrients, oxygen levels, toxins, other aspects of water chemistry, as
well as the introduction of target species and the exclusion of pest
biota. Removal of wastewater and replacement by freshwater from lakes or
streams, together with inputs of antibiotics and chemicals
(e.g.~pesticides and fertilisers) influence the physical, chemical, and
biological properties of the water column and substrate. When located
within cages in natural water bodies, freshwater aquafarms reflect the
hydrological and hydrochemical properties of their host waterbody.
Nutrient inputs drive the accumulation of ammonium and nitrite nitrogen,
as well as phosphorus and declining oxygen levels, which may lead to
eutrophication within aquaculture sites and receiving waters.

\subsection{Distribution}\label{distribution-20}

Concentrated in Asia but also in parts of northern and western Europe,
North and West Africa, South America, North America, and small areas of
southeast Australia and New Zealand.

\section{F3.5 Canals, ditches and
drains}\label{f3.5-canals-ditches-and-drains}

Belongs to biome F3. Artificial wetlands biome, part of the Freshwater
realm.

\subsection{Short description}\label{short-description-21}

Canals, ditches and drains are associated with agriculture and cities
throughout the world. They take freshwater to and from urban and rural
areas, particularly in temperate and subtropical regions. They can carry
high nutrient and pollutant loads. Diversity of organisms is generally
low, but may be high where there are earthen banks and fringing
vegetation. Algae and macrophytes (where present) support microbes and
waterbugs and other invertebrates often small fish, amphibians and
crustaceans. They are also important pathways for dispersal of some
aquatic species.

\subsection{Key Features}\label{key-features-21}

Artificial streams often with low horizontal and vertical heterogeneity,
but with productivity, diversity and trophic structure highly dependent
on fringing vegetation and subsidies of nutrients and carbon from
catchments.

\subsection{Ecological traits}\label{ecological-traits-21}

Canals, ditches and storm water drains are artificial streams with low
horizontal and vertical heterogeneity. They function as rivers or
streams and may have simplified habitat structure and trophic networks,
though some older ditches have fringing vegetation, which contributes to
structural complexity. The main primary producers are filamentous algae
and macrophytes that thrive on allochthonous subsidies of nutrients.
Subsidies of organic carbon from urban or rural landscapes support
microbial decomposers and mostly small invertebrate detritivores. While
earthen banks and linings may support macrophytes and a rich associated
fauna, sealed or otherwise uniform substrates limit the diversity and
abundance of benthic biota. Fish and crustacean communities, when
present, generally exhibit lower diversity and smaller body sizes
compared to natural systems, and are often dominated by introduced or
invasive species. Waterbirds, when present, typically include a low
diversity and density of herbivorous and piscivorous species. Canals,
ditches and drains may provide pathways for dispersal or colonisation of
native and invasive biota.

\subsection{Key Ecological Drivers}\label{key-ecological-drivers-21}

Engineered levees and channels enable managed water flow for human uses,
including water delivery for irrigation or recreation, water removal
from poorly drained sites or sealed surfaces (e.g.~storm water drains),
or routes for navigation. Deterministic water regimes and often shallow
depths have major influences on the physical, chemical, and biological
properties of the canals, ditches and drains. Flows in some ditches may
be very slow, approaching lentic regimes. Flows in storm water drains
vary with rain or other inputs. Irrigation, transport, or recreation
canals usually have steady perennial flows but may be seasonal for
irrigation or intermittent where the water source is small. Turbidity
varies but oxygen content is usually high. Substrates and banks vary
from earthen material or hard surfaces (e.g.~concrete, bricks),
affecting suitability for macrophytes and niche diversity. The water may
carry high levels of nutrients and pollutants due to inflow and
sedimentation from sealed surfaces, sewerage, other waste sources,
fertilised cropping, or pasture lands.

\subsection{Distribution}\label{distribution-21}

Urban landscapes and irrigation areas mostly in temperate and
subtropical latitudes. Several hundred thousand kilometres of ditches
and canals in Europe.

\section{FM1.1 Deepwater coastal
inlets}\label{fm1.1-deepwater-coastal-inlets}

Belongs to biome FM1. Semi-confined transitional waters biome, part of
the Freshwater, Marine realm.

\subsection{Short description}\label{short-description-22}

Ecosystems in these deep, narrow inlets were mostly formed by glaciers
and subsequently flooded (e.g.~fjords). They have some features of open
oceans, but are strongly influenced by freshwater inflows and the
surrounding coast. Productivity by phytoplankton is seasonal and limited
by cold, dark winters. Oxygen may be limited in the deepest parts of
these systems. The diverse biota includes invertebrates and fish, such
as jellyfish and salmon, and predatory marine mammals such as killer
whales.

\subsection{Key Features}\label{key-features-22}

Strong gradients between adjacent terrestrial and freshwater
systems,e.g.~fjords. Seasonaly abundant plankton, jellies, fish and
mammals..

\subsection{Ecological traits}\label{ecological-traits-22}

Deepwater coastal inlets (e.g.~fjords, sea lochs) are semi-confined
aquatic systems with many features of open oceans. Strong influences
from adjacent freshwater and terrestrial systems produce striking
environmental and biotic gradients. Autochthonous energy sources are
dominant, but allochthonous sources (e.g.~glacial ice discharge,
freshwater streams, and seasonal permafrost meltwater) may contribute
10\% or more of particulate organic matter. Phytoplankton, notably
diatoms, contribute most of the primary production, along with biofilms
and macroalgae in the epibenthic layer. Seasonal variation in inflow,
temperatures, ice cover, and insolation drives pulses of in situ and
imported productivity that generate blooms in diatoms, consumed in turn
by jellyfish, micronekton, a hierarchy of fish predators, and marine
mammals. Fish are limited by food, density-dependent predation, and
cannibalism. As well as driving pelagic trophic networks, seasonal
pulses of diatoms shape biogeochemical cycles and the distribution and
biomass of benthic biota when they senesce and sink to the bottom,
escaping herbivores, which are limited by predators. The vertical flux
of diatoms, macrophytes, and terrestrial detritus sustains a diversity
and abundance of benthic filter-feeders (e.g.~maldanids and oweniids).
Environmental and biotic heterogeneity underpins functional and
compositional contrasts between inlets and strong gradients within them.
Zooplankton, fish, and jellies distribute in response to resource
heterogeneity, environmental cues, and interactions with other
organisms. Deep inlets sequester more organic carbon into sediments than
other estuaries (FM1.2, FM1.3) because steep slopes enable efficient
influx of terrestrial carbon and low-oxygen bottom waters abate decay
rates. Inlets with warmer water have more complex trophic webs, stronger
pelagic-benthic coupling, and utilise a greater fraction of organic
carbon, sequestering it in sea-floor sediments at a slower rate than
those with cold water.

\subsection{Key Ecological Drivers}\label{key-ecological-drivers-22}

Deepwater coastal systems may exceed 300 km in length and 2 km in depth.
Almost all have glacial origins and many are fed by active glaciers. The
ocean interface at the mouth of the inlets, strongly influenced by
regional currents, interacts with large seasonal inputs of freshwater to
the inner inlet and wind-driven advection, to produce strong and dynamic
spatial gradients in nutrients, salinity and organic carbon. Advection
is critical to productivity and carrying capacity of the system.
Advection drives water movement in the upper and lower layers of the
water column in different directions, linking inlet waters with coastal
water masses. Coastal currents also mediate upwelling and downwelling
depending on the direction of flow. However, submerged glacial moraines
or sills at the inlet mouth may limit marine mixing, which can be
limited to seasonal episodes in spring and autumn. Depth gradients in
light typically extend beyond the photic zone and are exacerbated at
high latitudes by seasonal variation in insolation and surface ice.
Vertical fluxes also create strong depth gradients in nutrients, oxygen,
dissolved organic carbon, salinity, and temperature.

\subsection{Distribution}\label{distribution-22}

Historically or currently glaciated coastlines at polar and
cool-temperate latitudes.

\section{FM1.2 Permanently open riverine estuaries and
bays}\label{fm1.2-permanently-open-riverine-estuaries-and-bays}

Belongs to biome FM1. Semi-confined transitional waters biome, part of
the Freshwater, Marine realm.

\subsection{Short description}\label{short-description-23}

These coastal ecosystems are shifting mosaics of different habitats,
depending on the shape of the local coast, and proportional inflow of
freshwater and seawater. Combined nutrients from marine, freshwater and
land-based sources support very high productivity. Transient large
animals like dugongs, dolphins, turtles and shorebirds feed on abundant
fish, invertebrates and plant life, and they commonly serve as sheltered
nursery areas for fish. Many organisms are adapted to large variations
in salinity.

\subsection{Key Features}\label{key-features-23}

Productive mosaic systems with variable salinity, often nuseries for
fish and supporting abundant seabirds and mammals..

\subsection{Ecological traits}\label{ecological-traits-23}

These coastal water bodies are mosaic systems characterised by high
spatial and temporal variabilities in structure and function, which
depend on coastal geomorphology, ratios of freshwater inflows to marine
waters and tidal volume (hence residence time of saline water), and
seasonality of climate. Fringing shoreline systems may include
intertidal mangroves (MFT1.2), saltmarshes and reedbeds (MFT1.3), rocky
(MT1.1), muddy (MT1.2) or sandy shores (MT1.3), while seagrasses and
macrophytes (M1.1), shellfish beds (M1.4) or subtidal rocky reefs (M1.6)
may occur in shallow intertidal and subtidal areas. Water-column
productivity is typically higher than in nearby marine or freshwater
systems due to substantial allochthonous energy and nutrient subsidies
from shoreline vegetation and riverine and marine sources. This high
productivity supports a complex trophic network with relatively high
mosaic-level diversity and an abundance of aquatic organisms. Planktonic
and benthic invertebrates (e.g.~molluscs and crustaceans) often sustain
large fish populations, with fish nursery grounds being a common
feature. Waterbirds (e.g.~cormorants), seabirds (e.g.~gannets),
top-order predatory fish, mammals (e.g.~dolphins and dugongs), and
reptiles (e.g.~marine turtles and crocodilians) exploit these locally
abundant food sources. Many of these organisms in upper trophic levels
are highly mobile and move among different estuaries through connected
ocean waters or by flying. Others migrate between different ecosystem
types to complete their various life-history phases, although some may
remain resident for long periods. Most biota tolerate a broad range of
salinity or are spatially structured by gradients. The complex spatial
mixes of physical and chemical characteristics, alongside seasonal,
inter-annual, and sporadic variability in aquatic conditions, induce
correspondingly large spatial-temporal variability in food webs.
Low-salinity plumes, usually proportional to river size and discharge,
may extend far from the shore, producing tongues of ecologically
distinct conditions into the marine environment.

\subsection{Key Ecological Drivers}\label{key-ecological-drivers-23}

Characteristics of these coastal systems are governed by the relative
dominance of saline marine waters versus freshwater inflows (groundwater
and riverine), the latter depending on the seasonality of precipitation
and evaporative stress. Geomorphology ranges from wave-dominated
estuaries to drowned river valleys, tiny inlets, and enormous bays.
These forms determine the residence time, proportion, and distribution
of saline waters, which in turn affect salinity and thermal gradients
and stratification, dissolved O2 concentration, nutrients, and
turbidity. The water column is closely linked to mudflats and sandflats,
in which an array of biogeochemical processes occurs, including
denitrification and N-fixation, and nutrient cycling.

\subsection{Distribution}\label{distribution-23}

Coastlines of most landmasses but rarely on arid or polar coasts.

\section{FM1.3 Intermittently closed and open lakes and
lagoons}\label{fm1.3-intermittently-closed-and-open-lakes-and-lagoons}

Belongs to biome FM1. Semi-confined transitional waters biome, part of
the Freshwater, Marine realm.

\subsection{Short description}\label{short-description-24}

Opportunistic, short-lived organisms live in these ecosystems, where
conditions change rapidly as lagoon entrances to the open ocean open or
close. Periodic opening and closure influences dynamic gradients in
salinity, nutrients, temperature, and water level. Algae, invertebrates
like shrimps, and small fish rely on nutrients from land and, when open,
the sea. Timing of opening or closing depends on transport of sand and
mud by currents or freshwater inflow, or on anthropogenic processes.

\subsection{Key Features}\label{key-features-24}

Shallow water systems, highly variability depending on opening or
closing of lagoonal entrance. Detritus-based foodwebs with plankton,
invertebrates and small fish..

\subsection{Ecological traits}\label{ecological-traits-24}

These coastal water bodies have high spatial and temporal variability in
structure and function, which depends largely on the status of the
lagoonal entrance (open or closed). Communities have low species
richness compared to those of permanently open estuaries (FM1.2).
Lagoonal entrance closure prevents the entry of marine organisms and
resident biota must tolerate significant variation in salinity,
inundation, dissolved oxygen, and nutrient concentrations. Resident
communities are dominated by opportunists with short lifecycles. Trophic
networks are generally detritus-based, fuelled by substantial inputs of
organic matter from the terrestrial environment and, to a lesser extent,
from the sea. As net sinks of organic matter from the land, productivity
is often high, and lagoons may serve as nursery habitats for fish. High
concentrations of polyphenolic compounds (e.g.~tannins) in the water and
periods of low nutrient input limit phytoplankton populations. Benthic
communities dominate with attached algae, microphytobenthos and micro-
and macro-fauna being the dominant groups. The water column supports
plankton and small-bodied fish. Emergent and fringing vegetation is a
key source of detrital carbon to the food webs, and also provides
important structural habitats. Saltmarsh and reedbeds (MFT1.3) can
adjoin lagoons while seagrasses (M1.1) occupy sandy bottoms of some
lagoons, but mangroves (MFT1.2) are absent unless the entrance opens

\subsection{Key Ecological Drivers}\label{key-ecological-drivers-24}

These are shallow coastal water bodies that are intermittently connected
with the ocean. Some lagoons are mostly open, closing only once every
few decades. Some open and close frequently and some are closed most of
the time. The timing and frequency of entrance opening depend on
trade-offs between sedimentation from fluvial and shoreline processes
(which close the connection) and flushes of catchment inflow or erosive
wave action (which open the entrance). Opening leads to changes in water
level, tidal amplitude, salinity gradients, temperature, nutrients,
dissolved oxygen, and sources of organic carbon. Human-regulated opening
influences many of these processes.

\subsection{Distribution}\label{distribution-24}

Wave-dominated coastlines worldwide, but prevalent along microtidal to
low mesotidal mid-latitude coastlines with high inter-annual variability
in rainfall and wave climate. Intermittent closed open lakes and lagoons
(ICOLLs) are most prevalent in Australia (21\% of global occurrences),
South Africa (16\%), and Mexico (16\%).

\section{M1.1 Seagrass meadows}\label{m1.1-seagrass-meadows}

Belongs to biome M1. Marine shelf biome, part of the Marine realm.

\subsection{Short description}\label{short-description-25}

These shallow, subtidal systems are the only marine ecosystems with an
abundance of flowering plants. They are typically found mostly on soft,
sandy or muddy substrates around relatively sheltered coastlines. Extent
is limited in the shallows by wave action and tidal exposure, and at
depth by light availability. Productive ecosystems, their
three-dimensional structure provides shelter for juvenile fish,
invertebrates and epiphytic algae. Diverse organisms live in and around
seagrass beds including many grazers, from tiny invertebrates to
megafauna such as dugongs.

\subsection{Key Features}\label{key-features-25}

Soft, mostly subtidal substrates in low-energy waters with abundant
vascular macrophytes, associated epibiota, infauna and fish.

\subsection{Ecological traits}\label{ecological-traits-25}

Seagrass meadows are important sources of organic matter, much of which
is retained by seagrass sediments. Seagrasses are the only subtidal
marine flowering plants and underpin the high productivity of these
systems. Macroalgae and epiphytic algae, also contribute to
productivity, supporting both detritus production and autochthonous
trophic structures, but compete with seagrasses for light. The complex
three-dimensional structure of the seagrass provides shelter and cover
to juvenile fish and invertebrates, binds sediments and, at fine scales,
dissipates waves and currents. Seagrass ecosystems support infauna
living amongst their roots, epifauna, and epiflora living on their
shoots and leaves, as well as nekton in the water column. They have a
higher abundance and diversity of flora and fauna compared to
surrounding unvegetated soft sediments and comparable species richness
and abundances to most other marine biogenic habitats. Mutualisms with
lucinid molluscs may influence seagrass persistence. Mesograzers (such
as amphipods and gastropods) play an important role in controlling
epiphytic algal growth on seagrass. Grazing megafauna such as dugongs,
manatees and turtles can contribute to patchy seagrass distributions,
although they tend to `garden' rather than deplete seagrass.

\subsection{Key Ecological Drivers}\label{key-ecological-drivers-25}

Typically found in the subtidal zone on soft sedimentary substrates but
also occasionally on rocky substrates on low- to moderate-energy
coastlines with low turbidity and on intertidal shorelines.~Minimum
water depth is determined mainly by wave orbital velocity, tidal
exposure, and wave energy (i.e.~waves disturb seagrass and mobilise
sediment), while maximum depth is limited by the vertical diminution of
light intensity in the water column. Seagrass growth can be limited by
nitrogen and phosphorous availability, but in eutrophic waters, high
nutrient availability can lead to the overgrowth of seagrasses by
epiphytes and shading by algal blooms, leading to ecosystem collapse.
Large storm events and associated wave action lead to seagrass loss.

\subsection{Distribution}\label{distribution-25}

Widely distributed along the temperate and tropical coastlines of the
world.

\section{M1.10 Rhodolith/Maërl beds}\label{m1.10-rhodolithmauxebrl-beds}

Belongs to biome M1. Marine shelf biome, part of the Marine realm.

\subsection{Short description}\label{short-description-26}

These slow growing biogenic structures are formed by long-lived
coralline algae that absorb a wide spectrum of light, provide energy to
the system and contribute to nutrient cycles. They can occur in shallow
or intermediates depths with coarse gravel, sandy or mixed muddy
substrates. The carbonate structures of living and dead rhodoliths, and
the spatial (depth gradient) and temporal (diurnal and seasonal)
variation in the environmental conditions provide habitat for diverse
communities of macroinvertebrates and fish, along with other
characteristic sessile organisms like algae and sponges. Storms, waves
and other disturbances drive cycles of restructuring and slow recovery.

\subsection{Key Features}\label{key-features-26}

Biogenic beds formed by non-geniculate (non-jointed), free-living
coralline algae on soft substrates supporting diverse benthic and
demersal fauna and bacterial biofilms.

\subsection{Ecological traits}\label{ecological-traits-26}

Benthic carbonate ecosystems dominated by rhodoliths -- non-geniculate
(non-jointed), free-living, slow-growing, long-lived coralline algae --
cover 30-100\% of the seafloor within the beds, providing autochthonous
energy to the system. Their pigments enable red algae to absorb more
green - blue light efficiently, in addition to red-orange light.
Rhodolith primary productivity is likely to be lower than in sea grasses
(M1.1) and kelp forests (M1.2), although macrophytes add to primary
production in shallow waters. They play a role in benthic nutrient
cycling and represent significant long-term carbonate stores. Rhodoliths
vary from smooth semi-spherical to complex fruticose structures that may
form mono- or multi- specific aggregations typically composed of living
and dead rhodoliths, as well as calcic sediments produced by breakdown.
They can form 3-dimensional biogenic structures that facilitate
coexistence of a diversity of benthic and demersal organisms, including
algae, ascidians, sponges, macroinvertebrates and fish. Compared to
coral reefs (M1.3), shellfish beds (M1.4) or marine animal forests
(M1.5), where rhodoliths may be minor components, they are usually less
rugose and less stable, due displacement or aggregation by water motion
and bioturbators such as fish and macroinvertebrates. Large rhodoliths
appear to facilitate deepwater kelp as well as feeding and reproduction
in fish and invertebrates, supporting high species richness. High
abundance of larval stages in these groups, suggests the intermediate
rugosity of the beds is important for age-dependent predator evasion.
Macroinvertebrate detritivores and herbivores well represented in
rhodolith beds include crustaceans, molluscs, echinoderms and
polychaetes. Closely associated microinvertebrates and microbes include
small gastropods, ostracods, diatoms, foraminifera and bacteria.
Bacterial guilds on rhodolith surfaces include photolithoautotrophs,
anoxygenic phototrophs, anaerobic heterotrophs, sulfide oxidizers and
methanogens, suggesting important roles in biomineralization. The biotic
assemblages of rhodolith beds vary spatially, with depth gradients and
temporally over diurnal and seasonal time scales. Fish and sponges that
aggregate and agglutinate individual rhodoliths are thought to promote
development of reefs from rhodolith beds, counter-balancing slow
recovery from disturbance.

\subsection{Key Ecological Drivers}\label{key-ecological-drivers-26}

Rhodolith beds occur on coarse gravel, sandy or mixed muddy substrates.
They are most common at depths of 5-150m, but may occur from the
subtidal zone down to 270m below the ocean surface. Light availability,
pH and hydrodynamics are important drivers of variation in biotic
assemblages, as are temperatures. Rhodoliths form extensive beds on open
coasts on the mid shelf and in tide-swept channels where the water
column and suspended sediment diminish red light. Recurring disturbances
such as bioturbation, wave action or storms physically restructure the
system and initiate successional recovery.

\subsection{Distribution}\label{distribution-26}

Tropical to subpolar coastal waters, extensive areas in the north and
southwest Atlantic, Mediterranean, Gulf of California and southern
Australia.

\section{M1.2 Kelp forests}\label{m1.2-kelp-forests}

Belongs to biome M1. Marine shelf biome, part of the Marine realm.

\subsection{Short description}\label{short-description-27}

Kelps (large, brown macroalgae up to 30m in length) form the basis of
these highly productive systems found on shallow, subtidal rocky reefs
around cold temperate and polar coastlines. Their forest-like structure
and vertical habitat supports diverse epiflora and --fauna living on the
kelp itself, as well as rich communities of invertebrates, fish and
marine birds and mammals living and foraging in and around these
ecosystems. High nutrient requirements mean these ecosystems are often
associated with upwelling water, while wave action and currents are
important for replenishing oxygen.

\subsection{Key Features}\label{key-features-27}

Hard subtidal substrates in cold, clear nutrient-rich waters with
dominant brown algal macrophytes, associated epibiota, benthic
macrofauna, fish \& mammals.

\subsection{Ecological traits}\label{ecological-traits-27}

Kelps are benthic brown macroalgae (Order Laminariales) forming canopies
that shape the structure and function of these highly productive,
diverse ecosystems. These large (up to 30 m in length), fast-growing (up
to 0.5 m/day) autotrophs produce abundant consumable biomass, provide
vertical habitat structure, promote niche diversity, alter light-depth
gradients, dampen water turbulence, and moderate water temperatures.
Traits such as large, flexible photosynthetic organs, rapid growth, and
strong benthic holdfasts enable kelps to persist on hard substrates in
periodically turbulent waters. These kelps may occur as scattered
individuals in other ecosystem types, but other macroalgae (e.g.~green
and coralline) rarely form canopies with similar function and typically
form mixed communities with sessile invertebrates (see M1.5 and M1.6).
Some kelps are fully submerged, while others form dense canopies on the
water surface, which profoundly affect light, turbulence, and
temperature in the water column. Interactions among co-occurring kelps
are generally positive or neutral, but competition for space and light
is an important evolutionary driver. Kelp canopies host a diverse
epiflora and epifauna, with some limpets having unique kelp hosts.
Assemblages of benthic invertebrate herbivores and detritivores inhabit
the forest floor, notably echinoderms and crustaceans. The structure and
diversity of life in kelp canopies provide forage for seabirds and
mammals, such as gulls and sea otters, while small fish find refuge from
predators among the kelp fronds. Herbivores keep epiphytes in check, but
kelp sensitivity to herbivores makes the forests prone to complex
trophic cascades when declines in top predators release herbivore
populations from top-down regulation. This may drastically reduce the
abundance of kelps and dependent biota and lead to replacement of the
forests by urchin barrens, which persist as an alternative stable state.

\subsection{Key Ecological Drivers}\label{key-ecological-drivers-27}

Kelp forests are limited by light, nutrients, salinity, temperature, and
herbivory. Growth rates are limited by light and proximity to sediment
sources. High nutrient requirements are met by terrestrial runoff or
upwelling currents, although eutrophication can lead to transition to
turf beds. Truncated thermal niches limit the occurrence of kelps in
warm waters. Herbivory on holdfasts influences recruitment and can
constrain reversals of trophic cascades, even when propagules are
abundant. Kelp forests occur on hard substrates in the upper photic zone
and rely on wave action and currents for oxygen. Currents also play
important roles in dispersing the propagules of kelps and associated
organisms. Storms may dislodge kelps, creating gaps that may be
maintained by herbivores or rapidly recolonized.

\subsection{Distribution}\label{distribution-27}

Nearshore rocky reefs to depths of 30 m in temperate and polar waters.
Absent from warm tropical waters but present in upwelling zones off
Oman, Namibia, Cape Verde, Peru, and the Galapagos.

\section{M1.3 Photic coral reefs}\label{m1.3-photic-coral-reefs}

Belongs to biome M1. Marine shelf biome, part of the Marine realm.

\subsection{Short description}\label{short-description-28}

These slow growing biogenic structures are formed by the calcium
carbonate skeletons of certain coral species that depend on symbiotic
relationships with algae. They occur in warm, shallow, low-nutrient
waters and provide complex three-dimensional habitat for a highly
diverse community across all trophic levels, from algae to sharks, along
with other characteristic sessile organisms like coralline algae and
sponges. Niche habitats produce specialist behaviours and diets, like
the symbiotic relationship between clown fish and anemones. Storms and
marine heat waves drive cycles of reef destruction and renewal.

\subsection{Key Features}\label{key-features-28}

Biogenic reefs formed by hard coral-algal symbionts with phylogentically
\& functionally diverse biota in clear, warm subtidal waters.

\subsection{Ecological traits}\label{ecological-traits-28}

Coral reefs are biogenic structures that have been built up and continue
to grow over decadal timescales as a result of the accumulation of
calcium carbonate laid down by hermatypic (scleractinian) corals and
other organisms. Reef-building corals are mixotrophic colonies of coral
polyps in endosymbiotic relationships with photosynthesizing
zooxanthellae that assimilate solar energy and nutrients, providing
almost all of the metabolic requirements for their host. The corals
develop skeletons by extracting dissolved carbonate from seawater and
depositing it as aragonite crystals. Corals reproduce asexually,
enabling the growth of colonial structures. They also reproduce
sexually, with mostly synchronous spawning related to annual lunar cues.
Other sessile organisms including sponges, soft corals, gorgonians,
coralline algae, and other algae add to the diversity and structural
complexity of coral reef ecosystems. The complex three-dimensional
structure provides a high diversity of habitat niches and resources that
support a highly diverse and locally endemic marine biota, including
crustaceans, polychaetes, holothurians, echinoderms, and other groups,
with one-quarter of marine life estimated to depend on reefs for food
and/or shelter. Diversity is high at all taxonomic levels relative to
all other ecosystems. The trophic network is highly complex, with
functional diversity represented on the benthos and in the water column
by primary producers, herbivores, detritivores, suspension-feeders, and
multiple interacting levels of predators. Coral diseases also play a
role in reef dynamics. The vertebrate biota includes fish, snakes,
turtles, and mammals. The fish fauna is highly diverse, with herbivores
and piscivores displaying a wide diversity of generalist and specialist
diets (including parrot fish that consume corals), feeding strategies,
schooling and solitary behaviours, and reproductive strategies. The
largest vertebrates include marine turtles and sharks.

\subsection{Key Ecological Drivers}\label{key-ecological-drivers-28}

Coral reefs are limited to warm, shallow (rarely \textgreater60 m
depth), clear, relatively nutrient-poor, open coastal waters, where
salinity is 3.0--3.8\% and sea temperatures vary (17--34°C). Cooler
temperatures are insufficient to support coral growth, while warmer
temperatures cause coral symbiosis to break down (i.e.~bleaching). Reef
geomorphology varies from atolls, barrier reefs, fringing reefs and
lagoons to patch reefs depending upon hydrological and geological
conditions. Reef structure and composition vary with depth gradients
such as light intensity and turbulence, exposure gradients, such as
exposure itself and sedimentation. Storm regimes and marine heat waves
(thermal anomalies) drive cycles of reef destruction and renewal.

\subsection{Distribution}\label{distribution-28}

Tropical and subtropical waters on continental and island shelves,
mostly within latitudes of 30°N and 30°S.

\section{M1.4 Shellfish beds and
reefs}\label{m1.4-shellfish-beds-and-reefs}

Belongs to biome M1. Marine shelf biome, part of the Marine realm.

\subsection{Short description}\label{short-description-29}

These productive intertidal or subtidal biogenic ecosystems are formed
and dominated by sessile molluscs like mussels or oysters, around
temperate or tropical coasts and estuaries globally. They filter
plankton from the water column, acting as carbon sinks and modifying
local physical environments by changing currents and dampening wave
action. Distribution is limited by available rocky substrates on
low-energy coastlines, as well as requirements for high water quality
and oxygen availability. Many organisms are adapted to the extreme range
of conditions typical of the intertidal zone (e.g.~shellfish closing
valves to avoid adverse desiccation).

\subsection{Key Features}\label{key-features-29}

Intertidal or subtidal three-dimensional stuctures, formed primarily by
oysters and mussels, and supporting algae, invertebrates and fishes..

\subsection{Ecological traits}\label{ecological-traits-29}

These ecosystems are founded on intertidal or subtidal 3-dimensional
biogenic structures formed primarily by high densities of oysters and/or
mussels, which provide habitat for a moderate diversity of algae,
invertebrates, and fishes, few of which are entirely restricted to
oyster reefs. Structural profiles may be high (i.e.~reefs) or low
(i.e.~beds). Shellfish reefs are usually situated on sedimentary or
rocky substrates, but pen shells form high-density beds of vertically
orientated non-gregarious animals in soft sediments. Sessile
filter-feeders dominate these strongly heterotrophic but relatively
high-productivity systems. Tides bring in food and carry away waste.
Energy and matter in waste is processed by a subsystem of
deposit-feeding invertebrates. Predators are a small component of the
ecosystem biomass, but are nevertheless important in influencing the
recruitment, biomass, and diversity of prey organisms (e.g.~seastar
predation on mussels). Shellfish beds and reefs may influence adjoining
estuaries and coastal waters physically and biologically. Physically,
they modify patterns of currents, dampen wave energy and remove
suspended particulate matter through filter-feeding. Biologically, they
remove phytoplankton and produce abundant oyster biomass. They function
in biogeochemical cycling as carbon sinks, by increasing denitrification
rates, and through N burial/sequestration. Relatively (or entirely)
immobile and thin-shelled juveniles are highly susceptible to benthic
predators such as crabs, fish, and birds. Recruitment can depend on
protective microhabitats provided either by abiogenic or biogenic
structures. In intertidal environments, the survival of shellfish can
increase with density due to self-shading and moisture retention.

\subsection{Key Ecological Drivers}\label{key-ecological-drivers-29}

The availability of hard substrate (including shells of live or dead
conspecifics) can limit the establishment of reef-forming shellfish,
though a few occur on soft substrates. Many shellfish are robust to
changes in salinity, closing their valves for days to weeks to avoid
adverse conditions, but salinity may indirectly influence survival by
determining susceptibility to parasites. High suspended sediment loads
caused by high energy tides, rainfall, and run-off events or the erosion
of coastal catchments can smother larvae and impede filter-feeding. Most
reef- or bed-building shellfish cannot survive prolonged periods of low
dissolved oxygen. They are also sensitive to climate change stressors
such as temperature (and associated increased risk of desiccation for
intertidal species), as well as lowered pH as they are calcifiers. In
subtidal environments, the formation of reefs can help elevate animals
above anoxic bottom waters.

\subsection{Distribution}\label{distribution-29}

Estuarine and coastal waters of temperate and tropical regions,
extending from subtidal to intertidal zones. Present-day distributions
are shaped by historic overharvest, which has removed 85\% of reefs
globally.

\section{M1.5 Photo-limited marine animal
forests}\label{m1.5-photo-limited-marine-animal-forests}

Belongs to biome M1. Marine shelf biome, part of the Marine realm.

\subsection{Short description}\label{short-description-30}

These subtidal biogenic ecosystems, formed by either a single species or
a community of sessile filter feeders such as sponges, ascidians or
aphotic corals, are found mostly on hard substrates. Low light due to
depth, turbidity, ice cover or tannins from terrestrial runoff means
that photosynthesis is limited to microphytobenthos (like microalgae and
bacteria) or coralline algae. As a result, nutrient flux from surface
waters is important. A high diversity of invertebrates and fish are also
associated with these complex, three-dimensional forest-like habitats.

\subsection{Key Features}\label{key-features-30}

Largely heterotrophic systems dominated by megabenthic suspension
feeders and associated diverse epifauna, microphytobenthos and fish.

\subsection{Ecological traits}\label{ecological-traits-30}

These benthic systems are characterised by high densities of
megabenthic, sessile heterotrophic suspension feeders or coralline algae
that act as habitat engineers and dominate a subordinate autotrophic
biota. Unlike coral reefs and shellfish beds, the major sessile animals
in these animal forests include sponges, aphotic corals, hydroids,
ascidians, hydrocorals, bryozoans, polychaetes, and bivalves (the latter
only dominate in M1.4). Various coralline algae may be present in Marine
animal forests, but rhodoliths, are never dominant (cf.~M1.10). All
these organisms engineer complex three-dimensional biogenic structures,
sometimes of a single species or distinct phylogenetic groups. The
structural complexity generates environmental heterogeneity and habitat,
promoting a high diversity of invertebrate epifauna, with
microphytobenthos and fish. Endemism may be high. Low light limits
primary productivity especially by macroalgae, although
microphytobenthos can be important. Energy flow and depth-related
processes distinguish these systems from their deepwater aphotic
counterparts (M3.7). Nonetheless, these systems are strongly
heterotrophic, relying on benthic-pelagic coupling processes.
Consequently, these systems are generally of moderate productivity and
are often found near shallower, less photo-limited, high-productivity
areas. Complex biogeochemical cycles govern nutrient release, particle
retention, and carbon fixation. Biodiversity is enhanced by secondary
consumers (i.e.~deposit-feeding and filter-feeding invertebrates).
Predators may influence the biomass and diversity of epifaunal prey
organisms. Recruitment processes in benthic animals can be episodic and
highly localised and, together with slow growth rates, limit recovery
from disturbance.

\subsection{Key Ecological Drivers}\label{key-ecological-drivers-30}

Light is generally insufficient to support abundant macroalgae but is
above the photosynthetic threshold for coralline algae and
cyanobacteria. Light is limited by diffusion through deepwater, surface
ice cover, turbidity from river outflow, or tannins in terrestrial
runoff. Low to moderate temperatures may further limit productivity.
These systems are generally found on hard substrates but can occur on
soft substrates. Currents or resuspension and lateral transport
processes are important drivers of benthic-pelagic coupling, hence food
and nutrient supply. Natural physical disturbances are generally of low
severity and frequency, but ice scour can generate successional mosaics
where animal forests occur on subpolar shelves.

\subsection{Distribution}\label{distribution-30}

Tropical to polar coastal waters extending from the shallow subtidal to
the `twilight' zone on the shelf. Present-day distributions are likely
to have been modified by benthic trawling.

\section{M1.6 Subtidal rocky reefs}\label{m1.6-subtidal-rocky-reefs}

Belongs to biome M1. Marine shelf biome, part of the Marine realm.

\subsection{Short description}\label{short-description-31}

These rocky subtidal ecosystems are widespread globally on ocean shelves
. They are distinguished from kelp forests (M1.2) by their lack of a
dense macroalgal canopy. Indeed, their complex habitat structure is
derived mostly from irregular rock forms, rather than biogenic features,
and supports a diverse epibenthic fauna, with a range of mobile benthic
animals (e.g.~anemones), while truly sessile organisms tend to be small
(e.g.~turf algae, barnacles). Community structure depends on depth, wave
action, currents and light: for example, turbulence specialists like
barnacles are more prolific on shallower, higher energy reefs. Storms
impact structure by shifting sand and dislodging larger organisms
episodically.

\subsection{Key Features}\label{key-features-31}

Productive systems with functionally diverse sessile and mobile biota,
and a strong depth gradient.

\subsection{Ecological traits}\label{ecological-traits-31}

Submerged rocky reefs host trophically complex communities lacking a
dense macroalgal canopy (cf.~M1.2). Sessile primary producers and
invertebrate filter-feeders assimilate autochthonous and allochthonous
energy, respectively. Mobile biota occur in the water column.
Reef-associated organisms have diverse dispersal modes. Some disperse
widely as adults, some have non-dispersing larvae, others with sessile
adult phases develop directly on substrates, or have larval stages or
spores dispersed widely by currents or turbulence. Sessile plants
include green, brown, and red algae. To reduce dislodgement in storms,
macroalgae have holdfasts, while smaller species have low-growing `turf'
life forms. Many have traits such as air lacunae or bladders that
promote buoyancy. Canopy algae are sparse at the depths or levels of
wave exposure occupied by this functional group (cf.~kelp forests in
M1.2). Algal productivity and abundance decline with depth due to
diminution of light and are also kept in check by periodic storms and a
diversity of herbivorous fish, molluscs, and echinoderms. The latter two
groups and some fish are benthic and consume algae primarily in turf
form or at its juvenile stage before stipes develop. Sessile
invertebrates occur throughout. Some are high-turbulence specialists
(e.g.~barnacles, ascidians and anemones), while others become dominant
at greater depths as the abundance of faster-growing algae diminishes
(e.g.~sponges and red algae). Fish include both herbivores and
predators. Some are specialist bottom-dwellers, while others are more
generalist pelagic species. Herbivores promote diversity through
top-down regulation, influencing patch dynamics, trophic cascades and
regime shifts involving kelp forests in temperate waters (M1.2). Mosaics
of algal dominance, sessile invertebrate dominance, and barrens may
shift over time. Topographic variation in the rocky substrate promotes
habitat diversity and spatial heterogeneity. This provides refuges from
predators but also hiding places for ambush predators including
crustaceans and fish.

\subsection{Key Ecological Drivers}\label{key-ecological-drivers-31}

Minerogenic rocky substrates with variable topography and cobbles are
foundational to the habitats of many plants and animals, influencing how
they capture resources and avoid predation. A strong depth gradient and
substrate structures (e.g.~overhangs and caves) limit light
availability, as does turbidity. Currents and river outflows are crucial
to the delivery of resources, especially nutrients, and also play a key
role in biotic dispersal. Animal waste is a key nutrient source
sustaining algal productivity in more nutrient-limited systems. Salinity
is relatively constant through time (3.5\% on average). Turbulence from
subsurface wave action promotes substrate instability and maintains high
levels of dissolved oxygen. Episodic storms generating more extreme
turbulence shift sand and dislodge larger sessile organisms, create gaps
that may be maintained by herbivores or rapidly recolonized.

\subsection{Distribution}\label{distribution-31}

Widespread globally on rocky parts of continental and island shelves.

\section{M1.7 Subtidal sand beds}\label{m1.7-subtidal-sand-beds}

Belongs to biome M1. Marine shelf biome, part of the Marine realm.

\subsection{Short description}\label{short-description-32}

These relatively unstable shelf ecosystems in turbulent waters support
moderately diverse communities made up largely of consumers, like
invertebrate detritivores and filter-feeders, including burrowing
polychaetes, crustaceans, echinoderms, and molluscs. Filter feeders are
most common in higher energy areas of currents and wave action. Primary
producers are limited by substrate instability or light, with seagrass
ecosystems (M1.1) occurring where these factors are not limiting to
plant establishment and persistence. Low structural habitat complexity
means a lack of shelter, and many organisms display predator avoidance
traits like burrowing, shells or camouflage (e.g.~sole).

\subsection{Key Features}\label{key-features-32}

Medium to coarse-grained soft sediment with burrowing invertebrate
detrivores and suspension-feeders mostly relying on allochthonous
energy..

\subsection{Ecological traits}\label{ecological-traits-32}

Medium to coarse-grained, unvegetated, and soft minerogenic sediments
show moderate levels of biological diversity. The trophic network is
dominated by consumers with very few in situ primary producers.
Interstitial microalgae and planktonic algae are present, but larger
benthic primary producers are limited either by substrate instability or
light, which diminishes with depth. In shallow waters where light is
abundant and soft substrates are relatively stable, this group of
systems is replaced by group M1.1, which is dominated by vascular marine
plants. In contrast to those autochthonous systems, Subtidal sand beds
rely primarily on allochthonous energy, with currents generating strong
bottom flows and a horizontal flux of food. Sandy substrates tend to
have less organic matter content and lower microbial diversity and
abundance than muddy substrates (M1.8). Soft sediments may be dominated
by invertebrate detritivores and suspension-feeders including burrowing
polychaetes, crustaceans, echinoderms, and molluscs. Suspension-feeders
tend to be most abundant in high-energy environments where waves and
currents move sandy sediments, detritus, and living organisms. The
homogeneity and low structural complexity of the substrate exposes
potential prey to predation, especially from pelagic fish. Large
bioturbators such as dugongs, stingrays and whales may also be important
predators. Consequently, many benthic animals possess specialised traits
that enable other means of predator avoidance, such as burrowing,
shells, or camouflage.

\subsection{Key Ecological Drivers}\label{key-ecological-drivers-32}

The substrate is soft, minerogenic, low in organic matter, relatively
homogeneous, structurally simple, and mobile. The pelagic waters are
moderate to high-energy environments, with waves and currents promoting
substrate instability. Nonetheless, depositional processes dominate over
erosion, leading to net sediment accumulation. Fluvial inputs from land
and the erosion of headlands and sea cliffs contribute sediment,
nutrients, and organic matter, although significant lateral movement is
usually driven by longshore currents. Light availability diminishes with
depth. Mixing is promoted by waves and currents, ensuring low temporal
variability in salinity, which averages 3.5\%.

\subsection{Distribution}\label{distribution-32}

Globally widespread around continental and island shelves.

\section{M1.8 Subtidal mud plains}\label{m1.8-subtidal-mud-plains}

Belongs to biome M1. Marine shelf biome, part of the Marine realm.

\subsection{Short description}\label{short-description-33}

These low energy, muddy ocean shelf ecosystems are moderately productive
and typically dominated by microalgal and bacterial primary producers,
microbial decomposers, and larger deposit feeders like burrowing
polychaete worms and molluscs. Unlike subtidal sand beds (M1.7), the
microbial community has a strong influence on biogeochemical cycles. Low
oxygen zones can form where concentrations of organic matter and
associated high bacterial activity deplete this limited resource.

\subsection{Key Features}\label{key-features-33}

Soft sediment with limited primary production, abundant micro- and
macro-detritivores and associated foraging predators.

\subsection{Ecological traits}\label{ecological-traits-33}

The muddy substrates of continental and island shelves support
moderately productive ecosystems based on net allochthonous energy
sources. In situ primary production is contributed primarily by
microphytobenthos, mainly benthic diatoms with green microalgae, as
macrophytes are scarce or absent. Both decline with depth as light
diminishes. Drift algae can be extensive over muddy sediments,
particularly in sheltered waters. Abundant heterotrophic microbes
process detritus. The microbial community mediates most of the
biogeochemical cycles in muddy sediments, a feature distinguishing these
ecosystems from subtidal sand beds (M1.7). Deposit feeders (notably
burrowing polychaetes, crustaceans, echinoderms, and molluscs) are
important components of the trophic network as the low kinetic energy
environment promotes vertical food fluxes, which they are able to
exploit more effectively than suspension-feeders. The latter are less
abundant on subtidal mud plains than on rocky reefs (M1.6) and Subtidal
sand beds (M1.7) where waters are more turbulent and generate stronger
lateral food fluxes. Deposit feeders may also constrain the abundance of
co-occurring suspension-feeders by disturbing benthic sediment that
resettles and smothers their larvae and clogs their filtering
structures. Nonetheless, suspension-feeding tube worms may be common
over muddy sediments when settlement substrates are available. Although
such interference mechanisms may be important, competition is generally
weak. In contrast, foraging predators, including demersal fish, may have
a major structuring influence on these systems through impacts on the
abundance of infauna, particularly on settling larvae and recently
settled juveniles, but also adults. Burrowing is a key mechanism of
predator avoidance and the associated bioturbation is influential on
microhabitat diversity and resource availability within the sediment.

\subsection{Key Ecological Drivers}\label{key-ecological-drivers-33}

These depositional systems are characterised by low kinetic energy (weak
turbulence and currents), which promotes the accumulation of
fine-textured, stable sediments that are best developed on flat bottoms
or gentle slopes. The benthic surface is relatively homogeneous, except
where structure is engineered by burrowing organisms. The small particle
size and poor interchange of interstitial water limit oxygen supply, and
anaerobic conditions are especially promoted where abundant in-fall of
organic matter supports higher bacterial activity that depletes
dissolved oxygen. On the other hand, the stability of muddy substrates
makes them more conducive to the establishment of permanent burrows.
Bioturbation and bio-irrigation activities by a variety of benthic fauna
in muddy substrates are important contributors to marine nutrient and
biogeochemical cycling as well as the replenishment of oxygen.

\subsection{Distribution}\label{distribution-33}

Globally distributed in the low-energy waters of continental and island
shelves.

\section{M1.9 Upwelling zones}\label{m1.9-upwelling-zones}

Belongs to biome M1. Marine shelf biome, part of the Marine realm.

\subsection{Short description}\label{short-description-34}

These productive regions are often associated with eastern-boundary
current systems on the transition between marine shelves and the open
ocean, forming where divergence of surface water causes upwelling of
cold, nutrient-rich water. Bursts of primary productivity are associated
with naturally variable, often wind-driven upwelling events and support
a very high biomass of plankton, fish, and marine birds and mammals.
Small species like sardine and anchovy that operate at low trophic
levels dominate fish communities, may vary greatly in abundance through
time, and play important roles in food webs.

\subsection{Key Features}\label{key-features-34}

Cool, wind-driven systems with high productivity and variability,
supporting abundant plankton, fish, mammals and seabirds.

\subsection{Ecological traits}\label{ecological-traits-34}

Upwelled, nutrient-rich water supports very high net autochthonous
primary production, usually through diatom blooms. These, in turn,
support high biomass of copepods, euphausiids (i.e.~krill), pelagic and
demersal fish, marine mammals, and birds. Fish biomass tends to be
dominated by low- to mid-trophic level species such as sardine, anchovy,
and herring. The abundance of these small pelagic fish has been
hypothesised to drive ecosystem dynamics through `wasp-waist' trophic
control. Small pelagic fish exert top-down control on the copepod and
euphausiid plankton groups they feed on and exert bottom-up control on
predatory fish, although diel-migrant mesopelagic fish (M2.2) may have
important regulatory roles. Abundant species of higher trophic levels
include hake and horse mackerel, as well as pinnipeds and seabirds.
Highly variable reproductive success of planktivorous fish reflects the
fitness of spawners and suitable conditions for concentrating and
retaining eggs and larvae inshore prior to maturity.

\subsection{Key Ecological Drivers}\label{key-ecological-drivers-34}

Upwelling is a wind-driven process that draws cold, nutrient-rich water
towards the surface, displacing warmer, nutrient-depleted waters. The
strength of upwelling depends on interactions between local current
systems and the Coriolis effect that causes divergence, generally on the
eastern boundaries of oceans. The rate of upwelling, the offshore
transportation of nutrients, and the degree of stratification in the
water column once upwelling has occurred all determine the availability
of nutrients to plankton, and hence the size and structure of the
community that develops after an event. The main upwelling systems
around the world extend to depths of up to 500 m at the shelf break,
although primary production is restricted to the epipelagic zone
(\textless200 m). Upwelling zones are characterised by low sea-surface
temperatures and high chlorophyll a concentrations, high variability due
to large-scale interannual climate cycles (e.g El Niño Southern
Oscillation), as well as the pulsed and seasonal nature of the driving
winds, and periodic low-oxygen, low pH events due to high biological
activity and die-offs.

\subsection{Distribution}\label{distribution-34}

The most productive upwelling zones are coastal, notably in four major
eastern-boundary current systems (the Canary, Benguela, California, and
Humboldt). Weaker upwelling processes occurring in the open ocean are
included in M2.1 (e.g.~along the intertropical convergence zone).

\section{M2.1 Epipelagic ocean
waters}\label{m2.1-epipelagic-ocean-waters}

Belongs to biome M2. Pelagic ocean waters biome, part of the Marine
realm.

\subsection{Short description}\label{short-description-35}

This uppermost ocean layer (0-200m depth) is the most influenced by the
atmosphere, and is defined and structured by light availability.
Photosynthesis in these ecosystems accounts for half of all global
carbon fixation. That productivity supports diverse marine life,
including many visual predators, like tuna, that rely on the high light
environment. Migration is a common life history trait across all groups:
either vertical -- rising from the depths to feed at the surface at
night to evade daytime predators; or horizontal -- between breeding and
feeding grounds. Detritus from this zone is an important nutrient source
for lower oceanic layers.

\subsection{Key Features}\label{key-features-35}

Uppermost euphotic ocean, where phytoplankton production supports
abundant mobile zooplankton, fish, cephalopods, mammals and seabirds.

\subsection{Ecological traits}\label{ecological-traits-35}

The epipelagic or euphotic zone of the open ocean is the uppermost layer
that is penetrated by enough light to support photosynthesis. The vast
area of the ocean means that autochthonous productivity in the
epipelagic layer, largely by diatoms, accounts for around half of all
global carbon fixation. This in turn supports a complex trophic network
and high biomass of diatoms, copepods (resident and vertical migrants),
fish, cephalopods, marine mammals, and seabirds, including fast-swimming
visual predators taking advantage of the high-light environment. The
suitability of conditions for recruitment and reproduction depends on
the characteristics of the water column, which vary spatially and impact
productivity rates, species composition, and community size structure.
Mid-ocean subtropical gyres, for example, are characteristically
oligotrophic, with lower productivity than other parts of the ocean
surface. In contrast to the rest of the epipelagic zone, upwelling zones
are characterised by specific patterns of water movement that drive high
nutrient levels, productivity, and abundant forage fish, and are
therefore included in a different functional group (M1.9). Seasonal
variation in productivity is greater at high latitudes due to lower
light penetration and duration in winter compared to summer. The habitat
and lifecycle of some specialised pelagic species (e.g.~herbivorous
copepods, flying fish) are entirely contained within epipelagic ocean
waters, but many commonly occurring crustaceans, fish, and cephalopods
undertake either diel or ontogenetic vertical migration between the
epipelagic and deeper oceanic layers. These organisms exploit the food
available in the productive epipelagic zone either at night (when
predation risk is lower) or for the entirety of their less mobile,
juvenile life stages. Horizontal migration is also common and some
species (e.g.~tuna and migratory whales) swim long distances to feed and
reproduce. Other species use horizontal currents for passive migration,
particularly smaller planktonic organisms or life stages, e.g.~copepods
and small pelagic fish larvae moving between spawning and feeding
grounds. Unconsumed plankton and dead organisms sink from this upper
oceanic zone, providing an important particulate source of nutrients to
deeper, aphotic zones.

\subsection{Key Ecological Drivers}\label{key-ecological-drivers-35}

The epipelagic zone is structured by a strong depth gradient in light,
which varies seasonally at high latitudes. Light also varies with local
turbidity, but at lower latitudes may extend to \textasciitilde200 m
where light attenuates to 1\% of surface levels. Interaction at the
surface between the ocean and atmosphere leads to increased seasonality,
mixing, and warming, and makes this the most biologically and
physicochemically variable ocean layer. Nutrient levels are spatially
variable as a result. Salinity varies with terrestrial freshwater
inputs, evaporation, and mixing, with greater variation in semi-enclosed
areas (e.g.~the Mediterranean Sea) than the open ocean.

\subsection{Distribution}\label{distribution-35}

The surface layer of the entire open ocean beyond the near-shore zone.

\section{M2.2 Mesopelagic ocean
water}\label{m2.2-mesopelagic-ocean-water}

Belongs to biome M2. Pelagic ocean waters biome, part of the Marine
realm.

\subsection{Short description}\label{short-description-36}

This very low light `twilight zone' (\textasciitilde200-1000m depth)
divides the surface epipelagic waters from the deep ocean. Sunlight is
too dim for photosynthesis, but in the upper mesopelagic zone there is
enough to enable some visual predators to exploit their prey. Often used
as a refuge by species that migrate upward at night to feed in more
productive epipelagic waters when predation risk is lower, it supports a
high but unknown biomass of fish and planktonic detritivores, relying on
flux of nutrients from upper oceanic layers. Low oxygen zones can form
where patches of high biological activity deplete this limited resource.
Bioluminescence is a common trait in mesopelagic organisms.

\subsection{Key Features}\label{key-features-36}

Dimly lit `twilight' zone below the epipelagic with a high biomass of
diverse detrivores and predators and where bioliuminescence is common.

\subsection{Ecological traits}\label{ecological-traits-36}

The mesopelagic, dysphotic, or `twilight' zone begins below the
epipelagic layer and receives enough light to discern diurnal cycles but
too little for photosynthesis. The trophic network is therefore
dominated by detritivores and predators. The diverse organisms within
this layer consume and reprocess allochthonous organic material sinking
from the upper, photosynthetic layer. Hence, upper mesopelagic waters
include layers of concentrated plankton, bacteria, and other organic
matter sinking from the heterogeneous epipelagic zone (M2.1). Consumers
of this material including detritivorous copepods deplete oxygen levels
in the mesopelagic zone, more so than in other layers where oxygen can
be replenished via diffusion and mixing at the surface or photosynthesis
(as in the epipelagic zone), or where lower particulate nutrient levels
limit biological processes (as in the deeper layers). Many species
undertake diel vertical migration into the epipelagic zone during the
night to feed when predation risk is lower. These organisms use the
mesopelagic zone as a refuge during the day and increase the flow of
carbon between ocean layers. Bioluminescence is a common trait present
in more than 90\% of mesopelagic organisms often with silvery reflective
skin (e.g.~lantern fish). Fish in the lower mesopelagic zone
(\textgreater700 m) are less reflective and mobile due to reduced
selection pressure from visual predators in low light conditions. These
systems are difficult to sample, but advances in estimating fish
abundances indicate that biomass is very high, possibly two orders of
magnitude larger than global fisheries landings (1 × 1010 t).

\subsection{Key Ecological Drivers}\label{key-ecological-drivers-36}

Nutrient and energy availability depend on allochthonous fluxes of
carbon from the upper ocean. Energy assimilation from sunlight is
negligible. This is characteristically episodic and linked to events in
the epipelagic zone. Buffered from surface forcing by epipelagic waters,
the mesopelagic zone is less spatially and temporally variable, but the
interface between the two zones is characterised by heterogeneous
regions with greater biotic diversity. Areas of physicochemical
discontinuity (e.g.~current and water-mass boundaries and eddies) also
result in niche habitats with increased local diversity. Oxygen minimum
zones are formed in mesopelagic waters when biological activity reduces
oxygen levels in a water mass that is then restricted from mixing by
physical processes or features. Oxygen minimum zones support specialised
biota and have high levels of biological activity around their borders.

\subsection{Distribution}\label{distribution-36}

Global oceans from a depth of \textasciitilde200 m or where \textless1\%
of light penetrates, down to 1,000 m.

\section{M2.3 Bathypelagic ocean
waters}\label{m2.3-bathypelagic-ocean-waters}

Belongs to biome M2. Pelagic ocean waters biome, part of the Marine
realm.

\subsection{Short description}\label{short-description-37}

These deep (\textasciitilde1000-3000m depth), open-ocean ecosystems
receive no sunlight and rely on detritus from upper layers for
nutrients. Other resources such as oxygen are replenished via the
`global ocean conveyer belt' (thermohaline circulation) that starts when
distant, polar surface waters cool and sink. With no primary producers,
life is limited to groups like zooplankton, jellyfish, crustaceans,
cephalopods and fish like the gulper eel. Common adaptations that enable
animals to live under high pressure and no light include slow
metabolism, long generation lengths and low density bodies.

\subsection{Key Features}\label{key-features-37}

Lightless, high pressure depths where adapted zooplankton, crustaceans,
jellies, cephalopods and fish rely on nutrients falling from above.

\subsection{Ecological traits}\label{ecological-traits-37}

These are deep, open-ocean ecosystems in the water column, generally
between 1,000--3,000 m in depth. Energy sources are allochthonous,
derived mainly from the fallout of particulate organic matter from the
epipelagic horizon (M2.1). Total biomass declines exponentially from an
average of 1.45 mgC m-3 at 1,000 m depth to 0.16 mgC m-3 at 3,000 m.
Trophic structure is truncated, with no primary producers. Instead, the
major components are zooplankton, micro-crustaceans (e.g.~shrimps),
medusozoans (e.g.~jellyfish), cephalopods, and four main guilds of fish
(gelativores, zooplanktivores, micronektivores, and generalists). These
organisms generally do not migrate vertically, in contrast to those in
the mesopelagic zone (M2.2). Larvae often hatch from buoyant egg masses
at the surface to take advantage of food sources. Long generation
lengths (\textgreater20 years in most fish) and low fecundity reflect
low energy availability. Fauna typically have low metabolic rates, with
bathypelagic fish having rates of oxygen consumption \textasciitilde10\%
of that of epipelagic fish. Fish are consequently slow swimmers with
high water content in muscles and relatively low red-to-white muscle
tissue ratios. They also have low-density bodies, reduced skeletons,
and/or specialised buoyancy organs to achieve neutral buoyancy for
specific depth ranges. Traits related to the lack of light include
reduced eyes, lack of pigmentation, and enhanced vibratory and
chemosensory organs. Some planktonic forms, medusas, and fish have
internal light organs that produce intrinsic or bacterial
bioluminescence to attract prey items or mates or to defend themselves.
Most of the biota possess cell membranes with specialised phospholipid
composition, intrinsic protein modifications, and protective osmolytes
(i.e.~organic compounds that influence the properties of biological
fluids) to optimise protein function at high pressure.

\subsection{Key Ecological Drivers}\label{key-ecological-drivers-37}

No light penetrates from the ocean surface to bathypelagic waters.
Oxygen concentrations are not limiting to aerobic respiration (mostly
3--7 mL.L-1) and are recharged through thermohaline circulation by
cooling. Oxygenated water is circulated globally from two zones (the
Weddell Sea and the far North Atlantic Ocean) where ice formation and
surface cooling create high-salinity, oxygenated water that sinks and is
subsequently circulated globally via the `great ocean conveyor'.
Re-oxygenation frequency varies from 300 to 1,000 years, depending on
the circulation route. More local thermohaline circulation occurs by
evaporation in the Mediterranean and Red Seas, resulting in warm
temperatures (13--15°C) at great depths. Otherwise, bathypelagic
temperatures vary from −1°C in polar waters to 2--4°C in tropical and
temperate waters. Nutrient levels are low and derive from the fall of
organic remains from surface horizons. Pressure varies with depth from
100 to 300 atmospheres.

\subsection{Distribution}\label{distribution-37}

All oceans and deep seas beyond the continental slope and within a depth
range of 1,000 -- 3,000 m.

\section{M2.4 Abyssopelagic ocean
waters}\label{m2.4-abyssopelagic-ocean-waters}

Belongs to biome M2. Pelagic ocean waters biome, part of the Marine
realm.

\subsection{Short description}\label{short-description-38}

At greater depths (\textasciitilde3,000-6,000m) than bathypelagic
systems, these very deep open ocean ecosystems receive no light and rely
solely on debris from upper layers for nutrients. Other resources such
as oxygen are replenished via the `global ocean conveyer belt'
(thermohaline circulation) that starts when distant, polar surface
waters cool and sink. There is a low diversity and low density of life,
largely planktonic detritivores, along with some gelatinous
invertebrates and scavenging or predatory fish like the anglerfish. Life
histories body structures and physiological traits are adapted to the
very high pressure and lack of light (e.g.~non-visual sensory organs,
specialised metabolic proteins, and low density body structures).

\subsection{Key Features}\label{key-features-38}

Lightless, high pressure depths with limited nutrients and low
biodiversity of adapted detrivores, jellies, scavengers and predatory
fish.

\subsection{Ecological traits}\label{ecological-traits-38}

These deep, open-ocean ecosystems span depths from 3,000 to 6,000 m.
Autotrophs are absent and energy sources are entirely allochthonous.
Particulate organic debris is imported principally from epipelagic
horizons (M2.1) and the flux of matter diminishing through the
mesopelagic zone (M2.2) and bathypelagic zone (M2.3). Food for
heterotrophs is therefore very scarce. Due to extreme conditions and
limited resources, biodiversity is very low. Total biomass declines
exponentially from an average of 0.16 mgC m-3 at 3,000 m in depth to
0.0058 mgC m-3 at 6,000 m. However, there is an order of magnitude
variation around the mean due to regional differences in the
productivity of surface waters. Truncated trophic networks are dominated
by planktonic detritivores, with low densities of gelatinous
invertebrates and scavenging and predatory fish. Fauna typically have
low metabolic rates and some have internal light organs that produce
bioluminescence to attract prey or mates or to defend themselves.
Vertebrates typically have reduced skeletons and watery tissues to
maintain buoyancy. Most of the biota possesses cell membranes with
specialised phospholipid composition, intrinsic protein modifications,
and protective osmolytes (i.e.~organic compounds that influence the
properties of biological fluids) to optimise protein function at high
pressure.

\subsection{Key Ecological Drivers}\label{key-ecological-drivers-38}

No light penetrates from the ocean surface to abyssopelagic waters.
Nutrient concentrations are very low and recharge is dependent on
organic flux and detrital fall from the epipelagic zone. Oxygen
concentrations, however, are not limiting to aerobic respiration (mostly
3--7mL.L-1) and are generally recharged through global thermohaline
circulation driven by cooling in polar regions. Water temperatures vary
from below zero in polar waters up to 3°C in parts of the Atlantic.
Hydrostatic pressure is extremely high (300--600 atmospheres). Currents
are weak, salinity is stable, and there is little spatial heterogeneity
in the water column.

\subsection{Distribution}\label{distribution-38}

All oceans and the deepest parts of the Mediterranean Sea beyond the
continental slope, mid-ocean ridges, and plateaus at depths of
3,000--6,000 m.

\section{M2.5 Sea ice}\label{m2.5-sea-ice}

Belongs to biome M2. Pelagic ocean waters biome, part of the Marine
realm.

\subsection{Short description}\label{short-description-39}

These seasonally frozen surface waters in polar oceans are one of the
most dynamic ecosystems on earth. The sea-ice itself provides habitat
for ice-dependent species, such as the microalgal and microbial
communities that form the basis of communities in waters below, while
plankton, fish and marine birds and mammals feed on and around the ice.
Sea-ice plays a crucial role in both pelagic marine ecosystems and
biogeochemical processes like ocean-atmosphere gas exchange.

\subsection{Key Features}\label{key-features-39}

Highly dynamic, seasonally frozen surface waters support diverse
ice-associated organisms from plankton to seabirds and whales.

\subsection{Ecological traits}\label{ecological-traits-39}

The seasonally frozen surface of polar oceans (1--2 m thick in the
Antarctic and 2--3 m thick in the Arctic) may be connected to land or
permanent ice shelves and is one of the most dynamic ecosystems on
earth. Sympagic (i.e.~ice-associated) organisms occur in all physical
components of the sea-ice system including the surface, the internal
matrix and brine channel system, the underside, and nearby waters
modified by sea-ice presence. Primary production by microalgal and
microbial communities beneath and within sea ice form the base of the
food web and waters beneath sea ice develop. The standing stocks
produced by these microbes are significantly greater than in ice-free
areas despite shading by ice and are grazed by diverse zooplankton
including krill. The sea ice underside provides refuge from surface
predators and is an important nursery for juvenile krill and fish.
Deepwater fish migrate vertically to feed on zooplankton beneath the sea
ice. High secondary production (particularly of krill) in sea ice and
around its edges supports seals, seabirds, penguins (in the Antarctic),
and baleen whales. The highest trophic levels include vertebrate
predators such as polar bears (in the Arctic), leopard seals, and
toothed whales. Sea ice also provides resting and/or breeding habitats
for pinnipeds (seals), polar bears, and penguins. As the sea ice decays
annually, it releases biogenic material consumed by grazers and
particulate and dissolved organic matter, nutrients, freshwater and
iron, which stimulate phytoplankton growth and have important roles in
biogeochemical cycling.

\subsection{Key Ecological Drivers}\label{key-ecological-drivers-39}

Sea ice is integral to the global climate system and has a crucial
influence on pelagic marine ecosystems and biogeochemical processes. Sea
ice limits atmosphere-ocean gas and momentum exchanges, regulates sea
temperature, reflects solar radiation, acquires snow cover, and
redistributes freshwater to lower latitudes. The annual retreat of sea
ice during spring and summer initiates high phytoplankton productivity
at the marginal ice zone and provides a major resource for grazing
zooplankton, including krill. Polynyas, where areas of low ice
concentration are bounded by high ice concentrations, have very high
productivity levels. Most sea ice is pack-ice transported by wind and
currents. Fast ice forms a stationary substrate anchored to the coast,
icebergs, glaciers, and ice shelves and can persist for decades.

\subsection{Distribution}\label{distribution-39}

Arctic Ocean 0--45°N (Japan) or only to 80°N (Spitsbergen). Southern
Ocean 55--70°S. At maximum extent, sea ice covers \textasciitilde5\% of
the Northern Hemisphere and 8\% of the Southern Hemisphere.

\section{M3.1 Continental and island
slopes}\label{m3.1-continental-and-island-slopes}

Belongs to biome M3. Deep sea floors biome, part of the Marine realm.

\subsection{Short description}\label{short-description-40}

These lightless slopes of sand, mud and rocky outrops run down from the
shallower shelf break to the very deep abyssal basins. Nutrients falling
from upper ocean layers and delivered by currents from the shelf support
diverse communities of microbial decomposers, detritivores like crabs
and demersal fish, and their predators, but sessile animals are rare and
algae are absent. Biomass is relatively low and peaks at mid-slope,
diminishing with depth as food and temperature decrease and bathymetric
pressure increases.

\subsection{Key Features}\label{key-features-40}

Large sedimentary, aphotic,~ and heterotrophic slopes where depth
gradients result in a bathymetric faunal zonation of high taxonomic
diverstiy..

\subsection{Ecological traits}\label{ecological-traits-40}

These aphotic heterotrophic ecosystems fringe the margins of continental
plates and islands, extending from the shelf break (\textasciitilde250 m
depth) to the abyssal basins (4,000 m). These large sedimentary slopes
with localised rocky outcrops are characterised by strong depth
gradients in the biota and may be juxtaposed with specialised ecosystems
such as submarine canyons (M3.2), deep-water biogenic systems (M3.6),
and chemosynthetic seeps (M3.7), as well as landslides and
oxygen-minimum zones. Energy sources are derived mostly from lateral
advection from the shelf and vertical fallout of organic matter
particles through the water column and pelagic fauna impinging on the
slopes, which varies seasonally with the productivity of the euphotic
layers. Other inputs of organic matter include sporadic pulses of large
falls (e.g.~whale falls and wood falls). Photoautotrophs and resident
herbivores are absent and the trophic network is dominated by microbial
decomposers, detritivores, and their predators. Depth-related gradients
result in a marked bathymetric zonation of faunal communities, and there
is significant basin-scale endemism in many taxa. The taxonomic
diversity of these heterotrophs is high and reaches a maximum at middle
to lower depths. The biomass of megafauna decreases with depth and the
meio-fauna and macro-fauna become relatively more important, but maximum
biomass occurs on mid-slopes in some regions. The megafauna is often
characterised by sparse populations of detritivores, including
echinoderms, crustaceans, and demersal fish, but sessile benthic
organisms are scarce and the bottom is typically bare, unconsolidated
sediments.

\subsection{Key Ecological Drivers}\label{key-ecological-drivers-40}

The continental slopes are characterised by strong environmental depth
gradients in pressure, temperature, light, and food. Limited sunlight
penetration permits some visual predation but no photosynthesis below
250 m and rapidly diminishes with depth, with total darkness (excluding
bioluminescence) below 1,000 m. Hydrostatic pressure increases with
depth (1 atmosphere every 10 m). Temperature drastically shifts below
the thermocline from warmer surface waters to cold, deep water (1--3°C),
except in the Mediterranean Sea (13°C) and the Red Sea (21°C). Food
quantity and quality decrease with increasing depth, as heterotrophic
zooplankton efficiently use the labile compounds of the descending
particulate organic matter. Sediments on continental slopes provide
important ecosystem services, including nutrient regeneration and carbon
sequestration.

\subsection{Distribution}\label{distribution-40}

Fringing the margins of all ocean basins and oceanic islands. Extending
beneath 11\% of the ocean surface at depths of 250--4,000 m.

\section{M3.2 Submarine canyons}\label{m3.2-submarine-canyons}

Belongs to biome M3. Deep sea floors biome, part of the Marine realm.

\subsection{Short description}\label{short-description-41}

Submarine canyons house some of the most productive and diverse deep sea
ecosystems. They can disrupt local currents and channel nutrients from
the continental shelves into ocean basins. The flux of nutrients
promotes productivity, with high densities of burrowing organisms can
live in muddy or sandy bottoms, and filter feeders like cold-water
corals inhabit the rocky walls. Canyons provide important shelter,
spawning and nursery areas and feeding grounds for many organisms
including non-resident megafauna like whales.

\subsection{Key Features}\label{key-features-41}

Dinamics and heterogenous geomorphic features,~ supporting highly
diverse heterotrophic communities through enhaced transport of energy
from the continents to the deep sea..

\subsection{Ecological traits}\label{ecological-traits-41}

Submarine canyons are major geomorphic features that function as dynamic
flux routes for resources between continental shelves and ocean basins.
As a result, canyons are one of the most productive and biodiverse
habitats in the deep sea. Habitat heterogeneity and temporal variability
are key features of submarine canyons, with the diversity of topographic
and hydrodynamic features and substrate types (e.g.~mud, sand, and rocky
walls) within and among canyons contributing to their highly diverse
heterotrophic faunal assemblages. Photoautotrophs are present only at
the heads of some canyons. Canyons are characterised by meio-, macro-,
and mega-fauna assemblages with greater abundances and/or biomass than
adjacent continental slopes (M3.1) due mainly to the greater quality and
quantity of food inside canyon systems. Habitat complexity and high
resource availability make canyons important refuges, nurseries,
spawning areas, and regional source populations for fish, crustaceans,
and other benthic biota. Steep exposed rock and strong currents may
facilitate the development of dense communities of sessile predators and
filter-feeders such as cold-water corals and sponges, engineering
complex three-dimensional habitats. Soft substrates favour high
densities of pennatulids and detritivores such as echinoderms. The role
of canyons as centres of carbon deposition makes them an extraordinary
habitat for deep-sea deposit-feeders, which represent the dominant
mobile benthic trophic guild. The high productivity attracts
pelagic-associated secondary and tertiary consumers, including
cetaceans, which may visit canyons for feeding and breeding.

\subsection{Key Ecological Drivers}\label{key-ecological-drivers-41}

Submarine canyons vary in their origin, length, depth range (mean: 2,000
m), hydrodynamics, sedimentation patterns, and biota. Their complex
topography modifies regional currents, inducing local upwelling,
downwelling, and other complex hydrodynamic processes (e.g.~turbidity
currents, dense shelf water cascading, and internal waves). Through
these processes, canyons act as geomorphic conduits of water masses,
sediments, and organic matter from the productive coastal shelf to deep
basins. This is particularly evident in shelf-incising canyons directly
affected by riverine inputs and other coastal processes. Complex
hydrodynamic patterns enhance nutrient levels and food inputs mostly
through downward lateral advection but also by local upwelling, active
biological transport by vertical migration of organisms, and passive
fall of organic flux of varied particles sizes. Differences among
canyons are driven primarily by variation in the abundance and quality
of food sources, as well as finer-scale drivers including the
variability of water mass structure (i.e.~turbidity, temperature,
salinity, and oxygen gradients), seabed geomorphology, depth, and
substratum.

\subsection{Distribution}\label{distribution-41}

Submarine canyons cover 11.2\% of continental slopes, with 9,000 large
canyons recorded globally. Most of their extent is distributed below 200
m, with a mean depth of 2,000 m.

\section{M3.3 Abyssal plains}\label{m3.3-abyssal-plains}

Belongs to biome M3. Deep sea floors biome, part of the Marine realm.

\subsection{Short description}\label{short-description-42}

These ecosystems on the very deep seafloors (3000-6000m) of all oceans
support a low biomass but high diversity of small invertebrates and
microbes, along with larger crustaceans, demersal fish and echinoderms
like starfish. Tracks and burrows of larger organisms in fine sediments
that may be up to thousands of metres thick, structure habitat for
smaller invertebrates. The absence of light, scarcity of food, and
extreme hydrostatic pressures limit the density and biomass of organisms
as well as the interactions among them. Inaccessible and little known,
exploration of these ecosystems continue to reveal large numbers of
species new to science.

\subsection{Key Features}\label{key-features-42}

Largest benthic heterotrophic system, mostly of fine sediment, supporing
high biodiversity of small organisms (microbes, meio- and macro-fauna).

\subsection{Ecological traits}\label{ecological-traits-42}

This is the largest group of benthic marine ecosystems, extending
between 3,000 and 6,000 m depth and covered by thick layers (up to
thousands of metres) of fine sediment. Less than 1\% of the seafloor has
been investigated biologically. Tests of giant protozoans and the
lebensspuren (i.e.~tracks, borrows, and mounds) made by megafauna
structure the habitats of smaller organisms. Ecosystem engineering
aside, other biotic interactions among large fauna are weak due to the
low densities of organisms. Abyssal communities are heterotrophic, with
energy sources derived mostly from the fallout of organic matter
particles through the water column. Large carrion falls are major local
inputs of organic matter and can later become important chemosynthetic
environments (M3.7). Seasonal variation in particulate organic matter
flux reflects temporal patterns in the productivity of euphotic layers.
Input of organic matter can also be through sporadic pulses of large
falls (e.g.~whale falls and wood falls). Most abyssal plains are
food-limited and the quantity and quality of food input to the abyssal
seafloor are strong drivers shaping the structure and function of
abyssal communities. Abyssal biomass is very low and dominated by
meio-fauna and microorganisms that play key roles in the function of
benthic communities below 3,000 m depth. The abyssal biota, however, is
highly diverse, mostly composed of macro- and meio-fauna with large
numbers of species new to science (up to 80\% in some regions). Many
species have so far been sampled only as singletons (only one specimen
per species) or as a few specimens. The megafauna is often characterised
by sparse populations of detritivores, notably echinoderms, crustaceans,
and demersal fish. Species distribution and major functions such as
community respiration and bioturbation are linked to particulate organic
carbon flux. These functions modulate the important ecosystem services
provided by abyssal plains, including nutrient regeneration and carbon
sequestration.

\subsection{Key Ecological Drivers}\label{key-ecological-drivers-42}

No light penetrates to abyssal depths. Hydrostatic pressure is very high
(300--600 atmospheres). Water masses above abyssal plains are well
oxygenated and characterised by low temperatures (−0.5--3°C), except in
the Mediterranean Sea (13°C) and the Red Sea (21°C). The main driver of
most abyssal communities is food, which mostly arrives to the seafloor
as particulate organic carbon or `marine snow'. Only 0.5--2\% of the
primary production in the euphotic zone reaches the abyssal seafloor,
with the quantity decreasing with increasing depth.

\subsection{Distribution}\label{distribution-42}

Seafloor of all oceans between 3,000 and 6,000 m depth, accounting for
76\% of the total seafloor area, segmented by mid-ocean ridges, island
arcs, and trenches.

\section{M3.4 Seamounts, ridges and
plateaus}\label{m3.4-seamounts-ridges-and-plateaus}

Belongs to biome M3. Deep sea floors biome, part of the Marine realm.

\subsection{Short description}\label{short-description-43}

These deep, lightless ecosystems are centred on major geomorphic
features of deep ocean floors. These elevated features interrupt lateral
ocean currents and generate upwelling of nutrients. This promotes
productivity in surface waters, which returns to depths as detritus. The
input of these resources, combined with varied rocky micohabitats,
unlike the sand and mud around them, supports diverse communities of
immobile filter feeders (e.g.~sponges), mobile benthic organisms
(e.g.~molluscs and starfish), and large aggregations of fish, especially
around seamounts.

\subsection{Key Features}\label{key-features-43}

Elevated geomorhic features with modified hydrography and heterogeneous
habitat supporting high bnethic and pelagic productivity.

\subsection{Ecological traits}\label{ecological-traits-43}

Seamounts, plateaus, and ridges are major geomorphic features of the
deep oceanic seafloor, characterised by hard substrates, elevated
topography, and often higher productivity than surrounding waters.
Topographically modified currents affect geochemical cycles, nutrient
mixing processes, and detrital fallout from the euphotic zone that
deliver allochthonous energy and nutrients to these
heterotroph-dominated systems. Suspension-feeders and their dependents
and predators dominate the trophic web, whereas deposit-feeders and
mixed-feeders are less abundant than in other deep-sea systems.
Autotrophs are generally absent. Summits that reach the euphotic zone
are included within functional groups of the Marine shelf biome.
Bathymetric gradients and local substrate heterogeneity support marked
variation in diversity, composition, and abundance. Rocky walls, for
example, may be dominated by sessile suspension-feeders including
cnidarians (especially corals), sponges, crinoids, and ascidians. High
densities of sessile animals may form deep-water biogenic beds (M3.5),
but those systems are not limited to seamounts or ridges. Among the
mobile benthic fauna, molluscs and echinoderms can be abundant.
Seamounts also support dense aggregations of large fish, attracted by
the high secondary productivity of lower trophic levels in the system,
as well as spawning and/or nursery habitats. Elevated topography affects
the distribution of both benthic and pelagic fauna. Seamounts and ridges
tend to act both as stepping stones for the dispersal of slope-dwelling
biota and as dispersal barriers between adjacent basins, while insular
seamounts may have high endemism.

\subsection{Key Ecological Drivers}\label{key-ecological-drivers-43}

Seamounts, rising more than 1,000 m above the sediment-covered seabed,
and smaller peaks, knobs, and hills are topographically isolated
features, mostly of volcanic origin. Mid-ocean ridges are
semi-continuous mountain chains that mark the spreading margins of
adjacent tectonic plates. These prominent topographic formations
interact with water masses and currents, increasing turbulence, mixing,
particle retention, and the upward movement of nutrients from large
areas of the seafloor. This enhances productivity on the seamounts and
ridges themselves and also in the euphotic zone above, some of which
returns to the system through detrital fallout. A diversity of
topographic, bathymetric, and hydrodynamic features and substrate types
(e.g.~steep rocky walls, flat muddy areas, and biogenic habitats at
varied depths) contribute to niche diversity and biodiversity. Major
bathymetric clines associated with elevated topography produce gradients
that shape ecological traits including species richness, community
structure, abundance, biomass, and trophic modes.

\subsection{Distribution}\label{distribution-43}

About 171,000 seamounts, knolls, and hills documented worldwide so far,
covering \textasciitilde2.6\% of the sea floor. Ridges cover
\textasciitilde9.2\% of the sea floor along a semi-continuous, 55,000km
long system.

\section{M3.5 Deepwater biogenic
beds}\label{m3.5-deepwater-biogenic-beds}

Belongs to biome M3. Deep sea floors biome, part of the Marine realm.

\subsection{Short description}\label{short-description-44}

Relatively complex three-dimensional structures are formed by
slow-growing, filter-feeders like sponges, corals and bivalves. Without
light, they rely on currents and fallout from upper ocean layers for
energy and also nutrients. However, their structural complexity provides
habitats for a great diversity of dependent species including symbionts,
microbial biofilms and associated grazers, and filter-feeding epifauna.
Mobile predators like crabs and benthic demersal fish contribute to
diverse communities.

\subsection{Key Features}\label{key-features-44}

Benthic sessile suspension feeders that crate structurally complex 3D
habitat, supporting high biodiversity.

\subsection{Ecological traits}\label{ecological-traits-44}

Benthic, sessile suspension-feeders such as aphotic corals, sponges, and
bivalves form structurally complex, three-dimensional structures or
`animal forests' in the deep oceans. In contrast to their shallow-water
counterparts in coastal and shelf systems (M1.5), these ecosystems are
aphotic and rely on allochthonous energy sources borne in currents and
pelagic fallout. The trophic web is dominated by filter-feeders,
decomposers, detritivores, and predators. Primary producers and
associated herbivores are only present at the interface with the photic
zone (\textasciitilde250 m depth). The biogenic structures are slow
growing but critical to local demersal biota in engineering shelter from
predators and currents, particularly in shallower, more dynamic waters.
They also provide stable substrates and enhance food availability. This
habitat heterogeneity becomes more important with depth as stable,
complex elevated substrate becomes increasingly limited. These
structures and the microenvironments within them support a high
diversity of associated species including symbionts, microorganisms in
coral biofilm, filter-feeding epifauna, biofilm grazers, mobile
predators (e.g.~polychaetes and crustaceans), and benthic demersal fish.
Diversity is positively related to the size, flexibility, and structural
complexity of habitat-forming organisms. Their impact on hydrography and
the flow of local currents increases retention of particulate matter,
zooplankton, eggs and larvae from the water column. This creates
positive conditions for suspension-feeders, which engineer their
environment and play important roles in benthic-pelagic coupling,
increasing the flux of matter and energy from the water column to the
benthic community.

\subsection{Key Ecological Drivers}\label{key-ecological-drivers-44}

The productivity of surface water, the vertical flux of nutrients, water
temperature, and hydrography influence the availability of food, and
hence the distribution and function of deep-water biogenic beds.
Although these systems occur on both hard and soft substrates, the
latter are less structurally complex and less diverse. Chemical
processes are important and ocean acidity is limiting. The presence of
cold-water corals, for example, has been linked to the depth of
aragonite saturation. Habitat-forming species prefer regions
characterised by oxygenation and currents or high flow, generally
avoiding oxygen-minimum zones. Benthic biogenic structures and their
dependents are highly dependent on low levels of physical disturbance
due to slow growth rates and recovery times.

\subsection{Distribution}\label{distribution-44}

Patchy but widespread distribution across the deep sea floor below 250 m
depth. Poorly explored and possibly less common on abyssal plains.

\section{M3.6 Hadal trenches and
troughs}\label{m3.6-hadal-trenches-and-troughs}

Belongs to biome M3. Deep sea floors biome, part of the Marine realm.

\subsection{Short description}\label{short-description-45}

The deepest ocean trenches, up to 11 km beneath the surface, are the
least explored marine ecosystems. They are also one of the most extreme,
with no sunlight, low temperatures, nutrient scarcity and hydrostatic
pressures of 600 to 1100 atmospheres, extending beyond the limits to
vertebrate life. The major sources of nutrients and carbon are fallout
from upper layers, drifts of fine sediment and landslides. Most
organisms are scavengers and detrivores, like the supergiant amphipod,
with abundance of predatory fish and crustaceans diminishing with depth.

\subsection{Key Features}\label{key-features-45}

Deepest ocean systems, poorly explored, mostly of fine nutrient-poor
sediment dominated by scavangers and detritivors.

\subsection{Ecological traits}\label{ecological-traits-45}

Hadal zones are the deepest ocean systems on earth and among the least
explored. They are heterotrophic, with energy derived from the fallout
of particulate organic matter through the water column, which varies
seasonally and geographically and accumulates in the deepest axes of the
trenches. Most organic matter reaching hadal depths is nutrient-poor
because pelagic organisms use the labile compounds from the particulate
organic matter during fallout. Hadal systems are therefore food-limited,
but particulate organic matter flux may be boosted by sporadic pulses
(e.g.~whale falls and wood falls) and sediment transported by advection
and seismically induced submarine landslides. Additional energy is
contributed by chemosynthetic bacteria that can establish symbiotic
relationships with specialised fauna. These are poorly known but more
are expected to be discovered in the future. Hadal trophic networks are
dominated by scavengers and detritivores, although predators (including
through cannibalism) are also represented. Over 400 species are
currently known from hadal ecosystems, with most metazoan taxa
represented including amphipods, polychaetes, gastropods, bivalves,
holothurians, and fish. These species possess physiological adaptations
to high hydrostatic pressure, darkness, low temperature, and low food
supply. These environmental filters, together with habitat isolation,
result in high levels of endemism. Gigantism in amphipods, mysids, and
isopods contrasts with the dwarfism in meio-fauna (e.g.~nematodes,
copepods, and kinorhynchs).

\subsection{Key Ecological Drivers}\label{key-ecological-drivers-45}

The hadal benthic zone extends from 6,000 to 11,000 m depth and includes
27 disjoint deep-ocean trenches, 13 troughs, and 7 faults. Sunlight is
absent, nutrients and organic carbon are scarce, and hydrostatic
pressure is extremely high (600--1,100 atmospheres). Water masses in
trenches and troughs are well oxygenated by deep currents and experience
constant, low temperatures (1.5--2.5°C). Rocky substrates outcrop on
steep slopes of trenches and faults, while the floors comprise large
accumulations of fine sediment deposited by mass movement, including
drift and landslides, which are important sources of organic matter.
Sediment, organic matter and pollutants tend to be ``funnelled'' and
concentrated in the axis of the trenches.

\subsection{Distribution}\label{distribution-45}

A cluster of isolated trenches in subduction zones, faults, and troughs
or basins, mostly in the Pacific Ocean, as well as the Indian and
Southern Oceans, accounting for 1--2\% of the total global benthic area.

\section{M3.7 Chemosynthetic-based-ecosystems
(CBE)}\label{m3.7-chemosynthetic-based-ecosystems-cbe}

Belongs to biome M3. Deep sea floors biome, part of the Marine realm.

\subsection{Short description}\label{short-description-46}

In these very deep, high pressure ecosystems, primary productivity is
fuelled by chemical compounds as energy sources instead of light
(chemoautotrophy). This group of productive deep sea ecosystems include:
1) hydrothermal vents on mid-ocean ridges and volcanically active
seamounts, where temperatures may reach 400°C; 2) cold seeps typically
on continental slopes; and 3) large organic falls of whales or wood.
These specialised environments have high biomass but low diversity of
organisms including microbes, tubeworms and shrimps, many of which are
locally unique.

\subsection{Key Features}\label{key-features-46}

Systems supported by microbial chemoautotrophy with high biomass of
relatively low diversity, highly speciliased, fauna.

\subsection{Ecological traits}\label{ecological-traits-46}

Chemosynthetic-based ecosystems (CBEs) include three major types of
habitats between bathyal and abyssal depths: 1) hydrothermal vents on
mid-ocean ridges, back-arc basins, and active seamounts; 2) cold seeps
on active and passive continental margins; and 3) large organic falls of
whales or wood. All these systems are characterised by microbial primary
productivity through chemoautotrophy, which uses reduced compounds (such
as H2S and CH4) as energy sources instead of light. Microbes form
bacterial mats and occur in trophic symbiosis with most megafauna. The
continuous sources of energy and microbial symbiosis fuel high faunal
biomass. However, specific environmental factors (e.g.~high temperature
gradients at vents, chemical toxicity, and symbiosis dependence) result
in a low diversity and high endemism of highly specialised fauna.
Habitat structure comprises hard substrate on vent chimneys and mostly
biogenic substrate at seeps and food-falls. Most fauna is sessile or
with low motility and depends on the fluids emanating at vents and seeps
or chemicals produced by microbes on food-falls, and thus is spatially
limited. Large tubeworms, shrimps, crabs, bivalves, and gastropods
dominate many hydrothermal vents, with marked biogeographic provinces.
Tubeworms, mussels, and decapod crustaceans often dominate cold seeps
with demersal fish. These are patchy ecosystems where connectivity
relies on the dispersal of planktonic larvae.

\subsection{Key Ecological Drivers}\label{key-ecological-drivers-46}

No light penetrates to deep-sea CBEs. Hydrostatic pressure is very high
(30--600 atmospheres). At hydrothermal vents, very hot fluids (up to
400°C) emanate from chimneys charged with metals and chemicals that
provide energy to chemoautotrophic microbes. At cold seeps, the fluids
are cold and reduced chemicals originate both biogenically and
abiotically. At food-falls, reduced chemicals are produced by
microorganisms degrading the organic matter of the fall. The main
drivers of CBEs are the chemosynthetically based primary productivity
and the symbiotic relationships between microorganisms and fauna.

\subsection{Distribution}\label{distribution-46}

Seafloor of all oceans. Vents (map) occur on mid-ocean ridges, back-arc
basins, and active seamounts. Cold seeps occur on active and passive
continental margins. Food-falls occur mostly along cetacean migration
routes (whale falls).

\section{M4.1 Submerged artificial
structures}\label{m4.1-submerged-artificial-structures}

Belongs to biome M4. Anthropogenic marine biome, part of the Marine
realm.

\subsection{Short description}\label{short-description-47}

Submerged structures, including rubble piles, ship wrecks, oil and gas
infrastructure and artificial reefs provide vertically oriented hard
substrates for marine organisms in coastal waters worldwide. Sedentary
filter feeders like sponges and barnacles take advantage of access to
plankton in ocean currents. Their excretions support high abundances of
other invertebrates and fish, while organisms beneath the structures
feed on nutrients falling to the bottom, particularly after storm
events. Some types of structure increase exposure to light, noise and
chemical pollution and may promote the spread of invasive species.

\subsection{Key Features}\label{key-features-47}

Hard surfaces of oil and gas infrastructure, artificial reefs and wrecks
form habitat for sessile filter feeders, invertebrates and some reef
fish..

\subsection{Ecological traits}\label{ecological-traits-47}

These deployments include submerged structures with high vertical relief
including ship wrecks, oil and gas infrastructure, and designed
artificial reefs, as well as some low-relief structures (e.g.~rubble
piles). The latter do not differ greatly from adjacent natural reefs,
but structures with high vertical relief are distinguished by an
abundance of zooplanktivorous fish, as well as reef-associated fishes.
Macroalgae are sparse or absent as the ecosystem is fed by currents and
ocean swell delivering phytoplankton to sessile invertebrates. Complex
surfaces quickly thicken with a biofouling community characterised by an
abundance of filter-feeding invertebrates (e.g.~sponges, barnacles,
bivalves, and ascidians) and their predators (e.g.~crabs and flatworms).
Invertebrate diversity is high, with representatives from every living
Phylum. Structures without complex surfaces, such as the smooth, wide
expanse of a hull, may suffer the sporadic loss of all biofouling
communities after storm events. This feeds the sandy bottom community,
evident as a halo of benthic invertebrates (e.g.~polychaetes and
amphipods), which also benefit from the plume of waste and detritus
drifting from the reef community. Artificial structures also provide a
visual focus attracting the occasional pelagic fish and marine mammals,
which respond similarly to fish-attraction devices and drift objects.

\subsection{Key Ecological Drivers}\label{key-ecological-drivers-47}

The high vertical relief of many artificial structures enables biota to
access plankton continuously transported by currents. They may be
situated on otherwise flat, soft-bottom habitats, isolated to varying
degrees from other hard substrates. High-energy waters experience low
variation in temperature and salinity (except near major river systems).
Currents and eddies cause strong horizontal flow, while ocean swell
creates orbital current velocities at least 10-fold greater. Near large
urban centres, fishing reduces populations of large predatory fish,
resulting in a continuum across species and deployments from purely fish
attraction to fish production (such as via the reef facilitating the
planktivorous food chain). The historical, opportunistic use of
materials (e.g.~rubber tyres, construction materials, or inadequately
decommissioned vessels) have left legacies of pollutants. Compared to
artificial reefs, oil and gas infrastructure is more exposed to
light/noise/chemical pollution associated with operations as well as the
spread of invasive species.

\subsection{Distribution}\label{distribution-47}

Millions of artificial reefs and fish-attraction devices are deployed in
coastal waters worldwide, including \textgreater10,000 oil and gas
structures, mostly in tropical and temperate waters. More than 500 oil
and gas platforms were decommissioned and left as artificial reefs in US
waters since 1940. Many others are candidates for reefing after
decommissioning in coming decades (\textgreater{} 600 in the
Asia-Pacific alone). Worldwide since 1984, over 130 ships and planes
have purposely been sunk for recreational SCUBA-diving. Map is
incomplete but shows areas with many documented wrecks and marine
infrastructure.

\section{M4.2 Marine aquafarms}\label{m4.2-marine-aquafarms}

Belongs to biome M4. Anthropogenic marine biome, part of the Marine
realm.

\subsection{Short description}\label{short-description-48}

High-productivity marine aquafarms are enclosed areas for the breeding,
rearing, and harvesting of marine plants and animals, including finfish
like salmon, molluscs, crustaceans, and algae. These low-diversity
communities are dominated by the harvested species, maintained at a high
densities. Non-harvested species may be controlled as pests, using
antibiotics, herbicides, or culling. Aquafarms in coastal or open ocean
waters are exposed to associated physical processes, but on land they
are in environmentally controlled tanks and ponds.

\subsection{Key Features}\label{key-features-48}

High density, productive, enclosed systems with variable permeability,
for breeding and harvesting marine species. Allochthonous nutrients from
human sources is common..

\subsection{Ecological traits}\label{ecological-traits-48}

Marine aquafarms (i.e.~mariculture) are localised, high-productivity
systems within and around enclosures constructed for the breeding,
rearing, and harvesting of marine plants and animals, including finfish,
molluscs, crustaceans, algae, and other marine plants. Allochthonous
energy and nutrient inputs are delivered by humans and by diffusion from
surrounding marine waters. Autochthonous inputs are small and produced
by pelagic algae or biofilms on the infrastructure, unless the target
species are aquatic macrophytes. More commonly, target species are
consumers that belong to middle or upper trophic levels. Diversity is
low across taxa, and the trophic web is dominated by a super-abundance
of target species. Where multiple target species are cultivated, they
are selected to ensure neutral or mutualistic interactions with one
another (e.g.~detritivores that consume the waste of a higher-level
consumer). Target biota are harvested periodically to produce food, fish
meal, nutrient agar, horticultural products, jewellery, and cosmetics.
Their high population densities are maintained by continual inputs of
food and regular re-stocking to compensate harvest. Target species may
be genetically modified and are often bred in intensive hatcheries and
then released into the enclosures. Food and nutrient inputs may promote
the abundance of non-target species including opportunistic microalgae,
zooplankton, and pathogens and predators of the target species. These
pest species or their impacts may be controlled by antibiotics or
herbicides or by culling (e.g.~pinnipeds around fish farms). The
enclosures constitute barriers to the movement of larger organisms, but
some cultivated stock may escape, while wild individuals from the
surrounding waters may invade the enclosure. Enclosures are generally
permeable to small organisms, propagules and waste products of larger
organisms, nutrients, and pathogens, enabling the ecosystem to extend
beyond the confines of the infrastructure.

\subsection{Key Ecological Drivers}\label{key-ecological-drivers-48}

Most marine farms are located in sheltered coastal waters but some are
located in the open ocean or on land in tanks or ponds filled with
seawater. Those in marine waters experience currents, tides, and
flow-through of marine energy, matter, and biota characteristic of the
surrounding environment. Those on land are more insular, with
intensively controlled light and temperature, recirculation systems that
filter and recycle water and waste, and intensive anthropogenic inputs
of food and nutrients, anti-fouling chemicals, antibiotics, and
herbicides. Marine enclosures have netting and frames that provide
substrates for biofilms and a limited array of benthic organisms, but
usually exclude the benthos. Land-based systems have smooth walls and
floors that provide limited habitat heterogeneity for benthic biota.

\subsection{Distribution}\label{distribution-48}

Rapidly expanding around coastal Asia, Europe, North America and
Mesoamerica, and southern temperate regions. Open-ocean facilities near
Hawaii and Puerto Rico.

\section{MFT1.1 Coastal river deltas}\label{mft1.1-coastal-river-deltas}

Belongs to biome MFT1. Brackish tidal biome, part of the Marine,
Freshwater, Terrestrial realm.

\subsection{Short description}\label{short-description-49}

At the convergence of terrestrial, freshwater and marine realms, coastal
river deltas are shaped by river inflows that deposit sediment and ocean
tides and currents that disperse it. Gradients of salinity and
submergence and dynamic substrates create shifting mosaics of channels,
islands, floodplains, intertidal and subtidal mud plains and sand beds
that may be regarded as embedded patches of other functional groups. The
dynamic mosaics support complex foodwebs. Planktonic algae, aquatic
plants and river inflows contribute detritus for in-sediment fauna. Fish
and zooplankton are diverse and abundant in the water column, providing
food for wading and fishing birds and marine and terrestrial predators.

\subsection{Key Features}\label{key-features-49}

Depositional, mosaic systems with strong gradients between terrestrial,
freshwater and marine elements. Productive with diverse plankton, fish,
birds and mammals..

\subsection{Ecological traits}\label{ecological-traits-49}

Coastal river deltas are prograding depositional systems, shaped by
freshwater flows and influenced by wave and tidal flow regimes and
substrate composition. The biota of these ecosystems reflects strong
relationships with terrestrial, freshwater, and marine realms at
different spatial scales. Consequently, they typically occur as
multi-scale mosaics comprised of unique elements juxtaposed with other
functional groups that extend far beyond the deltaic influence, such as
floodplain marshes (FT1.2), mangroves (MFT1.2), sandy shorelines
(TM1.3), and subtidal muddy plains (M1.8). Gradients of water
submergence and salinity structure these mosaics. Allochthonous
subsidies from riverine discharge and marine currents supplement
autochthonous sources of energy and carbon and contribute to high
productivity. Complex, multi-faceted trophic relationships reflect the
convergence and integration of three contrasting realms and the
resulting niche diversity. Autotrophs include planktonic algae and
emergent and submerged aquatic plants, which contribute to trophic
networks mostly through organic detritus (rather than herbivory). Soft
sediments and flowing water are critical to in-sediment fauna dominated
by polychaetes and molluscs. Freshwater, estuarine, and marine fish and
zooplankton are diverse and abundant in the water column. These provide
food for diverse communities of wading and fishing birds, itinerant
marine predators, and terrestrial scavengers and predators (e.g.~mammals
and reptiles). Virtually all biota have life-history and/or movement
traits enabling them to exploit highly dynamic ecosystem structures and
disturbance regimes. High rates of turnover in habitat and biota are
expressed spatially by large fluctuations in the mosaic of patch types
that make up deltaic ecosystems.

\subsection{Key Ecological Drivers}\label{key-ecological-drivers-49}

River inflows structure the dynamic mosaics of coastal river deltas.
Inflows depend on catchment geomorphology and climate and influence
water levels, nutrient input, turbidity (hence light penetration), tidal
amplitude, salinity gradients, temperature, dissolved oxygen, and
organic carbon. Rates of delta aggradation depend on interactions among
riverine sedimentation and ocean currents, tides, and wave action, which
disperse sediment loads. Coastal geomorphology influences depth
gradients. These processes result in complex, spatio-temporally variable
mosaics of distributary channels, islands, floodplains, mangroves,
subtidal mud plains, and sand beds. Regimes of floods and storm surges
driven by weather in the river catchment and ocean, respectively, have a
profound impact on patch dynamics.

\subsection{Distribution}\label{distribution-49}

Continental margins where rivers connect the coast to high-rainfall
catchments, usually with high mountains in their headwaters.

\section{MFT1.2 Intertidal forests and
shrublands}\label{mft1.2-intertidal-forests-and-shrublands}

Belongs to biome MFT1. Brackish tidal biome, part of the Marine,
Freshwater, Terrestrial realm.

\subsection{Short description}\label{short-description-50}

Mangroves create structurally complex and productive ecosystems in the
intertidal zone of depositional coasts, around tropical and warm
temperate regions. The biota includes aquatic and terrestrial species,
and intertidal specialists. Large volumes of mangrove leaves and twigs
are decomposed by fungi and bacteria, mobilising carbon and nutrients
for invertebrates such as crabs, worms and snails. Shellfish and
juvenile fish are protected from desiccation and predators amongst
mangrove roots. Mangrove canopies support many terrestrial species,
particularly birds. These forests are important carbon sinks, retaining
organic matter in sediments and living biomass.

\subsection{Key Features}\label{key-features-50}

Intertidal mangrove-dominated systems, producing high amounts of organic
matter that is both buried in situ and exported; sediments dominated by
detritivores and leaf shredders, with birds , mammals, reptiles and
terrestrial invertebrates occupying the canopy.

\subsection{Ecological traits}\label{ecological-traits-50}

Mangroves are structural engineers and possess traits including
pneumatophores, salt excretion glands, vivipary, and propagule buoyancy
that promote survival and recruitment in poorly aerated, saline, mobile,
and tidally inundated substrates. They are highly efficient in nitrogen
use efficiency and nutrient resorption. These systems are among the most
productive coastal environments. They produce large amounts of detritus
(e.g.~leaves, twigs, and bark), which is either buried in waterlogged
sediments, consumed by crabs, or more commonly decomposed by fungi and
bacteria, mobilising carbon and nutrients to higher trophic levels.
These ecosystems are also major blue carbon sinks, incorporating organic
matter into sediments and living biomass. Although highly productive,
these ecosystems are less speciose than other coastal biogenic systems.
Crabs are among the most abundant and important invertebrates. Their
burrows oxygenate sediments, enhance groundwater penetration, and
provide habitat for other invertebrates such as molluscs and worms.
Specialised roots (pneumatophores) provide a complex habitat structure
that protects juvenile fish from predators and serves as hard substrate
for the attachment of algae as well as sessile and mobile invertebrates
(e.g.~oysters, mussels, sponges, and gastropods). Mangrove canopies
support invertebrate herbivores and other terrestrial biota including
invertebrates, reptiles, small mammals, and extensive bird communities.
These are highly dynamic systems, with species distributions adjusting
to local changes in sediment distribution, tidal regimes, and local
inundation and salinity gradients.

\subsection{Key Ecological Drivers}\label{key-ecological-drivers-50}

Mangroves are physiologically intolerant of low temperatures, which
excludes them from regions where mean air temperature during the coldest
months is below 20°C, where the seasonal temperature range exceeds 10°C,
or where ground frost occurs. Many mangrove soils are low in nutrients,
especially nitrogen and phosphorus. Limited availability of nitrogen and
phosphorus Regional distributions are influenced by interactions among
landscape position, rainfall, hydrology, sea level, sediment dynamics,
subsidence, storm-driven processes, and disturbance by pests and
predators. Rainfall and sediment supply from rivers and currents promote
mangrove establishment and persistence, while waves and large tidal
currents destabilise and erode mangrove substrates, mediating
local-scale dynamics in ecosystem distributions. High rainfall reduces
salinity stress and increases nutrient loading from adjacent catchments,
while tidal flushing also regulates salinity.

\subsection{Distribution}\label{distribution-50}

Widely distributed along tropical and warm temperate coastlines of the
world. Large-scale currents may prevent buoyant seeds from reaching some
areas.

\section{MFT1.3 Coastal saltmarshes and
reedbeds}\label{mft1.3-coastal-saltmarshes-and-reedbeds}

Belongs to biome MFT1. Brackish tidal biome, part of the Marine,
Freshwater, Terrestrial realm.

\subsection{Short description}\label{short-description-51}

Coastal salt marshes and reedbeds are mosaics of salt-tolerant grasses
and low, typically succulent shrubs. structured by strong gradients of
salinity and tidal influence. Salts may approach hypersaline levels near
the limit of high spring tides, especially in the tropics. As well as
larger plants, algal mats and phytoplankton contribute to productivity,
while freshwater run-off and tides bring organic material and nutrients.
Bacteria and fungi decompose biomass in oxygen-poor subsoils, and
support a range of crustaceans, worms, snails and small fish. Shorebirds
breed and forage in saltmarshes, with migratory species dispersing
plants and animals.

\subsection{Key Features}\label{key-features-51}

Variable salinity tidal system dominated by salt-tolerant plants, with
invertebrates, small/juvenile fish and birds..

\subsection{Ecological traits}\label{ecological-traits-51}

Coastal saltmarshes are vegetated by salt-tolerant forbs, grasses, and
shrubs, with fine-scale mosaics related to strong local hydrological and
salinity gradients, as well as competition and facilitation. Plant
traits such as succulence, salt excretion, osmotic regulation, reduced
transpiration, C4 photosynthesis (among grasses), modular growth forms,
and aerenchymatous tissues confer varied degrees of tolerance to
salinity, desiccation, and substrate anoxia. Adjacent marine and
terrestrial ecosystems influence the complexity and function of the
trophic network, while freshwater inputs mediate resource availability
and physiological stress. Angiosperms are structurally dominant
autotrophs, but algal mats and phytoplankton imported by tidal waters
contribute to primary production. Cyanobacteria and rhizobial bacteria
are important N-fixers. Tides and run-off bring subsidies of organic
detritus and nutrients (including nitrates) from marine and terrestrial
sources, respectively. Nitrogen is imported into saltmarshes mainly as
inorganic forms and exported largely as organic forms, providing
important subsidies to the trophic networks of adjacent estuarine fish
nurseries (FM1.2). Fungi and bacteria decompose dissolved and
particulate organic matter, while sulphate-reducing bacteria are
important in the decay of substantial biomass in the anaerobic subsoil.
Protozoans consume microbial decomposers, while in situ detritivores and
herbivores include a range of crustaceans, polychaetes, and molluscs.
Many of these ingest a mixture of organic material and sediment,
structuring, aerating, and increasing the micro-scale heterogeneity of
the substrate with burrows and faecal pellets. Fish move through
saltmarsh vegetation at high tide, feeding mainly on algae. They include
small-bodied residents and juveniles of larger species that then move
offshore. Itinerant terrestrial mammals consume higher plants,
regulating competition and vegetation structure. Colonial and solitary
shorebirds breed and/or forage in saltmarsh. Migratory species that play
important roles in the dispersal of plants, invertebrates, and microbes,
while abundant foragers may force top-down transformational change.

\subsection{Key Ecological Drivers}\label{key-ecological-drivers-51}

High and variable salt concentration is driven by alternating episodes
of soil desiccation and flushing, associated with cycles of tidal
inundation and drying combined with freshwater seepage, rainfall, and
run-off in the upper intertidal zone. These interacting processes
produce dynamic fine-scale hydrological and salinity gradients, which
may drive transformation to intertidal forests (MFT1.2). Marshes are
associated with low-energy depositional coasts but may occur on sea
cliffs and headlands where wind deposits salt from wave splash
(i.e.~salt spray) and aerosol inputs. Salt approaches hypersaline levels
where flushing events are infrequent. Other nutrients make up a low
proportion of the total ionic content. Subsoils are generally anaerobic,
but this varies depending on seepage water and the frequency of tidal
inundation. Tidal cycles also influence temperature extremes,
irregularities in photoperiod, physical disturbance, and deposition of
sediment.

\subsection{Distribution}\label{distribution-51}

Widely distributed, mostly on low-energy coasts from arctic to tropical
and subantarctic latitudes.

\section{MT1.1 Rocky Shorelines}\label{mt1.1-rocky-shorelines}

Belongs to biome MT1. Shorelines biome, part of the Marine, Terrestrial
realm.

\subsection{Short description}\label{short-description-52}

Waves, tides and a gradient of exposure drive the structure and function
of these productive intertidal ecosystems found mostly on high energy
coasts. The biota includes filter feeders like barnacles, mussels and
sea squirts which compete for limited space. Grazers like limpets and
urchins consume small sessile algae, while predators such as crabs, fish
and birds consume a wide range of prey. Organisms use microhabitats
during low tide (e.g.~rockpools, crevices) or have adaptations like
shells to survive exposure to high temperatures, salinity and
desiccation.

\subsection{Key Features}\label{key-features-52}

Hard intertidal substrate, dominated by sessile and mobile
invertebrates, and macroalgae.

\subsection{Ecological traits}\label{ecological-traits-52}

These intertidal benthic systems, composed of sessile and mobile
species, are highly structured by fine-scale resource and stress
gradients, as well as trade-offs among competitive, facilitation, and
predatory interactions. Sessile algae and invertebrates form complex
three-dimensional habitats that provide microhabitat refugia from
desiccation and temperature stress for associated organisms; these
weaken competitive interactions. The biota exhibit behavioural and
morphological adaptions to minimise exposure to stressors, such as
seeking shelter in protective microhabitats at low tide, possessing
exoskeletons (e.g.~shells), or producing mucous to reduce desiccation.
Morphologies, such as small body sizes and small cross-sectional areas
to minimise drag, reflect adaptation to a wave-swept environment. Key
trophic groups include filter-feeders (which feed on phytoplankton and
dissolved organic matter at high tide), grazers (which scrape
microphytobenthos and macroalgal spores from rock or consume macroalgal
thalli), and resident (e.g.~starfish, whelks, and crabs) and transient
(e.g.~birds and fish) marine and terrestrial predators. Rocky shores
display high endemism relative to other coastal systems and frequently
display high productivity due to the large amounts of light they
receive, although this can vary according to nutrient availability from
upwelling.

\subsection{Key Ecological Drivers}\label{key-ecological-drivers-52}

Tides and waves are the key ecological drivers, producing resource
availability and physical disturbance gradients vertically and
horizontally, respectively. Across the vertical gradient of increasing
aerial exposure, desiccation and temperature stress increases, time
available for filter-feeding decreases, and interactions with marine and
terrestrial predators vary. The horizontal gradient of diminishing wave
exposure from headlands to bays or inlets influences community
composition and morphology. Many organisms rely on microhabitats formed
from natural rock features (e.g.~crevices, depressions, and rock pools)
or habitat-forming species (e.g.~canopy-forming algae, mussels, oysters,
and barnacles) to persist in an environment that would otherwise exceed
their environmental tolerances. Rocky shores are open systems, so
community structure can be influenced by larval supply, coastal
upwelling, and competition. Competition for space may limit the lower
vertical distributions of some sessile species. The limited space
available for the growth of marine primary producers can result in
competition for food among grazers. Disturbances (i.e.~storms, ice scour
on subpolar shores) that free-up space can have a strong influence on
community structure and diversity.

\subsection{Distribution}\label{distribution-52}

Found globally at the margins of oceans, where waves are eroding rocks.
They are the most common ecosystems on open, high-energy coasts and also
occur on many sheltered and enclosed coastlines, such as sea lochs,
fjords, and rias.

\section{MT1.2 Muddy Shorelines}\label{mt1.2-muddy-shorelines}

Belongs to biome MT1. Shorelines biome, part of the Marine, Terrestrial
realm.

\subsection{Short description}\label{short-description-53}

Mudflats occur on low-energy coastlines. Mud and silt, often from nearby
rivers, protect the burrowing organisms living in these ecosystems from
common shoreline stressors (e.g.~high temperatures and desiccation) and
predatory shorebirds, crabs and fish. These shorelines are critical
stopovers and foraging grounds for migratory shorebirds. Primary
productivity is mostly from diatoms (single-celled algae) that rely on
tides. Oxygen can be low where sediments are very fine or burrowing or
other disturbance is limited.

\subsection{Key Features}\label{key-features-53}

Intertidal soft-sediment, of fine particle-size, dependent on
allochtonous production and dominated by deposit feeding and
detritivorous invertebrates that provide a prey resource for shore birds
and fishes.

\subsection{Ecological traits}\label{ecological-traits-53}

Highly productive intertidal environments are defined by their fine
particle size (dominated by silts) and are fuelled largely by
allochthonous production. Benthic diatoms are the key primary producer,
although ephemeral intertidal seagrass may occur. Otherwise, macrophytes
are generally absent unlike other ecosystems on intertidal mudflats
(MFT1.2, MFT1.3). Fauna are dominated by deposit-feeding taxa (consuming
organic matter that accumulates in the fine-grained sediments) and
detritivores feeding on wrack (i.e.~drift algae deposited at the
high-water mark) and other sources of macro-detritus. Bioturbating and
tube-dwelling taxa are key ecosystem engineers, the former oxygenating
and mixing the sediments and the latter providing structure to an
otherwise sedimentary habitat. Infauna residing within sediments are
protected from high temperatures and desiccation by the surrounding
matrix and do not display the same marked patterns of zonation as rocky
intertidal communities. Many infaunal taxa are soft-bodied.
Nevertheless, competition for food resources carried by incoming tides
can lead to intertidal gradients in fauna. Predators include the
substantial shorebird populations that forage on infauna at low tide,
including migratory species that depend on these systems as stopover
sites. Fish, rays, crabs, and resident whelks forage around lugworm
bioturbation. Transitions to mangrove (MFT1.2), saltmarsh or reedbed
(MFT1.3) ecosystems may occur in response to isostatic or sea level
changes, freshwater inputs or changes in currents that promote
macrophyte colonisation.

\subsection{Key Ecological Drivers}\label{key-ecological-drivers-53}

These are depositional environments influenced by sediment supply and
the balance of erosion and sedimentation. They occur on lower wave
energy coastlines with lower slopes and larger intertidal ranges than
sandy shorelines, resulting in lower levels of sediment transport and
oxygenation by physical processes. In the absence of burrowing taxa,
sediments may display low rates of turnover, which may result in an
anoxic zone close to the sediment surface. Small particle sizes limit
interstitial spaces, further reducing aeration. The depth of the anoxic
zone can be a key structuring factor. In contrast to sandy shorelines,
they are organically rich and consequently higher in nutrients.
Generally, muddy shorelines are formed from sediments supplied by nearby
rivers, often remobilised from the seafloor throughout the tidal cycle.

\subsection{Distribution}\label{distribution-53}

Muddy shorelines occur along low-energy coastlines, in estuaries and
embayments where the velocity of water is so low that the finest
particles can settle to the bottom.

\section{MT1.3 Sandy Shorelines}\label{mt1.3-sandy-shorelines}

Belongs to biome MT1. Shorelines biome, part of the Marine, Terrestrial
realm.

\subsection{Short description}\label{short-description-54}

Beaches, sand bars and spits are exposed to waves and tides on
moderate-high energy coasts, and rely on drift seaweed and surf-zone
phytoplankton for nutrients. Polychaete worms, bivalve shellfish and a
range of smaller invertebrates burrow in the shifting sediments, while
larger vertebrate animals like seabirds, egg-laying turtles and
scavenging foxes can also be found at various times. Storm tides and
waves periodically restructure the sediments and profoundly influence
the traits of the organisms living in these highly dynamic systems.

\subsection{Key Features}\label{key-features-54}

Intertidal soft-sediment, of large particle-size, lacking conspicuous
macrophytes, and dominated by suspension-feeding invertebrates that
provide a prey resource for shore birds and fishes.

\subsection{Ecological traits}\label{ecological-traits-54}

Sandy shorelines include beaches, sand bars, and spits. These intertidal
systems typically lack macrophytes, with their low productivity largely
underpinned by detrital subsidies dominated by wrack (i.e.~drift seaweed
accumulating at the high-water mark) and phytoplankton, particularly in
the surf zone of dissipative beaches. Salt- and drought-tolerant primary
producers dominate adjacent dune systems (TM1.4). Meio-faunal biomass in
many instances exceeds macrofaunal biomass. In the intertidal zone,
suspension-feeding is a more common foraging strategy among
invertebrates than deposit-feeding, although detritivores may dominate
higher on the shore where wrack accumulates. Invertebrate fauna are
predominantly interstitial, with bacteria, protozoans, and small
metazoans contributing to the trophic network. Sediments are constantly
shifting and thus invertebrate fauna are dominated by mobile taxa that
display an ability to burrow and/or swash-ride up and down the beach
face with the tides. The transitional character of these systems
supports marine and terrestrial invertebrates and itinerant vertebrates
from marine waters (e.g.~egg-laying turtles) and from terrestrial or
transitional habitats (e.g.~shorebirds foraging on invertebrates or
foxes foraging on carrion).

\subsection{Key Ecological Drivers}\label{key-ecological-drivers-54}

Physical factors are generally more important ecological drivers than
biological factors. Particle size and wave and tidal regimes determine
beach morphology, all of which influence the spatial and temporal
availability of resources and niche diversity. Particle size is
influenced by sediment sources as well as physical conditions and
affects interstitial habitat structure. Wave action maintains substrate
instability and an abundant supply of oxygen through turbulence. Tides
and currents influence the dispersal of biota and regulate daily cycles
of desiccation and hydration as well as salinity. Beach morphology
ranges from narrow and steep (i.e.~reflective) to wide and flat
(i.e.~dissipative) as sand becomes finer and waves and tides larger.
Reflective beaches are accretional and more prevalent in the tropics;
dissipative beaches are erosional and more common in temperate regions.
Sands filter large volumes of seawater, with the volume greater on
reflective than dissipative beaches. Beaches are linked to nearshore
surf zones and coastal dunes through the storage, transport, and
exchange of sand. Sand transport is the highest in exposed surf zones
and sand storage the greatest in well-developed dunes.

\subsection{Distribution}\label{distribution-54}

Sandy shores are most extensive at temperate latitudes, accounting for
31\% of the ice-free global coastline, including 66\% of the African
coast and 23\% of the European coast.

\section{MT1.4 Boulder and cobble
shores}\label{mt1.4-boulder-and-cobble-shores}

Belongs to biome MT1. Shorelines biome, part of the Marine, Terrestrial
realm.

\subsection{Short description}\label{short-description-55}

Cobbled and boulder shores are exposed to wave action and tides, and are
periodically restructured by high-energy storm events. Drift seaweed and
local algae support communities of organisms adapted to regular
disturbance and grinding of rocks and cobbles, as well as the high
temperatures and desiccation common to all shorelines. For example, many
live largely below the surface layers. Stability of the substrate, and
hence the biota, depends on the size of the cobbles and boulders. Some
encrusting species or algae like cordgrass can stabilise these shores
and allow a wider range of plants and animals to establish.

\subsection{Key Features}\label{key-features-55}

Unstable intertidal hard substrate, that supports encrusting and fouling
species at low elevations and in some instances vegetation, though
largely dependent on allocthonous production.

\subsection{Ecological traits}\label{ecological-traits-55}

These low-productivity, net heterotrophic systems are founded on
unstable rocky substrates and share some ecological features with sandy
beaches (MT1.3) and rocky shores (MT1.1). Traits of the biota reflect
responses to regular substrate disturbance by waves and exposure of
particles to desiccation and high temperatures. For example, in the high
intertidal zone of boulder shores (where temperature and desiccation
stress is most pronounced), fauna may be predominantly nocturnal. On
cobble beaches, fauna are more abundant on the sub-surface because waves
cause cobbles to grind against each other, damaging or killing attached
fauna. Conversely, sandy beaches are where most fauna occupy surface
sediments. Intermediate frequencies of disturbance lead to the greatest
biodiversity. Only species with low tenacity (e.g.~top shells) are found
in surface sediments because they can detach and temporarily inhabit
deeper interstices during disturbance events. High-tenacity species
(e.g.~limpets) or sessile species (e.g.~macroalgae and barnacles) are
more readily damaged, hence rare on cobble shores. Large boulders,
however, are only disturbed during large storms and have more stable
temperatures, so more fauna can persist on their surface. Encrusting
organisms may cement boulders on the low shore, further stabilising them
in turbulent water. Allochthonous wrack is the major source of organic
matter on cobble beaches, but in situ autotrophs include superficial
algae and vascular vegetation dominated by halophytic forbs. On some
cobble beaches of New England, USA, extensive intertidal beds of the
cordgrass Spartina alterniflora stabilise cobbles and provide shade,
facilitating establishment of mussels, barnacles, gastropods, amphipods,
crabs, and algae. In stabilising cobbles and buffering wave energy,
cordgrass may also facilitate plants higher on the intertidal shore.

\subsection{Key Ecological Drivers}\label{key-ecological-drivers-55}

Particle size (e.g.~cobbles vs.~boulders) and wave activity determine
substrate mobility, hence the frequency of physical disturbance to
biota. Ecosystem engineers modify these relationships by stabilising the
substrate. Cobble beaches are typically steep because waves easily flow
through large interstices between coarse beach particles, reducing the
effects of backwash erosion. Hence swash and breaking zones tend to be
similar widths. The permeability of cobble beaches leads to desiccation
and heat stress at low tide along the beach surface gradient.
Desiccation stress is extreme on boulder shores, playing a similar role
in structuring communities as on rocky shores. The extent of the fine
sediment matrix present amongst cobbles, water supply (i.e.~rainfall),
and the frequency of physical disturbance all influence beach
vegetation. Alongshore grading of sediment by size could occur on long,
drift dominated shorelines, which may influence sediment calibre on the
beach.

\subsection{Distribution}\label{distribution-55}

Cobble beaches occur where rivers or glaciers delivered cobbles to the
coast or where they were eroded from nearby coastal cliffs. They are
most common in Europe and also occur in Bahrain, North America, and New
Zealand's South Island.

\section{MT2.1 Coastal shrublands and
grasslands}\label{mt2.1-coastal-shrublands-and-grasslands}

Belongs to biome MT2. Supralittoral coastal biome, part of the Marine,
Terrestrial realm.

\subsection{Short description}\label{short-description-56}

A low diversity of specialised plants and animals live in grasslands,
shrublands, and low forests on coastlines above the high tide mark where
they are exposed to harsh conditions of salt influx, desiccating winds
and sunshine, and disturbances associated with storms or unstable
substrates (e.g.~cliffs and dunes). Plants, for example, exhibit traits
such as succulence and rhizomes to promote persistence in these
conditions, and many organisms are widely dispersed by winds or ocean
currents. Consumers like seabirds or seals move between terrestrial and
marine environments.

\subsection{Key Features}\label{key-features-56}

Coastal scrub limited by salinity, water deficit and disturbances
(e.g.~cliff collapse). Strong gradients from sea to land and highly
mobile fauna..

\subsection{Ecological traits}\label{ecological-traits-56}

Relatively low productivity grasslands, shrublands, and low forests on
exposed coastlines are limited by salt influx, water deficit, and
recurring disturbances. Diversity is low across taxa and trophic
networks are simple, but virtually all plants and animals have strong
dispersal traits and most consumers move between adjacent terrestrial
and marine ecosystems. Vegetation and substrates are characterised by
strong gradients from sea to land, particularly related to aerosol salt
inputs, substrate instability and disturbance associated with sea storms
and wave action. Plant traits conferring salt tolerance (e.g.~succulent
and sub-succulent leaves and salt-excretion organs) are commonly
represented. Woody plants with ramulose and/or decumbent growth forms
and small (microphyll-nanophyll) leaves reflect mechanisms of
persistence under exposure to strong salt-laden winds, while modular and
rhizomatous growth forms of woody and non-woody plants promote
persistence, regeneration, and expansion under regimes of substrate
instability and recurring disturbance. These strong environmental
filters promote local adaptation, with specialised genotypes and
phenotypes of more widespread taxa commonly represented on the
strandline. Fauna are highly mobile, although some taxa such as
ground-nesting seabirds may be sedentary for some parts of their
lifecycles. Ecosystem dynamics are characterised by disturbance-driven
cycles of disruption and renewal, with early phases dominated by
colonists and in situ regenerators that often persist during the short
intervals between successive disturbances.

\subsection{Key Ecological Drivers}\label{key-ecological-drivers-56}

Desiccating winds promote an overall water deficit and appreciable
exposure to salinity due to aerosol influx and salt spray. Warm to mild
temperatures across the tropics to temperate zones and cold temperatures
in the cool temperate to boreal zones are moderated by direct maritime
influence. Above the regular intertidal zone, these systems are exposed
to periodic disturbance from exceptional tides, coastal storm events,
wind shear, bioturbation, and aeolian substrate mobility. Consolidated
substrates (headlands, cliffs) may differ from unconsolidated dunes in
their influence on function and biota. Geomorphological depositional and
erosional processes influence substrate stability and local vegetation
succession.

\subsection{Distribution}\label{distribution-56}

Coastal dunes and cliffs throughout tropical, temperate, and boreal
latitudes.

\section{MT2.2 Large seabird and pinniped
colonies}\label{mt2.2-large-seabird-and-pinniped-colonies}

Belongs to biome MT2. Supralittoral coastal biome, part of the Marine,
Terrestrial realm.

\subsection{Short description}\label{short-description-57}

Large concentrations of roosting or nesting seabirds and semiaquatic
mammals such as seals and walrus are found on relatively isolated
islands and shores. These animals consume large amounts of marine
resources but spend many weeks and months on land, accumulating high
concentrations of nitrogen, phosphorous and other nutrients. The
abundance and relatively large body size of individuals disrupt the
growth of vegetation. The combination of these factors means that
microbial activity is high and soil invertebrates are abundant, but
plant diversity is usually low, and land based grazers and predators are
usually absent.

\subsection{Key Features}\label{key-features-57}

Localised areas of bare or vegetated ground with diverse microbial
communities at the ocean interface receiving massive nutrient subsidies
and disturbance from large concentrations of roosting or nesting
seabirds and pinnipeds that function as mobile links between land and
sea.

\subsection{Ecological traits}\label{ecological-traits-57}

Large seabird and pinniped colonies are localised eutrophic terrestrial
ecosystems near the ocean interface that receive massive nutrient
subsidies from large concentrations of roosting or nesting seabirds and
pinnipeds that function as mobile links between land and sea. The
marine-derived subsidies and potentially massive physical disturbance to
vegetation and soils distinguish these colonies from otherwise similar
ecosystems in MT2.1. Subsidies are greatest where seabird body size is
typically larger (e.g.~penguins) and breeding seasons are longer,
particularly the sub-Antarctic and Antarctic. The waters around these
ecosystems may be locally depleted in seabird prey due to prolonged
predation. Colonies occupy diverse habitats, from sandy shores to rocky
islands and montane forests, with vegetation composition and structure
limited by physical disturbance, nutrient input, salt influx and
gradient (e.g., sea spray), water deficit, surface and subsurface
bioturbation-driven changes in soil condition and pH, avian seed
dispersal, unstable substrates, and high exposure, often exhibiting salt
tolerance and clonal reproduction. Plant assemblages exist across a
gradient, influenced by seabird/pinniped disturbance, nutrient input and
climate, whereby high-density colonies can completely suppress plant
growth, but where disturbance and nutrient load is lower, vegetation can
establish, typically in low richness but high abundance. Trophic
networks are characterized high microbial activity and abundant
invertebrates in soils which can lead to localised biodiversity
hotspots, in contrast to the low richness of plant communities under
high nutrient loading. There are typically low densities or a total
absence of terrestrial mammalian predators and grazers (limited by
dispersal barriers). Vibrant and specialised lichens can be abundant.
Plant dispersal linked to bird migration, and nutrient transport between
marine foraging areas and terrestrial breeding areas, may occur over
long distances.

\subsection{Key Ecological Drivers}\label{key-ecological-drivers-57}

Marine subsidies of nutrients, excreted by marine-foraging seabirds and
pinnipeds, drives eutrophication, resulting in the highest terrestrial
concentrations of nitrogen, phosphorus, and other nutrients on Earth's
surface. Nutrients may be derived from sources proximal to, or remote
from the colony, and may be continual or pulsed, depending on the colony
location, size, constituent species, and seasonal variation in
attendance. Substrates vary from sand to soil to rock to ice, and
desiccating winds add aerosol salts and limit water availability in
coastal colonies. Temperatures vary from warm to mild in
tropical/temperate/boreal latitudes to freezing in polar regions.
Bioturbation, coastal storms, and unstable substrates influence biotic
interactions and colony abundance and distribution.

\subsection{Distribution}\label{distribution-57}

Scattered globally on islands and coastlines, but most common in polar
and subpolar regions

\section{MT3.1 Artificial shorelines}\label{mt3.1-artificial-shorelines}

Belongs to biome MT3. Anthropogenic shorelines biome, part of the
Marine, Terrestrial realm.

\subsection{Short description}\label{short-description-58}

Constructed sea walls, breakwaters, piers, docks, tidal canals, islands
and other coastal infrastructure create habitat for marine plants and
animals around ports, harbours, and other intensively settled coastal
areas. The waters may be polluted by urban runoff or industrial
outflows. Opportunities for introduction of invasive species
e.g.~through shipping mean that non-native species are common. Foodwebs
are simple and dominated by species able to opportunistically colonise
these structures through dispersal of eggs, larvae and spores. Commonly,
these include algae and bacteria, and filter-feeders like sea-squirts
and barnacles.

\subsection{Key Features}\label{key-features-58}

Coastal infrastructure, such as seawalls, breakwaters, pilings and
piers, extending from the intertidal to subtidal, supporting
cosmopolitan sessile and mobile invertebrates and macroalgae on their
hard surfaces, and in some instances serving as artificial reefs for
fish.

\subsection{Ecological traits}\label{ecological-traits-58}

Constructed sea walls, breakwaters, piers, docks, tidal canals, islands
and other coastal infrastructure create substrates inhabited by
inter-tidal and subtidal, benthic and demersal marine biota around
ports, harbours, and other intensively settled coastal areas.
Structurally simple, spatially homogeneous substrates support a
cosmopolitan biota, with no endemism and generally lower taxonomic and
functional diversity than rocky shores (MT1.1). Trophic networks are
simple and dominated by filter-feeders (e.g.~sea squirts and barnacles)
and biofilms of benthic algae and bacteria. Low habitat heterogeneity
and the small surface area for attachment that the often vertical
substrate provides, regulate community structure by promoting
competition and limiting specialised niches (e.g.~crevices or pools) and
restricting refuges from predators. Small planktivorous fish may
dominate temperate harbours and ports. These can provide a trophic link,
but overharvest of predatory fish and sharks may destabilise food webs
and cause trophic cascades. Much of the biota possess traits that
promote opportunistic colonisation, including highly dispersive life
stages (e.g.~larvae, eggs, and spores), high fecundity, generalist
settlement niches and diet, wide ranges of salinity tolerance, and rapid
population turnover. These structures typically contain a higher
proportion of non-native species than the natural substrates they
replace.

\subsection{Key Ecological Drivers}\label{key-ecological-drivers-58}

The substrate material influences the texture, chemistry, and thermal
properties of the surface. Artificial structures of wood, concrete,
rock, or steel have flat, uniform, and vertical surfaces that limit
niche diversity and exacerbate inter-tidal gradients in desiccation and
temperature. Floating structures have downward-facing surfaces, rare in
nature. Some structures are ecologically engineered (designed for
nature) to provide more complex surfaces and ponds to enhance
biodiversity and ecosystem function. Structures may be located in high
(i.e.~breakwaters) or low (i.e.~harbours) energy waters. Tides and waves
are key drivers of onshore resource and kinetic energy gradients.
Brackish water plumes from polluted storm water and sewage overflows add
allochthonous nutrients, organic carbon, and open ecological space
exploited by invasive species introduced by shipping and ballast water.
The structures are often located close to vectors for invasive species
(e.g.~transport hubs). Boat traffic and storm water outflows cause
erosion and bank instability and maintain high turbidity in the water
column. This limits photosynthesis by primary producers, but nutrient
run-off may increase planktonic productivity. Maintenance regimes
(e.g.~scraping) reduce biomass and reset succession.

\subsection{Distribution}\label{distribution-58}

Urbanised coasts through tropical and temperate latitudes, especially in
North and Central America, Europe, and North and South Asia.

\section{S1.1 Aerobic caves}\label{s1.1-aerobic-caves}

Belongs to biome S1. Subterranean lithic biome, part of the Subterranean
realm.

\subsection{Short description}\label{short-description-59}

These air-filled voids beneath the ground are simple low-productivity
ecosystems limited by an absence of solar energy, except around their
openings to the surface. Food chains consequently lack plants and
herbivores. Microbes in biofilms are the dominant life forms, but some
caves have invertebrate detritivore and predators, or temporary
vertebrate inhabitants. Their limited energy comes from organic material
imported by seepage or animal movements, and bacteria that synthesise
chemical energy from rocks. They are found on all major land masses,
most commonly in carbonate rocks or lava tubes.

\subsection{Key Features}\label{key-features-59}

Dark dry or humid geological cavities with microbial chemoautotrophs,
detrivores, decomposers, endemic invertebrates \& no photoautotrophs.

\subsection{Ecological traits}\label{ecological-traits-59}

Dark subterranean air-filled voids support simple, low productivity
systems. The trophic network is truncated and dominated by heterotrophs,
with no representation of photosynthetic primary producers or
herbivores. Diversity is low, comprising detritivores and their
pathogens and predators, although there may be a few specialist
predators confined to resource-rich hotspots, such as bat latrines or
seeps. Biota include invertebrates (notably beetles, springtails, and
arachnids), fungi, bacteria, and transient vertebrates, notably bats,
which use surface-connected caves as roosts and breeding sites. Bacteria
and fungi form biofilms on rock surfaces. Fungi are more abundant in
humid microsites. Some are parasites and many are critical food sources
for invertebrates and protozoans. Allochthonous energy and nutrients are
imported via seepage moisture, tree roots, bats, and other winged
animals. This leads to fine-scale spatial heterogeneity in resource
distribution, reflected in patterns of biotic diversity and abundance.
Autochthonous energy can be produced by chemoautotrophs. For example,
chemoautotrophic Proteobacteria are prominent in subterranean caves
formed by sulphide springs. They fix carbon through sulphide oxidation,
producing sulphuric acid and gypsum residue in snottite draperies
(i.e.~microbial mats), accelerating chemical corrosion. The majority of
biota are obligate subterranean organisms that complete their life
cycles below ground. These are generalist detritivores and some are also
opportunistic predators, reflecting the selection pressure of food
scarcity. Distinctive traits include specialised non-visual sensory
organs, reduced eyes, pigmentation and wings, elongated appendages, long
lifespans, slow metabolism and growth, and low fecundity. Other cave
taxa are temporary below-ground inhabitants, have populations living
entirely above- or below-ground, or life cycles necessitating use of
both environments. The relative abundance and diversity of temporary
inhabitants decline rapidly with distance from the cave entrance. The
specialist subterranean taxa belong to relatively few evolutionary
lineages that either persisted as relics in caves after the extinction
of above-ground relatives or diversified after colonisation by
above-ground ancestors. Although diversity is low, local endemism is
high, reflecting insularity and limited connectivity between cave
systems.

\subsection{Key Ecological Drivers}\label{key-ecological-drivers-59}

Most caves form from the chemical weathering of limestone, dolomite or
gypsum, either from surface waters or from phreatic waters. Caves also
derive from lava tubes and other substrates. Characteristics include the
absence of light except at openings, low variability in temperature and
humidity, and scarcity of nutrients. The high physical fragmentation of
cave substrates limits biotic connectivity and promotes insular
evolution in stable conditions.

\subsection{Distribution}\label{distribution-59}

Scattered worldwide, but mostly in the Northern Hemisphere, in limestone
(map), basalt flows, and rarely in other lithic substrates.

\section{S1.2 Endolithic systems}\label{s1.2-endolithic-systems}

Belongs to biome S1. Subterranean lithic biome, part of the Subterranean
realm.

\subsection{Short description}\label{short-description-60}

The matrices and fissures of rocks host abundant microscopic life
throughout the Earth's crust, which is still at an early stage of
exploration. While some fissures support simple invertebrates, most
organisms are unicellular. Blue-green algae may occur in near-surface
layers of rock, but most energy comes from chemical synthesis of
minerals. Rates of growth and reproduction are slow, and limited by the
supply of energy. At depth, microbes tolerate high pressures and
temperatures.

\subsection{Key Features}\label{key-features-60}

Microbial systems within lithic matrices and interstitial spaces with
truncated trophic networks founded on lithautotrophs and lacking
photoautotrophs (except near surface) and high-order predators..

\subsection{Ecological traits}\label{ecological-traits-60}

Lithic matrices and their microscopic cracks and cavities host microbial
communities. Their very low productivity is constrained by the scarcity
of light, nutrients, and water, and sometimes also by high temperatures.
Diversity is low and the trophic network is truncated, supporting
microscopic bacteria, archaea, viruses, and unicellular eukaryotes. Most
are detritivores or lithoautotrophs, which derive energy, oxidants,
carbohydrates, and simple organic acids from carbon dioxide, geological
sources of hydrogen, and mineral compounds of potassium, iron and
sulphur. Some fissures are large enough to support small eukaryotic
predators such as nematodes. Photoautotrophs (i.e.~cyanobacteria) are
present only in the surface layers of exposed rocks. Sampling suggests
that these systems harbour 95\% of the world's prokaryote life (bacteria
and archaea), with rocks below the deep oceans and continents containing
similar densities of cells and potentially accounting for a significant
proportion of sequestered carbon. Endolithic microbes are characterised
by extremely slow reproductive rates, especially in deep sedimentary
rocks, which are the most oligotrophic substrates. At some depth within
both terrestrial and marine substrates, microbes are sustained by energy
from organic matter that percolates through fissures from surface
systems. In deeper or less permeable parts of the crust, however,
lithoautotrophic microbes are the primary energy synthesisers that
sustain heterotrophs in the food web. Methanogenic archaea and
iron-reducing bacteria appear to be important autotrophs in sub-oceanic
basalts. All endolithic microbes are characterised by slow metabolism
and reproduction rates. At some locations they tolerate extreme
pressures, temperatures (up to 125°C) and acidity (pH\textless2),
notably in crustal fluids. Little is currently known of endemism, but it
may be expected to be high based on the insularity of these ecosystems.

\subsection{Key Ecological Drivers}\label{key-ecological-drivers-60}

Endolithic systems are characterised by a lack of light, a scarcity of
nutrients, and high pressures at depth. Temperatures vary within the
crust from \textless20°C up to 125°C, but show little temporal
variation. The chemical properties and physical structure of lithic
matrices influence the supply of resources and the movement of biota.
Stable cratonic massifs have minimal pore space for microbial
occupation, which is limited to occasional cracks and fissures.
Sedimentary substrates offer more space, but nutrients may be scarce,
while fluids in basic volcanic and crustal rocks have more abundant
nutrients. Chemical and biogenic weathering occurs through biogenic
acids and other corrosive agents. The matrix is mostly stable, but
disturbances include infrequent and spatially variable earthquakes and
volcanic intrusions.

\subsection{Distribution}\label{distribution-60}

Throughout the earth's crust, from surface rocks to a predicted depth of
up to 4--4.5 km below the land surface and 7--7.5 km below ocean floors.

\section{S2.1 Anthropogenic subterranean
voids}\label{s2.1-anthropogenic-subterranean-voids}

Belongs to biome S2. Anthropogenic subterranean voids biome, part of the
Subterranean realm.

\subsection{Short description}\label{short-description-61}

Industrial excavations create artificial voids that resemble caves. Like
caves, they lack inputs of solar energy, and plants and herbivores are
absent, except at their openings or around artificial lighting. They are
colonised by opportunistic invertebrates and vertebrates, and microbes
from the rock matrix or imported materials. Diversity is low, but
depends on proximity to openings, human activity during occupation and
time since excavation.

\subsection{Key Features}\label{key-features-61}

Dry or humid subterranean voids created by mining or infrastructure
development and colonised by opportunistic microbes, invertebrates and
sometimes vertebrates..

\subsection{Ecological traits}\label{ecological-traits-61}

These low-productivity systems in subterranean air-filled voids are
created by excavation. Although similar to Aerobic caves (S1.1), these
systems are structurally simpler, younger, more geologically varied, and
much less biologically diverse with few evolutionary lineages and no
local endemism. Low diversity, low endemism, and opportunistic biotic
traits stem from founder effects related to their recent anthropogenic
origin (hence few colonisation events and little time for evolutionary
divergence), as well as low microhabitat niche diversity due to the
simple structure of void walls compared to natural caves. The trophic
network is truncated and dominated by heterotrophs, usually with no
representation of photosynthetic primary producers or herbivores.
Generalist detritivores and their pathogens and predators dominate,
although some specialists may be associated with bat dung deposits.
Biota include invertebrates (notably beetles, springtails, and
arachnids), fungi, bacteria, and transient vertebrates, notably bats,
which use the voids as roosts and breeding sites. Bacteria and fungi
form biofilms on void surfaces. Many are colonists of human
inoculations, with some microbes identified as ``human-indicator
bacteria'' (e.g.~E. coli, Staphylococcus aureus, and high-temperature
Bacillus spp.). Fungi are most abundant in humid microsites. Some are
parasites and many are critical food sources for invertebrates and
protozoans. Sources of energy and nutrients are allochthonous, imported
by humans, bats, winged invertebrates, other animals, and seepage
moisture. Many taxa have long life pans, slow metabolism and growth, and
low fecundity, but lack distinctive traits found in the biota of natural
caves. Some are temporary below-ground inhabitants, have populations
that live entirely above- or below-ground, or have life cycles
necessitating the use of both environments.

\subsection{Key Ecological Drivers}\label{key-ecological-drivers-61}

Excavations associated with tunnels, vaults and mines. While some are
abandoned, others are continuously accessed by humans, enhancing
connectivity with the surface, resource importation, and biotic
dispersal. Substrates include a range of rock types as well as
artificial surfaces on linings and debris piles. Air movement varies
from still to turbulent (e.g.~active train tunnels). Light is absent
except at openings and where artificial sources are maintained by
humans, sometimes supporting algae (i.e.~lampenflora). Humidity and
temperature are relatively constant, and nutrients are scarce except
where enriched by human sources.

\subsection{Distribution}\label{distribution-61}

Scattered worldwide, but mostly associated with urban centres, transit
corridors, and industrial mines.

\section{SF1.1 Underground streams and
pools}\label{sf1.1-underground-streams-and-pools}

Belongs to biome SF1. Subterranean freshwaters biome, part of the
Subterranean realm.

\subsection{Short description}\label{short-description-62}

These dark aquatic systems buried below ground have varied levels of
connectivity to surface waters, which influence their traits and
processes profoundly. Microbial mats composed of bacteria and aquatic
fungi on submerged rock surfaces are major food sources for protozoans
and invertebrates, especially in systems that are isolated from surface
waters. Larger water bodies support predatory fish. When connectivity is
strong, inflow brings a supply of nutrients and organic matter, as well
as itinerant biota.

\subsection{Key Features}\label{key-features-62}

Water-filled subterranean voids with low diversity of light-limited
bacteria, fungi, detrivores and predators..

\subsection{Ecological traits}\label{ecological-traits-62}

Subterranean streams, pools, and aquatic voids (flooded caves) are
low-productivity systems devoid of light. The taxonomic and functional
diversity of these water bodies is low, but they may host local
endemics, depending on connectivity with surface waters and between cave
systems. The truncated trophic network is entirely heterotrophic, with
no photosynthetic primary producers or herbivores. Detritivores and
their predators are dominant, although a few specialist predators may be
associated with resource-rich hotspots. Microbial mats composed of
bacteria and aquatic fungi covering submerged rock surfaces are major
food sources for protozoans and invertebrates. Other biota include
planktonic bacteria, crustaceans, annelids, molluscs, arachnids, and
fish in larger voids. Chemoautotrophic proteobacteria are locally
abundant in sulphur-rich waters fed by springs but not widespread.
Obligate denizens of subterranean waters complete their life cycles
entirely below ground and derive from relatively few evolutionary
lineages. These make up a variable portion of the biota, depending on
connectivity to surface waters. Most species are generalist detritivores
coexisting under weak competitive interactions. Some are also
opportunistic predators, reflecting selection pressures of food
scarcity. Distinctive traits include the absence of eyes and
pigmentation, long lifespans, slow metabolism and growth rates, and low
fecundity. Less-specialised biota include taxa that spend part of their
life cycles below ground and part above, as well as temporary
below-ground inhabitants. Transient vertebrates occur only in waters of
larger subterranean voids that are well connected to surface streams
with abundant food.

\subsection{Key Ecological Drivers}\label{key-ecological-drivers-62}

Most caves form from chemical weathering of soluble rocks such as
limestone or dolomite, and others include lava tunnels. Cave waters are
devoid of light, typically low in dissolved oxygen nutrients, and food,
and exhibit low variability in temperature. Water chemistry reflects
substrate properties (e.g.~high Calcium levels in limestone voids).
Resource supply and biotic dispersal depend on connectivity with surface
waters, flow velocity and turbulence. In the absence of light,
surface-connected streams are major allochthonous sources of energy and
nutrients. Disconnected systems are the most biologically insular and
oligotrophic, and may also be limited by nutrient imbalance. These
features promote insular evolution in stable conditions.

\subsection{Distribution}\label{distribution-62}

Scattered worldwide, mostly in the Northern Hemisphere in limestone and
more rarely in basalt flows and other lithic substrates.

\section{SF1.2 Groundwater
ecosystems}\label{sf1.2-groundwater-ecosystems}

Belongs to biome SF1. Subterranean freshwaters biome, part of the
Subterranean realm.

\subsection{Short description}\label{short-description-63}

Low productivity ecosystems occur within water-saturated permeable rock
strata or layers of buried unconsolidated sediments. Life in these dark,
confined or connected ecosystems is almost completely comprised of
microbial and small invertebrate detritivores and their protozoan
consumers. Diversity and productivity decline with depth and
connectivity to surface waters. Although groundwater ecosystems occur
almost everywhere in the near surface crust, they are prominent below
surface water and in artesian basins.

\subsection{Key Features}\label{key-features-63}

Saturated ecosystems at or below the watertable with low diversity
communities of heterotrophic microbes and invertebrates.

\subsection{Ecological traits}\label{ecological-traits-63}

These low-productivity ecosystems are found within or below groundwater
(phreatic) zones. They include aquifers (underground layers of
water-saturated permeable rock or unconsolidated gravel, sand, or silt)
and hyporheic zones beneath rivers and lakes (i.e.~where shallow
groundwater and surface water mix). Diversity and abundance of biota
decline with depth and connectivity to surface waters, as do nutrients
(e.g.~most meiofauna is limited to 100m depth). Microbial communities
are functionally diverse and invertebrate taxa exhibit high local
endemism where aquifers are poorly connected. Trophic networks are
truncated and comprised almost exclusively of heterotrophic microbes and
invertebrates. Chemoautotrophic bacteria are the only source of
autochthonous energy. Herbivores only occur where plant material enters
groundwater systems (e.g.~in well-connected hyporheic zones). Microbes
and their protozoan predators dwell on particle surfaces rather than in
pore water. They play key roles in weathering and mineral formation,
engineer chemically distinctive microhabitats through redox reactions,
and are repositories of Carbon, Nitrogen and Phosphorus within the
ecosystem. Meio-faunal detritivores and predators transfer Carbon and
nutrients from biofilms to larger invertebrate predators such as
crustaceans, annelids, nematodes, water mites, and beetles. These larger
trophic generalists live in interstitial waters, either browsing on
particle biofilms or ingesting sediment grains, digesting their surface
microbes, and excreting `cleaned' grains. They have morphological and
behavioural traits that equip them for life in dark, resource-scarce
groundwater where space is limited. These include slow metabolism and
growth, long lifespans without resting stages, low fecundity, lack of
pigmentation, reduced eyes, enhanced non-optic sensory organs, and
elongated body shapes with enhanced segmentation. Much of the biota
belongs to ancient subterranean lineages that have diverged
sympatrically within aquifers or allopatrically from repeated
colonisations or aquifer fragmentation.

\subsection{Key Ecological Drivers}\label{key-ecological-drivers-63}

Groundwater ecosystems are characterised by a scarcity of nutrients,
Carbon, dissolved oxygen and free space, and an absence of light. They
occur within basin fill or other porous geological strata. Groundwater
flow, pore size, interstitial biogeochemistry, and hydrological
conductivity to adjacent aquifers and surface waters determine ecosystem
properties. Subsurface water residence times vary from days in shallow,
well-connected, coarse-grained hyporheic systems to thousands of years
in deep, poorly connected aquifers confined between impermeable rock
strata. Lack of connectivity promotes insularity and endemism as well as
reductive biogeochemical processes that influence the availability of
food and nutrients.

\subsection{Distribution}\label{distribution-63}

Globally distributed. Map shows only the major groundwater basins by
recharge rates.

\section{SF2.1 Water pipes and subterranean
canals}\label{sf2.1-water-pipes-and-subterranean-canals}

Belongs to biome SF2. Anthropogenic subterranean freshwaters biome, part
of the Subterranean, Freshwater realm.

\subsection{Short description}\label{short-description-64}

Waters flowing rapidly through artificial conduits are typically bereft
of their own primary producers in the absence of light and rely on
imported algae and detritus as sources of energy. These support simple
bacterial and fungal communities in biofilms and largely itinerant
invertebrates. Diversity, abundance and productivity are low, but filter
feeders may colonise and productivity may be higher if source waters
supply nutrients and organic carbon.

\subsection{Key Features}\label{key-features-64}

Artificial flowing waterbodies that carry water with variable flow
regime, limited light, sometimes with high carbon and nutrients
supporting opportunities aquatic detritivores and predators.

\subsection{Ecological traits}\label{ecological-traits-64}

Constructed subterranean canals and water pipes are dark,
low-productivity systems acting as conduits for water, nutrients, and
biota between artificial or natural freshwater ecosystems. Energy
sources are therefore entirely or almost entirely allochthonous from
surface systems. Although similar to underground streams (S2.1), these
systems are structurally simpler, younger, and less biologically diverse
with few evolutionary lineages and no local endemism. Diversity and
abundance are low, often resulting from the accidental transport of
biota from source to sink ecosystems. Trophic networks are truncated,
with very few or no primary producers and no vertebrate predators except
incidental transients. The majority of the resident heterotrophic biota
are bacteria, aquatic fungi, and protists living in biofilms covering
mostly smooth artificial surfaces or cut rock faces. Biofilms constitute
food sources for detritivores and predators, including protozoans and
planktonic invertebrates as well as filter feeders such as molluscs. The
structure of the biofilm community varies considerably with hydraulic
regime, as does the biota in the water column. Transient vertebrates,
notably fish, occupy well-connected ecosystems with abundant food and
predominantly depend on transported nutrients and prey. A range of
organisms may survive in these environments but only some maintain
reproductive populations. All biota are capable of surviving under no or
low light conditions, at least temporarily while in transit. Other
traits vary with hydraulic regimes and hydrochemistry, with
physiological tolerance to toxins important in highly eutrophic,
slow-flowing drains and tolerance to low nutrients and turbulence
typical in high-velocity minerotrophic water pipes.

\subsection{Key Ecological Drivers}\label{key-ecological-drivers-64}

Subterranean canals and water pipes are engineered structures designed
to connect and move waters between artificial (or more rarely natural)
sources. They are united by an absence of light and usually low oxygen
levels and low variability in temperatures, but hydraulic regimes,
nutrient levels, water chemistry, flow and turbulence vary greatly among
ecosystems. Water supply pipes are extreme oligotrophic systems with
rapid flow, high turbulence, low nutrients and low connectivity to the
atmosphere, often sourced from de-oxygenated water at depth within large
reservoirs (F3.1). In contrast, subterranean wastewater or stormwater
canals have slower, more intermittent flows, low turbulence, and very
high nutrient levels and chemical pollutants including toxins. Many of
these eutrophic systems have an in situ atmosphere, but dissolved oxygen
levels are very low in connection with high levels of dissolved organic
Carbon and microbial activity.

\subsection{Distribution}\label{distribution-64}

Common in landscapes with urban or industrial infrastructure including
water supply and sewerage reticulation systems, hydroelectricity,
irrigation, and other intensive agricultural industries.

\section{SF2.2 Flooded mines and other
voids}\label{sf2.2-flooded-mines-and-other-voids}

Belongs to biome SF2. Anthropogenic subterranean freshwaters biome, part
of the Subterranean, Freshwater realm.

\subsection{Short description}\label{short-description-65}

Disused subterranean mines and other voids may fill with static or
slowing moving water from seepage. The inhabitants of these ecosystems
include biofilms comprising bacteria, aquatic fungi and protists that
originated as colonists from surrounding groundwaters, or imported with
people. As well as the lack of light and low nutrient levels,
productivity and diversity may be limited by heavy metals and toxins
liberated during construction and mining of the voids.

\subsection{Key Features}\label{key-features-65}

Underground largely static low-productivity waterbodies often with large
of warm groundwater or seepage, colonised by opportunistic microbes and
invertebrates.

\subsection{Ecological traits}\label{ecological-traits-65}

Abandoned and now flooded underground mines frequently contain extensive
reservoirs of geothermally warmed groundwater, colonized by stygobitic
invertebrates from nearby natural subterranean habitats. A fraction of
the biota is likely to have been introduced by mining activities. A lack
of light excludes photoautotrophs from these systems and low
connectivity limits inputs from allochthonous energy sources.
Consequently, overall productivity is low, and is likely to depend on
chemoautrophic microbes (e.g.~sulfate-reducing bacteria) as sources of
energy. Few studies have investigated the ecology of the aquatic biota
in quasi-stagnant water within mine workings, but trophic networks are
truncated and likely to be simple, with low diversity and abundance at
all trophic levels, and no endemism. Most of the resident heterotrophic
biota are bacteria, aquatic fungi, and protists living in biofilms on
artificial surfaces of abandoned infrastructure, equipment or cut rock
faces. Extremophiles are likely to dominate in waters that are highly
acidic or with high concentrations of heavy metals or other toxins.
Micro-invertebrates are most likely to be the highest-level predators.
Some voids may have simple assemblages of macroinverterbates, but few
are likely to support vertebrates unless they are connected with surface
waters that provide a means of colonization.

\subsection{Key Ecological Drivers}\label{key-ecological-drivers-65}

Like all subterranean ecosystems, light is absent or extremely dim in
flooded mines. Unlike subterranean canals and pipes (SF2.1), mine waters
are quasi-stagnant and not well connected to surface waters. During mine
operation, water is pumped out of the mine forming a widespread cone of
water table depression, with oxidation and hydrolysis of exposed
minerals changing groundwater chemistry. When mines close and dewatering
ceases, water table rebounds and the voids often flood. Some voids are
completely inundated, while others retain a subterranean atmosphere,
which may or may not be connected to the surface. Further changes in
water chemistry occur after flooding due to dissolution and flushing of
the oxidation products. Water is often warm due to geothermal heating.
After inundation has stabilised, seepage and mixing may be slow, and
stratification creates strong gradients in oxygen and solutes. Waters
are acidic in most flooded mines. The ionic composition varies depending
on mineralogy of the substrate, but ionic concentrations are typically
high, and often contain heavy metals at levels toxic to some aquatic
biota. Acid mine drainage is a common cause of pollution in surface
rivers and streams, where it seeps to the surface.

\subsection{Distribution}\label{distribution-65}

Common in in many mineral rich regions of the world.

\section{SM1.1 Anchialine caves}\label{sm1.1-anchialine-caves}

Belongs to biome SM1. Subterranean tidal biome, part of the
Subterranean, Marine realm.

\subsection{Short description}\label{short-description-66}

Anchialine caves contain water bodies that have a subterranean
connection to the sea and no or little direct connection to the
atmosphere. These ecosystems are structured by strong salinity
gradients. A primarily marine community influenced by tides and currents
at the cave entrance transitioning to a more insular and specialised
biota in still brackish waters influenced by rainfall percolation deep
in the landward section of the cave. They occur where carbonate rocks
and lava tubes meet coastlines worldwide.

\subsection{Key Features}\label{key-features-66}

Cave-bound waterbodies connected to the sea with a gradient of tidal
influence and salinity. Filter feeders, scavengers and predators limited
by light and nutrients.

\subsection{Ecological traits}\label{ecological-traits-66}

Anchialine caves contain bodies of saline or brackish waters with
subterranean connections to the sea. Since virtually all anchialine
biota are marine in origin, these caves have a larger and more diverse
species pool than underground freshwaters. The trophic network is
truncated and dominated by heterotrophs (scavenging and filter-feeding
detritivores and their predators), with photosynthetic primary producers
and herbivores only present where sinkholes connect caves to the surface
and sunlight. Productivity is limited by the scarcity of light and food,
but less so than in insular freshwater subterranean systems (SF1.1) due
to influx of marine detritus and biota. The dominant fauna includes
planktonic bacteria, protozoans, annelids, crustaceans, and fish.
Anchialine obligates inhabit locations deep within the caves, with
marine biota increasing in frequency with proximity to the sea. Caves
closely connected with the ocean tend to have stronger tidal currents
and biota such as sponges and hydroids commonly associated with sea
caves (SM1.3). Distinctive traits of cave obligates that reflect
selection under darkness and food scarcity include varying degrees of
eye loss and depigmentation, increased tactile and chemical sensitivity,
reproduction with few large eggs, long lifespans, and slow metabolism
and growth rates. Some anchialine biota are related to deep sea species,
including shrimps that retain red pigmentation, while others include
relict taxa inhabiting anchialine caves on opposite sides of ocean
basins. Characteristic anchialine taxa also occur in isolated water
bodies, far within extensive seafloor cave systems.

\subsection{Key Ecological Drivers}\label{key-ecological-drivers-66}

Anchialine caves originate from seawater penetration into faults,
fractures, and lava tubes as well as sea-level rise into limestone caves
formed by solution. Cave waters are characterised by an absence or
scarcity of light, low food abundance, and strong salinity gradients.
Sharp haloclines, which fluctuate with tides and rainfall percolation,
occur at deeper depths with increasing distance inland. Tidal
connections result in suck and blow phases of water movement that
diminish with increasing distance from the sea. In karst terrain with no
surface runoff, anchialine caves are closely linked via hydrology to
overlying subaerial coastal systems and can serve as subterranean rivers
with haloclines separating seaward flowing freshwater from underlying
saltwater. Temperatures are moderate, increasing at the halocline, then
stabilise with depth. Dissolved oxygen declines with depth.

\subsection{Distribution}\label{distribution-66}

Scattered worldwide, mostly in the Northern Hemisphere in limestone,
basalt flows, and more rarely other lithic substrates.

\section{SM1.2 Anchialine pools}\label{sm1.2-anchialine-pools}

Belongs to biome SM1. Subterranean tidal biome, part of the
Subterranean, Marine realm.

\subsection{Short description}\label{short-description-67}

Anchialine pools are brackish surface water bodies that have a
subterranean connection to the sea. They are associated with carbonate
substrates and lava flows on the coast and have a stronger terrestrial
influence than other subterranean systems. Exposure to the surface
enables algal primary producers to inhabit the water column and the
benthos. Diversity and productivity of these aquatic ecosystems
increases with age and connectivity to the sea. Some anchialine pools
are highly insular, with molluscs and crustacean species found nowhere
else.

\subsection{Key Features}\label{key-features-67}

Open pools with subterranean connections to the sea and groundwater, and
dynamic, diverse trophic networks.

\subsection{Ecological traits}\label{ecological-traits-67}

Anchialine pools, like anchialine caves (SM1.1), are tidally influenced
bodies of brackish water with subterranean connections to the sea and
groundwater, but with significant or full exposure to open air and
sunlight. They have no surface connection to the ocean or freshwater
ecosystems. Younger anchialine pools are exposed to abundant sunlight,
characterized by relatively low productivity, and tend to support only
benthic microalgae, cyanobacteria, and primary consumers. Older pools
with more established biological communities have higher productivity
with a wider range of autotrophs, including macroalgae, aquatic
monocots, established riparian and canopy vegetation, and primary and
secondary consumers. High productivity is attributed to a combination of
sunlight exposure, rugose substrates, and relatively high natural
concentrations of inorganic nutrients from groundwater. Anchialine pools
may support complex benthic microbial communities, primary consumers,
filter-feeders, detritivores, scavengers and secondary consumers. These
consumers are primarily molluscs and crustaceans, several of which are
anchialine obligates. Due to connections with deeper hypogeal habitats,
obligate species may display physical and physiological traits similar
to anchialine cave species. However, larger predatory fish and birds
also utilize anchialine pools for food and habitat. Anchialine pools are
ecologically dynamic systems due to their openness, connections with
surrounding terrestrial habitats and subterranean hydrologic
connections. Consequently, they are inherently sensitive to ecological
phase shifts throughout their relatively ephemeral existence, with
senescence initiating in as little as 100 years. However, new anchialine
pools may form within a few months after basaltic lava flows.

\subsection{Key Ecological Drivers}\label{key-ecological-drivers-67}

Anchialine pools form from subterranean mixing of seawater and
groundwater, primarily through porous basalt or limestone substrates,
and more rarely other lithic substrates. Tidal influences can drive
large fluctuations in water level and salinity on a daily cycle, but are
typically dampened with increased distance from the ocean. Sunlight, UV
exposure and other environmental characteristics vary within anchialine
pools and haloclines are common. The pools can also be connected to
anchialine cave systems (SM1.1) through tension fissures in basalt
flows, and collapsed openings in lava tubes.

\subsection{Distribution}\label{distribution-67}

Scattered worldwide, mostly in the northern hemisphere. Many well- known
examples occur in Hawaii, Palau and Indonesia, volcanic cracks or
grietas in the Galapagos Islands, and open-air entrance pools of
anchialine caves (e.g.~cenotes in Mexico's Yucatan Peninsula and blue
holes in the Bahamas).

\section{SM1.3 Sea caves}\label{sm1.3-sea-caves}

Belongs to biome SM1. Subterranean tidal biome, part of the
Subterranean, Marine realm.

\subsection{Short description}\label{short-description-68}

Sea caves are formed by waves action on fissures in a wide range of
rocky coastlines around the world. Unlike anchialine caves, salinity
gradients are weak and a strong marine influence is maintained
throughout their extent. They support a range of sessile invertebrates
(e.g.~sponges, cnidarians, bryozoans), mobile invertebrates
(e.g.~molluscs, crustaceans, annelids,) and fish. Some organisms appear
to be exclusive to sea caves, some shelter in sea caves diurnally, and
others are itinerant visitors.

\subsection{Key Features}\label{key-features-68}

Wave-exposed caves provide dim light and shelter to cave-exclusive,
resident and transient/ migratory invertebrates and fish..

\subsection{Ecological traits}\label{ecological-traits-68}

Sea caves (also known as marine or littoral caves) are usually formed by
wave action abrasion in various rock types. In contrast to anchialine
caves (SM1.1), sea caves are not isolated from the external marine
environment. Thus, the biota in sea caves are mostly stygophiles
(typical of dim-light cryptic and deep-water environments outside caves)
or stygoxenes (species sheltering in caves during daytime but foraging
outside at night). However, numerous taxa (mostly sessile invertebrates)
have so far been reported only from sea caves, and thus can be
considered as cave-exclusive sensu lato. Visitors often enter sea caves
by chance (e.g.~carried in by currents), and survive only for short
periods. The diverse sea-cave biota is dominated by sessile
(e.g.~sponges, cnidarians, bryozoans) and mobile invertebrates
(e.g.~molluscs crustaceans, annelids,) and fish. Photoautotrophs are
restricted close to cave openings, while chemoautotrophic bacteria form
extensive mats in sea caves with hydrothermal sulphur springs, similar
to those in some terrestrial caves (SF1.1) and deep sea vents (M3.7). In
semi-dark and dark cave sectors, the main trophic categories are
filter-feeders (passive and active), detritivores, carnivores, and
omnivores. Decomposers also play important roles. Filter-feeders consume
plankton and suspended organic material delivered by tidal currents and
waves. Other organisms either feed on the organic material produced by
filter-feeders or move outside caves in order to find food. These
``migrants'', especially swarm-forming crustaceans and schooling fish,
can be significant import pathways for organic matter, mitigating
oligotrophy in confined cave sectors.

\subsection{Key Ecological Drivers}\label{key-ecological-drivers-68}

Sea caves openings vary from fully submerged and never exposed to the
atmosphere to partially submerged and exposed to waves and tides. Sea
caves are generally shorter and receive less input of freshwater from
terrestrial sources than anchialine caves (SM1.1). Sea caves thus lack
haloclines, a defining feature of anchialine caves, and are influenced
more strongly by marine waters and biota throughout their extent. While
salinity gradients are weak, the decrease of light and sea water renewal
from the opening to the cave interior drive marked zonation of biota by
creating oligotrophic conditions and limiting larval supply. Submersion
level, cave morphology and micro-topography play key roles in forming
such gradients.

\subsection{Distribution}\label{distribution-68}

Globally distributed in coastal headlands, rocky reefs and in coral
reefs.

\section{T1.1 Tropical/Subtropical lowland
rainforests}\label{t1.1-tropicalsubtropical-lowland-rainforests}

Belongs to biome T1. Tropical-subtropical forests biome, part of the
Terrestrial realm.

\subsection{Short description}\label{short-description-69}

A huge diversity of species occupy niches within a complex
vertically-layered structure of plant forms. High productivity is
fuelled by rapid-growing plants including buttressed trees, bamboos,
epiphytes, lianas and ferns. Forest canopies sustain moist soils and
abundant leaf litter decomposed by fungi and bacteria. High diversity of
invertebrates at all levels of the forest supports diverse vertebrate
life forms, particularly mammals and birds, which play critical roles in
plant dispersal and pollination. Conditions near the equator are stable
and humid year-round (up to 6000 mm rain per annum), but become more
seasonal with mild winter frosts in the subtropics.

\subsection{Key Features}\label{key-features-69}

Tall closed-canopy evergreen forests in warm wet climates,
phylogenetically \& functionally highly diverse life forms.

\subsection{Ecological traits}\label{ecological-traits-69}

These closed-canopy forests are renowned for their complex structure and
high primary productivity, which support high functional and taxonomic
diversity. At subtropical latitudes they transition to warm temperate
forests (T2.4). Bottom-up regulatory processes are fuelled by large
autochthonous energy sources that support very high primary
productivity, biomass and LAI. The structurally complex, multi-layered,
evergreen tree canopy has a large range of leaf sizes (typically
macrophyll-notophyll) and high SLA, reflecting rapid growth and
turnover. Diverse plant life forms include buttressed trees, bamboos
(sometimes abundant), palms, epiphytes, lianas and ferns, but grasses
and hydrophytes are absent or rare. Trophic networks are complex and
vertically stratified with low exclusivity and diverse representation of
herbivorous, frugivorous, and carnivorous vertebrates. Tree canopies
support a vast diversity of invertebrate herbivores and their predators.
Mammals and birds play critical roles in plant diaspore dispersal and
pollination. Growth and reproductive phenology may be seasonal or
unseasonal, and reproductive masting is common in trees and regulates
diaspore predation. Fungal, microbial, and diverse invertebrate
decomposers and detritivores dominate the forest floor and the subsoil.
Diversity is high across taxa, especially at the upper taxonomic levels
of trees, vertebrates, fungi, and invertebrate fauna. Neutral processes,
as well as micro-niche partitioning, may have a role in sustaining high
diversity, but evidence is limited. Many plants are in the shade,
forming seedling banks that exploit gap-phase dynamics initiated by
individual tree-fall or stand-level canopy disruption by tropical storms
(e.g.~in near-coastal forests). Seed banks regulated by dormancy are
uncommon. Many trees exhibit leaf plasticity enabling photosynthetic
function and survival in deep shade, dappled light or full sun, even on
a single individual. A few species germinate on tree trunks, gaining
quicker access to canopy light, while roots absorb microclimatic
moisture until they reach the soil.

\subsection{Key Ecological Drivers}\label{key-ecological-drivers-69}

Precipitation exceeds evapo-transpiration with low intra- and
inter-annual variability, creating a reliable year-round surplus, while
closed tree canopies maintain humid microclimate and shade. Temperatures
are warm with low-moderate diurnal and seasonal variation (mean winter
minima rarely \textless10°C except in subtropical transitional zones).
Soils are moist but not regularly inundated or peaty (see TF1.1) and
vary widely in nutrient status. Most nutrient capital is sequestered in
vegetation or cycled through the dynamic litter layer, critical for
retaining nutrients that would otherwise be leached or lost to runoff.
In some coastal regions outside equatorial latitudes (mostly
\textgreater10° and excluding extensive forests in continental America
and Africa), decadal regimes of tropical storms drive cycles of canopy
destruction and renewal.

\subsection{Distribution}\label{distribution-69}

Humid tropical and subtropical regions in Central and West Africa,
Southeast Asia, Oceania, northeast Australia, Central and tropical South
America and the Caribbean.

\section{T1.2 Tropical/Subtropical dry forests and
thickets}\label{t1.2-tropicalsubtropical-dry-forests-and-thickets}

Belongs to biome T1. Tropical-subtropical forests biome, part of the
Terrestrial realm.

\subsection{Short description}\label{short-description-70}

Tropical-subtropical dry forests and thickets are characterised by
fertile substrates and seasonally dry conditions (\textasciitilde1800 mm
per year, with a period of 3 to 6 months receiving less than 100 mm per
month) that drive markedly seasonal patterns of productivity.
Tree-diversity is high, with a variable mixture of evergreen and
drought-deciduous trees, with abundant vines, but fewer epiphytes,
ferns, mosses, and forbs, limited by seasonal drought. Abundant leaf
litter is decomposed by fungi and microbes within complex foodwebs
dominated by vertebrates of all kinds.

\subsection{Key Features}\label{key-features-70}

Closed-canopy deciduous and semi-deciduous forests in warm seasonally
wet/dry climates, diverse life forms.

\subsection{Ecological traits}\label{ecological-traits-70}

These closed-canopy forests and thickets have drought-deciduous or
semi-deciduous phenology in at least some woody plants (rarely fully
evergreen), and thus seasonally high LAI. Strongly seasonal
photoautotrophic productivity is limited by a regular annual water
deficit/surplus cycle. Diversity is lower across most taxa than T1.1,
but tree and vertebrate diversity is high relative to most other forest
systems. Plant growth forms and leaf sizes are less diverse than in
T1.1. Grasses are rare or absent, except on savanna ecotones, due to
canopy shading and/or water competition, while epiphytes, ferns,
bryophytes, and forbs are present but limited by seasonal drought.
Trophic networks are complex with low exclusivity and diverse
representation of herbivorous, frugivorous, and carnivorous vertebrates.
Fungi and other microbes are important decomposers of abundant leaf
litter and N-fixing plants can be abundant. Many woody plants are
dispersed by wind and some by vertebrates. Most nutrient capital is
sequestered in vegetation or cycled through the litter layer. Trees
typically have thin bark and low fire tolerance and can recruit in
shaded microsites, unlike many in savannas. Plants are tolerant of
seasonal drought but can exploit moisture when it is seasonally
available through high SLA and plastic productivity. Gap-phase dynamics
are driven primarily by individual tree-fall and exploited by seedling
banks and vines (seedbanks are uncommon). These forests may be involved
in fire-regulated stable-state dynamics with savannas.

\subsection{Key Ecological Drivers}\label{key-ecological-drivers-70}

Overall water surplus (or small deficit \textless100 mm), but a
substantial seasonal deficit in winter in which little or no rain falls
within a 4--7-month period. Warm temperatures (minima rarely
\textless10°C) with low-moderate diurnal and seasonal variability in the
tropics, but greater seasonal variability in subtropical continental
areas. Diverse substrates generally produce high levels of nutrients.
Tropical storms may be important disturbances in some areas but
flammability is low due to limited ground fuels except on savanna
ecotones.

\subsection{Distribution}\label{distribution-70}

Seasonally dry tropical and subtropical regions in Central and West
Africa, Madagascar, southern Asia, north and northern and eastern
Australia, the Pacific, Central and South America and the Caribbean.

\section{T1.3 Tropical/Subtropical montane
rainforests}\label{t1.3-tropicalsubtropical-montane-rainforests}

Belongs to biome T1. Tropical-subtropical forests biome, part of the
Terrestrial realm.

\subsection{Short description}\label{short-description-71}

These mountain rainforests are characterised by a single-layered tree
canopy, with epiphytic ferns, bryophytes, lichens, orchids, and
bromeliads draping tree branches. Grasses are rare or absent. At high
altitude forest structure becomes less complex, with dwarf tree forms.
Although rainfall is abundant (up to 6000 mm per year), productivity is
limited by cool temperatures, wind exposure and shallow soils, although
under the canopy a moist shady microclimate provides stable habitat for
a frog, bird, plant and invertebrate species that are found nowhere
else.

\subsection{Key Features}\label{key-features-71}

Closed-canopy evergreen forests with abundant non-vascular epiphytes in
warm/cool wet cloudy climates, diverse life forms.

\subsection{Ecological traits}\label{ecological-traits-71}

Closed-canopy evergreen forests on tropical mountains usually have a
single-layer low tree canopy (\textasciitilde5--20m tall) with small
leaf sizes (microphyll-notophyll) and moderate-high SLA. They transition
to lowland rainforests (T1.1) with decreasing altitude and to warm
temperate forests (T2.4) at higher latitudes. Structure and taxonomic
diversity become more diminutive and simpler with altitude, culminating
in elfinwood forms. Conspicuous epiphytic ferns, bryophytes, lichens,
orchids, and bromeliads drape tree branches and exploit atmospheric
moisture (cloud stripping), but grasses are rare or absent, except for
bamboos in some areas. Moderate productivity fuelled by autochthonous
energy is limited by high exposure to UV-B radiation, cool temperatures,
and sometimes by shallow soil or wind exposure. Limited energy and
sequestration in humic soils may limit N and P uptake. Growth and
reproductive phenology is usually seasonal. Plant propagules are
dispersed mostly by wind and territorial birds and mammals. Tree
diversity is moderate to low, while epiphytes are diverse, but there is
often high local endemism at higher altitudes in most groups, especially
amphibians, birds, plants, and invertebrates. Gap-phase dynamics are
driven by tree-fall, landslides, lightning strikes, or in some areas
more rarely by extreme wind storms. Seedling banks are common (seedbanks
are uncommon) and most plants are shade tolerant and can recruit in the
shade.

\subsection{Key Ecological Drivers}\label{key-ecological-drivers-71}

Substantial cloud moisture and high humidity underpin a reliable
year-round rainfall surplus over evapotranspiration. Altitudinal
gradients in temperature, precipitation, and exposure are pivotal in
ecosystem structure and function. Frequent cloud cover from orographic
uplift and closed tree canopies maintain a moist microclimate and shady
conditions. Temperatures are mild-cool with occasional frost. Seasonal
variability is low-moderate but diurnal variability is moderate-high.
Winter monthly mean minima may be around 0°C in some areas. Landslides
are a significant form of disturbance that drives successional dynamics
on steep slopes and is exacerbated by extreme rainfall events. Mountains
experience elevated UV-B radiation with altitude and, in some regions,
are exposed to local or regional storms.

\subsection{Distribution}\label{distribution-71}

Humid tropical and subtropical regions in East Africa, East Madagascar,
Southeast Asia, west Oceania, northeast Australia, Central and tropical
South America.

\section{T1.4 Tropical heath forests}\label{t1.4-tropical-heath-forests}

Belongs to biome T1. Tropical-subtropical forests biome, part of the
Terrestrial realm.

\subsection{Short description}\label{short-description-72}

These structurally unique forests are restricted to less fertile soils
on acidic sandy substrates best known in Amazonia and Southeast Asia.
They are characterised by high densities of low, slender trees that
allow light penetration to an open forest floor covered with mosses.
They have low productivity compared to other tropical forests, with a
limited diversity of plants and animals forming a simple foodweb,
including amphibians and reptiles.

\subsection{Key Features}\label{key-features-72}

Low closed-canopy evergreen forests in warm wet climates on low-nutrient
substrates, structurally simple cf T1.

\subsection{Ecological traits}\label{ecological-traits-72}

Structurally simple evergreen forests with high densities of thin stems,
closed to open uniform canopies, typically 5--20 m tall and uniform with
a moderate to high LAI. Productivity is lower than in other tropical
forests, weakly seasonal and limited by nutrient availability and in
some cases by soil anoxia, but decomposition is rapid. Plant traits such
as insectivory, N-fixing microbial associations and ant mutualisms are
well represented, suggesting adaptive responses to nitrogen deficiency.
Plant insectivory aside, trophic networks are simple compared to other
tropical forests. Diversity of plant and animal taxa is also relatively
low, but dominance and endemism are proportionately high. Tree foliage
is characterised by small (microphyll-notophyll) leaves with lower SLA
than other tropical forests. Leaves are leathery and often ascending
vertically, enabling more light penetration to ground level than in
other tropical forests. Tree stems are slender (generally \textless20 cm
in diameter), sometimes twisted, and often densely packed and without
buttresses. Epiphytes are usually abundant but lianas are rare and
ground vegetation is sparse, with the forest floor dominated by
insectivorous vascular plants and bryophytes.

\subsection{Key Ecological Drivers}\label{key-ecological-drivers-72}

These forests experience an overall water surplus, but productivity is
limited by deep sandy low-nutrient acidic substrates, which are leached
by high rainfall. Acidity promotes high Al levels that inhibit root
growth. Most nutrients are retained in vegetation. Downward movement of
clay and organic particles through the soil profile results in a deep,
white sandy horizon capped by a thin grey surface horizon (typical of
podzols), limiting the capacity of the soil to retain nutrients
(especially nitrogen) and moisture within the shallow rooting zone.
Hence they are prone to inter-annual droughts, but waterlogging may
occur where the water table is close to the surface, resulting in
periodic anoxia within the root zone. Landscape water-table gradients
result in surface mosaics in which heath forests may be juxtaposed with
more waterlogged peat forests (TF1.1) and palustrine wetland systems
(TF1.2).

\subsection{Distribution}\label{distribution-72}

Scattered through northwest and west Amazonia, possibly Guiana, and
Southeast Asia, notably in the Rio Negro catchment and southern
Kalimantan. Poorly known in Africa, but possibly in the Gabon region.

\section{T2.1 Boreal and temperate high montane forests and
woodlands}\label{t2.1-boreal-and-temperate-high-montane-forests-and-woodlands}

Belongs to biome T2. Temperate-boreal forests and woodlands biome, part
of the Terrestrial realm.

\subsection{Short description}\label{short-description-73}

In boreal and mountainous cold, seasonally snow-prone climates, acid
soils support structurally simple forests made up of needle-leaf conifer
trees, sometimes with broad-leaf deciduous trees. Large forest trees
provide habitat for fungi, mosses and liverworts. Seasonal understorey
growth sustains high densities of herbivores such as bear, deer, and a
variety of insects, with predators such as lynx, wolves and raptors.
Plants and animals survive cold winters through freeze-tolerant organs,
hibernation or migratory movements.

\subsection{Key Features}\label{key-features-73}

Closed to open, evergreen (conifers) or deciduous forests in cold
climates with short growth periods, low vascular plant species
diversity, but abundant cryptogams.

\subsection{Ecological traits}\label{ecological-traits-73}

Evergreen, structurally simple forests and woodlands in cold climates
are dominated by needle-leaf conifers and may include a subdominant
component of deciduous trees, especially in disturbed sites, accounting
for up to two-thirds of stand-level leaf biomass. Boreal forests are
generally less diverse, more cold-tolerant and support a more migratory
fauna than temperate montane forests. Structure varies from dense forest
up to 30 m tall to stunted open woodlands \textless5 m tall. Large trees
engineer habitats of many fungi, non-vascular plants, invertebrates, and
vertebrates that depend on rugose bark, coarse woody debris, or large
tree canopies. Energy is mainly from autochthonous sources but may
include allochthonous subsidies from migratory vertebrates. Primary
productivity is limited by seasonal cold and may also be limited by
water deficit on coarse textured soils. Forested bogs occupy peaty soils
(TF1.6). Seasonal primary productivity may sustain a trophic web with
high densities of small and large herbivores (e.g.~hare, bear, deer, and
insects), with feline, canine, and raptor predators. Browsers are
top-down regulators of plant biomass and cyclers of nitrogen, carbon,
and nutrients. Forest structure may be disrupted by insect defoliation
or fires on multi-decadal cycles. Tree recruitment occurs
semi-continuously in gaps or episodically after canopy fires and may be
limited by spring frost, desiccation, permafrost fluctuations,
herbivory, and surface fires. Plants and animals have strongly seasonal
growth and reproductive phenology and possess morphological,
behavioural, and ecophysiological traits enabling cold-tolerance and the
exploitation of short growing seasons. Plant traits include bud
protection, extra-cellular freezing tolerance, hardened evergreen needle
leaves with low SLA or deciduous leaves with high SLA,
cold-stratification seed dormancy, seasonal geophytic growth forms, and
vegetative storage organs. Tracheids in conifers confer resistance to
cavitation in drought by compartmentalising water transport tissues.
Some large herbivores and most birds migrate to winter habitats from the
boreal zone, and thus function as mobile links, dispersing other biota
and bringing allochthonous subsidies of energy and nutrients into the
system. Hibernation is common among sedentary vertebrates, while insect
life cycles have adult phases cued to spring emergence.

\subsection{Key Ecological Drivers}\label{key-ecological-drivers-73}

These systems are driven by large seasonal temperature ranges, cold
winters with prolonged winter snow, low light, short growing seasons
(1--3 months averaging \textgreater10°C) and severe post-thaw frosts.
There is an overall water surplus, but annual precipitation can be
\textless200 mm. Soil moisture recharged by winter snow sustains the
system through evapotranspiration peaks in summer, but moisture can be
limiting where these systems extend to mountains in warm semi-arid
latitudes. The acid soils usually accumulate peat and upper horizons may
be frozen in winter. Forests may be prone to lightning-induced canopy
fires on century time scales and surface fires on multi-decadal scales.

\subsection{Distribution}\label{distribution-73}

Boreal distribution across Eurasia and North America, extending to
temperate (rarely subtropical) latitudes on mountains.

\section{T2.2 Deciduous temperate
forests}\label{t2.2-deciduous-temperate-forests}

Belongs to biome T2. Temperate-boreal forests and woodlands biome, part
of the Terrestrial realm.

\subsection{Short description}\label{short-description-74}

At cool temperate latitudes in the Northern hemisphere, fertile soils
and high precipitation support forests dominated by broadleaf deciduous
trees, although evergreen needleleaf trees may account for up to
one-third of the canopy. Cold snow-prone winters punctuate a limited but
highly productive growing season. Fungi and bacteria play vital roles in
decomposition of the seasonal leaf fall on the forest floor, with
insects and browsing herbivores important in carbon and nutrient
cycling. Herbivores such as deer and hares are prey to feline, canine
and avian predators. Winter dormancy, hibernation and migration are key
strategies enabling survival of plants and animals.

\subsection{Key Features}\label{key-features-74}

Closed canopy broadleaved forests in seasonally warm and cold humid
climates, with low to moderate woody species diversity.

\subsection{Ecological traits}\label{ecological-traits-74}

These structurally simple, winter deciduous forests have high
productivity and LAI in summer. Winter dormancy, hibernation and
migration are common life histories among plants and animals enabling
cold avoidance. Local endemism is comparatively low and there are modest
levels of diversity across major taxa. The forest canopy comprises at
least two-thirds deciduous broad-leaf foliage (notophylll-mesophyll)
with high SLA and up to one-third evergreen (typically needleleaf)
cover. As well as deciduous woody forms, annual turnover of above-ground
biomass also occurs some in non-woody geophytic and other ground flora,
which are insulated from the cold beneath winter snow and flower soon
after snowmelt before tree canopy closure. Annual leaf turnover is
sustained by fertile substrates and water surplus, with nutrient
withdrawal from foliage and storage of starch prior to fall. Tissues are
protected from cold by supercooling rather than extra-cellular
freeze-tolerance. Dormant buds are insulated from frost by bracts or by
burial below the soil in some non-woody plants. Fungal and microbial
decomposers play vital roles in cycling carbon and nutrients in the soil
surface horizon. Despite highly seasonal primary productivity, the
trophic network includes large browsing herbivores (deer), smaller
granivores and herbivores (rodents and hares), and mammalian predators
(canids and felines). Most invertebrates are seasonally active.
Behavioural and life-history traits allow animals to persist through
cold winters, including through dense winter fur, food caching, winter
foraging, hibernation, dormant life phases, and migration. Migratory
animals provide allochthonous subsidies of energy and nutrients and
promote incidental dispersal of other biota. Browsing mammals and
insects are major consumers of plant biomass and cyclers of nitrogen,
carbon, and nutrients. Deciduous trees may be early colonisers of
disturbed areas (later replaced by evergreens) but are also stable
occupants across large temperate regions. Tree recruitment is limited by
spring frost, allelopathy, and herbivory, and occurs semi-continuously
in gaps. Herbivores may influence densities of deciduous forest canopies
by regulating tree regeneration. Deciduous leaf fall may exert
allelopathic control over tree seedlings and seasonal ground flora.

\subsection{Key Ecological Drivers}\label{key-ecological-drivers-74}

Phenological processes in these forests are driven by large seasonal
temperature ranges, (mean winter temperatures \textless−1°C, summer
means up to 22°C), typically with substantial winter snow and limited
growing season, with 4--6 months \textgreater10°C, and severe post-thaw
frosts. Fertile soils with high N levels and an overall water surplus
support deciduous leaf turnover. Fires are uncommon.

\subsection{Distribution}\label{distribution-74}

Cool temperate Europe (southwest Russia to British Isles), northeast
Asia (northeast China, southern Siberia Korea, and Japan), and northeast
America. Limited occurrences in warm-temperate zones of south Europe and
Asia and the Midwest USA.

\section{T2.3 Oceanic cool temperate
rainforests}\label{t2.3-oceanic-cool-temperate-rainforests}

Belongs to biome T2. Temperate-boreal forests and woodlands biome, part
of the Terrestrial realm.

\subsection{Short description}\label{short-description-75}

Oceanic cool temperate rainforests have evergreen or semi-deciduous,
small-leaved trees, with conifers in some regions. These forests occupy
a cooler, wetter climate than warm temperate forests (T2.4), but ocean
influence promotes very high precipitation and limits persistence of
winter snow (compared with T2.1 or T2.2). Tree diversity is low, but
abundant epiphytic and terrestrial mosses and liverworts, ferns,
lichens, and conspicuous fungi contribute to seasonal productivity.
Foodwebs are simple, with low vertebrate diversity and fewer large
herbivores and predators than T2.1 and T2.2, but with many species found
nowhere else.

\subsection{Key Features}\label{key-features-75}

Closed canopy evergreen or semi-deciduous forests in cool wet climates,
high endemism with low tree diversity and abundant epiphytes.

\subsection{Ecological traits}\label{ecological-traits-75}

Broadleaf and needleleaf rainforests in cool temperate climates have
evergreen or semi-deciduous tree canopies with high LAI and mostly
nanophyll-microphyll foliage. Productivity is moderate to high and
constrained by strongly seasonal growth and reproductive phenology and
moderate levels of frost tolerance. SLA may be high but lower than in
T2.2. Evergreen trees typically dominate, but deciduous species become
more abundant in sites prone to severe frost and/or with high soil
fertility and moisture surplus. The smaller range of leaf sizes and SLA,
varied phenology, frost tolerance, broader edaphic association, and
wetter, cooler climate distinguish these forests from warm temperate
forests (T2.4). Local or regional endemism is significant in many taxa.
Nonetheless, energy sources are primarily autochthonous. Trophic
networks are less complex than in other cool-temperate or boreal forests
(T2.1 and T2.2), with weaker top-down regulation due to the lower
diversity and abundance of large herbivores and predators. Tree
diversity is low (usually \textless8--10 spp./ha), with abundant
epiphytic and terrestrial bryophytes, pteridophytes, lichens, a modest
range of herbs, and conspicuous fungi, which are important decomposers.
The vertebrate fauna is mostly sedentary and of low-moderate diversity.
Most plants recruit in the shade and some remain in seedling banks until
gap-phase dynamics are driven by individual tree-fall, lightning
strikes, or by extreme wind storms in some areas. Tree recruitment
varies with tree masting events, which strongly influence trophic
dynamics, especially of rodents and their predators.

\subsection{Key Ecological Drivers}\label{key-ecological-drivers-75}

There is a large water surplus, rarely with summer deficits. Rainfall is
seasonal, borne on westerly winds peaking in winter months and
inter-annual variability is relatively low. Cool winters (minima
typically \textless0--5°C for 3 months) limit the duration of the
growing season. Maritime air masses are the major supply of climatic
moisture and moderate winters and summer temperatures. Light may be
limited in winter by frequent cloud cover and high latitude.
Intermittent winter snow does not persist for more than a few days or
weeks. Soils are moderately fertile to infertile and may accumulate
peat. Exposure to winter storms and landslides leaves imprints on forest
structure in some regions. Fires are rare, occurring on century time
scales when lightning (or human) ignitions follow extended droughts.

\subsection{Distribution}\label{distribution-75}

Cool temperate coasts of Chile and Patagonia, New Zealand, Tasmania and
the Pacific Northwest, rarely extending to warm-temperate latitudes on
mountains in Chile, southeast Australia, and outliers above 2,500-m
elevation in the New Guinea highlands. Some authors extend the concept
to wet boreal forests on the coasts of northwest Europe, Japan, and
northeast Canada.

\section{T2.4 Warm temperate laurophyll
forests}\label{t2.4-warm-temperate-laurophyll-forests}

Belongs to biome T2. Temperate-boreal forests and woodlands biome, part
of the Terrestrial realm.

\subsection{Short description}\label{short-description-76}

With a patchy warm-temperate distribution, laurophyll forests are
extensive in some regions, but more typically occupy topographic refugia
in a matrix of drier, more fire-prone ecosystems. They have glossy- or
leathery-leaved dense evergreen canopies, with moderate tree diversity.
A mild climate, and more acid soils, distinguish them from oceanic cool
temperate forests (T2.3). Primary productivity is high, but can be
limited by mild summer drought. Decomposers such as invertebrates,
fungi, and microbes on the forest floor are critical to nutrient
cycling. Insects are the major consumers of primary production and a
major food source for birds and bats, which can be important
seed-dispersers and pollinators. Vertebrate herbivores are relatively
uncommon.

\subsection{Key Features}\label{key-features-76}

Simple, closed-canopy mostly evergreen forests in warm environments with
modest summer rainfall deficits; moderate diversity and endemism.

\subsection{Ecological traits}\label{ecological-traits-76}

Relatively productive but structurally simple closed-canopy forests with
high LAI occur in humid warm-temperate to subtropical climates. The tree
canopies are more uniform than most tropical forests (T1.1 and T1.2) and
usually lack large emergents. Their foliage is often leathery and glossy
(laurophyll) with intermediate SLA values, notophyll-microphyll sizes,
and prodigiously evergreen. Deciduous species are rarely scattered
within the forest canopies. These features, and drier, warmer climates
and often more acid soils distinguish them from oceanic cool temperate
forests (T2.3), while in the subtropics they transition to biome T1
forests. Autochthonous energy supports relatively high primary
productivity, weakly limited by summer drought and sometimes by acid
substrates. Forest function is regulated mainly by bottom-up processes
related to resource competition rather than top-down trophic processes
or disturbance regimes. Trophic structure is simpler than in tropical
forests, with moderate levels of diversity and endemism among major taxa
(e.g.~typically \textless20 tree spp./ha), but local assemblages of
birds, bats, and canopy invertebrates may be abundant and species-rich
and play important roles in pollination and seed dispersal. Canopy
insects are the major consumers of primary production and a major food
source for birds. Decomposers and detritivores such as invertebrates,
fungi, and microbes on the forest floor are critical to nutrient
cycling. Vertebrate herbivores are relatively uncommon, with
low-moderate mammalian diversity. Although epiphytes and lianas are
present, plant life-form traits that are typical of tropical forests
(T1.1 and T1.2) such as buttress roots, compound leaves, monopodial
growth, and cauliflory are uncommon or absent in warm-temperate
rainforests. Some trees have ecophysiological tolerance of acid soils
(e.g.~through aluminium accumulation). Gap-phase dynamics are driven by
individual tree-fall and lightning strikes, but many trees are
shade-tolerant and recruit slowly in the absence of disturbance. Ground
vegetation includes varied growth forms but few grasses.

\subsection{Key Ecological Drivers}\label{key-ecological-drivers-76}

The environmental niche of these forests is defined by a modest overall
water surplus with no distinct dry season, albeit moderate summer water
deficits in some years. Mean annual rainfall is typically 1,200--2,500
mm, but topographic mesoclimates (e.g.~sheltered gullies and orographic
processes) sustain reliable moisture at some sites. Temperatures are
mild with moderate seasonality and a growing season of 6--8 months, and
mild frosts occur. Substrates may be acidic with high levels of Al and
Fe that limit the uptake of nutrients. These forests may be embedded in
fire-prone landscapes but are typically not flammable due to their moist
microclimates .

\subsection{Distribution}\label{distribution-76}

Patchy warm temperate-subtropical distribution at 26--43° latitude,
north or south of the Equator.

\section{T2.5 Temperate pyric humid
forests}\label{t2.5-temperate-pyric-humid-forests}

Belongs to biome T2. Temperate-boreal forests and woodlands biome, part
of the Terrestrial realm.

\subsection{Short description}\label{short-description-77}

Forests with the tallest flowering trees on earth are complex in
structure, with an open canopy of sclerophyll trees 40-90m tall allowing
light to filter through to multiple understorey layers and a ground
flora of ferns, grasses and mosses. The complex forest structure
sustains a high diversity of birds, reptiles mammals, and canopy
invertebrates, as well as a moist microclimate and deep, moist leaf
litter on the forest floor that supports a diversity of soil
invertebrates, fungi, and other microbes. These highly productive,
fast-growing forests are notable carbon sinks, but extreme droughts make
them prone to the most ferocious forest fires on earth at century-scale
intervals. Seedbanks are crucial to the forest persisting after fire.

\subsection{Key Features}\label{key-features-77}

Tall, moist and complex multi-layered forests in wet-temperate climates;
characterised by sclerophyll dominant trees and diverse mesophyll
understorey; population processes driven by fire regimes.

\subsection{Ecological traits}\label{ecological-traits-77}

This group includes the tallest forests on earth. They are moist,
multi-layered forests in wet-temperate climates with complex spatial
structure and very high biomass and LAI. The upper layer is an open
canopy of sclerophyllous trees 40--90-m tall with long, usually
unbranched trunks. The open canopy structure allows light transmission
sufficient for the development of up to three subcanopy layers,
consisting mostly of non-sclerophyllous trees and shrubs with higher SLA
than the upper canopy species. These forests are highly productive, grow
rapidly, draw energy from autochthonous sources and store very large
quantities of carbon, both above and below ground. They have complex
trophic networks with a diverse invertebrate, reptile, bird, and mammal
fauna with assemblages that live primarily in the tree canopy or the
forest floor, and some that move regularly between vertical strata. Some
species are endemic and have traits associated with large trees,
including the use of wood cavities, thick or loose bark, large canopies,
woody debris, and deep, moist leaf litter. There is significant
diversification of avian foraging methods and hence a high functional
and taxonomic diversity of birds. High deposition rates of leaf litter
and woody debris sustain diverse fungal decomposers and invertebrate
detritivores and provide nesting substrates and refuges for ground
mammals and avian insectivores. The shade-tolerant ground flora may
include a diversity of ferns forbs, grasses (mostly C3), and bryophytes.
The dominant trees are shade-intolerant and depend on tree-fall gaps or
periodic fires for regeneration. In cooler climates, trees are killed by
canopy fires but may survive surface fires, and canopy seedbanks are
crucial to persistence. Epicormic resprouting (i.e.~from aerial stems)
is more common in warmer climates. Subcanopy and ground layers include
both shade-tolerant and shade-intolerant plants, the latter with
physically and physiologically dormant seedbanks that cue episodes of
mass regeneration to fire. Multi-decadal or century-scale canopy fires
consume biomass, liberate resources, and trigger life-history processes
in a range of biota. Seedbanks sustain plant diversity through storage
effects.

\subsection{Key Ecological Drivers}\label{key-ecological-drivers-77}

There is an annual water surplus with seasonal variation (peak surplus
in winter) and rare major summer deficits associated with inter-annual
drought cycles. Multiple tree layers produce a light diminution gradient
and moist micro-climates at ground level. Winters are cool and summers
are warm with occasional heatwaves that dry out the moist micro-climate
and enable periodic fires, which may be extremely intense and consume
the canopy. The growing season is 6--8 months. Snow is uncommon and
short-lived. Soils are relatively fertile, but often limited in
Nitrogen.

\subsection{Distribution}\label{distribution-77}

Subtropical - temperate southeast and temperate southwest Australia.

\section{T2.6 Temperate pyric sclerophyll forests and
woodlands}\label{t2.6-temperate-pyric-sclerophyll-forests-and-woodlands}

Belongs to biome T2. Temperate-boreal forests and woodlands biome, part
of the Terrestrial realm.

\subsection{Short description}\label{short-description-78}

In fire-prone temperate regions, temperate pyric sclerophyll forests are
characterised by an open canopy of hard-leaved trees with a shrub layer
underneath, sometimes with grasses and forbs. Productivity is limited by
seasonal drought, hot summers, and low nutrients in sandy and loamy
soils. Some groups of plants, birds, reptiles, and invertebrates have
high diversity and uniqueness, many with locally restricted
distributions. Plants and animals are adapted for persistence through
successive summer droughts and fires and, although sensitive to fire
frequency and season, periodic disturbance by fire is critical to
maintaining forest diversity.

\subsection{Key Features}\label{key-features-78}

Sclerophyll forests and woodlands in warm climates with winter
precipitation and a canopy-fire regime.

\subsection{Ecological traits}\label{ecological-traits-78}

Forests and woodlands, typically 10--30-m tall with an open evergreen
sclerophyllous tree canopy and low-moderate LAI grow in fire-prone
temperate landscapes. Productivity is lower than other temperate and
tropical forest systems, limited by low nutrient availability and summer
water deficits. Abundant light and water (except in peak summer) enable
the development of substantial biomass with high C:N ratios. Trees have
microphyll foliage with low to very low SLA. Sclerophyll or
subsclerophyll shrubs with low to very low SLA foliage form a prominent
layer between the trees. A sparse ground layer of C3 and C4 tussock
grasses and forbs becomes more prominent on soils of loamy texture.
Diversity and local endemism may be high among some taxa including
plants, birds, and some invertebrates such as dipterans and hemipterans.
Low nutrients and summer droughts limit the diversity and abundance of
higher trophic levels. Plant traits (e.g.~sclerophylly, stomatal
invagination, tubers, and seedbanks) confer tolerance to pronounced but
variable summer water deficits. Plants possess traits that promote the
efficient capture and retention of nutrients, including specialised root
structures, N-fixing bacterial associations, slow leaf turnover, and
high allocation of photosynthates to structural tissues and exudates.
Consumers have traits that enable the consumption of high-fibre biomass.
Mammalian herbivores (e.g.~the folivorous koala) can exploit high-fibre
content and phenolics. Plants and animals have morphological and
behavioural traits that allow tolerance or avoidance of fire and the
life-history processes of many taxa are cued to fire (especially plant
recruitment). Key fire traits in plants include recovery organs
protected by thick bark or burial, serotinous seedbanks (i.e.~held in
plant canopies), physical and physiological seed dormancy and pyrogenic
reproduction. Almost all plants are shade-intolerant and fire is a
critical top-down regulator of diversity through storage effects and the
periodic disruption of plant competition.

\subsection{Key Ecological Drivers}\label{key-ecological-drivers-78}

Hot summers generate a marked but variable summer water deficit, usually
with a modest winter surplus, irrespective of whether rainfall is highly
seasonal with winter maximum, aseasonal, or weakly seasonal with
inter-annually variable summer maxima. Soils are acidic, sandy, or loamy
in texture, and low to very impoverished in P and N. Hot summers define
a marked season for canopy or surface fires at decadal to multi-decadal
intervals. Light frost occurs periodically in some areas but snow is
rare.

\subsection{Distribution}\label{distribution-78}

Temperate regions of Australia, the Mediterranean, and central
California.

\section{T3.1 Seasonally dry tropical
shrublands}\label{t3.1-seasonally-dry-tropical-shrublands}

Belongs to biome T3. Shrublands and shrubby woodlands biome, part of the
Terrestrial realm.

\subsection{Short description}\label{short-description-79}

Occurring on nutrient-deficient soils of tropical regions, these
fire-prone shrublands and low forests are associated with dry tropical
winters, often occurring in a matrix with savannas (T4.2) or tropical
dry forests (T1.2). Dominated by small-leaved sclerophyll shrubs and
grasses, plants have traits to capture and conserve nutrients, such as
cluster roots and carnivorous forms. Birds, reptiles and seed-eating
small mammals dominate the vertebrate fauna, with few vertebrate
herbivores. Periodic fires are cues for life-history processes of plants
and animals, and help maintain species composition and nutrient cycling.

\subsection{Key Features}\label{key-features-79}

Mostly evergreen, sclerophyll shrublands on nutrient-poor soils, C4
grasses can be important.

\subsection{Ecological traits}\label{ecological-traits-79}

These moderate-productivity, mostly evergreen shrublands, shrubby
grasslands and low, open forests (generally \textless6-m tall) are
limited by nutritional poverty and strong seasonal drought in the
tropical winter months. Taxonomic and functional diversity is moderate
in most groups but with high local endemism in plants, invertebrates,
birds, and other taxa. Vegetation is spatially heterogeneous in a matrix
of savannas (T4.2) or tropical dry forests (T1.2) and dominated by
sclerophyllous shrubs with small leaf sizes (nanophyll-microphyll) and
low SLA. C4 grasses may be conspicuous or co-dominant (unlike in most
temperate heathlands, T3.2) but generally do not form a continuous
stratum as in savannas (biome T4). These systems have relatively simple
trophic networks fuelled by autochthonous energy sources. Productivity
is low to moderate and constrained by seasonal drought and nutritional
poverty. Shrubs are the dominant primary producers and show traits
promoting the capture and conservation of nutrients (e.g.~sclerophylly,
cluster roots, carnivorous structures, and microbial and fungal root
mutualisms) and tolerance to severe seasonal droughts (e.g.~stomatal
invagination). Nectarivorous and/or insectivorous birds and reptiles and
granivorous small mammals dominate the vertebrate fauna, but vertebrate
herbivores are sparse. Recurring fires play a role in the top-down
regulation of ecosystem structure and composition.

\subsection{Key Ecological Drivers}\label{key-ecological-drivers-79}

A severe seasonal climatic water deficit during tropical winter months
is exacerbated by sandy or shallow rocky substrates with low moisture
retention. Nutritional poverty (especially N and P) stems from
oligotrophic, typically acid substrates such as sandstones, ironstones,
leached sand deposits, or rocky volcanic or ultramafic substrates.
Vegetation holds the largest pool of nutrients. Temperatures are warm,
rarely \textless10°C, with low diurnal and seasonal variation.
Dry-season fires recur on decadal or longer time scales, but they are
rare in table-top mountains (tepui).

\subsection{Distribution}\label{distribution-79}

Brazilian campos rupestres (where grasses are important), Venezuelan
tepui, Peruvian tabletops, Florida sands, and scattered in northern
Australia and montane oceanic islands.

\section{T3.2 Seasonally dry temperate heath and
shrublands}\label{t3.2-seasonally-dry-temperate-heath-and-shrublands}

Belongs to biome T3. Shrublands and shrubby woodlands biome, part of the
Terrestrial realm.

\subsection{Short description}\label{short-description-80}

These temperate ecosystems are dominated by sclerophyll shrubs with
small or ericoid leaves. A low sparse tree canopy may or may not be
present. Low-moderate productivity is limited by summer droughts and low
nutrient availability. and sandy or loamy soils, with many diverse plant
specialisations to low nutrient, and regular fires, accelerated by slow
decomposition rates. Foodwebs vary from complex to simple, but most lack
large herbivores and predators. Vertebrate herbivores have
specialisations to exploit low nutrient vegetation and avoid recurring
fires, which are influential on plant and animal life histories.
Specific plant-invertebrate relationships (e.g.~as larval hosts and
pollinators) are common (moths and butterflies larval hosts, wasp
pollinators).

\subsection{Key Features}\label{key-features-80}

Sclerophyll evergreen shrublands of humid and subhumid mid-latitudes
with a canopy-fire regime.

\subsection{Ecological traits}\label{ecological-traits-80}

Sclerophyllous, evergreen shrublands are distinctive ecosystems of humid
and subhumid climates in mid-latitudes. Their low-moderate productivity
is fuelled by autochthonous energy sources and is limited by resource
constraints and/or recurring disturbance. Vegetation is dominated by
shrubs with very low SLA, high C:N ratios, shade-intolerance, and
long-lived, small, often ericoid leaves, sometimes with a low, open
canopy of sclerophyll trees. The ground layer may include geophytes and
sclerophyll graminoids, though less commonly true grasses. Trophic webs
are simple, with large mammalian predators scarce or absent, and low
densities of vertebrate herbivores. Native browsers may have local
effects on vegetation. Diversity and local endemism may be high among
vascular plants and invertebrate consumers. Plants and animals have
morphological, ecophysiological, and life-history traits that promote
persistence under summer droughts, nutrient poverty, and recurring
fires, which play a role in top-down regulation. Stomatal regulation and
root architecture promote drought tolerance in plants. Cluster roots and
acid exudates, mycorrhizae, and insectivory promote nutrient capture,
while cellulose, lignin, exudate production, and leaf longevity promote
nutrient conservation in plants. Vertebrate herbivores and granivores
possess specialised dietary and digestive traits enabling consumption of
foliage with low nutrient content and secondary compounds. Slow
decomposition rates are slow, allowing litter-fuel accumulation to add
to well-aerated fine fuels in shrub canopies. Life-history traits such
as recovery organs, serotiny, post-fire seedling recruitment, pyrogenic
flowering, and fire-related germination cues promote plant survival,
growth, and reproduction under recurring canopy fires. Animals evade
fires in burrows or through mobility. Animal pollination syndromes are
common (notably dipterans, lepidopterans, birds, and sometimes mammals)
and ants may be prominent in seed dispersal.

\subsection{Key Ecological Drivers}\label{key-ecological-drivers-80}

A marked summer water deficit and a modest winter surplus is driven by
high summer temperatures and evapotranspiration with winter-maximum or
aseasonal rainfall patterns. Winters are mild, or cool at high
elevations. Sandy soil textures or reverse-texture effects of clay-loams
exacerbate an overall water deficit. Soils are typically acid, derived
from siliceous sand deposits, sandstones, or acid intrusives or
volcanics, and are low to very low in P, N, and mineral cations (though
this varies between regions, e.g, base-rich limestones, marl and
dolomites in southern Europe). The climate, soils, and vegetation
promote summer canopy fires at decadal to multi-decadal intervals.
Positive feedbacks between fire and vegetation may be important in
maintaining flammability.

\subsection{Distribution}\label{distribution-80}

Mediterranean-type climate regions of Europe, north and south Africa,
southern Australia, western North and South America, and occurrences in
non-Mediterranean climates in eastern Australia, the USA, and Argentina.

\section{T3.3 Cool temperate
heathlands}\label{t3.3-cool-temperate-heathlands}

Belongs to biome T3. Shrublands and shrubby woodlands biome, part of the
Terrestrial realm.

\subsection{Short description}\label{short-description-81}

In cool temperate, humid, maritime environments, a dense cover of low
shrubs with small tough leaves is interspersed with grasses and ferns.
Cold temperatures and low-fertility acid soils limit productivity, with
wet subsoils limiting decomposition so that organic matter accumulates.
Low intensity fires may occur in the warmer months. Browsing mammals,
such as rabbits and deer, bring nutrients from more productive systems
and maintain the shrubby composition. Canids and raptors are common
predators of ground-nesting birds and rodents, in a relatively simple
foodweb.

\subsection{Key Features}\label{key-features-81}

Low-diversity, low productivity mixed graminoid ericoid shrublands of
maratime environments, supporting mammalian browsers.

\subsection{Ecological traits}\label{ecological-traits-81}

These mixed graminoid shrublands are restricted to cool-temperate
maritime environments. Typically, the vegetation cover is
\textgreater70\% and mostly less than 1-m tall, dominated by low,
semi-sclerophyllous shrubs with ferns and C3 graminoids. Shrub foliage
is mostly evergreen and ericoid, with low SLA or reduced to spiny stems.
Modular growth forms are common among shrubs and grasses. Diversity and
local endemism are low across taxa and the trophic network is relatively
simple. Primary productivity is low, based on autochthonous energy
sources and limited by cold temperatures and low-fertility acid soils
rather than by water deficit (as in other heathlands, biome T3).
Seasonally low light may limit productivity at the highest latitudes.
Cool temperatures and low soil oxygen due to periodically wet subsoil
limit decomposition by microbes and fungi so that soils accumulate
organic matter despite low productivity. Mammalian browsers including
cervids, lagomorphs, and camelids (South America) consume local plant
biomass but subsidise autochthonous energy with carbon and nutrients
consumed in more productive forest or anthropogenic ecosystems adjacent
to the heathlands. Browsers and recurring low-intensity fires appear to
be important in top-down regulatory processes that prevent the
transition to forest, as is anthropogenic fire, grazing, and tree
removal. Canids and raptors are the main vertebrate predators. Other
characteristic vertebrate fauna include ground-nesting birds and
rodents. At least some communities exhibit autogenic cyclical patch
dynamics in which shrubs and grasses are alternately dominant,
senescent, and regenerating.

\subsection{Key Ecological Drivers}\label{key-ecological-drivers-81}

Unlike most other heathlands, these ecosystems have an overall water
surplus, though sometimes with small summer deficits. Mild summers and
cold winters with periodic snow are tempered by maritime climatic
influences. A short day length and low solar angle limits energy influx
at the highest latitudes. Severe coastal storms with high winds occur
periodically. Acid soils, typically with high humic content in upper
horizons, are often limited in N and P. Low-intensity fires recur at
decadal time scales or rarely. Some northern European heaths were
derived from forest and return to forest when burning and grazing
ceases.

\subsection{Distribution}\label{distribution-81}

Boreal and cool temperate coasts of western Europe and America, the
Azores, and the Magellanic region of South America, mostly at
\textgreater40° latitude, except where transitional with warm-temperate
heaths (e.g.~France and Spain).

\section{T3.4 Young rocky pavements, lava flows and
screes}\label{t3.4-young-rocky-pavements-lava-flows-and-screes}

Belongs to biome T3. Shrublands and shrubby woodlands biome, part of the
Terrestrial realm.

\subsection{Short description}\label{short-description-82}

With a scattered distribution globally, these young rocky ecosystems are
exposed to extreme temperatures, weathering and disturbance, and have
limited capacity to retain water and nutrients. Analogues in icy
environments belong to T6.2. Productivity and diversity are consequently
low. Lichens and mosses are often abundant and important to ecosystem
development, slowly building soils through incremental retention of
moisture and nutrients. Successional development of soils and vegetation
may be interrupted by landslides, eruptions and other mass movements.
Small-leaved pioneer shrubs and grasses are sparse, often growing in
crevices. The simple foodwebs are comprised mainly of microbes and
itinerant organisms, with few resident vertebrates other than reptiles
and ground-nesting birds.

\subsection{Key Features}\label{key-features-82}

Low-diversity cryptogam-dominated systems with scattered herbs and
shrubs on skeletal substrates with limited nutrients and moisture.

\subsection{Ecological traits}\label{ecological-traits-82}

Vegetation dominated by cryptogams (lichens, bryophytes) develops on
skeletal rocky substrates and may have scattered shrubs with very low
LAI. These low-productivity systems are limited by moisture and nutrient
scarcity, temperature extremes, and periodic disturbance through mass
movement. Diversity and endemism is low across taxa and the trophic
structure is simple. Reptiles and ground-nesting birds are among the few
resident vertebrates. Lichens and bryophytes may be abundant and perform
critical roles in moisture retention, nutrient acquisition, energy
capture, surface stabilisation, and proto-soil development, especially
through carbon accumulation. N-fixing lichens and cyanobacteria, nurse
plants, and other mutualisms are critical to ecosystem development.
Rates of ecosystem development are linked to substrate weathering,
decomposition, and soil development, which mediate nutrient supply,
moisture retention, and temperature amelioration. Vascular plants have
nanophyll-microphyll leaves and low SLA. Their cover is sparse and
comprises ruderal pioneer species (shrubs, grasses, and forbs) that
colonise exposed surfaces and extract moisture from rock crevices.
Species composition and vegetation structure are dynamic in response to
surface instability and show limited differentiation across
environmental gradients and microsites due to successional development,
episodes of desiccation, and periodic disturbances that destroy biomass.
Rates of vegetation development, soil accumulation, and compositional
change display amplified temperature-dependence due to
resource-concentration effects. Older rocky systems have greater
micro-habitat diversity, more insular biota, and higher endemism and are
classified in other functional groups.

\subsection{Key Ecological Drivers}\label{key-ecological-drivers-82}

Skeletal substrates (e.g.~lava pavements, scree slopes, and rock
outcrops) limit water retention and nutrient capital and increase heat
absorption, leading to periodically extreme temperatures. High summer
temperatures and solar exposure concentrate resources and increase the
temperature-sensitivity of biogeochemical processes. Winter temperatures
may be cold at high elevations (see T6.2). Recurring geophysical
disturbances such as lava flow, mass movement, and geothermal activity
as well as desiccation episodes periodically destroy biomass and reset
successional pathways.

\subsection{Distribution}\label{distribution-82}

Localised areas scattered around the Pacific Rim, African Rift Valley,
Mediterranean and north Atlantic.

\section{T4.1 Trophic savannas}\label{t4.1-trophic-savannas}

Belongs to biome T4. Savannas and grasslands biome, part of the
Terrestrial realm.

\subsection{Short description}\label{short-description-83}

An unparalleled abundance and diversity of large herbivores maintain an
open structure of these tree-grass ecosystems in the seasonal tropics of
Africa and south Asia. Large herbivores are pivotal in maintaining short
rhizomatous and tussock grasses, limiting recruitment of trees, cycling
nutrients, sustaining complex food webs of invertebrate detritivores and
diverse assemblages of mammalian and avian predators and scavengers.
Seasonally high productivity coincides with the summer rainy season.
Fires occur in some years, but are less influential than herbivores on
ecosystem function. Trees and grasses have adaptations to seasonal
drought and heavy browsing, such as rhizomes and stolons that allow
grasses to spread under grazing pressure.

\subsection{Key Features}\label{key-features-83}

Grassy woodlands and grasslands dominated by C4 grasses in seasonal
climates with lower rainfall and higher soil fertility..

\subsection{Ecological traits}\label{ecological-traits-83}

These grassy woodlands and grasslands are dominated by C4 grasses with
stoloniferous, rhizomatous and tussock growth forms that are kept short
by vertebrate grazers. Trophic savannas (relative to pyric savannas,
T4.2) have unique plant and animal diversity within a complex trophic
structure dominated by abundant mammalian herbivores and predators.
These animals are functionally differentiated in body size, mouth
morphology, diet, and behaviour. They promote fine-scale vegetation
heterogeneity and dominance of short grass species, sustaining the
system through positive feedbacks and limiting fire fuels. Trees and
grasses possess functional traits that promote tolerance to chronic
herbivory as well as seasonal drought. Seasonal high productivity
coincides with summer rains. The dry season induces grass drying and
leaf fall in deciduous and semi-deciduous woody plants. Trees are
shade-intolerant during their establishment and most develop chemical
(e.g.~phenolics) or physical (e.g.~spinescence) herbivory defence traits
and an ability to re-sprout as they enter the juvenile phase. Their soft
microphyll-notophyll foliage has relatively high SLA and low C:N ratios,
as do grasses. Robust root systems and stolons/rhizomes enable
characteristic grasses to survive and spread under heavy grazing. As
well as vertebrate herbivores and predators, vertebrate scavengers and
invertebrate detritivores are key components of the trophic network and
carbon cycle. Nitrogen fixation, recycling, and deposition by animals
exceeds volatilisation.

\subsection{Key Ecological Drivers}\label{key-ecological-drivers-83}

Trophic savannas like pyric savannas are driven by seasonal climates but
generally occupy environmental niches with lower rainfall and higher
soil fertility. High annual rainfall deficit of 400 mm to
\textgreater1,800 mm. Annual rainfall generally varies from 300 mm to
700 mm, always with strong seasonal (winter) drought, but these savanna
types are restricted to landscapes with sufficient water bodies (rivers
and lakes) to sustain high densities of large mammals. Temperatures are
warm-hot with low-moderate variability through the year. Low intensity
fires have return intervals of 5--50 years, depending on animal
densities and inter-annual rainfall variation, usually after the growing
season, removing much of the remaining biomass not consumed by
herbivores. Soils are moderately fertile and often have a significant
clay component.

\subsection{Distribution}\label{distribution-83}

Seasonal tropics and subtropics of Africa and Asia.

\section{T4.2 Pyric tussock savannas}\label{t4.2-pyric-tussock-savannas}

Belongs to biome T4. Savannas and grasslands biome, part of the
Terrestrial realm.

\subsection{Short description}\label{short-description-84}

In pyric savannas, recurring fire is the principal agent that limits
tree dominance and maintains tree-grass coexistence. Dominated by
tussock grasses that grow tall during high-productivity wet summers and
cure over dry winter seasons, these ecosystems occur on all major land
masses at tropical and subtropical latitudes around the world. Large
mammalian herbivores are usually present, but not at densities that
limit grass growth or mediate tree-grass coexistence (unlike T4.1). Many
plants have traits that promote tolerate of seasonal drought, such as
deciduous leaf phenology, subterranean storage organs and deep roots.
Invertebrate detritivores, notably termites, and vertebrate scavengers
are key groups in the foodweb.

\subsection{Key Features}\label{key-features-84}

Grasslands and grassy woodlands dominated by C4 tussock grasses. Strong
seasonal (winter) drought, low fertility, and fires major consumer of
biomass..

\subsection{Ecological traits}\label{ecological-traits-84}

Grassy woodlands and grasslands are dominated by C4 tussock grasses,
with some C3 grasses in the Americas and variable tree cover. In the
tropics, seasonally high productivity coincides with the timing of
summer rains and grasses cure in dry winters, promoting flammability.
This pattern also occurs in the subtropics but transitions occur with
temperate woodlands (T4.4), which have different seasonal phenology,
tree and grass dominance, and fire regimes. Tree basal area, abundance
of plants with annual semelparous life cycles and abundant grasses with
tall tussock growth forms are strongly dependent on mean annual rainfall
(i.e.~limited by seasonal drought). Local endemism is low across all
taxa but regional endemism is high, especially in the Americas and
Australasia. Plant traits such as deciduous leaf phenology or deep roots
promote tolerance to seasonal drought and rapid resource exploitation.
Woody plants have microphyll-notophyll foliage with moderate-high SLA
and mostly high C:N ratios. Some C4 grasses nonetheless accumulate high
levels of rubisco, which may push down C:N ratios. Nitrogen
volatilisation exceeds deposition because fire is the major consumer of
biomass. Woody plant species are shade-intolerant during their
establishment and develop fire-resistant organs (e.g.~thick bark and
below-ground bud banks). The contiguous ground layer of erect tussock
grasses creates an aerated flammable fuel bed, while grass architecture
with tightly clustered culms vent heat away from meristems. Patchy fires
promote landscape-scale vegetation heterogeneity (e.g.~in tree cover)
and maintain the dominance of flammable tussock grasses over shrubs,
especially in wetter climates, and hence sustain the system through
positive feedbacks. Fires also enhance efficiency of predators.
Vertebrate scavengers and invertebrate detritivores are key components
of the trophic network and carbon cycle. Mammalian herbivores and
predators are present but exert less top-down influence on the diverse
trophic structure than fire. Consequently, plant physical defences
against herbivores, such as spinescence are less prominent than in T4.1.

\subsection{Key Ecological Drivers}\label{key-ecological-drivers-84}

An overall rainfall deficit up to \textasciitilde1,200 mm or a modest
surplus of up to 500 mm, always with strong seasonal (winter) drought
with continuously warm-hot temperatures through the year, even though
rainfall becomes less seasonal in the subtropics. Mean annual rainfall
varies from 650 mm to 1,500 mm. Sub-decadal fire regimes of surface
fires occur throughout the dry season, while canopy fires occur rarely,
late in the dry season. Soils are of low-moderate fertility, often with
high Fe and Al.

\subsection{Distribution}\label{distribution-84}

Seasonally dry tropics and subtropics of the Americas, Australia, Asia,
and Africa.

\section{T4.3 Hummock savannas}\label{t4.3-hummock-savannas}

Belongs to biome T4. Savannas and grasslands biome, part of the
Terrestrial realm.

\subsection{Short description}\label{short-description-85}

Found only in northern Australia, hummock savannas are distinguished by
a ground layer of slow-growing, domed hummock grasses interspersed with
bare ground and some trees and shrubs. These habitats are less strongly
seasonal than other savannas, but still with winter droughts and summer
rains, and many plant adaptations to seasonal drought. Recurring fires
are an important factor promoting patchiness of vegetation, but
post-fire recovery is slower than other savanna ecosystems. Rocky
coarse-textured substrates are low in nutrients. Foodwebs are
correspondingly simple, with large numbers of invertebrates and low
numbers of mammalian herbivores and vertebrate predators.

\subsection{Key Features}\label{key-features-85}

Sparse to open low-productivity woodlands in nutrient poor often rocky
landscapes with C4 hummock grasses, rich reptile fauna, abundant
termites, moderate herbivore densities and irregular fires..

\subsection{Ecological traits}\label{ecological-traits-85}

These open woodlands are dominated by C4 hummock grasses (C3 and
stoloniferous grasses are absent) with sclerophyllous trees and shrubs.
Their primary productivity is lower and less regularly seasonal than in
other savannas of the subtropics (T4.1 and T4.2), but the seasonal peak
nonetheless coincides with summer monsoonal rains. Plant traits promote
tolerance to seasonal drought, including reduced leaf surfaces, thick
cuticles, sunken stomata, and deep root architecture to access subsoil
moisture. Deciduous leaf phenology is less common than in other
savannas, likely due to selection pressure for nutrient conservation
associated with oligotrophic substrates. A major feature distinguishing
this group of savannas from others is its ground layer of slow-growing
sclerophyllous, spiny, domed hummock grasses interspersed with bare
ground. Woody biomass and LAI decline along rainfall gradients.
Sclerophyll shrubs and trees are shade-intolerant during establishment
and most possess fire-resistant organs (e.g.~thick bark, epicormic
meristematic tissues, and below-ground bud banks). Their notophyll
foliage and that of hummock grasses have low SLA and mostly high C:N
ratios, although N may be elevated in rubisco-enriched C4 grasses.
Trophic structure is therefore simpler than in other savannas. Mammalian
herbivores and their predators are present in low densities, but fire
and invertebrates are the major biomass consumers. Fire promotes
landscape-scale vegetation heterogeneity but occurs less frequently than
in other savannas due to slow recovery of perennial hummock grass fuels.
Nitrogen volatilisation exceeds deposition due to recurring fires.

\subsection{Key Ecological Drivers}\label{key-ecological-drivers-85}

Large overall rainfall deficit up to \textasciitilde2,000 mm, always
with a seasonal (winter) drought, but in drier areas seasonality is
weaker than in other savanna groups. Mean annual rainfall is generally
400--1,000 mm. Climatic water deficit is exacerbated by coarse-textured,
usually shallow, rocky soils. These are characteristically infertile.
Temperatures are warm-hot with moderate seasonal and diurnal
variability. Fires promoted by flammable hummocks may consume the low
tree canopies and occur at variable decadal intervals any time when it
is dry, but fire spread depends on ground fuel continuity which is
limited by rainfall and rocky terrain.

\subsection{Distribution}\label{distribution-85}

Rocky areas of the seasonal Australian tropics, extending to the
semi-arid zone.

\section{T4.4 Temperate woodlands}\label{t4.4-temperate-woodlands}

Belongs to biome T4. Savannas and grasslands biome, part of the
Terrestrial realm.

\subsection{Short description}\label{short-description-86}

Temperate woodlands are structurally simple, with widely-spaced trees
and a ground layer of grasses with scattered shrubs. They are globally
distributed in temperate climates with warm-season droughts. Tree
foliage is typically evergreen, but may be deciduous in cold dry
climates. During warm summers, productive grasses on fertile soils
sustain a complex foodweb of insects, reptiles, birds and mammals. The
ground flora varies with rainfall and tree cover, which creates diverse
microhabitats beneath. Large herbivores and their predators are
important to maintaining woodland composition, with burrowing mammals
influencing soil and nutrient cycling. Fires occur periodically, but
have less influence than in pyric savannas (T4.2) or forests (T2.6).

\subsection{Key Features}\label{key-features-86}

Open-canopy woodlands, trees microphyll and evergreen,~ with herbaceous
understory including C3 and/or C4 grasses.

\subsection{Ecological traits}\label{ecological-traits-86}

These structurally simple woodlands are characterised by space between
open tree crowns and a ground layer with tussock grasses, interstitial
forbs, and a variable shrub component. Grasses with C3 and C4
photosynthetic pathways are common, but C4 grasses may be absent from
the coldest and wettest sites or where rain rarely falls in the summer.
In any given area, C4 grasses are most abundant in summer or on dry
sites or areas with summer-dominant rainfall, while C3 grasses
predominate in winter, locally moist sites, cold sites, or areas without
summer rainfall. The ground flora also varies inter-annually depending
on rainfall. Trees generate spatial heterogeneity in light, water, and
nutrients, which underpin a diversity of microhabitats and mediate
competitive interactions among plants in the ground layer. Foliage is
mostly microphyll and evergreen (but transmitting abundant light) or
deciduous in colder climates. Diversity of plant and invertebrate groups
may therefore be relatively high at local scales, but local endemism is
limited due to long-distance dispersal. Productivity is relatively high
as grasses rapidly produce biomass rich in N and other nutrients after
rains. This sustains a relatively complex trophic network of
invertebrate and vertebrate consumers. Large herbivores and their
predators are important top-down regulators. Bioturbation by fossorial
mammals influences soil structure, water infiltration, and nutrient
cycling. The fauna is less functionally and taxonomically diverse than
in most tropical savannas (T4.1 and T4.2), but includes large and small
mammals, reptiles, and a high diversity of birds and
macro-invertebrates, including grasshoppers, which are major consumers
of biomass. Plants and animals tolerate and persist through periodic
ground fires that consume cured-grass fuels, but few have specialised
traits cued to fire (cf.~pyric ecosystems such as T2.6).

\subsection{Key Ecological Drivers}\label{key-ecological-drivers-86}

A water deficit occurs seasonally in summer, driven primarily by peak
evapotranspiration under warm-hot temperatures and, in some regions,
seasonal (winter-maximum) rainfall patterns. Mean annual rainfall is
350--1,000 mm. Low winter temperatures and occasional frost and snow may
limit the growing season to 6--9 months. Soils are usually fine-textured
and fertile, but N may be limiting in some areas. Fires burn mostly in
the ground layers during the drier summer months at decadal intervals.

\subsection{Distribution}\label{distribution-86}

Temperate southeast and southwest Australia, Patagonia and Pampas of
South America, western and eastern North America, the Mediterranean
region, and temperate Eurasia.

\section{T4.5 Temperate subhumid
grasslands}\label{t4.5-temperate-subhumid-grasslands}

Belongs to biome T4. Savannas and grasslands biome, part of the
Terrestrial realm.

\subsection{Short description}\label{short-description-87}

Temperate subhumid grasslands are simple in structure, composed of
tussock grasses with scattered forbs, with few isolated trees and
shrubs. Cold winters with occasional to frequent snow and frost limit
the growing season, but hot dry summers create water stress.
Nonetheless, fertile soils enable high productivity after rains,
supporting a complex foodweb of invertebrates, ground-nesting birds,
burrowing mammals, large herbivores, reptiles and predators. Large
herbivores graze heavily, range widely, and are important in maintaining
coexistence of plant species and nutrient cycling, as are periodic
fires.

\subsection{Key Features}\label{key-features-87}

Tussock grasslands with mixtures of C3 and C4 grasses and interstitial
forbs, high productivity and complex trophic networks.

\subsection{Ecological traits}\label{ecological-traits-87}

Structurally simple tussock grasslands with interstitial forbs occur in
subhumid temperate climates. Isolated trees or shrubs may be present in
very low densities, but are generally excluded by heavy soil texture,
summer drought, winter frost, or recurring summer fires. Unlike tropical
savannas (T4.1--T4.3), these systems are characterised by a mixture of
both C3 and C4 grasses, with C4 grasses most abundant in summer or on
dry sites and C3 grasses predominating in winter or locally moist sites.
There are also strong latitudinal gradients, with C3 grasses more
dominant towards the poles. Diversity of plant and invertebrate groups
may be high at small spatial scales, but local endemism is limited due
to long-distance dispersal. Productivity is high as grasses rapidly
produce biomass rich in N and other nutrients after rains. This sustains
a complex trophic network in which large herbivores and their predators
are important top-down regulators. Fossorial mammals are important in
bioturbation and nutrient cycling. Mammals are less functionally and
taxonomically diverse than in most savannas. Taxonomic affinities vary
among regions (e.g.~ungulates, cervids, macropods, and camelids), but
their life history and dietary traits are convergent. Where grazing is
not intense and fire occurs infrequently, leaf litter accumulates from
the tussocks, creating a thatch that is important habitat for
ground-nesting birds, small mammals, reptiles, and macro-invertebrates,
including grasshoppers, which are major consumers of plant biomass.
Dense thatch limits productivity. Plant competition plays a major role
in structuring the ecosystem and its dynamics, with evidence that it is
mediated by resource ratios and stress gradients, herbivory, and fire
regimes. Large herbivores and fires both interrupt competition and
promote coexistence of tussocks and interstitial forbs.

\subsection{Key Ecological Drivers}\label{key-ecological-drivers-87}

A strong seasonal water deficit in summer driven by peak
evapotranspiration under warm-hot temperatures, despite an unseasonal or
weakly seasonal rainfall pattern. Mean annual rainfall varies from 250
mm to 750 mm. Cold winter temperatures limit the growing season to 5--7
months, with frost and snow frequent in continental locations. Summers
are warm. Soils are deep, fertile and organic and usually fine-textured.
Fires ignited by lightning occur in the drier summer months at
sub-decadal or decadal intervals.

\subsection{Distribution}\label{distribution-87}

Subhumid and semi-arid regions of western Eurasia, northeast Asia,
Midwest North America, Patagonia and Pampas regions of South America,
southeast Africa, southeast Australia, and southern New Zealand.

\section{T5.1 Semi-desert steppe}\label{t5.1-semi-desert-steppe}

Belongs to biome T5. Deserts and semi-deserts biome, part of the
Terrestrial realm.

\subsection{Short description}\label{short-description-88}

Semi-desert steppes on all continents are dominated by perennial shrubs,
often with semi-fleshy or velvety foliage, and tussock grasses
interspersed with bare ground. Low variable rainfall and extreme
temperatures favour flora and fauna with drought and stress-tolerance
adaptations, such as deep roots and nomadism. Growth and reproduction of
shrubs and grasses is varies with rainfall, the cover of grass
diminishing to near zero in extended droughts. Grass cover also depends
on soil fertility and grazing animals, which may also limit shrub
recruitment and growth. These steppes are among the most productive of
desert ecosystems, with relatively abundant small and large herbivorous
and seed-eating mammals supporting bird and mammal predators and
scavengers.

\subsection{Key Features}\label{key-features-88}

Low-productivity and low-stature shrublands, tussock-grass and mixed,
with episodic trophic pulses driven by variable rainfall.

\subsection{Ecological traits}\label{ecological-traits-88}

These mixed semi-deserts are dominated by suffrutescent (i.e.~with a
woody base) or subsucculent (semi-fleshy) perennial shrubs and tussock
grasses. Productivity and biomass are limited by low average
precipitation, extreme temperatures and, to a lesser extent, soil
nutrients, but vary temporally in response to water availability.
Vegetation takes a range of structural forms including open shrublands,
mixed shrublands with a tussock grass matrix, prairie-like tall forb
grasslands, and very low dwarf shrubs interspersed with forbs or
grasses. Total cover varies from 10\% to 30\% and the balance between
shrubs and grasses is mediated by rainfall, herbivory, and soil
fertility. Stress-tolerator and ruderal life-history types are strongly
represented in flora and fauna. Trait plasticity and nomadism are also
common. Traits promoting water capture and conservation in plants
include xeromorphy, deep roots, and C4 photosynthesis. Shrubs have small
(less than nanophyll), non-sclerophyll, often hairy leaves with moderate
SLA. Shrubs act as resource-accumulation sites, promoting heterogeneity
over local scales. C3 photosynthesis is represented in short-lived
shrubs, forbs, and grasses, enabling them to exploit pulses of winter
rain. Consumers include small mammalian and avian granivores,
medium-sized mammalian herbivores, and wide-ranging large mammalian and
avian predators and scavengers. Abundant detritivores consume dead
matter and structure resource availability and habitat characteristics
over small scales. Episodic rainfall initiates trophic pulses with rapid
responses by granivores and their predators, but less so by herbivores,
which show multiple traits promoting water conservation.

\subsection{Key Ecological Drivers}\label{key-ecological-drivers-88}

Semi-desert steppes are associated with fine-textured, calcareous soils
of low-moderate fertility, and may contain appreciable levels of
magnesium or sodium. Clay particles exchange mineral ions with plant
roots and have `reverse texture effects', limiting moisture extraction
as soils dry. Indurated subsoils influence infiltration/runoff
relationships and vegetation patterns. Semi-desert steppes are not
typically fire-prone and occur in temperate-arid climates. Mean annual
rainfall (\textasciitilde150--300 mm), with and has a winter maximum.
Evapotranspiration is 2-20 times greater than precipitation, but large
rain events bring inter-annual pulses of water surplus. Temperatures are
highly variable diurnally and seasonally, often exceeding 40°C in summer
and reaching 0°C in winters but rarely with snow.

\subsection{Distribution}\label{distribution-88}

Extensive areas across the Sahara, the Arabian Peninsula, west Asia,
southwest Africa, southern Australia, Argentina, and the Midwest USA.

\section{T5.2 Succulent or Thorny deserts and
semi-deserts}\label{t5.2-succulent-or-thorny-deserts-and-semi-deserts}

Belongs to biome T5. Deserts and semi-deserts biome, part of the
Terrestrial realm.

\subsection{Short description}\label{short-description-89}

Succulent or Thorny deserts and semi-deserts are restricted in their
occurrence to parts of the Americas, Africa, Madagascar and south Asia.
They have a sparse cover of typical cactus-like plants and other
slow-growing spiny and succulent species, on stony low-nutrient soils.
Many of these plants store water in their stems and have deep roots.
Short-lived plant species emerge after rains from dormant organs or soil
seed banks. Plants and animals tolerate extreme summer temperatures and
mild winters. Nocturnal and burrowing mammals are able to avoid extreme
temperatures. Diverse opportunistic invertebrates and reptiles occur,
along with small numbers of large-ranging ungulates.

\subsection{Key Features}\label{key-features-89}

Characterized by tall succulent plants, diverse annuals and geophytes,
supporting diverse mammals, reptiles and invertebrates.

\subsection{Ecological traits}\label{ecological-traits-89}

These deserts are characterised by long-lived perennial plants, many
with spines and/or succulent stem tissues or leaves. Local endemism is
prominent among plants and animals. Productivity is low but relatively
consistent through time and limited by precipitation and extreme summer
temperatures. Vegetation cover is sparse to moderate (10--30\%) and up
to several metres tall. Dominant plants are stress-tolerators with slow
growth and reproduction, many exhibiting CAM physiology and traits that
promote water capture, conservation, and storage. These include deep
root systems, suffrutescence, plastic growth and reproduction, succulent
stems and/or foliage, thickened cuticles, sunken stomata, and deciduous
or reduced foliage. Spinescence in many species is likely a physical
defence to protect moist tissues from herbivores. Annuals and geophytes
constitute a variable proportion of the flora exhibiting rapid
population growth or flowering responses to semi-irregular rainfall
events, which stimulate germination of soil seed banks or growth from
dormant subterranean organs. Mammalian, reptilian, and invertebrate
faunas are diverse, with avian fauna less well represented. Faunal
traits adaptive to drought and heat tolerance include physiological
mechanisms (e.g.~specialised kidney function and reduced metabolic
rates) and behavioural characters (e.g.~nocturnal habit and burrow
dwelling). Many reptiles and invertebrates have ruderal life histories,
but fewer mammals and birds do. Larger ungulate fauna exhibit flexible
diets and forage over large areas. Predators are present in low
densities due to the low productivity of prey populations.

\subsection{Key Ecological Drivers}\label{key-ecological-drivers-89}

These systems occur in subtropical arid climates with large overall
water deficits. Precipitation is 5--20\% of potential
evapotranspiration, but exhibits low inter-annual variability relative
to other desert systems. Inter-annual pulses of surplus are infrequent
and atmospheric moisture from fogs may contribute significantly to
available water. Temperatures are hot with relatively large diurnal
ranges, but seasonal variation is less than in other deserts, with very
hot summers and mild winters. Substrates are stony and produce soils of
moderate to low fertility. Thorny deserts are generally not fire-prone.

\subsection{Distribution}\label{distribution-89}

Mostly subtropical latitudes in the Americas, southern Africa, and
southern Asia.

\section{T5.3 Sclerophyll hot deserts and
semi-deserts}\label{t5.3-sclerophyll-hot-deserts-and-semi-deserts}

Belongs to biome T5. Deserts and semi-deserts biome, part of the
Terrestrial realm.

\subsection{Short description}\label{short-description-90}

Sclerophyll hot deserts and semi-deserts occur across central and
western Australia on sandy soils. Dry and very low nutrient conditions
favour dominance of long-lived hard-leaved shrubs and hummock grasses.
This vegetation concentrates scarce resources in patches that provide
critical refuges for invertebrates, reptiles, ground-nesting birds and
small mammals, whose digging activity contributes to nutrient cycling.
Fires periodically liberate resources and restructure these ecosystems.
Episodic rain storms regulated by regional climate cycles produce a
`boom' in productivity, with emergence of short-lived plants and high,
but transient abundance of small mammals.

\subsection{Key Features}\label{key-features-90}

Perennial sclerophyll shrubs and Hummock C4 grasses on nutrient-poor
soils; highly variable rainfall, high diversity and endemism.

\subsection{Ecological traits}\label{ecological-traits-90}

Arid systems dominated by hard-leaved (sclerophyll) vegetation have
relatively high diversity and local endemism, notably among plants,
reptiles, and small mammals. Large moisture deficits and extremely low
levels of soil nutrients limit productivity, however, infrequent
episodes of high rainfall drive spikes of productivity and boom-bust
ecology. Spatial heterogeneity is also critical in sustaining diversity
by promoting niche diversity and resource-rich refuges during `bust'
intervals. Stress-tolerator and ruderal life-history types are strongly
represented in both flora and fauna. Perennial, long-lived,
slow-growing, drought-tolerant, sclerophyll shrubs and hummock (C4)
grasses structure the ecosystem by stabilising soils, acting as
nutrient-accumulation sites and providing continuously available
habitat, shade, and food for fauna. Strong filtering by both nutritional
poverty and water deficit promote distinctive scleromorphic and
xeromorphic plant traits. They include low SLA, high C:N ratios, reduced
foliage, stomatal regulation and encryption, slow growth and
reproduction rates, deep root systems, and trait plasticity. Perennial
succulents are absent. Episodic rains initiate emergence of a prominent
ephemeral flora, with summer and winter rains favouring grasses and
forbs, respectively. This productivity `boom' triggers rapid responses
by granivores and their predators. Herbivore populations also fluctuate
but less so due to ecophysiological traits that promote water
conservation. Abundant detritivores support a diverse and abundant
resident reptilian and small-mammal fauna. Small mammals and some
macro-invertebrates are nocturnal and fossorial, with digging activity
contributing to nutrient and carbon cycling, as well as plant
recruitment. The abundance and diversity of top predators is low.
Nomadism and ground-nesting are well represented in birds. Periodic
fires reduce biomass, promote recovery traits in plants
(e.g.~re-sprouting and fire-cued recruitment) and initiate successional
processes in both flora and fauna.

\subsection{Key Ecological Drivers}\label{key-ecological-drivers-90}

Resource availability is limited by a large overall water deficit
(rainfall \textless250 mm p.a., 5--50\% of potential evapotranspiration)
and acid sandy soils with very low P and N, together with high diurnal
and seasonal variation in temperatures. Summers have runs of extremely
hot days (\textgreater40°C) and winters have cool nights (0°C), rarely
with snow. Long dry spells are punctuated by infrequent inter-annual
pulses of water surplus, driving ecological booms and transient periods
of fuel continuity. Fires occur at decadal- or century-scale return
intervals when lightning or human ignitions coincide with fuel
continuity.

\subsection{Distribution}\label{distribution-90}

Mid-latitudes on sandy substrates of central and northwestern Australia.

\section{T5.4 Cool deserts and
semi-deserts}\label{t5.4-cool-deserts-and-semi-deserts}

Belongs to biome T5. Deserts and semi-deserts biome, part of the
Terrestrial realm.

\subsection{Short description}\label{short-description-91}

Cool deserts and semi-deserts occur on cool temperate plains and
plateaus in central Eurasia and temperate parts of the Americas from sea
level up to 4,000 m. Strong winds and freezing temperatures prevail,
with low annual precipitation falling as winter snow or sleet.
Productivity is low on infertile sandy and clay soils, often with high
salinity. Vegetation comprises a sparse cover of low grasses and dwarf
shrubs, interspersed with bare patches, with some areas having only
lichens and mosses or no vegetation at all. Fauna includes large nomadic
herbivores including antelopes, wild horses and camels, which control
composition of vegetation. Predators include raptors, snakes, bears, and
cats.

\subsection{Key Features}\label{key-features-91}

Xeromorphic suffratescent or non-sclerophyll shrublands or grasslands;
freezing temperatures in winter low, rainfall offset by reduced
evapotranspiration burdon; low diversity and endemism.

\subsection{Ecological traits}\label{ecological-traits-91}

In these arid systems, productivity is limited by both low precipitation
and cold temperatures but varies spatially in response to soil texture,
salinity, and water table depth. Vegetation cover varies with soil
conditions from near zero (on extensive areas of heavily salinized soils
or mobile dunes) to \textgreater50\% in upland grasslands and
shrublands, but is generally low in stature (\textless1 m tall). The
dominant plants are perennial C3 grasses and xeromorphic suffrutescent
or non-sclerophyllous perennial shrubs. Dwarf shrubs, tending to
prostrate or cushion forms occur in areas exposed to strong, cold winds.
Plant growth occurs mainly during warming spring temperatures after
winter soil moisture recharges. Eurasian winter annuals grow rapidly in
this period after developing extensive root systems over winter.
Diversity and local endemism are low across all taxa relative to other
arid ecosystems. Trophic networks are characterised by large nomadic
mammalian herbivores. Vertebrate herbivores including antelopes,
equines, camelids, and lagomorphs are important mediators of shrub-grass
dynamics, with heavy grazing promoting replacement of grasses by
N-fixing shrubs. Grasses become dominant with increasing soil fertility
or moisture but may be replaced by shrubs as grazing pressure increases.
Fossorial lagomorphs and omnivorous rodents contribute to soil
perturbation. Predator populations are sparse but taxonomically diverse.
They include raptors, snakes, bears, and cats. Bio-crusts with
cyanobacteria, mosses, and lichens are prominent on fine-textured
substrates and become dominant where it is too cold for vascular plants.
They play critical roles in soil stability and water and nutrient
availability.

\subsection{Key Ecological Drivers}\label{key-ecological-drivers-91}

Mean annual precipitation is similar to most warm deserts (\textless250
mm) due to rain shadows and continentality, however, in cool deserts
this falls mainly as snow or sleet in winter rather than rain.
Evapotranspiration is less severe than in hot deserts, but a substantial
water deficit exists due to low precipitation (mostly 10--50\% of
evapotranspiration) and strong desiccating winds that may occasionally
propagate fires. Mean monthly temperatures may fall below −20°C in
winter (freezing the soil surface) and exceed 15°C in summer. Substrates
vary from stony plains and uplands to extensive dune fields, with
mosaics of clay and sandy regolith underpinning landscape-scale
heterogeneity. Large regions were submerged below seas or lakes in past
geological eras with internal drainage systems leaving significant
legacies of salinity in some lowland areas, especially in clay
substrates.

\subsection{Distribution}\label{distribution-91}

Cool temperate plains and plateaus from sea level to 4,000 m elevation
in central Eurasia, western North America, and Patagonia. Extreme cold
deserts are placed in the polar/alpine biome.

\section{T5.5 Hyper-arid deserts}\label{t5.5-hyper-arid-deserts}

Belongs to biome T5. Deserts and semi-deserts biome, part of the
Terrestrial realm.

\subsection{Short description}\label{short-description-92}

The most extreme of all desert ecosystems, hyper-arid deserts are
limited by very dry and often windy conditions with high temperatures
and sandy or stony soils. Dry periods may be prolonged for several
years. Vegetation is characterised by very low densities of small
drought-tolerant perennial plants known as xerophytes, very slow growing
species with adaptations to drought such as extensive root systems and
water storage tissues. Some of these plants may acquire much of their
moisture from fogs. Ephemeral plants occur in some regions, but are less
common than in other desert systems. Microbial biofilms are important
decomposers. Drought tolerant reptiles and invertebrates are the main
faunal groups, along with occasional nomadic mammals and birds, in
simple foodwebs.

\subsection{Key Features}\label{key-features-92}

Very sparsely vegetated ecosystems in areas with very low or no
precipitation; very low productivity and simple trophic structures; low
diversitybut high endemism.

\subsection{Ecological traits}\label{ecological-traits-92}

Hyper-arid deserts show extremely low productivity and biomass and are
limited by low precipitation and extreme temperatures. Vegetation cover
is very sparse (\textless1\%) and low in stature (typically a few
centimetres tall), but productivity and biomass may be marginally
greater in topographically complex landscapes within patches of rising
ground-water or where runoff accumulates or cloud cover intersects.
Trophic networks are simple because autochthonous productivity and
allochthonous resources are very limited. Rates of decomposition are
slow and driven by microbial activity and UV-B photodegradation, both of
which decline with precipitation. Microbial biofilms play important
decomposition roles in soils and contain virus lineages that are
putatively distinct from other ecosystems. Although diversity is low,
endemism may be high because of strong selection pressures and
insularity resulting from the large extent of these arid regions and
limited dispersal abilities of most organisms. Low densities of
drought-tolerant perennial plants (xerophytes) characterise these
systems. The few perennials present have very slow growth and tissue
turnover rates, low fecundity, generally long life spans, and water
acquisition and conservation traits (e.g.~extensive root systems, thick
cuticles, stomatal regulation, and succulent organs). Ephemeral plants
with long-lived soil seed banks are well represented in hyper-arid
deserts characterised by episodic rainfall, but they are less common in
those that are largely reliant on fog or groundwater. Fauna include both
ruderal and drought-tolerant species. Thermoregulation is strongly
represented in reptiles and invertebrates. Birds and large mammals are
sparse and nomadic, except in areas with reliable standing water.
Herbivores and granivores have boom-bust population dynamics coincident
with episodic rains.

\subsection{Key Ecological Drivers}\label{key-ecological-drivers-92}

Extreme rainfall deficit arising from very low rainfall (150 mm to
almost zero and \textless5\% of potential evapotranspiration),
exacerbated by extremely hot temperatures and desiccating winds.
Principal sources of moisture may include moisture-laden fog, irregular
inter-annual or decadal rainfall events, and capillary rise from deep
water tables. UV-B radiation is extreme except where moderated by fogs.
Temperatures exhibit high diurnal and seasonal variability with extreme
summer maxima and sub-zero winter night temperatures. Hyper-arid deserts
occur on extensive low-relief plains (peneplains) and mountainous
terrain. Substrates may be extensive sheets of unstable, shifting sand
or stony gibber with no soil profile development and low levels of
nutrients.

\subsection{Distribution}\label{distribution-92}

Driest parts of the Sahara-Arabian, Atacama, and Namib deserts in
subtropical latitudes.

\section{T6.1 Ice sheets, glaciers and perennial
snowfields}\label{t6.1-ice-sheets-glaciers-and-perennial-snowfields}

Belongs to biome T6. Polar/alpine (cryogenic) biome, part of the
Terrestrial realm.

\subsection{Short description}\label{short-description-93}

Found in polar regions and on high mountains, ice sheets, glaciers and
perennial snowfields make up around 10\% of the earth's surface. They
have very low productivity and diversity in extreme cold conditions.
Nutrients are in short supply, and generally come from glacial debris,
seawater or guano. At the base of simple foodwebs, micro-organisms such
as bacteria, viruses and algae are the dominant life forms, although
itinerant vertebrates make important contributions to nutrient and
carbon subsidies. Micro-organisms are often dispersed by wind, and
accumulate organic matter at the surface, fuelling microbial activity
both at the surface, and below the ice. Productivity is restricted to
summer months, when migratory birds and mammals visit.

\subsection{Key Features}\label{key-features-93}

Permanent, dynamic ice cover where extreme cold limits productivity and
diversity, biota dominated by microorganisms, migratory/overwintering
birds may occur.

\subsection{Ecological traits}\label{ecological-traits-93}

In these icy systems, extreme cold and periodic blizzards limit
productivity and diversity to very low levels, and trophic networks are
truncated. Wherever surface or interstitial water is available, life is
dominated by micro-organisms including viruses, bacteria, protozoa, and
algae, which may arrive by Aeolian processes. Bacterial densities vary
from 107 to 1011 cells.L-1. On the surface, the main primary producers
are snow (mainly Chlamydomonadales) and ice algae (mainly Zygnematales)
with contrasting traits. Metabolic activity is generally restricted to
summer months at temperatures close to zero and is enabled by
exopolymeric substances, cold-adapted enzymes, cold-shock proteins, and
other physiological traits. N-fixing cyanobacteria are critical in the
N-cycle, especially in late summer. Surface heterogeneity and dynamism
create cryoconite holes, rich oases for microbial life (especially
cyanobacteria, prokaryotic heterotrophs and viruses) and active
biogeochemical cycling. Most vertebrates are migratory birds with only
the emperor penguin over-wintering on Antarctic ice. Mass movement and
snow burial also places severe constraints on establishment and
persistence of life. Snow and ice algae and cyanobacteria on the surface
are ecosystem engineers. Their accumulation of organic matter leads to
positive feedbacks between melting and microbial activity that
discolours snow and reduces albedo. Organic matter produced at the
surface can also be transported through the ice to dark subglacial
environments, fuelling microbial processes involving heterotrophic and
chemoautotrophic prokaryotes and fungi.

\subsection{Key Ecological Drivers}\label{key-ecological-drivers-93}

Permanent but dynamic ice cover accumulates by periodic snow fall and is
reduced in summer by melting, sublimation, and calving (i.e.~blocks of
ice breaking free) in the ablation zone. Slow lateral movement occurs
downslope or outwards from ice cap centres with associated cracking.
Precipitation may average several metres per year on montane glaciers or
less than a few hundred millimetres on extensive ice sheets. Surface
temperatures are extremely cold in winter (commonly −60°C in Antarctica)
but may rise above 0°C in summer. Desiccating conditions occur during
high winds or when water is present almost entirely in solid form.
Nutrients, especially N and P, are extremely scarce, the main inputs
being glacial moraines, aerosols, and seawater (in sea ice), which may
be supplemented locally by guano. Below the ice, temperatures are less
extreme, there is greater contact between ice, water, and rock
(enhancing nutrient supply), a diminished light intensity, and redox
potential tends towards anoxic conditions, depending on hydraulic
residence times.

\subsection{Distribution}\label{distribution-93}

Polar regions and high mountains in the western Americas, central Asia,
Europe, and New Zealand, covering \textasciitilde10\% of the earth's
surface.

\section{T6.2 Polar/alpine cliffs, screes, outcrops and lava
flows}\label{t6.2-polaralpine-cliffs-screes-outcrops-and-lava-flows}

Belongs to biome T6. Polar/alpine (cryogenic) biome, part of the
Terrestrial realm.

\subsection{Short description}\label{short-description-94}

Polar-alpine rocky outcrops occur in permanently ice-free areas of polar
regions and high mountains. Productivity and biomass are limited by
extreme cold, rocky substrate and strong winds. Algae, lichens, mosses
and bacteria support a short and simple foodweb in the summer months,
with cold-tolerant invertebrates such as tardigrades. Some rocky sites
provide nesting sites for birds in summer. Substrate weathering and
guano are major nutrient inputs. These systems are periodically
disturbed as accumulated snow and ice collapses down steep slopes.

\subsection{Key Features}\label{key-features-94}

Environments free of permanent ice where extreme cold, winds, skeletal
substrates and periodic mass movement limit biota to cryptogams,
invertebrates and microorganisms, nesting birds may occur..

\subsection{Ecological traits}\label{ecological-traits-94}

Low biomass systems with very low productivity constrained by extreme
cold, desiccating winds, skeletal substrates, periodic mass movement,
and, in polar regions, by seasonally low light intensity. The dominant
lifeforms are freeze-tolerant crustose lichens, mosses, and algae that
also tolerate periodic desiccation, invertebrates such as tardigrades,
nematodes, and mites, micro-organisms including bacteria and protozoa,
and nesting birds that forage primarily in other (mostly marine)
ecosystems. Diversity and endemism are low, likely due to intense
selection pressures and wide dispersal. Trophic networks are simple and
truncated. Physiological traits such as cold-adapted enzymes and
cold-shock proteins enable metabolic activity, which is restricted to
summer months when temperatures are close to or above zero. Nutrient
input occurs primarily through substrate weathering supplemented by
guano, which along with cyanobacteria is a major source of N. Mass
movement of snow and rock, with accumulation of snow and ice during the
intervals between collapse events, promotes disequilibrium ecosystem
dynamics.

\subsection{Key Ecological Drivers}\label{key-ecological-drivers-94}

Extremely cold winters with wind-chill that may reduce temperatures
below −80°C in Antarctica. In contrast, insolation and heat absorption
on rocky substrates may increase summer temperatures well above 0°C.
Together with the impermeable substrate and intermittently high winds,
exposure to summer insolation may produce periods of extreme water
deficit punctuated by saturated conditions associated with meltwater and
seepage. Periodic burial by snow reduces light availability, while mass
movement through landslides, avalanches, or volcanic eruptions maintain
substrate instability and destroy biomass, limiting the persistence of
biota.

\subsection{Distribution}\label{distribution-94}

Permanently ice-free areas of Antarctica, Greenland, the Arctic Circle,
and high mountains in the western Americas, central Asia, Europe,
Africa, and New Zealand.

\section{T6.3 Polar tundra and
deserts}\label{t6.3-polar-tundra-and-deserts}

Belongs to biome T6. Polar/alpine (cryogenic) biome, part of the
Terrestrial realm.

\subsection{Short description}\label{short-description-95}

Polar tundra and deserts have continuous to sparse cover of
cold-tolerant mosses, liverworts, lichens, grasses, low shrubs and other
flowering plants. They occur primarily in the Arctic circle, but polar
desert is found in dry coastal lowlands of Antarctica. Precipitation
falls as snow, with seasonal snow cover limiting the growing season.
Extreme cold temperatures and short growing seasons exclude trees, as
well as vascular plants in the coldest and driest locations. Permafrost
substrates accumulate peat through slow decomposition rates. Migratory
birds feed in distant wetlands or open oceans, and contribute nutrients
to the system through guano, as well as dispersing seeds and other
organisms. Migratory or hibernating mammals include seals, and, in the
north, polar bears, foxes and wolves.

\subsection{Key Features}\label{key-features-95}

Open and low vegetation of herbaceous plants (e.g.~tussocks, cushions,~
rosette plants) and abundant kryptogams in very cold climates with
permafrost.

\subsection{Ecological traits}\label{ecological-traits-95}

These low productivity autotrophic ecosystems are limited by winter
dormancy during deep winter snow cover, extreme cold temperatures and
frost during spring thaw, short growing seasons, desiccating winds, and
seasonally low light intensity. Microbial decomposition rates are slow,
promoting accumulation of peaty permafrost substrates in which only the
surface horizon thaws seasonally. Vegetation is treeless and dominated
by a largely continuous cover of cold-tolerant bryophytes, lichens, C3
grasses, sedges, forbs, and dwarf and prostrate shrubs. Tundra around
the world, is delimited by the physiological temperature limits of
trees, which are excluded where the growing season (i.e.~days
\textgreater0.9°C) is less than 90-94 days duration, with mean
temperatures less than 6.5°C across the growing season. In the coldest
and/or driest locations, vascular plants are absent and productivity
relies on bryophytes, lichens, cyanobacteria, and allochthonous energy
sources such as guano. Aestivating insects (i.e.~those that lay dormant
in hot or dry seasons) dominate the invertebrate fauna. Vertebrate fauna
is dominated by migratory birds, some of which travel seasonal routes
exceeding several thousand kilometres. Many of these feed in distant
wetlands or open oceans. These are critical mobile links that transfer
nutrients and organic matter and disperse the propagules of other
organisms, both externally on plumage or feet and endogenously. A few
mammals in the Northern Hemisphere are hibernating residents or
migratory herbivores. Pinnipeds occur in near-coast tundras and may be
locally important marine subsidies of nutrients and energy. Predatory
canids and polar bears are nomadic or have large home ranges.

\subsection{Key Ecological Drivers}\label{key-ecological-drivers-95}

Winters are very cold and dark and summers define short, cool growing
seasons with long hours of low daylight. Precipitation falls as snow
that persists through winter months. In most areas, there is an overall
water surplus, occasionally with small summer deficit, but some areas
are ice-free, extremely dry (annual precipitation \textless150mm p.a.)
polar deserts with desiccating winds. Substrates are peaty or gravelly
permafrost, which may partially thaw on the surface in summer, causing
cryoturbation.

\subsection{Distribution}\label{distribution-95}

Primarily within the Arctic Circle and adjacent subarctic regions, with
smaller occurrences on subantarctic islands and the Antarctic coast.

\section{T6.4 Temperate alpine grasslands and
shrublands}\label{t6.4-temperate-alpine-grasslands-and-shrublands}

Belongs to biome T6. Polar/alpine (cryogenic) biome, part of the
Terrestrial realm.

\subsection{Short description}\label{short-description-96}

Temperate alpine grasslands and shrublands occur above the treeline, on
temperate and boreal mountains worldwide. Seasonal productivity is
limited by cold and snow cover, often with strong winds. Mosses,
liverworts, lichens, flowering plants and low shrubs generally form a
continuous cover, except where strong winds and dry conditions limit
vegetation to sparse lichens and dwarf shrubs. Most fauna is seasonally
active during warmer summer months, with insects and vertebrates having
adaptations to extreme cold, including hibernation. Many species have
restricted distributions, with strong barriers to dispersal between
mountains.

\subsection{Key Features}\label{key-features-96}

Mountain systems above the physiological limits of trees, with sparse to
continuous cover of herbaceous plants, cryptogams and dwarf shrubs that
may be morphologically adaptated to extreme cold..

\subsection{Ecological traits}\label{ecological-traits-96}

Mountain systems beyond the cold climatic treeline are dominated by
grasses, herbs, or low shrubs (typically \textless1 m tall).
Moderate-low and strictly seasonal productivity is limited by deep
winter snow cover, extreme cold and frost during spring thaw, short
growing seasons, desiccating winds, and, in some cases, by mass
movement. Vegetation comprises a typically continuous cover of plants
including bryophytes, lichens, C3 grasses, sedges, forbs, and dwarf
shrubs including cushion growth forms. However, the cover of vascular
plants may be much lower in low-rainfall regions or in sites exposed to
strong desiccating winds and often characterised by dwarf shrubs and
lichens that grow on rocks (e.g.~fjaeldmark). Throughout the world,
alpine ecosystems are defined by the physiological temperature limits of
trees, which are excluded where the growing season (i.e.~days
\textgreater0.9°C) is less than 90-94 days, with mean temperatures less
than 6.5°C across the growing season. Other plants have morphological
and ecophysiological traits to protect buds, leaves, and reproductive
tissues from extreme cold, including growth forms with many branches,
diminutive leaf sizes, sclerophylly, vegetative propagation, and
cold-stratification dormancy. The vertebrate fauna includes a few
hibernating residents and migratory herbivores and predators that are
nomadic or have large home ranges. Aestivating insects include katydids,
dipterans, and hemipterans. Local endemism and beta-diversity may be
high due to steep elevational gradients, microhabitat heterogeneity, and
topographic barriers to dispersal between mountain ranges, with evidence
of both facilitation and competition.

\subsection{Key Ecological Drivers}\label{key-ecological-drivers-96}

Winters are long and cold, while summers are short and mild. Seasonal
snow up to several metres deep provides insulation to over-wintering
plants and animals. Severe frosts and desiccating winds characterise the
spring thaw and exposed ridges and slopes. Severe storms may result from
orographic-atmospheric instability. Typically there is a large
precipitation surplus, but deficits occur in some regions. Steep
elevational gradients and variation in micro-topography and aspect
promote microclimatic heterogeneity. Steep slopes are subjected to
periodic mass movements, which destroy surface vegetation.

\subsection{Distribution}\label{distribution-96}

Mountains in the temperate and boreal zones of the Americas, Europe,
central Eurasia, west and north Asia, Australia, and New Zealand.

\section{T6.5 Tropical alpine grasslands and
herbfields}\label{t6.5-tropical-alpine-grasslands-and-herbfields}

Belongs to biome T6. Polar/alpine (cryogenic) biome, part of the
Terrestrial realm.

\subsection{Short description}\label{short-description-97}

Tropical alpine grasslands and herbfields are limited to a few
mountainous areas of Africa, the Americas and southeast Asia.
Productivity is low, limited by dry conditions, rocky substrate and
nightly cold, but is not strongly seasonal. Snow and fog are common.
Typical flora includes mosses, lichens, and flowering plants including
distinctive megaherbs and low shrubs. Plant species have adaptations for
cold dry conditions, such as tiny leaves, cushion and rosette growth
forms and slow growth. Plant composition is affected by competition and
facilitation between species, as well as grazing and occasional fires.
Simple foodwebs include invertebrates, small mammals and reptiles, along
with visiting predators and occasional herbivores from lowland savannas.

\subsection{Key Features}\label{key-features-97}

Dense perennial C3 cold tolerant tussock grasslands, with distinctive
arborescent rosette and cushion growth forms, treeless except for
sheltered gullies..

\subsection{Ecological traits}\label{ecological-traits-97}

Treeless mountain systems dominated by an open to dense cover of
cold-tolerant C3 perennial tussock grasses, herbs, small shrubs, and
distinctive arborescent rosette or cushion growth forms. Lichens and
bryophytes are also common. Productivity is low, dependent on
autochthonous energy, and limited by cold temperatures, diurnal
freeze-thaw cycles, and desiccating conditions, but not by a short
growing season (as in T6.4). Elfin forms of tropical montane forests
(T1.3) occupy sheltered gullies and lower elevations. Diversity is low
to moderate but endemism is high among some taxa, reflecting steep
elevational gradients, microhabitat heterogeneity, and topographic
insularity, which restricts dispersal. Solifluction (i.e.~the slow flow
of saturated soil downslope) restricts seedling establishment to stable
microsites. Plants have traits to protect buds, leaves, and reproductive
tissues from diurnal cold and transient desiccation stress, including
ramulose (i.e.~many-branched), cushion, and rosette growth forms,
insulation from marcescent (i.e.~dead) leaves or pectin fluids,
diminutive leaf sizes, leaf pubescence, water storage in stem-pith, and
vegetative propagation. Most plants are long-lived and some rosette
forms are semelparous. Cuticle and epidermal layers reduce UV-B
transmission to photosynthetic tissues. Plant coexistence is mediated by
competition, facilitation, herbivory (vertebrate and invertebrate), and
fire regimes. Simple trophic networks include itinerant large herbivores
and predators from adjacent lowland savannas as well as resident
reptiles, small mammals, and macro-invertebrates.

\subsection{Key Ecological Drivers}\label{key-ecological-drivers-97}

Cold nights (as low as −10°C) and mild days (up to 15°C) produce low
mean temperatures and diurnal freeze-thaw cycles, but seasonal
temperature range is small and freezing temperatures are short-lived.
Cloud cover and precipitation are unseasonal in equatorial latitudes or
seasonal in the monsoonal tropics. Strong orographic effects result in
an overall precipitation surplus and snow and fog are common, but
desiccating conditions may occur during intervals between precipitation
events, with morning insolation also increasing moisture stress when
roots are cold. Exposure to UV-B radiation is very high. Substrates are
typically rocky and shallow (with low moisture retention capacity) and
exposed to solifluction. Micro-topographic heterogeneity influences
fine-scale spatial variation in moisture availability. Steep slopes are
subjected to periodic mass movements, which destroy surface vegetation.
Low-intensity fires may be ignited by lightning or spread upslope from
lowland savannas, but these occur infrequently at multi-decadal
intervals.

\subsection{Distribution}\label{distribution-97}

Restricted mountainous areas of tropical Central and South America, East
and West Africa, and Southeast Asia.

\section{T7.1 Annual croplands}\label{t7.1-annual-croplands}

Belongs to biome T7. Intensive land-use biome, part of the Terrestrial
realm.

\subsection{Short description}\label{short-description-98}

Croplands are intensively managed agricultural ecosystems maintained by
supplementation of nutrients and water, sowing and harvesting, soil
cultivation and control of non-target plants and animals (weeds and
pests). They currently cover 11\% of the world's land surface. Some
systems of management include domestic herbivores introduced on
harvested stubble or in `fallow' years. These are structurally simple,
very low-diversity, high-productivity systems, dominated by one or few
non-woody, shallow-rooted annual plant species such as grains (mostly C3
grasses), vegetables, `cut flowers', legumes, or fibre plants harvested
annually by humans for commercial or subsistence production of food,
materials, or ornamental displays.

\subsection{Key Features}\label{key-features-98}

Structurally simple, very low- diversity, high-productivity annual
croplands are maintained by the intensive anthropogenic supplementation
of nutrients, water and artificial disturbance regimes.

\subsection{Ecological traits}\label{ecological-traits-98}

High-productivity croplands are maintained by the intensive
anthropogenic supplementation of nutrients, water, and artificial
disturbance regimes (e.g.~annual cultivation), translocation
(e.g.~sowing), and harvesting of annual plants. These systems are
typically dominated by one or few shallow-rooted short-lived plant
species such as grains (mostly C3 grasses), vegetables, `flowers',
legumes, or fibre species harvested annually by humans for the
commercial or subsistence production of food, materials, or ornamental
displays. Disequilibrium community structure and composition is
maintained by translocations and/or managed reproduction of target
species and usually by periodic application of herbicides and pesticides
and/or culling to exclude competitors, predators, herbivores, and/or
pathogens. Consequently, compared to antecedent `natural' systems,
croplands are structurally simple, have low functional, genetic, and
taxonomic diversity and no local endemism. Subsistence croplands,
including Swidden rotation systems, are typically more diverse than
industrial croplands. Productivity is highly sensitive to variations in
resource availability. Target biota are genetically manipulated by
selective breeding or molecular engineering to promote rapid growth
rates, efficient resource capture, enhanced resource allocation to
production tissues, and tolerance to harsh environmental conditions,
insect predators, and diseases. Typically, at least 40\% of net primary
productivity is appropriated by humans. Croplands may be rotated
inter-annually with livestock pastures or fallow fields (T7.2) or may be
integrated into mixed cropping-livestock systems. Target biota coexists
with a cosmopolitan ruderal biota (e.g.~weedy plants, mice, and
starlings) that exploits production landscapes opportunistically through
efficient dispersal, itinerant foraging, rapid establishment, high
fecundity, and rapid population turnover. Native biota from adjoining
non-anthropogenic systems may also interact with croplands. When
actively managed systems are abandoned or managed less intensively,
these non-target biota, especially non-woody plants, become dominant and
may form a steady, self-maintaining state or a transitional phase to
novel ecosystems.

\subsection{Key Ecological Drivers}\label{key-ecological-drivers-98}

The high to moderate natural availability of water (from at least
seasonally high rainfall) and nutrients (from fertile soils) is often
supplemented by human inputs via irrigation, landscape drainage
modifications (e.g.~surface earthworks), and/or fertiliser application
by humans. Intermittent flooding may occur where croplands replace
palustrine wetlands. Temperatures are mild to warm, at least seasonally.
These systems are typically associated with flat to moderate terrain
accessible by machinery. Artificial disturbance regimes (e.g.~annual
ploughing) maintain soil turnover, aeration, nutrient release, and
relatively low soil organic carbon content.

\subsection{Distribution}\label{distribution-98}

Tropical to temperate humid climatic zones or river flats in dry
climates across south sub-Saharan and North Africa, Europe, Asia,
southern Australia, Oceania, and the Americas.

\section{T7.2 Sown pastures and
fields}\label{t7.2-sown-pastures-and-fields}

Belongs to biome T7. Intensive land-use biome, part of the Terrestrial
realm.

\subsection{Short description}\label{short-description-99}

In these intensively managed agricultural systems, grasses and legumes
are sown and cultivated, with regular inputs of nutrients and
(sometimes) water, primary for the mostly commercial production of
livestock or food (hay) for livestock. Sown pastures are structurally
simple ecosystems with low-diversity and high-productivity. They are
dominated by one or few selected plant species as primary food sources
for one animal species (usually large mammalian herbivores). Management
includes chemical or physical treatments to exclude competitors,
predators, herbivores, or pathogens. They differ from less intensively
managed rangeland (e.g.~biome T5) and semi-natural grasslands (T7.5),
where livestock graze in predominantly native ecosystems.

\subsection{Key Features}\label{key-features-99}

Structurally simple, very low- diversity, high-productivity grasslands
dominated by one or few species of perennial grasses (Poacaeae)
maintained by intensive addition of nutrients, water and artificial
disturbance regimes (mowing or grazing).

\subsection{Ecological traits}\label{ecological-traits-99}

Structurally simple, high-productivity pastures are maintained by the
intensive anthropogenic supplementation of nutrients (more rarely water)
and artificial disturbance regimes (e.g.~periodic ploughing,),
translocation (e.g.~livestock movement and sowing), and harvesting of
animals or plants. The magnitude of these inputs distinguish these
systems from semi-natural pastures and rangelands in biomes biome T4 and
biome T5 used for less intense livestock production. They are dominated
by one or few selected plant species (C3 and C4 perennial pasture
grasses and/or herbaceous legumes) and animal species (usually large
mammalian herbivores) for commercial production of food or materials,
ornamental displays, or sometimes subsistence. Their composition and
structure is maintained by the translocation and/or managed reproduction
of target species and the periodic application of herbicides and
pesticides and/or culling to exclude competitors, predators, herbivores,
or pathogens. Consequently, compared to `natural' rangeland systems and
semi-natural pastures, these systems have low functional and taxonomic
diversity and little or no local endemism. Target biota are genetically
manipulated to promote rapid growth rates, efficient resource capture,
enhanced resource allocation to production tissues, and tolerance to
harsh environmental conditions, diseases, and predators, . They are
harvested by humans continuously or periodically for consumption or
maintenance. Typically, at least 40\% of net primary productivity is
appropriated by humans. Major examples include intensively managed
production pastures for livestock or forage (e.g.~hay). Livestock
pastures may be rotated inter-annually with non-woody crops (T7.1), or
they may be managed as mixed silvo-pastoral systems (T7.3). Target biota
coexist with native and cosmopolitan ruderal biota that exploits
production landscapes through efficient dispersal, rapid establishment,
high fecundity, and rapid population turnover. When the ecosystem is
abandoned or managed less intensively, non-target biota become dominant
and may form a steady, self-maintaining state or a transitional phase to
novel ecosystems.

\subsection{Key Ecological Drivers}\label{key-ecological-drivers-99}

High to moderate natural availability of water and nutrients is
typically supplemented by human inputs via water management, landscape
drainage modifications (e.g.~surface earthworks), and/or fertiliser
application at varied rates. Intermittent flooding may occur where
pastures replace palustrine wetlands. Temperatures are mild to warm, at
least seasonally. Typically associated with moderately fertile
substrates and flat to undulating terrain accessible by machinery.
Artificial disturbance regimes (e.g.~ploughing for up to 5 years/decade)
maintain soil turnover, aeration, and nutrient release.

\subsection{Distribution}\label{distribution-99}

Mostly in tropical to temperate climatic zones and developed countries
across Europe, east and south Asia, subtropical and temperate Africa,
southern Australasia, north and central America, and temperate south
America. See map caveats (Table S4.1)

\section{T7.3 Plantations}\label{t7.3-plantations}

Belongs to biome T7. Intensive land-use biome, part of the Terrestrial
realm.

\subsection{Short description}\label{short-description-100}

Plantations are generally long-rotation perennial woody crops
established and maintained for a variety of food and materials. The
harvested products include wood, various fruits, tea, coffee, palm oil
and other food additives, materials such as rubber, ornamental materials
(cut flowers), etc. The vegetation of most plantations comprises at
least two vertical strata (the managed woody species and a ruderal
ground layer), although mixed plantings may be more complex and host a
relatively diverse flora and fauna if managed to promote habitat
features. Fertilisers and water subsidies are applied, and harvesting
occurs at intervals depending on the crop.

\subsection{Key Features}\label{key-features-100}

Structurally simple, low-diversity forests of one (rarely, a few)
planted tree species of mostly same age, lack of structural elements of
old-growth forests such as deadwood or cavities.

\subsection{Ecological traits}\label{ecological-traits-100}

These moderate to high productivity autotrophic systems are established
by the translocation (i.e.~planting or seeding) of woody perennial
plants. Target biota may be genetically manipulated by selective
breeding or molecular engineering to promote rapid growth rates,
efficient resource capture, enhanced resource allocation to production
tissues, and tolerance of harsh environmental conditions, insect
predators, and diseases. The diversity, structure, composition,
function, and successional trajectory of the ecosystem depends on the
identity, developmental stage, density, and traits (e.g.~phenology,
physiognomy, and growth rates) of planted species, as well as the
subsequent management of plantation development. Most plantations
comprise at least two vertical strata (the managed woody species and a
ruderal ground layer). Mixed forest plantings may be more complex and
host a relatively diverse flora and fauna if managed to promote habitat
features. Cyclical harvest may render the habitat periodically
unsuitable for some biota. Mixed cropping systems may comprise two
vertical strata of woody crops or a woody and herbaceous layer.
Secondary successional processes involve colonisation and regeneration,
initially of opportunistic biota. Successional feedbacks occur as
structural complexity increases, promoting visits or colonisation by
vertebrates and the associated dispersal of plants and other organisms.
Crop replacement (which may occur on inter-annual or decadal cycles),
the intensive management of plantation structure, or the control of
non-target species may reset, arrest, or redirect successional
processes. Examples with increasing management intervention include:
environmental plantations established for wildlife or ecosystem
services; agroforestry plantings for subsistence products or livestock
benefits; forestry plantations for timber, pulp, fibre, bio-energy,
rubber, or oils; and vineyards, orchards, and other perennial food crops
(e.g.~cassava, coffee, tea, palm oil, and nuts). Secondary (regrowth)
forests and shrublands are not included as plantations even where
management includes supplementary translocations.

\subsection{Key Ecological Drivers}\label{key-ecological-drivers-100}

High to moderate natural availability of water and nutrients is
supplemented by human inputs of fertiliser or mulch, landscape drainage
modifications (e.g.~surface earthworks), and, in intensively managed
systems, irrigation. Rainfall is at least seasonally high. Temperatures
are mild to warm, at least seasonally. Artificial disturbance regimes
involving the complete or partial removal of biomass and soil turnover
are implemented at sub-decadal to multi-decadal frequencies.

\subsection{Distribution}\label{distribution-100}

Tropical to cool temperate humid climatic zones or river flats in dry
climates across south sub-Saharan and Mediterranean Africa, Europe,
Asia, southern Australia, Oceania, and the Americas.

\section{T7.4 Urban and industrial
ecosystems}\label{t7.4-urban-and-industrial-ecosystems}

Belongs to biome T7. Intensive land-use biome, part of the Terrestrial
realm.

\subsection{Short description}\label{short-description-101}

Cities, smaller settlements and industrial areas are structurally
complex ecosystems and characterised by their highly dynamic spatial
structure. Diverse patch types include buildings, paved surfaces,
transport infrastructure, parks and gardens; excavations, bare ground
and refuse areas. Patches undergo periodic destruction and renewal.
Human population density is high, relative to other ecosystems, and
dependent on large subsidies of imported resources (particularly water,
nutrients and food). Interactions among patch types and human social
behaviours produce emergent properties and complex feedbacks among
ecosystem components.

\subsection{Key Features}\label{key-features-101}

Ecosystems dominated by anthroipogenic structures (e.g.~buildings,
roads, wastelands) associated with human infrastructures, intensive
anthropogenic disturbance regimes, and severely altered biogeochemical
site conditions.

\subsection{Ecological traits}\label{ecological-traits-101}

These systems are structurally complex and highly heterogeneous
fine-scale spatial mosaics of diverse patch types that may be recognised
in fine-scale land use classifications. These include: a) buildings; b)
paved surfaces; c) transport infrastructure: d) treed areas; e) grassed
areas; f) gardens; g) mines or quarries; h) bare ground; and i) refuse
areas. Patch mosaics are dynamic over decadal time scales and driven by
socio-ecological feedbacks and a human population that is highly
stratified, functionally, socially and economically. Interactions among
patch types and human social behaviours produce emergent properties and
complex feedbacks among components within each system and interactions
with other ecosystem types. Unlike most other terrestrial ecosystems,
the energy, water and nutrient sources of urban/industrial village
systems are highly allochthonous and processes within urban systems
drive profound and extensive global changes in land use, land cover,
biodiversity, hydrology, and climate through both resource consumption
and waste discharge. Biotic community structure is characterised by low
functional and taxonomic diversity, highly skewed rank-abundance
relationships and relict local endemism. Trophic networks are simplified
and sparse and each node is dominated by few taxa. Urban/village biota
include humans, dependents (e.g.~companion animals and cultivars),
opportunists and vagrants, and legacy biota whose establishment
pre-dates settlement. Many biota have highly plastic realised niches,
traits enabling wide dispersal, high fecundity, and short generation
times. The persistence of dependent biota is maintained by
human-assisted migration, managed reproduction, genetic manipulation,
amelioration of temperatures, and intensive supplementation of
nutrients, food, and water. Pest biota are controlled by the application
of herbicides and pesticides or culling with collateral impacts on
non-target biota.

\subsection{Key Ecological Drivers}\label{key-ecological-drivers-101}

Humans influence the availability of water, nutrients, and energy
through governance systems for resource importation and indirectly
through interactions and feedbacks. Light is enhanced artificially at
night. Urban temperature regimes are elevated by the anthropogenic
conversion of chemical energy to heat and the absorption of solar energy
by buildings and paved surfaces. However, temperatures may be locally
ameliorated within buildings. Surface water runoff is enhanced and
percolation is reduced by sealed surfaces. Chemical and particulate air
pollution, as well as light and noise pollution may affect biota.
Infrastructure development and renewal, driven by socio-economic
processes, as well as natural disasters (e.g.~storms, floods,
earthquakes, and tsunami) create recurring disturbances. There is
frequent movement of humans and associated biota and matter between
cities.

\subsection{Distribution}\label{distribution-101}

Extensively scattered through equatorial to subpolar latitudes from
sea-level to submontane altitudes, mostly in proximity to the coast,
rivers or lakes, especially in North America, Western Europe and Japan,
as well as India, China, and Brazil. Land use maps depict fine-scale
patch types listed above.

\section{T7.5 Derived semi-natural pastures and old
fields}\label{t7.5-derived-semi-natural-pastures-and-old-fields}

Belongs to biome T7. Intensive land-use biome, part of the Terrestrial
realm.

\subsection{Short description}\label{short-description-102}

These managed ecosystems are derived from a range of other ecosystems
(mostly from biome T1 - biome T4, a few from biome T5) by the removal or
modification of woody plant components. The remaining vegetation
includes both local indigenous species and introduced species, providing
habitat for a mixed indigenous and non-indigenous fauna. They are used
mainly for livestock grazing, which is essential to maintaining the
structure of the system. Unlike sown pastures, inputs of water and
nutrients are limited. Although structurally simpler than the systems
from which they were derived, they often harbour an appreciable
diversity of native organisms.

\subsection{Key Features}\label{key-features-102}

Extensively used, low-input grasslands (no or moderate fertilizer
application, no sowing), rich in vascular plant species.

\subsection{Ecological traits}\label{ecological-traits-102}

Extensive `semi-natural' grasslands and open shrublands exist where
woody components of vegetation have been removed or greatly modified for
agricultural land uses. Hence they have been `derived' from a range of
other ecosystems (mostly from biomes biome T1, biome T2, biome T3, biome
T4, a few from biome T5). Remaining vegetation includes a substantial
component of local indigenous species, as well as an introduced exotic
element, providing habitat for a mixed indigenous and non-indigenous
fauna. Although structurally simpler at site scales than the systems
from which they were derived, spatial complexity may be greater in
fragmented landscapes and they often harbour appreciable diversity of
native organisms, including some no longer present in `natural'
ecosystems. Dominant plant growth forms include tussock or stoloniferous
grasses and forbs, with or without non-vascular plants, shrubs and
scattered trees. These support microbial decomposers and diverse
invertebrate groups that function as detritivores, herbivores and
predators, as well as vertebrate herbivores and predators characteristic
of open habitats. Energy sources are primarily autochthonous, with
varying levels of indirect allochthonous subsidies (e.g.~via surface
water sheet flows), but few managed inputs (cf.~T7.2). Productivity can
be low or high, depending on climate and substrate, but is generally
lower and more stable than more intensive anthropogenic systems
(T7.1-T7.3). Trophic networks include all levels, but complexity and
diversity depends on the species pool, legacies from antecedent
ecosystems, successional stage, and management regimes. These novel
ecosystems may persist in a steady self-maintaining state, or undergo
passive transformation (e.g.~oldfield succession) unless actively
maintained in disequilibrium. For example, removal of domestic
herbivores may initiate transition to tree-dominated ecosystems.

\subsection{Key Ecological Drivers}\label{key-ecological-drivers-102}

Availability of water and nutrients varies depending on local climate,
substrate and terrain (hence surface water movement and infiltration).
The structure, function and composition of these ecosystems are shaped
by legacy features of antecedent systems from which they were derived,
as well as ongoing and past human activities. These activities may
reflect production and/or conservation goals, or abandonment. They
include active removal of woody vegetation, management of vertebrate
herbivores, introductions of biota, control of `pest' biota,
manipulation of disturbance regimes, drainage and earthworks, etc.
Fertilisers and pesticides are not commonly applied.

\subsection{Distribution}\label{distribution-102}

Mostly in temperate to tropical climates across all land masses. See map
caveats (Table S4.1).

\section{TF1.1 Tropical flooded forests and peat
forests}\label{tf1.1-tropical-flooded-forests-and-peat-forests}

Belongs to biome TF1. Palustrine wetlands biome, part of the
Terrestrial, Freshwater realm.

\subsection{Short description}\label{short-description-103}

These tropical swamps have closed forest canopies and experience high
rainfall and consistent temperatures all year. In some, peat accumulates
in anaerobic black water conditions, while others are highly productive
white-water systems, with frequent refilling and turnover of nutrients.
Trees and other plants, such as palms, pitcher plants, epiphytic mosses
and ferns grow in soils that are waterlogged or periodically inundated.

\subsection{Key Features}\label{key-features-103}

Evergreen closed-canopy forests in tropical swamps and riparian zones,
differing between high and low nutrients waters, and supporting complex
trophic networks.

\subsection{Ecological traits}\label{ecological-traits-103}

Closed-canopy forests in tropical swamps and riparian zones have high
biomass and LAI, with unseasonal growth and reproductive phenology. The
canopy foliage is evergreen, varying in size from mesophyll to notophyll
with moderate SLA. Productivity differs markedly between high-nutrient
`white water' riparian systems and low-nutrient `black water' systems.
In the latter, most of the nutrient capital is sequestered in plant
biomass, litter, or peat, whereas in white water systems, soil nutrients
are replenished continually by fluvial subsidies. Some trees have
specialised traits conferring tolerance to low-oxygen substrates, such
as surface root mats, pneumatophores, and stilt roots. Palms (sometimes
in pure stands), hydrophytes, pitcher plants, epiphytic mosses, and
ferns may be abundant, but lianas and grasses are rare or absent. The
recent origin of these forests has allowed limited time for evolutionary
divergence from nearby lowland rainforests (T1.1), but strong filtering
by saturated soils has resulted in low diversity and some endemism. The
biota is spatially structured by local hydrological gradients. Riparian
galleries of floodplain forests also occur within savanna matrices.
Trophic networks are complex but with less diverse representation of
vertebrate consumers and predators than T1.1, although avian frugivores,
primates, amphibians, macro-invertebrates, and crocodilian predators are
prominent. Plant propagules are dispersed mostly by surface water or
vertebrates. Seed dormancy and seedbanks are rare. Gap-phase dynamics
are driven by individual treefall, storm events, or floods in riparian
forests, but many plants exhibit leaf-form plasticity and can recruit in
the shade.

\subsection{Key Ecological Drivers}\label{key-ecological-drivers-103}

High rainfall, overbank flows or high water tables maintain an abundant
water supply. Continual soil profile saturation leads to anaerobic black
water conditions and peat accumulation. In contrast, white water
riparian zones undergo frequent fluvial disturbance and drain rapidly.
Peat forests often develop behind lake shore vegetation or mangroves,
which block lateral drainage. Black water peatlands may become domed,
ombrogenous (i.e.~rain-dependent), highly acidic, and nutrient-poor,
with peat accumulating to depths of 20 m. In contrast, white water
riparian forests are less permanently inundated and floods continually
replenish nutrients, disturb vegetation, and rework sediments.
Hummock-hollow micro-topography is characteristic of all forested
wetlands and contributes to niche diversity. Light may be limited by
dense tree canopies. There is low diurnal, intra- and inter-annual
variability in rainfall and temperature, with the latter rarely
\textless10°C, which promotes microbial activity when oxygen is
available.

\subsection{Distribution}\label{distribution-103}

Flat equatorial lowlands of Southeast Asia, South America, and Central
and West Africa, notably in Borneo and the Amazonian lowlands.

\section{TF1.2 Subtropical/temperate forested
wetlands}\label{tf1.2-subtropicaltemperate-forested-wetlands}

Belongs to biome TF1. Palustrine wetlands biome, part of the
Terrestrial, Freshwater realm.

\subsection{Short description}\label{short-description-104}

Forested wetlands in temperate and subtropical climates undergo periodic
flooding. One or two tree species dominate the canopy. Trees shape the
flow of flood waters, the ground surface, and the understorey, as well
as animal habitats. With flooding, complex aquatic food webs support
turtles, frogs, fish and birds, but can produce microbial blooms with
nutrients flushed from the floodplain. Many vertebrates use these
wetlands as refuges during dry times.

\subsection{Key Features}\label{key-features-104}

Permently to seasonally wet (or flooded), nutrient poor, to nutrient
rich, open to closed canopy forests, often on organic soils (peat); poor
in woody species, high abundance of mosses and sedges and no to open
woody species cover.

\subsection{Ecological traits}\label{ecological-traits-104}

These hydrophilic forests and thickets have an open to closed tree or
shrub canopy, 2--40 m tall, dependent on flood regimes or groundwater
lenses. Unlike tropical forests (TF1.1), they typically are dominated by
one or very few woody species. Trees engineer fine-scale spatial
heterogeneity in resource availability (water, nutrients, and light) and
ecosystem structure, which affects the composition, form, and functional
traits of understorey plants and fauna. Engineering processes include
the alteration of sediments, (e.g.~surface micro-topography by the
growth of large roots), the deposition of leaf litter and woody debris,
canopy shading, creation of desiccation refuges for fauna and the
development of foraging or nesting substrates (e.g.~tree hollows).
Forest understories vary from diverse herbaceous assemblages to simple
aquatic macrophyte communities in response to spatial and temporal
hydrological gradients, which influence the density and relative
abundance of algae, hydrophytes and dryland plants. Primary production
varies seasonally and inter-annually and can be periodically high due to
the mobilisation of nutrients on floodplains during inundation.
Nutrients accumulate on floodplains during low flows, and may drive
microbial blooms, leading to aquatic anoxia, and fish kills, which may
be extensive when flushing occurs. Plant and animal life histories are
closely connected to inundation (e.g.~seed-fall, germination
fish-spawning and bird breeding are stimulated by flooding).
Inundation-phase aquatic food webs are moderately complex. Turtles,
frogs, birds and sometimes fish exploit the alternation between aquatic
and terrestrial phases. Waterbirds forage extensively on secondary
production, stranded as floodplains recede, and breed in the canopies of
trees or mid-storey. Forested wetlands are refuges for many vertebrates
during droughts. Itinerant mammalian herbivores (e.g.~deer and
kangaroos) may have locally important impacts on vegetation structure
and recruitment.

\subsection{Key Ecological Drivers}\label{key-ecological-drivers-104}

These forests occur on floodplains, riparian corridors, and disconnected
lowland flats. Seasonally and inter-annually variable water supply
influences ecosystem dynamics. Allochthonous water and nutrient
subsidies from upstream catchments supplement local resources and
promote the extension of floodplain forests and their biota into arid
regions (`green tongues'). Water movement is critical for the
connectivity and movement of biota, while some groundwater-dependent
forests are disconnected. High-energy floods in riparian corridors
displace standing vegetation and woody debris, redistribute nutrients,
and create opportunities for dispersal and recruitment. Low-energy
environments with slow drainage promote peat accumulation. Extreme
drying and heat events may generate episodes of tree dieback and
mortality. Fires may occur depending on the frequency of fire weather,
ignition sources, and landscape context.

\subsection{Distribution}\label{distribution-104}

Temperate and subtropical floodplains, riparian zones and lowland flats
worldwide.

\section{TF1.3 Permanent marshes}\label{tf1.3-permanent-marshes}

Belongs to biome TF1. Palustrine wetlands biome, part of the
Terrestrial, Freshwater realm.

\subsection{Short description}\label{short-description-105}

Permanent marshes occur throughout tropical and temperate regions of the
world in flat areas with stable water levels close to the surface. They
are essentially treeless, with extensive reedbeds and aquatic grasses,
interspersed with patches of open water. Food webs are strongly
influenced by highly productive algae and plants, providing food for
large numbers of invertebrates, waterbirds, reptiles, and mammals.

\subsection{Key Features}\label{key-features-105}

Shallow permanently inundated freshwater wetlands, dominated by
herbaceous macrophytes, supporting high primary productivity and complex
trophic networks with abundant insects, birds and amphibians.

\subsection{Ecological traits}\label{ecological-traits-105}

These shallow, permanently inundated freshwater wetlands lack woody
vegetation but are dominated instead by emergent macrophytes growing in
extensive, often monospecific groves of rhizomatous grasses, sedges,
rushes, or reeds in mosaics with patches of open water. These plants,
together with phytoplankton, algal mats, epiphytes, floating, and
amphibious herbs, sustain high primary productivity and strong bottom-up
regulation. Although most of the energy comes from these functionally
diverse autotrophs, inflow and seepage from catchments may contribute
allochthonous energy and nutrients. Plant traits including
aerenchymatous stems and leaf tissues (i.e.~with air spaces) enable
oxygen transport to roots and rhizomes and into the substrate.
Invertebrate and microbial detritivores and decomposers inhabit the
water column and substrate. Air-breathing invertebrates are more common
than gill-breathers, due to low dissolved oxygen. The activity of
microbial decomposers is also limited by low oxygen levels and organic
deposition continually exceeds decomposition. Their aquatic predators
include invertebrates, turtles, snakes and sometimes small fish. The
emergent vegetation supports a complex trophic web including insects
with winged adult phases, waterbirds, reptiles, and mammals, which feed
in the vegetation and also use it for nesting (e.g.~herons, muskrat, and
alligators). Waterbirds include herbivores, detritivores, and predators.
Many plants and animals disperse widely beyond the marsh through the
air, water and zoochory (e.g.~birds, mammals). Reproduction and
recruitment coincide with resource availability and may be cued to
floods. Most macrophytes spread vegetatively with long rhizomes but also
produce an abundance of wind- and water-dispersed seeds.

\subsection{Key Ecological Drivers}\label{key-ecological-drivers-105}

These systems occur in several geomorphic settings including lake
shores, groundwater seeps, river floodplains, and deltas, always in
low-energy depositional environments. Shallow but perennial inundation
and low variability are maintained by frequent floods and lake waters,
sometimes independently of local climate. This sustains high levels of
water and nutrients but also generates substrate anoxia. Substrates are
typically organic. Their texture varies, but silt and clay substrates
are associated with high levels of P and N. Salinity is low but may be
transitional where wetlands connect with brackish lagoons (FM1.2,
FM1.3). Surface fires may burn vegetation in some permanent marshes, but
rarely burn the saturated substrate, and are less pervasive drivers of
these ecosystems than seasonal floodplain marshes (TF1.4).

\subsection{Distribution}\label{distribution-105}

Scattered throughout the tropical and temperate regions worldwide.

\section{TF1.4 Seasonal floodplain
marshes}\label{tf1.4-seasonal-floodplain-marshes}

Belongs to biome TF1. Palustrine wetlands biome, part of the
Terrestrial, Freshwater realm.

\subsection{Short description}\label{short-description-106}

Seasonal flooding and drying regimes characterise high productivity
floodplain marshes in the seasonal tropics and subhumid temperate
regions. Typically, different plants respond to the mosaic of variable
flooding regimes, supporting complex networks of invertebrates,
waterbirds, reptiles, and mammals. Prey concentrate as the wetlands dry,
and many plants and animals use specialised adaptations such as seed
banks or egg banks, to survive drying.

\subsection{Key Features}\label{key-features-106}

High productivity wetlands with strongly seasonal water regimes,
supporting functionally diverse mosaics of aquatic plants and seasonally
variable trophic networks of invertebrates, amphibians, crocodilians and
birds.

\subsection{Ecological traits}\label{ecological-traits-106}

This group includes high-productivity floodplain wetlands fed regularly
by large inputs of allochthonous resources that drive strong bottom-up
regulation, and smaller areas of disconnected oligotrophic wetlands.
Functionally diverse autotrophs include phytoplankton, algal mats and
epiphytes, floating and amphibious herbs and graminoids, and
semi-terrestrial woody plants. Interactions of fine-scale spatial
gradients in anoxia and desiccation are related to differential
flooding. These gradients shape ecosystem assembly by enabling species
with diverse life-history traits to exploit different niches, resulting
in strong local zonation of vegetation and high patch-level diversity of
habitats for consumers. Wetland mosaics include very productive and
often extensive grasses, sedges and forbs (sedges dominate oligotrophic
systems) that persist through dry seasons largely as dormant seeds or
subterranean organs, as well as groves of woody perennials that are less
tolerant of prolonged anoxia but access ground water or arrest growth
during dry phases. Productive and functionally diverse autotrophs
support complex trophic networks with zooplankton, aquatic
invertebrates, fish, amphibians, reptiles, aquatic mammals, waterbirds,
and terrestrial animals with diverse dietary and foraging strategies.
During dry phases, obligate aquatic organisms are confined to wet
refugia. Others, including many invertebrates, have dormancy traits
allowing persistence during dry phases. Very high abundances and
diversities of invertebrates, waterbirds, reptiles, and mammals exploit
resource availability, particularly when prey are concentrated during
drawdown phases of floods. Reproduction and recruitment, especially of
fish, coincide with food availability cued by flood regimes.

\subsection{Key Ecological Drivers}\label{key-ecological-drivers-106}

Regular seasonal flooding and drying is driven by river flow regimes,
reflecting seasonal precipitation or melt patterns in catchments.
Salinity gradients and tides influence these marshes where they adjoin
estuaries, with brackish marshes on transitions to TF1.2, TF1.3 and
MFT1.3. Disconnected oligotrophic systems rely on rainfall and low
substrate permeability for seasonal waterlogging. Seasonal flood extent
and duration vary inter-annually, especially in temperate zones.
Geomorphic heterogeneity in the depositional floodplains promote spatial
and temporal variability in moisture status, creating contrasting
patches including perennially inundated refuges and dry `islands' that
seldom flood and dry rapidly. Substrates are fertile alluvia or
infertile white sands with variable grain sizes, moisture, and organic
content that reflect fine-scale depositional patterns and hydrological
gradients. Fires may occur in dry seasons, releasing resources, changing
vegetation structure and composition, consuming organic substrates and
lowering the wetland surface.

\subsection{Distribution}\label{distribution-106}

Throughout the seasonal tropics and subhumid temperate regions of the
world.

\section{TF1.5 Episodic arid
floodplains}\label{tf1.5-episodic-arid-floodplains}

Belongs to biome TF1. Palustrine wetlands biome, part of the
Terrestrial, Freshwater realm.

\subsection{Short description}\label{short-description-107}

Episodic arid floodplains rarely flood and are predominantly dry,
sometimes for years. They are supplied by temporary rivers in semi-arid
and arid regions of all continents. When floods come, there is a spike
in productivity as nutrients mobilise from leaf litter and organic
matter. At such times, dormant plants and animals form complex food
webs, capitalising on short periods of high productivity.

\subsection{Key Features}\label{key-features-107}

Highly productive floodplains when flooded, supporting highly diverse
and complex trophic networks, followed by long periods of low
productivity when dry.

\subsection{Ecological traits}\label{ecological-traits-107}

Highly episodic freshwater floodplains are distinct from, but associated
with, adjacent river channels, which provide water and sediment during
flooding. These are low-productivity systems during long, dry periods
(maybe years), with periodic spikes of very high productivity when first
inundated. These floodplains have a high diversity of aquatic and
terrestrial biota in complex trophic networks, with ruderal life-history
traits enabling the exploitation of transient water and nutrient
availability. Primary producers include flood-dependent macrophytes and
algae with physiological traits for water conservation or drought
avoidance. Lower trophic levels (e.g.~algae, invertebrate consumers)
avoid desiccation with traits such as dormant life-cycle phases,
deposition of resting eggs (e.g.~crustaceans and rotifers), and burial
in sediments banks (e.g.~larvae of cyclopoid copepods). Higher trophic
levels (e.g.~fish, amphibians, reptiles, and waterbirds) are highly
mobile in large numbers or with resting strategies (e.g.~burrowing
frogs). These taxa can be important mobile links for the movement of
biota and resources, but floods are the primary allochthonous sources of
energy and nutrients. Floods are important triggers for life-history
processes such as seed germination, emergence from larval stages,
dispersal, and reproduction. Common lifeforms include detritus-feeding
invertebrate collector-gatherers, indicating a reliance on heterotrophic
energy pathways.

\subsection{Key Ecological Drivers}\label{key-ecological-drivers-107}

Multi-year dry periods are punctuated by brief intervals of shallow
inundation caused by the overspill from flooding river channels. These
boom-bust systems have temporarily high productivity driven by water and
partly by elevated levels of dissolved Carbon and nutrients (notably N
and P) released from leaf litter, oxygen, and organic matter in newly
inundated, shallow areas. High temperatures promote productivity and
rapid drying in arid environments. Water may be turbid or clear, which
affects light environments and may limit benthic algal production to the
shallow littoral margins of small channels. This in turn affects aquatic
food webs and Carbon dynamics. Drainage is predominantly horizontal and
bidirectional (i.e.~in and out of the river), but infiltration and
evapotranspiration can be significant in the flat terrain and may
influence salinity if there are sources of salt in the catchment or
ground water.

\subsection{Distribution}\label{distribution-107}

Connected to ephemeral rivers in semi-arid and arid regions of all
continents.

\section{TF1.6 Boreal, temperate and montane peat
bogs}\label{tf1.6-boreal-temperate-and-montane-peat-bogs}

Belongs to biome TF1. Palustrine wetlands biome, part of the
Terrestrial, Freshwater realm.

\subsection{Short description}\label{short-description-108}

Peat bogs in the boreal-subarctic and temperate areas of the world
account for up to 40\% of the world's soil carbon. They are landscape
sponges, with highly specialised plant life including shrubs, sedges and
mosses equipped to grow in acidic, nutrient-poor, low-oxygen,
waterlogged soils. Sphagnum moss and other peat-forming plants are
foundational to these ecosystems. Insects are the dominant animal group,
along with amphibians, reptiles, rodents and a few visiting birds.

\subsection{Key Features}\label{key-features-108}

Permanently ground water-logged (by rainwater-fed ground water,)
nutrient poor, acidic sites on organic soils (peat); species poor, but
high abundance of mosses, sedges and no to open woody species cover.

\subsection{Ecological traits}\label{ecological-traits-108}

These patterned peatlands account for up to 40\% of global soil carbon
are dominated by a dense cover (high LAI) of hydrophytic mosses,
graminoids, and shrubs, sometimes with scattered trees. Positive
feedbacks between dense ground vegetation, hydrology, and substrate
chemistry promote peat formation through water retention and inhibition
of microbial decomposition. Moderate to low primary production is
partially broken down at the soil surface by anamorphic fungi and
aerobic bacteria. Burial by overgrowth and saturation by the water table
promotes anaerobic conditions, limiting subsurface microbial activity,
while acidity, nutrient scarcity, and low temperatures enhance the
excess of organic deposition over decomposition. Plant diversity is low
but fine-scale hydrological gradients structure vegetation mosaics,
which may include fens (TF1.7). Mosses (notably Sphagnum spp.) and
graminoids with layering growth forms promote peat formation. Their
relative abundance influences microbial communities and peat
biochemistry. Plant traits such as lacunate stem tissues, aerenchyma,
and surface root mats promote oxygen transport into the anaerobic
substrate. Woody plant foliage is small (leptophyll-microphyll) and
sclerophyllous, reflecting excess carbohydrate production in
low-nutrient conditions. Plants and fungi reproduce primarily by
cloning, except where disturbances (e.g.~fires) initiate gaps enabling
recruitment. Pools within the bogs have specialised aquatic food webs
underpinned by algal production and allochthonous carbon. Invertebrate
larvae are prominent consumers in the trophic network of bog pools, and
as adults they are important pollinators and predators. Assemblages of
flies, dragonflies, damselflies, caddisflies and other invertebrates
vary with the number, size and stability of pools. Carnivorous plants
(e.g.~sundews) support N cycling. Vertebrates are mostly itinerant but
include specialised resident amphibians, reptiles, rodents, and birds.
Some regions are rich in locally endemic flora and fauna, particularly
in the Southern Hemisphere.

\subsection{Key Ecological Drivers}\label{key-ecological-drivers-108}

Bogs are restricted to cool humid climates where moisture inputs
(e.g.~precipitation, seepage, and surface inflow) exceed outputs
(e.g.~evapotranspiration, percolation, and run-off) for extended
periods, enabling these systems to function as landscape sponges.
Seasonally low temperatures and/or frequent cloud cover limit
evapotranspiration. Substrates are waterlogged, anaerobic, highly
organic (usually \textgreater30\% dry weight), acidic (pH 3.5--6), and
nutrient-poor. Peat growth may produce raised ombrotrophic bogs entirely
fed by rain, but if minerotrophic inflows from catchments occur, they
provide limited nutrient subsidies (cf.~TF1.7). Fires may occur in dry
summers, sometimes igniting peat with long-term consequences for
ecosystem function and stability.

\subsection{Distribution}\label{distribution-108}

Extensive across boreal-subarctic latitudes, with small areas on
tropical mountains of South America, New Guinea, and Central Africa and
at cool, temperate southern latitudes in Patagonia and Australasia.

\section{TF1.7 Boreal and temperate
fens}\label{tf1.7-boreal-and-temperate-fens}

Belongs to biome TF1. Palustrine wetlands biome, part of the
Terrestrial, Freshwater realm.

\subsection{Short description}\label{short-description-109}

Fens occur extensively in boreal-subarctic and cool temperate regions.
Like peat bogs, with which they may form mosaics, they have waterlogged
organic soils, but they are rich in mineral nutrients and typically
neutral or alkaline in pH. The vegetation comprises a low diversity of
small plants, fungi and brown mosses, but woody plants are generally
absent. They support insects, specialised frogs and some birds. Shallow
standing water or permafrost may be present.

\subsection{Key Features}\label{key-features-109}

Permanently groundwater-logged, nutrient poor to (moderately)
nutrient-rich sites, often organic soils; high abundance of mosses,
sedges and no to open woody species cover.

\subsection{Ecological traits}\label{ecological-traits-109}

Fens are peatland ecosystems dominated by hydrophytic grasses, sedges,
or forbs. Fens have higher productivity but lower functional diversity
than bogs (TF1.6). Productivity is subsidised by inflow of minerotrophic
waters and limited by anoxic substrates. Plant diversity is very low
where surface hydrology varies temporally from complete saturation to
desiccation but can be high in mineral-rich fens with stable
near-surface water tables. Some regions are rich in locally endemic
flora and fauna. Woody plants are typically scarce or absent, though
some boreal forests (T2.1) develop on minerotrophic peats. Sphagnum
mosses and hummock-forming sedges are absent from rich fens but `brown
mosses' are common. Primary production is partly broken down on
soil-surface layers by anamorphic fungi and aerobic bacteria. Anaerobic
conditions due to high water tables limit subsurface microbial activity
so that organic deposition exceeds decomposition and peat accumulates.
Plant traits such as lacunate stem tissues, aerenchyma, and surface root
mats promote oxygen transport into the anaerobic substrate. Methanogenic
archaea and anaerobic bacteria may occur in the subsoil if N, Fe, and S
are sufficient to sustain them. Fens may be spatially homogeneous or
form string mosaics with bogs (e.g.~aapa mires of Finland) but often
display zonation reflecting differences in water chemistry (notably pH)
or saturation. Patches of fen and bogs may be juxtaposed within peatland
mosaics. Ongoing peat build-up may lead to transition from fen to bog
systems. Plants and fungi reproduce locally by cloning, but seed and
spore production enables dispersal and the colonisation of new sites.
Invertebrates are dominant consumers in the trophic network, including
dragonflies, caddisflies, flies, as well as calcareous specialists such
as snails. Vertebrates are mostly itinerant but include specialised
resident amphibians and birds.

\subsection{Key Ecological Drivers}\label{key-ecological-drivers-109}

Moisture inputs (precipitation, seepage, and surface inflow) exceed
outputs (evapotranspiration, percolation, and run-off) for extended
periods, enabling these systems to function as landscape sponges.
Seasonally low temperatures and/or frequent cloud cover limit
evapotranspiration. Fens typically develop by the paludification
(i.e.~peat accumulation) of shallow lakes or around springs and thus
shallow standing water is present frequently. Such lakes may be abundant
in post-glacial landscapes. Substrates are waterlogged, anaerobic,
highly organic (usually \textgreater30\% dry weight), slightly acidic or
alkaline, and rich in mineral nutrients. Minerotrophic water
(i.e.~inflow from catchments) provides significant nutrient subsidies
that vary with catchment geology. Fens on the arctic circle (palsa
mires) have subsurface permafrost. Fires may occur in dry summers,
rarely consuming peat, lowering the surface and degrading permafrost.

\subsection{Distribution}\label{distribution-109}

Extensive across boreal-subarctic latitudes and cool temperate regions,
especially mountains. Very restricted in the Southern Hemisphere. Fens
may also occur in tropical mountains (e.g.~Andes), but are poorly known
there.

\section{F1.2 Permanent lowland
rivers}\label{f1.2-permanent-lowland-rivers-1}

Belongs to biome F1. Rivers and streams biome, part of the Freshwater
realm.

\subsection{Short description}\label{short-description-110}

Lowland rivers with slow continuous flows up to 10,000m3/s are common at
low elevations throughout tropical and temperate parts of the world.
These are productive ecosystems with major energy and fine sediment
inputs from floodplains and upper catchments. Zooplankton can be
abundant, along with aquatic plants and diverse communities of fish able
to tolerate a range of temperatures and oxygen concentrations, as well
as reptiles, birds, and mammals that depend wholly or partly on lowland
lotic aquatic habitats.

\subsection{Key Features}\label{key-features-110}

Low-medium velocity, high volume, perennial flows with abundant
zooplankton, fish, macrophytes, macroinvertebrates \& piscivores.

\subsection{Ecological traits}\label{ecological-traits-110}

Small-medium lowland rivers (stream orders 4-9) are productive
depositional ecosystems with trophic webs that are less diverse than
large lowland rivers (F1.7). Macrophytes rooted in benthos or along the
river margins contribute most primary production, but allochthonous
inputs from floodplains and upper catchments generally dominate energy
flow in the system. The biota tolerates a range of temperatures, which
vary with catchment climate. Aquatic biota have physiological,
morphological and even behavioural adaptations to lower oxygen
concentrations, which may vary seasonally and diurnally. Zooplankton can
be abundant in slower deeper rivers. Sessile (e.g.~mussels) and
scavenging (e.g.~crayfish) macroinvertebrates are associated with the
hyporheic zone and structurally complex microhabitats in moderate flow
environments, including fine sediment and woody debris. Fish communities
are diverse and may contribute to complex trophic networks. They include
large predatory fish (e.g.~sturgeons), smaller predators of
invertebrates, herbivores, and detritivores. The feeding activities and
movement of piscivorous birds (e.g.~cormorants), diadromous fish
(seawater-freshwater migrants), mammals (e.g.~otters), and reptiles
(e.g.~turtles) extend trophic network beyond instream waters. Riparian
zones vary in complexity from forested banks to shallow areas where
emergent, floating and submerged macrophyte vegetation grows.
Intermittently connected oxbow lakes or billabongs increase the
complexity of associated habitats, providing more lentic waters for a
range of aquatic fauna and flora.

\subsection{Key Ecological Drivers}\label{key-ecological-drivers-110}

These rivers are distinguished by shallow gradients, low turbulence, low
to moderate flow velocity and moderate flow volumes
(\textless10,000m3/s). Flows are continuous but may vary seasonally
depending on catchment precipitation. This combination of features is
most common at low altitudes below 200 m and rarely occurs above 1,500
m. River channels are tens to a few hundred metres wide and up to tens
of metres deep with mostly soft sediment substrates. They are dominated
by depositional processes. Surface water and groundwater mix in the
alluvium in the hyporheic zone, which plays an important role in
nutrient cycling. Overbank flows increase turbulence and turbidity.
Locally or temporally important erosional processes redistribute
sediment and produce geomorphically dynamic depositional features
(e.g.~braided channels and point bars). Nutrient levels depend on
riparian/floodplain inputs and vary with catchment geochemistry. Oxygen
and temperatures also vary with climate and catchment features. For
catchments with extensive peatlands, waters may be tannin-rich, poorly
oxygenated, acidic and dark, thus reducing productivity and diversity.

\subsection{Distribution}\label{distribution-110}

Distributed throughout tropical and temperate lowlands but very uncommon
in arid zones. They are absent from boreal zones, where they are
replaced by F1.3.

\section{F1.3 Freeze-thaw rivers and
streams}\label{f1.3-freeze-thaw-rivers-and-streams-1}

Belongs to biome F1. Rivers and streams biome, part of the Freshwater
realm.

\subsection{Short description}\label{short-description-111}

In cold climates at high latitudes or altitudes, the surfaces of both
small streams and large rivers freeze in winter. In winter, the layer of
surface ice reduces nutrient inputs and light penetration, limiting the
productivity of these ecosystems and the diversity of their biota. In
spring, meltwaters transport increased organic matter and nutrients,
producing seasonal peaks in abundance of algae and phytoplankton.
Animals, such as fish and beavers, tolerate near-freezing water
temperatures, while a range of invertebrates and other vertebrates come
to forage from spring to autumn.

\subsection{Key Features}\label{key-features-111}

Cold-climate streams with seasonally frozen surface water and variable
melt flows and aquatic biota with cold-resistance and/or seasonal
dormancy.

\subsection{Ecological traits}\label{ecological-traits-111}

In seasonally cold montane and boreal environments, the surfaces of both
small streams and large rivers freeze in winter. These systems have
relatively simple trophic networks with low functional and taxonomic
diversity, but the biota may include local endemics. In small, shallow
streams, substrate algae are the principal autotrophs, while
phytoplankton occur in larger rivers and benthic macrophytes are rare.
All are seasonally inactive or curtailed when temperatures are cold and
surface ice reduces light penetration through the water. Bottom-up
regulatory processes dominate. Subsidies of dissolved organic carbon and
nutrients from spring meltwaters and riparian vegetation along smaller
streams are crucial to maintaining the detritivores that dominate the
trophic network. Overall decomposition rates of coarse particles are
low, but can exceed rates per degree day in warmer climates as the fauna
are adapted to cold temperatures. Microbial decomposers often dominate
small streams, but in larger rivers, the massive increase in fine
organic particles in spring meltwaters can support abundant filter
feeders which consume huge quantities of suspended particles and
redeposit them within the river bed. Resident invertebrates survive cold
temperatures through dormant life stages, extended life cycles and
physiological adaptations. Vertebrate habitat specialists (e.g.~dippers,
small fish, beavers, and otters) tolerate low temperatures with traits
such as subcuticular fat, thick hydrophobic, and/or aerated fur or
feathers. Many fish disperse from frozen habitat to deeper water refuges
during the winter (e.g.~deep pools) before foraging in the meltwater
streams from spring to autumn. In the larger rivers, fish, and
particularly migratory salmonids returning to their natal streams and
rivers for breeding, are a food source for itinerant terrestrial
predators such as bears. When they die after reproduction, their
decomposition in turn provides huge inputs of energy and nutrients to
the system.

\subsection{Key Ecological Drivers}\label{key-ecological-drivers-111}

These rivers experience low winter temperatures and seasonal freeze-thaw
regimes. Winter freezing is generally limited to the surface but can
extend to the substrate forming `anchor ice'. Flows may continue below
the ice or may be intermittent in smaller streams or dry climates.
Freezing reduces resource availability by reducing nutrient inputs,
allochthonous organic matter and light penetration through the water.
Light may also be attenuated at high latitudes and by high turbidity in
erosional streams. Meltwaters drive increased flow and flooding in
spring and summer. Carbon and nutrient concentrations are greatest
during spring floods, and pH tends to decrease with flow during spring
and autumn. When catchments include extensive peatlands, waters may be
tannin-rich, acidic and dark, thereby reducing light penetration and
productivity.

\subsection{Distribution}\label{distribution-111}

Restricted to boreal, subarctic, alpine and subalpine regions, with
limited examples in the subantarctic and Antarctic.

\section{F1.4 Seasonal upland
streams}\label{f1.4-seasonal-upland-streams-1}

Belongs to biome F1. Rivers and streams biome, part of the Freshwater
realm.

\subsection{Short description}\label{short-description-112}

Seasonal rainfall patterns in large parts of the tropics and temperate
regions generate flows that are hugely variable in narrow and steep
upland streams. Globally, these streams account for the greatest stream
length of any flowing ecosystem. During the dry season, flows in some
streams are reduced to very levels, while in others flow ceases
altogether and water persists only in isolated stagnant pools. Algae and
leaf fall support moderate productivity, with seasonal floods sending
organic matter downstream. The diversity of organisms fluctuates
seasonally, with many localised (endemic) species, and specialised
adaptations that enable animals to survive both flooding and dry
conditions.

\subsection{Key Features}\label{key-features-112}

High-medium velocity, low-medium volume, highly seasonal flows with
abundant benthic filter feeders, algal biofilms \& small fish.

\subsection{Ecological traits}\label{ecological-traits-112}

Upland streams (orders 1-4) with highly seasonal flows generally have
low to moderate productivity and a simpler trophic structure than
lowland rivers. They tend to be shallow, hence benthic algae are major
contributors to in-stream food webs and productivity, but riparian zones
and catchments both contribute allochthonous energy and organic carbon
through leaf fall, which may include an annual deciduous component.
Primary production also varies with light availability and flow.
Taxonomic diversity varies between streams, but can be lower than
permanent streams and relatively high in endemism. Traits that enable
biota to persist in narrow and shallow channels with large seasonal
variations in flow velocity, episodes of torrential flow, and seasonal
desiccation include small body sizes (especially in resident fish),
dormant life phases and/or burrowing (crustaceans), omnivorous diets and
high dispersal ability, including seasonal migration. Compared to
lowland rivers, the trophic structure has a higher representation of
algal and omnivorous feeders and low numbers of larger predators. Birds
show specialist feeding strategies (e.g.~dippers). Diversity and
abundance of invertebrates and their predators (e.g.~birds) fluctuate in
response to seasonal flood regimes.

\subsection{Key Ecological Drivers}\label{key-ecological-drivers-112}

Flow and flood regimes in these rivers are highly variable between
marked wet and dry seasons, with associated changes in water quality as
solute concentration varies with volume. They may be perennial, with
flows much-reduced in the dry season, or seasonally intermittent with
flows ceasing and water persisting in isolated stagnant pools. Channels
are narrow with steep to moderate gradients, seasonally high velocity
and sometimes large volumes of water, producing overbank flows. This
results in considerable turbulence, turbidity, and erosion during the
wet season and coarse substrates (cobbles and boulders). Seasonal floods
are critical to allochtonous subsidies and downstream exports of organic
matter and nutrients.

\subsection{Distribution}\label{distribution-112}

Elevated regions in seasonal tropical, subtropical and temperate
climates worldwide.

\section{F1.5 Seasonal lowland
rivers}\label{f1.5-seasonal-lowland-rivers-1}

Belongs to biome F1. Rivers and streams biome, part of the Freshwater
realm.

\subsection{Short description}\label{short-description-113}

These medium to large rivers in tropical, subtropical and temperate
lowlands have markedly seasonal flows due to seasonal water supply in
the catchments. Their single or multi-channelled forms link to
floodplain wetlands, and transport large floods during wet seasons:
summer in the tropics or winter-spring in temperate latitudes.
Productivity is high, both within channels and on connected floodplains,
with algae and aquatic plants supporting complex food webs, and
providing seasonal nurseries for breeding animals.

\subsection{Key Features}\label{key-features-113}

Highly productive large rivers with seasonal hydrology large floodplain
subsidies. Short food chains support large mobile predaors.

\subsection{Ecological traits}\label{ecological-traits-113}

These large riverine systems (stream orders 5-9) can be highly
productive with trophic structures and processes shaped by seasonal
hydrology and linkages to floodplain wetlands. In combination with
biophysical heterogeneity, this temporal variability promotes functional
diversity in the biota. Although trophic networks are complex due to the
diversity of food sources and the extent of omnivory amongst consumers,
food chains tend to be short and large mobile predators such as otters,
large piscivorous waterbirds, sharks, dolphins, and crocodilians (in the
tropics) can have a major impact on the food webs. Benthic algae are key
contributors to primary productivity, although macrophytes become more
important during the peak and late wet season when they also provide
substrate for epiphytic algae. Rivers receive very significant resource
subsidies from both algae and macrophytes on adjacent floodplains when
they are connected by flows. Enhanced longitudinal hydrological
connectivity during the wet season enables fish and other large aquatic
consumers to function as mobile links, extending floodplain and
estuarine resource subsidies upstream. Life cycle processess including
reproduction, recruitment, and dispersal in most biota are tightly cued
to seasonally high flow periods, often with floodplain nursery areas for
river fish, amphibians and larger invertebrates.

\subsection{Key Ecological Drivers}\label{key-ecological-drivers-113}

These rivers are driven by cyclical, seasonal flow regimes. High-volume
flows and floods occur during summer in the tropics or winter-spring at
temperate latitudes, with two peaks in some areas. A decline of flows
and reduced flood residence times during the transition to the dry
season is followed by low and disconnected flows during the dry season.
Turbidity, light availability, erosion, sedimentation, lateral and
longitudinal connectivity, biological activity, dissolved oxygen and
solute concentrations all vary with this seasonal cycle. The
inter-annual variability of this pattern depends on the catchment
precipitation and sources of inflow that offset or mute the influences
of rainfall seasonality (e.g.~snow melt in South Asia). Streams may be
single, multi-channelled or complex anabranching systems.

\subsection{Distribution}\label{distribution-113}

Tropical, subtropical and temperate lowlands with seasonal inflow
patterns worldwide.

\section{F1.6 Episodic arid rivers}\label{f1.6-episodic-arid-rivers-1}

Belongs to biome F1. Rivers and streams biome, part of the Freshwater
realm.

\subsection{Short description}\label{short-description-114}

These desert rivers occur mostly in flat areas of arid and semi-arid
mid-latitudes. Channels are typically broad, flat, and often branching,
with soft sandy sediments. They are dry most of the time, but punctuated
by high-volume, short duration flows that transport nutrients and
stimulate high productivity by algae and zooplankton. Plants and animals
can either tolerate or avoid long, dry periods and then exploit short
pulses of abundant resources, producing hotspots of biodiversity and
ecological activity in arid landscapes.

\subsection{Key Features}\label{key-features-114}

Rivers with high temporal flow variability which determines periods of
high and low productivity, supporting high levels of biodiversity and
complex trophic networks during floods and simple trophic networks
during dry periods.

\subsection{Ecological traits}\label{ecological-traits-114}

Episodic rivers have high temporal variability in flows and resource
availability, shaping a low-diversity biota with periodically high
abundance of some organisms. Productivity is episodically high and
punctuated by longer periods of low productivity (i.e.~boom-bust
dynamics). The trophic structure can be complex and dominated by
autochthonous primary production. Even though riparian vegetation is
sparse, allochthonous inputs from connected floodplains may be
important. Top-down control of ecosystem structure is evident in some
desert streams. Episodic rivers are hotspots of biodiversity and
ecological activity in arid landscapes, acting as both evolutionary and
ecological refuges. Most biota have ruderal life cycles, dormancy
phases, or high mobility enabling them to tolerate or avoid long, dry
periods and to exploit short pulses of high resource availability during
flooding. During dry periods, many organisms survive as dormant life
phases (e.g.~eggs or seeds), by reducing metabolism, or by persisting in
perennial refugia (e.g.~waterholes, shallow aquifers). They may rapidly
recolonise the channel network during flow (networkers). Waterbirds
survive dry phases by moving elsewhere, returning to breed during flows.
The abundance of water, nutrients and food during flows and floods
initiates rapid primary production (especially by algae), breeding and
recruitment. Zooplankton are abundant in slower reaches during periods
of flow. Macroinvertebrates such as sessile filter-feeders
(e.g.~mussels) and scavengers (e.g.~crayfish) may occur in moderate flow
environments with complex microhabitats in fine sediment and amongst
woody debris. Assemblages of fish and amphibians are dominated by small
body sizes. Most fish species use inundated floodplains in larval,
juvenile and mature life stages, and produce massive biomass after large
floods. Organisms generally tolerate wide ranges of temperature,
salinity, and oxygen.

\subsection{Key Ecological Drivers}\label{key-ecological-drivers-114}

These mostly lowland systems are distinguished by highly episodic flows
and flood regimes that vary with catchment size and precipitation.
High-volume, short duration flows (days to weeks, rarely months)
punctuate long dry periods fill channels and flood wetlands. Low
elevational gradients and shallow channels result in low turbulence and
low to moderate flow velocity. Lowland stream channels are broad, flat,
and often anastomising, with mostly soft sandy sediments. Groundwater is
usually within rooting zones of perennial plants, which may establish in
channels after flow events. Sediment loads drive periodically high
turbidity. Locally or temporally important erosional processes have
roles in geomorphic dynamism redistributing sediment in depositional
features (e.g.~braided channels and point bars). Upland streams are
prone to erosive flash floods. High nutrient levels are due to large
catchments and riparian inputs but depend on catchment geochemistry.
These rivers often flow over naturally saline soils. Salinity can thus
be high and increases in drying phases.

\subsection{Distribution}\label{distribution-114}

Arid and semi-arid mid-latitudes, in lowlands, and some uplands, but
rarely above 1,500 m elevation.

\section{F1.7 Large lowland rivers}\label{f1.7-large-lowland-rivers-1}

Belongs to biome F1. Rivers and streams biome, part of the Freshwater
realm.

\subsection{Short description}\label{short-description-115}

These very large rivers transport massive volumes of freshwater
(\textgreater10,000m3/s) through flat lowlands, mostly in tropical or
subtropical regions. Their very large flow volumes, diverse habitats and
slow to moderate flows make them highly productive. High nutrient levels
come from upstream catchments and floodplains, with additional
productivity contributed by in-channel algae and aquatic plants. Their
food webs are complex, with a high diversity of plants and animals,
including large-bodied fish, reptiles and mammals.

\subsection{Key Features}\label{key-features-115}

Large highly productive rivers with megaflow rates and complex food
webs, reflecting the extent of habitat, connections with floodplains and
available niches for plants, invertebrates and large vertebrates
including aquatic mammals..

\subsection{Ecological traits}\label{ecological-traits-115}

Large lowland rivers (typically stream orders 8-12) are highly
productive environments with complex trophic webs which are supported by
very large flow volumes. Primary production is mostly from autochthonous
phytoplankton and riparian macrophytes, with allochthonous inputs from
floodplains and upper catchments generally dominating energy flow in the
system. The fauna includes a significant diversity of pelagic organisms.
Zooplankton are abundant, while sessile (e.g.~mussels), burrowing
(e.g.~annelids) and scavenging (e.g.~crustaceans) macroinvertebrates
occur in the fine sediment and amongst woody debris. Fish communities
are diverse and contribute to complex trophic networks. They include
large predatory fish (e.g.~freshwater sawfish, Pirhana, Alligator Gar)
and in some rivers endemic River Dolphins, smaller predators of
invertebrates (benthic and pelagic feeders), phytoplankton herbivores,
and detritivores. The feeding activities and movement of semi-aquatic
piscivorous birds (e.g.~cormorants), mammals (e.g.~otters), and reptiles
(e.g.~turtles, crocodilians) connect the trophic network to other
ecosystems beyond instream waters. Riparian and large floodplain zones
vary in complexity from forested banks, to productive lentic oxbow lakes
and extensive and complex flooded areas where emergent and floodplain
vegetation grows (e.g.~reeds and macrophytes, shrubs, trees). Riparian
zones can be complex but have less direct influence on large rivers than
on smaller river ecosystems.

\subsection{Key Ecological Drivers}\label{key-ecological-drivers-115}

These rivers have shallow gradients with low turbulence, low to moderate
flow velocity and very high flow volumes (\textgreater10,000m3/s), which
are continuous but may vary seasonally depending on catchment area and
precipitation (e.g.~Congo up to 41,000 m3/s, Amazon up to 175,000 m3/s).
River channels are wide (e.g.~Amazon River; 11 km in dry season, up to
25km when flooded at its widest point) and deep (e.g.~Congo up to 200m;
Mississippi up to 60m) with mostly soft sediment substrates. They are
dominated by depositional processes so turbidity may be high. Overbank
flows increase turbulence and turbidity. Locally or temporally important
erosional processes redistribute sediment and produce geomorphically
dynamic depositional features (e.g.~braided channels, islands and point
bars). Nutrient levels are high due to large catchments and
riparian/floodplain inputs but vary with catchment geochemistry.
Moderate water temperatures are buffered due to large catchments.

\subsection{Distribution}\label{distribution-115}

Tropical and subtropical lowlands, with a few extending to temperate
zones. They are absent from arid regions, and in boreal zones are
replaced by F1.3.

\section{F2.1 Large permanent freshwater
lakes}\label{f2.1-large-permanent-freshwater-lakes-1}

Belongs to biome F2. Lakes biome, part of the Freshwater realm.

\subsection{Short description}\label{short-description-116}

Large volumes of permanent water in these lakes buffers water
temperatures and effects of nutrient input on water quality. The spatial
extent and range of habitats support very large numbers of species in
some groups such as fish, some of which are unique to a single lake,
often composed of closely related species, endemic to a lake. The high
primary productivity from algae and aquatic plants lake supports diverse
foodwebs. High numbers of plankton support large numbers of waterbugs,
fish, frogs, reptiles, waterbirds, and mammals. Bacteria play key roles
in cycling organic matter.

\subsection{Key Features}\label{key-features-116}

Large (usually \textgreater100km2) permanent freshwater lakes connected
to rivers, with high spatial and bathymetric niche diversity supporting
complex trophic networks supported by planktonic algae, high diversity
and endemism.

\subsection{Ecological traits}\label{ecological-traits-116}

Large permanent freshwater lakes, generally exceeding 100 km2, are
prominent landscape features connected to one or more rivers either
terminally or as flow-through systems. Shoreline complexity, depth,
bathymetric stratification, and benthic topography promote niche
diversity and zonation. High niche diversity and large volumes of
permanent water (extensive, stable, connected habitat) support complex
trophic webs with high diversity and abundance. High primary
productivity may vary seasonally, driving succession, depending on
climate, light availability, and nutrient regimes. Autochthonous energy
from abundant pelagic algae (mainly diatoms and cyanobacteria) and from
benthic macrophytes and algal biofilms (in shallow areas) is
supplemented by allochthonous inflows that depend on catchment
characteristics, climate, season, and hydrological connectivity.
Zooplankton, invertebrate consumers, and herbivorous fish sustain high
planktonic turnover and support upper trophic levels with abundant and
diverse predatory fish, amphibians, reptiles, waterbirds, and mammals.
This bottom-up web is coupled to a microbial loop, which returns
dissolved organic matter to the web (rapidly in warm temperatures) via
heterotrophic bacteria. Obligate freshwater biota in large lakes,
including aquatic macrophytes and macroinvertebrates (e.g.~crustaceans)
and fish, often display high catchment-level endemism, in part due to
long histories of environmental variability in isolation. Marked niche
differentiation in life history and behavioural feeding and reproductive
traits enables sympatric speciation and characterises the most diverse
assemblages of macroinvertebrates and fish (e.g.~\textasciitilde500
cichlid fish species in Lake Victoria). Large predators are critical in
top-down regulation of lower trophic levels. Large lake volume buffers
against nutrient-mediated change from oligotrophic to eutrophic states.
Recruitment of many organisms is strongly influenced by physical
processes such as large inflow events. Mobile birds and terrestrial
mammals use the lakes as breeding sites and/or sources of drinking water
and play key roles in the inter-catchment transfer of nutrients and
organic matter and the dispersal of biota.

\subsection{Key Ecological Drivers}\label{key-ecological-drivers-116}

Large water volumes influence resource availability, environmental
stability (through thermal buffering), and niche diversity. Water is
from catchment inflows, which may vary seasonally with climate. Large
lakes influence regional climate through evaporation, cooling, and
convection feedbacks. These processes also influence nutrient
availability, along with catchment and lake substrates and vertical
mixing. Mixing may be monomictic (i.e.~annual) or meromictic
(i.e.~seldom), especially in large tropical lakes, depending on inflow,
depth, wind regimes, and seasonal temperature variation. Light varies
with lake depth, turbidity, cloud cover, and latitude.

\subsection{Distribution}\label{distribution-116}

Humid temperate and tropical regions on large land masses.

\section{F2.10 Subglacial lakes}\label{f2.10-subglacial-lakes-1}

Belongs to biome F2. Lakes biome, part of the Freshwater realm.

\subsection{Short description}\label{short-description-117}

These hidden lakes exist beneath permanent ice sheets, sometimes tens to
thousands of metres below, mostly in Antarctica and Greenland. Bacteria
and other microbes are the only forms of life, but there is a surprising
diversity of them. Productivity is very low in the dark and freezing
conditions, with species relying on metabolism of chemicals such as
methane, iron and sulphur to support the simple foodweb.

\subsection{Key Features}\label{key-features-117}

Lakes beneath permanent ice sheets with a truncated microbial food web,
including chemoautotrophic and heterotrophic of bacteria and archaea.

\subsection{Ecological traits}\label{ecological-traits-117}

Remarkable lacustrine ecosystems occur beneath permanent ice sheets.
They are placed within the Lakes biome (biome F2) due to their
relationships with some Freeze-thaw lakes (F2.4), but they share several
key features with the Subterranean freshwater biome (biome SF1).
Evidence of their existence first emerged in 1973 from airborne
radar-echo sounding imagery, which penetrates the ice cover and shows
lakes as uniformly flat structures with high basal reflectivity. The
biota of these ecosystems is very poorly known due to technological
limitations on access and concerns about the risk of contamination from
coring. Only a few shallow lakes up to 1 km beneath ice have been
surveyed (e.g.~Lake Whillams in West Antarctica and Grímsvötn Lake in
Iceland). The exclusively microbial trophic web is truncated, with no
photoautotrophs and apparently few multi-cellular predators, but
taxonomic diversity is high across bacteria and archaea, with some
eukaryotes also represented. Chemosynthesis form the base of the trophic
web, chemolithoautotrophic species using reduced N, Fe and S and methane
in energy-generating metabolic pathways. The abundance of
micro-organisms is comparable to that in groundwater (SF1.2) (104 -- 105
cells.ml-1), with diverse morphotypes represented including long and
short filaments, thin and thick rods, spirals, vibrio, cocci and
diplococci. Subglacial lakes share several biotic traits with
extremophiles within ice (T6.1), subterranean waters (SF1.1, SF1.2) and
deep oceans (e.g.~M2.3, M2.4, M3.3), including very low productivity,
slow growth rates, large cell sizes and aphotic energy synthesis.
Although microbes of the few surveyed subglacial lakes, and from
accreted ice which has refrozen from lake water, have DNA profiles
similar to those of other contemporary microbes, the biota in deeper
disconnected lake waters and associated lake-floor sediments, could be
highly relictual if it evolved in stable isolation over millions of
years under extreme selection pressures.

\subsection{Key Ecological Drivers}\label{key-ecological-drivers-117}

Subglacial lakes vary in size from less than 1 km2 to
\textasciitilde10,000 km2, and most are 10-20 m deep, but Lake Vostok
(Antarctica) is at least 1,000 m deep. The environment is characterised
by high isostatic pressure (up to \textasciitilde350 atmospheres),
constant cold temperatures marginally below 0°C, low-nutrient levels,
and an absence of sunlight. Oxygen concentrations can be high due to
equilibration with gas hydrates from the melting ice sheet base ice, but
declines with depth in amictic lakes due to limited mixing, depending on
convection gradients generated by cold meltwater from the ice ceiling
and geothermal heating from below. Chemical weathering of basal debris
is the main source of nutrients supplemented by ice melt.

\subsection{Distribution}\label{distribution-117}

Some \textasciitilde400 subglacial lakes in Antarctica,
\textasciitilde60 in Greenland and a few in Iceland and Canada have been
identified from radar remote sensing and modelling.

\section{F2.2 Small permanent freshwater
lakes}\label{f2.2-small-permanent-freshwater-lakes-1}

Belongs to biome F2. Lakes biome, part of the Freshwater realm.

\subsection{Short description}\label{short-description-118}

With a surface area of up to 100 km2, the diversity of small permanent
lakes, ponds and pools depends on their size, depth and connectivity.
Littoral vegetation and benthic energy pathways are critical to
productivity and food web complexity. Deep lakes have plankton,
supporting fish, birds and frogs, in different habitats of the lake.
Shallow lakes are often more productive, providing breeding habitat for
birds, frogs and reptiles, but limited buffering against nutrient inputs
may result in regime shifts between alternative stable states dominated
either by large aquatic plants or phytoplankton.

\subsection{Key Features}\label{key-features-118}

Small permanent freshwater lakes or ponds with niche diversity strongly
related to size and depth, and resource subsidies from catchments.
Littoral zones and benthic macrophytes are important contributors to
productivity.

\subsection{Ecological traits}\label{ecological-traits-118}

Small permanent freshwater lakes, pools or ponds are lentic environments
with relatively high perimeter-to-surface area and
surface-area-to-volume ratios. Most are \textless1 km2 in area, but this
functional group includes lakes of transitional sizes up to 100 km2,
while the largest lakes (\textgreater100 km2) are classified in F2.1.
Niche diversity increases with lake size. Although less diverse than
larger lakes, these lakes may support phytoplankton, zooplankton,
shallow-water macrophytes, invertebrates, sedentary and migratory fish,
reptiles, waterbirds, and mammals. Primary productivity, dominated by
cyanobacteria, algae, and macrophytes, arises from allochthonous and
autochthonous energy sources, which vary with lake and catchment
features, climate, and hydrological connectivity. Productivity can be
highly seasonal, depending on climate, light, and nutrients. Permanent
water and connectivity are critical to obligate freshwater biota, such
as fish, invertebrates, and aquatic macrophytes. Trophic structure and
complexity depend on lake size, depth, location, and connectivity.
Littoral zones and benthic pathways are integral to overall production
and trophic interactions. Shallow lakes tend to be more productive (by
volume and area) than deep lakes because light penetrates to the bottom,
establishing competition between benthic macrophytes and phytoplankton,
more complex trophic networks and stronger top-down regulation leading
to alternative stable states and possible regime shifts between them.
Clear lakes in macrophyte-dominated states support higher biodiversity
than phytoplankton-dominated eutrophic lakes. Deep lakes are more
dependent on planktonic primary production, which supports zooplankton,
benthic microbial and invertebrate detritivores. Herbivorous fish and
zooplankton regulate the main primary producers (biofilms and
phytoplankton). The main predators are fish, macroinvertebrates,
amphibians and birds, many of which have specialised feeding traits tied
to different habitat niches (e.g.~benthic or pelagic), but there are few
filter-feeders. In many regions, shallow lakes provide critical breeding
habitat for waterbirds, amphibians, and reptiles, while visiting mammals
transfer nutrients, organic matter, and biota.

\subsection{Key Ecological Drivers}\label{key-ecological-drivers-118}

These lakes may be hydrologically isolated, groundwater-dependent or
connected to rivers as terminal or flow-through systems. Nutrients
depend on catchment size and substrates. Some lakes (e.g.~on leached
coastal sandplains or peaty landscapes) have dystrophic waters. The
seasonality and amount of inflow, size, depth (mixing regime and light
penetration), pH, nutrients, salinity, and tanins shape lake ecology and
biota. Seasonal cycles of temperature, inflow and wind (which drives
vertical mixing) may generate monomictic or dimictic temperature
stratification regimes in deeper lakes, while shallow lakes are
polymicitic, sometimes with short periods of multiple stratification.
Seasonal factors such as light, increases in temperature, and flows into
lakes can induce breeding and recruitment.

\subsection{Distribution}\label{distribution-118}

Mainly in humid temperate and tropical regions, rarely semi-arid or arid
zones.

\section{F2.3 Seasonal freshwater
lakes}\label{f2.3-seasonal-freshwater-lakes-1}

Belongs to biome F2. Lakes biome, part of the Freshwater realm.

\subsection{Short description}\label{short-description-119}

Small seasonal lakes, pools and rock holes have plants and animals
specialised to seasonally changing wet and dry conditions in temperate
and wet-dry tropical regions. Their energy comes mostly from algae and
plants. To survive the annual wet/dry cycles, the animals and plants
have dormant life stages, such as eggs or seeds, within the lake
sediments, or they shelter in damp burrows or other refuges. Plants and
animals can build up high abundances during wet seasons, supporting
plankton and waterbugs, frogs, birds and mammals but, in most cases, no
fish.

\subsection{Key Features}\label{key-features-119}

Mostly small and shallow well mixed freshwater lakes with seasonal
patterns of filling and seasonally variable abundance and composition of
aquatic biota, including species with dormant life phases and some that
retreat to refuges in dry seasons.

\subsection{Ecological traits}\label{ecological-traits-119}

These small (mostly \textless5 km2 in area) and shallow (\textless2 m
deep) seasonal freshwater lakes, vernal pools, turloughs, or gnammas
(panholes, rock pools), have a seasonal aquatic biota. Hydrological
isolation promotes biotic insularity and local endemism, which occurs in
some Mediterranean climate regions. Autochthonous energy sources are
supplemented by limited allochthonous inputs from small catchments and
groundwater. Seasonal variation in biota and productivity outweighs
inter-annual variation, unlike in ephemeral lakes (F2.5 and F2.7).
Filling induces microbial activity, the germination of seeds and algal
spores, hatching and emergence of invertebrates, and growth and
reproduction by specialists and opportunistic colonists. Wind-induced
mixing oxygenates the water, but eutrophic or unmixed waters may become
anoxic and dominated by air-breathers as peak productivity and biomass
fuel high biological oxygen demand. Anoxia may be abated diurnally by
photosynthetic activity. Resident biota persists through seasonal drying
on lake margins or in sediments as desiccation-resistant dormant or
quiescent life stages, e.g.~crayfish may retreat to burrows that extend
to the water table, turtles may aestivate in sediments or fringing
vegetation, amphibious perennial plants may persist on lake margins or
in seedbanks. Trophic networks and niche diversity are driven by
bottom-up processes, especially submerged and emergent macrophytes, and
depend on productivity and lake size. Cyanobacteria, algae, and
macrophytes are the major primary producers, while annual grasses may
colonise dry lake beds. The most diverse lakes exhibit zonation and
support phytoplankton, zooplankton, macrophytes, macroinvertebrate
consumers, and seasonally resident amphibians (especially juvenile
aquatic phases), waterbirds, and mammals. Rock pools have simple trophic
structure, based primarily on epilithic algae or macrophytes, and
invertebrates, but no fish. Invertebrates and amphibians may reach high
diversity and abundance in the absence of fish.

\subsection{Key Ecological Drivers}\label{key-ecological-drivers-119}

Seasonal rainfall, surface flows, groundwater fluctuation and seasonally
high evapo-transpiration drive annual filling and drying. These lakes
are polymicitc, mixing continuously when filled. Impermeable substrates
(e.g.~clay or bedrock) impede infiltration in some lakes; in others
groundwater percolates up through sand, peat or fissures in karstic
limestone (turloughs). Small catchments, low-relief terrain, high
area-to-volume ratios, and hydrological isolation promote seasonal
fluctuation. Most lakes are hydrologically isolated, but some become
connected seasonally by sheet flows or drainage lines. These
hydrogeomorphic features also limit nutrient supply, in turn limiting pH
buffering. Water fluctuations drive high rates of organic decomposition,
denitrification, and sediment retention. High alkalinity reflects high
anaerobic respiration. Groundwater flows may ameliorate hydrological
isolation. Seasonal filling and drying induce spatio-temporal
variability in temperature, depth, pH, dissolved oxygen, salinity, and
nutrients, resulting in zonation within lakes and high variability among
them.

\subsection{Distribution}\label{distribution-119}

Subhumid temperate and wet-dry tropical regions in monsoonal and
Mediterranean-type climates but usually not semi-arid or arid regions.

\section{F2.4 Freeze-thaw freshwater
lakes}\label{f2.4-freeze-thaw-freshwater-lakes-1}

Belongs to biome F2. Lakes biome, part of the Freshwater realm.

\subsection{Short description}\label{short-description-120}

Many plants and animals survive surface freezing of freshwater lakes, in
dormant life stages, by reducing activity beneath the ice, or by moving.
Such freshwater lakes vary enormously in size and distribution,
providing a wide range of habitats for many organisms, which undergo a
succession of emergence during lake thaw. The annual thaw triggers
highly productive plant and animal activity, beginning with diatom algae
and then zooplankton. Habitat diversity increases with lake size,
increasing the variety of plankton, aquatic plants, waterbugs, birds,
and sometimes fish.

\subsection{Key Features}\label{key-features-120}

Waterbodies with frozen surfaces for at least one month of the year,
with spring thaw initiating trophic successional dynamics beginning with
a flush of diatom productivity. Deeper lakes may be cold stratified and
fish tolerate oxygen depletion in winter.

\subsection{Ecological traits}\label{ecological-traits-120}

The majority of the surface of these lakes is frozen for at least a
month in most years. Their varied origins (tectonic, riverine,
fluvioglacial), size and depth affect composition and function.
Allochthonous and autochthonous energy sources vary with lake and
catchment features. Productivity is highly seasonal, sustained in winter
largely by the metabolism of microbial photoautotrophs, chemautotrophs
and zooplankton that remain active under low light, nutrients, and
temperatures. Spring thaw initiates a seasonal succession, increasing
productivity and re-establishing complex trophic networks, depending on
lake area, depth, connectivity, and nutrient availability. Diatoms are
usually first to become photosynthetically active, followed by small and
motile zooplankton, which respond to increased food availability, and
cyanobacteria later in summer when grazing pressure is high. Large lakes
with high habitat complexity (e.g.~Lake Baikal) support phytoplankton,
zooplankton, macrophytes (in shallow waters), invertebrate consumers,
migratory fish (in connected lakes), waterbirds, and mammals. Their
upper trophic levels are more abundant, diverse, and endemic than in
smaller lakes. Herbivorous fish and zooplankton are significant top-down
regulators of the main primary producers (i.e.~biofilms and
phytoplankton). These, in turn, are regulated by predatory fish, which
may be limited by prey availability and competition. The biota is
spatially structured by seasonally dynamic gradients in cold
stratification, light, nutrient levels, and turbulence. Traits such as
resting stages, dormancy, freeze-cued spore production in phytoplankton,
and the ability of fish to access low oxygen exchange enable persistence
through cold winters under the ice and through seasonal patterns of
nutrient availability.

\subsection{Key Ecological Drivers}\label{key-ecological-drivers-120}

Seasonal freeze-thaw cycles typically generate dimictic temperature
stratification regimes (i.e.~mixing twice per year), where cold water
lies above warm water in winter and vice versa in summer. Shallow lakes
may mix continuously (polymicitic) during the summer and may freeze
completely during winter. Mixing occurs in autumn and spring. Freezing
reduces light penetration and turbulence, subduing summer depth
gradients in temperature, oxygen, and nutrients. Ice also limits
atmospheric inputs, including gas exchange. Very low temperatures reduce
the growth rates, diversity, and abundance of fish. Many lakes are
stream sources. Lake sizes vary from \textless1 ha to more than 30,000
km2, profoundly affecting niche diversity and trophic complexity.
Freezing varies with the area and depth of lakes. Thawing is often
accompanied by flooding in spring, ameliorating light and temperature
gradients, and increasing mixing. Dark-water inflows from peatlands in
catchments influence water chemistry, light penetration, and
productivity.

\subsection{Distribution}\label{distribution-120}

Predominantly across the high latitudes of the Northern Hemisphere and
high altitudes of South America, New Zealand and Tasmania.

\section{F2.5 Ephemeral freshwater
lakes}\label{f2.5-ephemeral-freshwater-lakes-1}

Belongs to biome F2. Lakes biome, part of the Freshwater realm.

\subsection{Short description}\label{short-description-121}

These are shallow lakes that are mostly dry, and then fill for weeks or
months, before drying again. During dry periods, many animals or plants
survive as eggs, seeds or other dormant forms, while other species
disperse. Floods, bring water from surrounding catchments and
floodplains with organic matter, nutrients and fine sediments, and
trigger movement of birds and mammals. Floods activate simple foodwebs,
comprising abundant algae, zooplankton, waterbugs and crustaceans, which
have rapid life-cycles able to exploit windows of productivity. This
produces food for frogs and visiting waterbirds.

\subsection{Key Features}\label{key-features-121}

Shallow temporary lakes, depressions or pans with long dry periods of
low productivity, punctuated by episodes of inflow that bring large
resource subsidies from catchments, resulting in high productivity,
population turnover and trophic connectivity.

\subsection{Ecological traits}\label{ecological-traits-121}

Shallow ephemeral freshwater bodies are also known as depressions,
playas, clay pans, or pans. Long periods of low productivity during dry
phases are punctuated by episodes of high production after filling.
Trophic structure is relatively simple with mostly benthic, filamentous,
and planktonic algae, detritivorous and predatory zooplankton
(e.g.~rotifers and Daphnia), crustaceans, insects, and in some lakes,
molluscs. The often high invertebrate biomass provides food for
amphibians and itinerant waterbirds. Terrestrial mammals use the lakes
to drink and bathe and may transfer nutrients, organic matter, and
'hitch-hiking' biota. Diversity may be high in boom phases but there are
only a few local endemics (e.g.~narrow-ranged charophytes). Specialised
and opportunistic biota exploit boom-bust resource availability through
life-cycle traits that confer tolerance to desiccation
(e.g.~desiccation-resistant eggs in crustaceans) and/or enable rapid
hatching, development, breeding, and recruitment when water arrives.
Much of the biota (e.g.~opportunistic insects) have widely dispersing
adult phases enabling rapid colonisation and re-colonisation. Filling
events initiate succession with spikes of primary production, allowing
short temporal windows for consumers to grow and reproduce, and for
itinerant predators to aggregate. Drying initiates senescence,
dispersal, and dormancy until the next filling event.

\subsection{Key Ecological Drivers}\label{key-ecological-drivers-121}

Arid climates have highly variable hydrology. Episodic inundation after
rain is relatively short (days to months) due to high evaporation rates
and infiltration. Drainage systems are closed or nearly so, with
channels or sheet inflow from flat, sparsely vegetated catchments.
Inflows bring allochthonous organic matter and nutrients and are
typically turbid with fine particles. Clay-textured lake bottoms hold
water by limiting percolation but may include sand particles. Bottom
sediments release nutrients rapidly after filling. Lakes are shallow,
flat-bottomed and polymicitic when filled with small volumes, so light
and oxygen are generally not limiting. Persistent turbidity may limit
light but oxygen production by macrophytes and flocculation
(i.e.~clumping) from increasing salinity during drying reduce turbidity
over time. Shallow depth promotes high daytime water temperatures (when
filling in summer) and high diurnal temperature variability.

\subsection{Distribution}\label{distribution-121}

Semi-arid and arid regions at mid-latitudes of the Americas, Africa,
Asia, and Australia.

\section{F2.6 Permanent salt and soda
lakes}\label{f2.6-permanent-salt-and-soda-lakes-1}

Belongs to biome F2. Lakes biome, part of the Freshwater realm.

\subsection{Short description}\label{short-description-122}

These lakes are usually large and shallow in semi-arid regions, with
high concentrations of salts, mediated by inflows of water. Their
productivity from growth of algae and plants can support large numbers,
but low diversity of organisms equipped with tolerance to high salinity
and other solutes. They have relatively simple foodwebs, with high
numbers of microbes and plankton, crustaceans, insect larvae, fish and
specialised waterbirds such as flamingos.

\subsection{Key Features}\label{key-features-122}

Permanent waterbodies with high inorganic solute concentrations
(particularly sodium), supporting simple trophic networks, including
cyanobacteria and algae, invertebrates and specialist birds.

\subsection{Ecological traits}\label{ecological-traits-122}

Permanent salt lakes have waters with periodically or permanently high
sodium chloride concentrations. This group includes lakes with high
concentrations of other ions (e.g.~carbonate in soda lakes). Unlike in
hypersaline lakes, productivity is not suppressed and autotrophs may be
abundant, including phytoplankton, cyanobacteria, green algae, and
submerged and emergent macrophytes. These, supplemented by allochthonous
energy and C inputs from lake catchments, support relatively simple
trophic networks characterised by few species in high abundance and some
regional endemism. The high biomass of archaeal and bacterial
decomposers and phytoplankton in turn supports abundant consumers
including brine shrimps, copepods, insects and other invertebrates,
fish, and waterbirds (e.g.~flamingos). Predators and herbivores that
become dominant at low salinity exert top-down control on algae and
low-order consumers. Species niches are structured by spatial and
temporal salinity gradients. Species in the most saline conditions tend
to have broader ranges of salinity tolerance. Increasing salinity
generally reduces diversity and the importance of top-down trophic
regulation but not necessarily the abundance of organisms, except at
hypersaline levels. Many organisms tolerate high salinity through
osmotic regulation (at a high metabolic cost), limiting productivity and
competitive ability.

\subsection{Key Ecological Drivers}\label{key-ecological-drivers-122}

Permanent salt lakes tend to be large and restricted to semi-arid
climates with high evaporation but with reliable inflow sources
(e.g.~snowmelt). They may be thousands of hectares in size and several
metres deep. A few are much larger and deeper (e.g.~Caspian Sea), while
some volcanic lakes are small and deep. Endorheic drainage promotes salt
accumulation, but lake volume and reliable water inflows are critical to
buffering salinity below extreme levels. Salinity varies temporally from
0.3\% to rarely more than 10\% depending on lake size, temperature, and
the balance between freshwater inflows, precipitation, and evaporation.
Inflow is critical to ecosystem dynamics, partly by driving the indirect
effects of salinity on trophic or engineering processes. Within lakes,
salt concentrations may be vertically stratified (i.e.~meromictic) due
to the higher density of saltwater compared to freshwater inflow and
slow mixing. Dissolved oxygen is inversely related to salinity, hence
anoxia is common at depth in meromictic lakes. Ionic composition and
concentration varies greatly among lakes due to differences in substrate
and inflow, with carbonate, sulphate, sulphide, ammonia, and/or
phosphorus sometimes reaching high levels, and pH varying from 3 to 11.

\subsection{Distribution}\label{distribution-122}

Mostly in semi-arid regions of Africa, southern Australia, Eurasia, and
western parts of North and South America.

\section{F2.7 Ephemeral salt lakes}\label{f2.7-ephemeral-salt-lakes-1}

Belongs to biome F2. Lakes biome, part of the Freshwater realm.

\subsection{Short description}\label{short-description-123}

Ephemeral salt lakes in semi-arid and arid regions are shallow, with
extreme variation in salinity during wet-dry cycles that limits life to
a low diversity of specialised salt-tolerant species. The lakes are dry
and salt-encrusted most of the time, but episodic inundation dilutes
salt, allowing high growth of algae and larger plants which support
crustaceans, insect larvae, fish and specialist waterbirds. These
species use dormant life stages to survive drying, or disperse rapidly
to other habitats when the lake dries.

\subsection{Key Features}\label{key-features-123}

Salt lakes with salt crusts in long dry phases and short productive wet
phases. Trophic networks are simple but high productivity is driven by
bacteria and phytoplankton, supporting specialist birds.

\subsection{Ecological traits}\label{ecological-traits-123}

Ephemeral salt lakes or playas have relatively short-lived wet phases
and long dry periods of years to decades. During filling phases, inflow
dilutes salinity to moderate levels, and allochthonous energy and carbon
inputs from lake catchments supplement autochthonous energy produced by
abundant phytoplankton, cyanobacteria, diatoms, green algae, submerged
and emergent macrophytes, and fringing halophytes. In drying phases,
increasing salinity generally reduces diversity and top-down trophic
regulation, but not necessarily the abundance of organisms -- except at
hypersaline levels, which suppress productivity. Trophic networks are
simple and characterised by few species that are often highly abundant
during wet phases. The high biomass of archaeal and bacterial
decomposers and phytoplankton in turn support abundant consumers,
including crustaceans (e.g.~brine shrimps and copepods), insects and
other invertebrates, fish, and specialist waterbirds (e.g.~banded
stilts, flamingos). Predators and herbivores that dominate at low
salinity levels exert top-down control on algae and low-order consumers.
Species niches are strongly structured by spatial and temporal salinity
gradients and endorheic drainage promotes regional endemism. Species
that persist in the most saline conditions tend to have broad salinity
tolerance. Many organisms regulate salinity osmotically at a high
metabolic cost, limiting productivity and competitive ability. Many
specialised opportunists are able to exploit boom-bust resource cycles
through life-cycle traits that promote persistence during dry periods
(e.g.~desiccation-resistant eggs in crustaceans and/or rapid hatching,
development, breeding, and recruitment). Much of the biota (e.g.~insects
and birds) have widely dispersed adult phases enabling rapid
colonisation. Filling events drive specialised succession, with short
windows of opportunity to grow and reproduce reset by drying until the
next filling event.

\subsection{Key Ecological Drivers}\label{key-ecological-drivers-123}

Ephemeral salt lakes are up to 10,000 km2 in area and usually less than
a few metres deep. They may be weakly vertically stratified
(i.e.~meromictic) due to the slow mixing of freshwater inflow with
higher density saltwater. Endorheic drainage promotes salt accumulation.
Salinity varies temporally from 0.3\% to over 26\% depending on lake
size, depth temperature, and the balance between freshwater inflows,
precipitation, and evaporation. Inflow is critical to ecosystem
dynamics, mediates wet-dry phases, and drives the indirect effects of
salinity on trophic and ecosystem processes. Dissolved oxygen is
inversely related to salinity, hence anoxia is common in hypersaline
lake states. Ionic composition varies, with carbonate, sulphate,
sulphide, ammonia, and/or phosphorus sometimes at high levels, and pH
varying from 3 to 11.

\subsection{Distribution}\label{distribution-123}

Mostly in arid and semi-arid Africa, Eurasia, Australia, and North and
South America.

\section{F2.8 Artesian springs and
oases}\label{f2.8-artesian-springs-and-oases-1}

Belongs to biome F2. Lakes biome, part of the Freshwater realm.

\subsection{Short description}\label{short-description-124}

Surface waterbodies fed by (often warm) groundwaters rising to the
surface are scattered in dry landscapes of Africa, the Middle East,
Eurasia, North America and Australia, but also occur in humid
landscapes. Algae, floating plants and leaf fall support waterbugs,
crustaceans and small fish making simple foodwebs with some locally
restricted species found nowhere else. These ecosystems are sometimes
important waterholes for birds and mammals, in otherwise dry landscapes.

\subsection{Key Features}\label{key-features-124}

Groundwater dependent ecosystems from artesian waters discharged to the
surface, maintaining relatively stable water levels. Often insular
systems with high endemism.

\subsection{Ecological traits}\label{ecological-traits-124}

These groundwater-dependent systems are fed by artesian waters that
discharge to the surface. They are.surrounded by dry landscapes and
receive little surface inflow, being predominantly disconnected from
surface-stream networks. Insularity from the broader landscape results
in high levels of endemism in sedentary aquatic biota, which are likely
descendants of relic species from a wetter past. Springs may be
spatially clustered due to their association with geological features
such as faults or outcropping aquifers. Even springs in close proximity
may have distinct physical and biological differences. Some springs have
outflow streams, which may support different assemblages of plants and
invertebrates to those in the spring orifice. Artesian springs and oases
tend to have simple trophic structures. Autotrophs include aquatic algae
and floating vascular plants, with emergent amphibious plants in shallow
waters. Terrestrial plants around the perimeter contribute subsidies of
organic matter and nutrients through litter fall. Consumers and
predators include crustaceans, molluscs, arachnids, insects, and
small-bodied fish. Most biota are poorly dispersed and have continuous
life cycles and other traits specialised for persistence in
hydrologically stable, warm, or hot mineral-rich water. Springs and
oases are reliable watering points for wide-ranging birds and mammals,
which function as mobile links for resources and promote the dispersal
of other biota between isolated wetlands in the dryland matrix.

\subsection{Key Ecological Drivers}\label{key-ecological-drivers-124}

Flow of artesian water to the surface is critical to these wetlands,
which receive little input from precipitation or runoff. Hydrological
variability is low compared to other wetland types, but hydrological
connections with deep regional aquifers, basin-fill sediments and local
watershed recharge drive lagged flow dynamics. Flows vary over
geological timeframes, with evidence of cyclic growth, waning, and
extinction. Discharge waters tend to have elevated temperatures, are
polymicitic and enriched in minerals that reflect their geological
origins. The precipitation of dissolved minerals (e.g.~carbonates) and
deposition by wind and water form characteristic cones or mounds known
as ``mound springs''. Perennial flows and hydrological isolation from
other spatially and temporally restricted surface waters make these
wetlands important ecological refuges in arid landscapes.

\subsection{Distribution}\label{distribution-124}

Scattered throughout arid regions in southern Africa, the Sahara, the
Middle East, central Eurasia, southwest of North America, and
Australia's Great Artesian Basin.

\section{F2.9 Geothermal pools and
wetlands}\label{f2.9-geothermal-pools-and-wetlands-1}

Belongs to biome F2. Lakes biome, part of the Freshwater realm.

\subsection{Short description}\label{short-description-125}

Geothermal pools and associated wetlands are fed by deeply circulating
groundwater that mixes with magma and hot rocks in volcanically active
regions. Mineral concentrations are therefore high and produce
chemically precipitated substrates as waters cool. The extreme
temperatures and water chemistry limit life to a low diversity of
specialised bacteria, extensive algal mats and insect larvae which can
live in warm acid or alkaline water with high mineral content. Away from
their hottest waters, aquatic plants, crustaceans, frogs, fish, snakes
and birds can all occur.

\subsection{Key Features}\label{key-features-125}

Hot springs, geysers and mud pots dependent on groundwater interactions
with magma and hot rocks, supporting highly specialised low diversity
biota tolerate of high temperatures and high concentrations of inorganic
salts.

\subsection{Ecological traits}\label{ecological-traits-125}

These hot springs, geysers, mud pots and associated wetlands result from
interactions of deeply circulating groundwater with magma and hot rocks
that produce chemically precipitated substrates. They support a
specialised but low-diversity biota structured by extreme thermal and
geochemical gradients. Energy is almost entirely autochthonous,
productivity is low, and trophic networks are very simple. Primary
producers include chemoautotrophic bacteria and archaea, as well as
photoautotrophic cyanobacteria, diatoms, algae, and macrophytes.
Thermophilic and metallophilic microbes dominate the most extreme
environments in vent pools, while mat-forming green algae and
animal-protists occur in warm acidic waters. Thermophilic blue-green
algae reach optimum growth above 45°C. Diatoms occur in less acidic warm
waters. Aquatic macrophytes occur on sinter aprons and wetlands with
temperatures below 35°C. Herbivores are scarce, allowing thick algal
mats to develop. These are inhabited by invertebrate detritivores,
notably dipterans and coleopterans, which may tolerate temperatures up
to 55°C. Molluscs and crustaceans occupy less extreme microhabitats
(notably in hard water hot springs), as do vertebrates such as
amphibians, fish, snakes and visiting birds. Microinvertebrates such as
rotifers and ostracods are common. Invertebrates, snakes and fish
exhibit some endemism due to habitat insularity. Specialised
physiological traits enabling metabolic function in extreme temperatures
include thermophilic proteins with short amino-acid lengths, chaperone
molecules that assist protein folding, branched chain fatty acids and
polyamines for membrane stabilisation, DNA repair systems, and
upregulated glycolysis providing energy to regulate heat stress. Three
mechanisms enable metabolic function in extremely acidic (pH\textless3)
geothermal waters: proton efflux via active transport pumps that counter
proton influx, decreased permeability of cell membranes to suppress
proton entry into the cytoplasm, and strong protein and DNA repair
systems. Similar mechanisms enable metabolic function in waters with
high concentrations of metal toxins. A succession of animal and plant
communities occur with distance from the spring source as temperatures
cool and minerals precipitate.

\subsection{Key Ecological Drivers}\label{key-ecological-drivers-125}

Continual flows of geothermal groundwater sustain these polymicitic
water bodies. Permanent surface waters may be clear or highly turbid
with suspended solids as in `mud volcanoes'. Water temperatures vary
from hot (\textgreater44°C) to extreme (\textgreater80°C) on local
gradients (e.g.~vent pools, geysers, mounds, sinter aprons, terraces,
and outflow streams). The pH is either extremely acid (2--4) or
neutral-alkaline (7--11). Mineral salts are concentrated, but
composition varies greatly among sites with properties of the underlying
bedrock. Dissolved and precipitated minerals include very high
concentrations of silicon, calcium or iron, but also arsenic, antimony,
copper, zinc, cadmium, lead, polonium or mercury, usually as oxides,
sulphides, or sulphates, but nutrients such as nitrogen and phosphorus
may be scarce.

\subsection{Distribution}\label{distribution-125}

Tectonically or volcanically active areas from tropical to subpolar
latitudes. Notable examples in Yellowstone (USA), Iceland, New Zealand,
Atacama (Chile), Japan and east Africa.

\section{F3.1 Large reservoirs}\label{f3.1-large-reservoirs-1}

Belongs to biome F3. Artificial wetlands biome, part of the Freshwater
realm.

\subsection{Short description}\label{short-description-126}

Large dams or reservoirs occur in humid, populated areas of the world.
Their biological productivity and diversity is generally limited due to
their depth and frequent and large changes in water level. Shallow zones
have the highest diversity, with simple food webs of algae, waterbugs,
birds, frogs and aquatic plants, often supporting introduced fish
species. Plankton live at the surface, but life is scarce in the depths.
Algal blooms may be common if there are high nutrient inputs from
rivers.

\subsection{Key Features}\label{key-features-126}

Large, usually deep stratifed waterbodies impounded by walls across
outflow channels. Productivity and biotic diversity are lower than
unregulated lakes of simila rsize and complexity. Trophic networks are
simple.

\subsection{Ecological traits}\label{ecological-traits-126}

Rivers are impounded by the construction of dam walls, creating large
freshwater reservoirs, mostly 15--250 m deep. Primary productivity is
low to moderate and restricted to the euphotic zone (limnetic and
littoral zones), varying with turbidity and associated light
penetration, nutrient availability, and water temperature. Trophic
networks are simple with low species diversity and endemism. Shallow
littoral zones have the highest species diversity including benthic
algae, macroinvertebrates, fish, waterbirds, aquatic reptiles, aquatic
macrophytes, and terrestrial or amphibious vertebrates. Phytoplankton
and zooplankton occur through the littoral and limnetic zones. The
profundal zone lacks primary producers and, if oxygenated, is dominated
by benthic detritivores and microbial decomposers. Fish communities
inhabit the limnetic and littoral zones and may be dominated by managed
species and opportunists. Reservoirs may undergo eutrophic succession
due to inflow from catchments with sustained fertiliser application or
other nutrient inputs.

\subsection{Key Ecological Drivers}\label{key-ecological-drivers-126}

Reservoirs receive water from the rivers they impound. Managed release
or diversion of water alters natural variability. Large variations in
water level produce wide margins that are intermittently inundated or
dry, limiting productivity and the number of species able to persist
there. Inflow volumes may be regulated. Inflows may contain high
concentrations of phosphorus and/or nitrogen (e.g.~from sewerage
treatment effluents or fertilised farmland), leading to eutrophication.
Reservoirs in upper catchments generally receive less nutrients and
cooler water (due to altitude) than those located downstream.
Geomorphology, substrate, and land use of the river basin influence the
amount of inflowing suspended sediment, and hence turbidity, light
penetration, and the productivity of planktonic and benthic algae, as
well as rates of sediment build-up on the reservoir floor. Depth
gradients in light and oxygen, as well as thermal stratification,
strongly influence the structure of biotic communities and trophic
interactions, as do human introductions of fish, aquatic plants, and
other alien species.

\subsection{Distribution}\label{distribution-126}

Large reservoirs are scattered across all continents with the greatest
concentrations in Asia, Europe, and North America. Globally, there are
more than 3000 reservoirs with a surface area ≥ 50km2. Spatial data are
incomplete for some countiries.

\section{F3.2 Constructed lacustrine
wetlands}\label{f3.2-constructed-lacustrine-wetlands-1}

Belongs to biome F3. Artificial wetlands biome, part of the Freshwater
realm.

\subsection{Short description}\label{short-description-127}

Small farm dams, wastewater ponds and mine pits generally form lake-like
environments, common in humid and semi-arid climates world-wide.
Nutrient inputs vary greatly depending on purpose and surrounding land
uses. They are often warm and shallow, and biological productivity and
diversity vary widely depending on the cover and state of fringing
vegetation, with the most diverse examples rivalling equivalent natural
wetlands. Aquatic plants, plankton, algae and waterbugs may dominate in
the shallows, supporting amphibians, turtles, fish, and waterbirds.

\subsection{Key Features}\label{key-features-127}

Small, shallow open waterbodies with high or low productivity depending
on nutrient subsidies and complexity of littoral zones and benthos
Relatively simple trophic networks with algae, macrophytes, zooplankton,
aquatic invertebrates and amphibians.

\subsection{Ecological traits}\label{ecological-traits-127}

Shallow, open water bodies have been constructed in diverse landscapes
and climates. They may be fringed by amphibious vegetation, or else
bedrock or bare soil maintained by earthworks or livestock trampling.
Emergents rarely extend throughout the water body, but submerged
macrophytes are often present. Productivity ranges from very high in
wastewater ponds to low in mining and excavation pits, depending on
depth, shape, history and management. Taxonomic and functional diversity
range from levels comparable to natural lakes to much less, depending on
productivity, complexity of aquatic or fringing vegetation, water
quality, management and proximity to other waterbodies or vegetation.
Trophic structure includes phytoplankton and microbial detritivores,
with planktonic and invertebrate predators dominating limnetic zones.
Macrophytes may occur in shallow littoral zones or submerged habitats,
and some artificial water bodies include higher trophic levels including
macroinvertebrates, amphibians, turtles, fish, and waterbirds. Fish may
be introduced by people or arrive by flows connected to source
populations, where these exist. Endemism is generally low, but these
waterbodies may be important refuges for some species now highly
depleted in their natural habitats. Life histories often reflect those
found in natural waterbodies nearby, but widely dispersed opportunists
dominate where water quality is poor. Intermittent water bodies support
biota with drought resistance or avoidance traits, while permanently
inundated systems provide habitat for mobile species such as waterbirds.

\subsection{Key Ecological Drivers}\label{key-ecological-drivers-127}

Water bodies are constructed for agriculture, mining, stormwater,
ornamentation, wastewater, or other uses, or fill depressions left by
earthworks, obstructing surface flow or headwater channels. Humans may
directly or indirectly regulate inputs of water and chemicals
(e.g.~fertilisers, flocculants, herbicides), as well as water drawdown.
Climate and weather also affect hydrology. Shallow depth and lack of
shade may expose open water to rapid solar heating and hence diurnally
warm temperatures. Substrates include silt, clay, sand, gravel, cobbles
or bedrock, and fine sediments of organic material may build up over
time. Nutrient levels are highest in wastewater or with run-off from
fertilised agricultural land or urban surfaces. Some water bodies
(e.g.~mines and industrial wastewaters) have concentrated chemical
toxins, extremes of pH or high salinities. Humans may actively introduce
and remove the biota of various trophic levels (e.g.~bacteria, algae,
fish, and macrophytes) for water quality management or human
consumption.

\subsection{Distribution}\label{distribution-127}

Scattered across most regions of the world occupied by humans. Farm dams
covered an estimated 77,000km2 globally in 2006.

\section{F3.3 Rice paddies}\label{f3.3-rice-paddies-1}

Belongs to biome F3. Artificial wetlands biome, part of the Freshwater
realm.

\subsection{Short description}\label{short-description-128}

Rice paddies cover more than a million square kilometres mostly in
tropical to warm temperate climates, especially Southeast Asia. They are
filled by rainfall or river water diversions. Their levees and channels
retain shallow water areas, with nutrients inputs from inflows and
fertilisers. Planting and harvest establishes a regular cycle of
disturbance, with many paddies also supporting production of fish and
crustaceans, Their simple foodwebs are adapted to temporary flooding and
the harvest cycle, including algae and plankton, waterbugs, frogs and
waterbirds.

\subsection{Key Features}\label{key-features-128}

Artificial wetlands with limited horizontal and vertical heterogeneity,
filled seasonally with water from rivers or rainfall and frequently
disturbed by planting and harvest of rice. Simple trophic networks with
colonists from rivers and wetlands that may also include managed fish
populations.

\subsection{Ecological traits}\label{ecological-traits-128}

Rice paddies are artificial wetlands with low horizontal and vertical
heterogeneity fed by rain or irrigation water diverted from rivers. They
are predominantly temporary wetlands, regularly filled and dried,
although some are permanently inundated, functioning as simplified
marshes. Allochthonous inputs come from water inflow but also include
the introduction of rice, other production organisms (e.g.~fish and
crustaceans), and fertilisers that promote rice growth. Simplified
trophic networks are sustained by highly seasonal, deterministic
flooding and drying regimes and the agricultural management of harvest
crops, weeds, and pests. Cultivated macrophytes dominate primary
production, but other autotrophs including archaea, cyanobacteria,
phytoplankton, and benthic or epiphytic algae also contribute. During
flooded periods, microbial changes produce anoxic soil conditions and
emissions by methanogenic archaea. Opportunistic colonists include
consumers such as invertebrates, zooplankton, insects, fish, frogs, and
waterbirds, as well as other aquatic plants. Often they come from nearby
natural wetlands or rivers and may breed within rice paddies. During dry
phases, obligate aquatic organisms are confined to wet refugia away from
rice paddies. These species possess traits that promote tolerance to low
water quality and predator avoidance. Others organisms, including many
invertebrates and plants, have rapid life cycles and dormancy traits
allowing persistence as eggs or seeds during dry phases.

\subsection{Key Ecological Drivers}\label{key-ecological-drivers-128}

Engineering of levees and channels enables the retention of standing
water a few centimetres above the soil surface and rapid drying at
harvest time. This requires reliable water supply either through summer
rains in the seasonal tropics or irrigation in warm-temperate or
semi-arid climates. The water has high oxygen content and usually warm
temperatures. Deterministic water regimes and shallow depths limit niche
diversity and have major influences on the physical, chemical, and
biological properties of soils, which contain high nutrient levels. Rice
paddies are often established on former floodplains but may also be
created on terraced hillsides. Other human interventions include
cultivation and harvest, aquaculture, and the addition of fertilisers,
herbicides, and pesticides.

\subsection{Distribution}\label{distribution-128}

More than a million square kilometres, mostly in tropical and
subtropical Southeast Asia, with small areas in Africa, Europe, South
America, North America, and Australia.

\section{F3.4 Freshwater aquafarms}\label{f3.4-freshwater-aquafarms-1}

Belongs to biome F3. Artificial wetlands biome, part of the Freshwater
realm.

\subsection{Short description}\label{short-description-129}

Freshwater aquafarms are ponds constructed from earthworks or cages
built within freshwater lakes, rivers and reservoirs. They are most
common in Asia. and used to produce species. Their commercial production
of fish and crustacean involves intensive interventions, including
focussed inputs of food and nutrients, and control of competitors,
predators and diseases that may limit production of target species.
Consequently habitat diversity and primary production are low, and
non-target biota is limited to opportunistic colonisers from adjacent
water sources, including insects, fish, frogs, waterbirds and some
aquatic plants.

\subsection{Key Features}\label{key-features-129}

Artificial mostly permanent waterbodies managed for production of fish
or crustaceans with managed inputs of nutrients and energy Simple
trophic networks of opportunistic colonists supported mainly be algal
productivity.

\subsection{Ecological traits}\label{ecological-traits-129}

Freshwater aquaculture systems are mostly permanent water bodies in
either purpose-built ponds, tanks, or enclosed cages within artificial
reservoirs (F3.1), canals (F3.5), freshwater lakes (F2.1 and F2.2), or
lowland rivers (F1.2). These systems are shaped by large allochthonous
inputs of energy and nutrients to promote secondary productivity by one
or a few target consumer species (mainly fish or crustaceans), which are
harvested as adults and restocked as juveniles on a regular basis. Fish
are sometimes raised in mixed production systems within rice paddies
(F3.3), but aquaculture ponds may also be co-located with rice paddies,
which are centrally located and elevated above the level of the ponds.
The enclosed structures exclude predators of the target species, while
intensive anthropogenic management of hydrology, oxygenation, toxins,
competitors, and pathogens maintains a simplified trophic structure and
near-optimal survival and growth conditions for the target species.
Intensive management and low niche diversity within the enclosures limit
the functional diversity of biota within the system. However, biofilms
and phytoplankton contribute low levels of primary production,
sustaining zooplankton and other herbivores, while microbial and
invertebrate detritivores break down particulate organic matter. Most of
these organisms are opportunistic colonists, as are insects, fish,
frogs, and waterbirds, as well as aquatic macrophytes. Often these
disperse from nearby natural wetlands, rivers, and host waterbodies.

\subsection{Key Ecological Drivers}\label{key-ecological-drivers-129}

Aquafarms are small artificial water bodies with low horizontal and
vertical heterogeneity. Water regimes are mostly perennial but may be
seasonal (e.g.~when integrated with rice production). Engineering of
tanks, channels, and cages enables the intensive management of water,
nutrients, oxygen levels, toxins, other aspects of water chemistry, as
well as the introduction of target species and the exclusion of pest
biota. Removal of wastewater and replacement by freshwater from lakes or
streams, together with inputs of antibiotics and chemicals
(e.g.~pesticides and fertilisers) influence the physical, chemical, and
biological properties of the water column and substrate. When located
within cages in natural water bodies, freshwater aquafarms reflect the
hydrological and hydrochemical properties of their host waterbody.
Nutrient inputs drive the accumulation of ammonium and nitrite nitrogen,
as well as phosphorus and declining oxygen levels, which may lead to
eutrophication within aquaculture sites and receiving waters.

\subsection{Distribution}\label{distribution-129}

Concentrated in Asia but also in parts of northern and western Europe,
North and West Africa, South America, North America, and small areas of
southeast Australia and New Zealand.

\section{F3.5 Canals, ditches and
drains}\label{f3.5-canals-ditches-and-drains-1}

Belongs to biome F3. Artificial wetlands biome, part of the Freshwater
realm.

\subsection{Short description}\label{short-description-130}

Canals, ditches and drains are associated with agriculture and cities
throughout the world. They take freshwater to and from urban and rural
areas, particularly in temperate and subtropical regions. They can carry
high nutrient and pollutant loads. Diversity of organisms is generally
low, but may be high where there are earthen banks and fringing
vegetation. Algae and macrophytes (where present) support microbes and
waterbugs and other invertebrates often small fish, amphibians and
crustaceans. They are also important pathways for dispersal of some
aquatic species.

\subsection{Key Features}\label{key-features-130}

Artificial streams often with low horizontal and vertical heterogeneity,
but with productivity, diversity and trophic structure highly dependent
on fringing vegetation and subsidies of nutrients and carbon from
catchments.

\subsection{Ecological traits}\label{ecological-traits-130}

Canals, ditches and storm water drains are artificial streams with low
horizontal and vertical heterogeneity. They function as rivers or
streams and may have simplified habitat structure and trophic networks,
though some older ditches have fringing vegetation, which contributes to
structural complexity. The main primary producers are filamentous algae
and macrophytes that thrive on allochthonous subsidies of nutrients.
Subsidies of organic carbon from urban or rural landscapes support
microbial decomposers and mostly small invertebrate detritivores. While
earthen banks and linings may support macrophytes and a rich associated
fauna, sealed or otherwise uniform substrates limit the diversity and
abundance of benthic biota. Fish and crustacean communities, when
present, generally exhibit lower diversity and smaller body sizes
compared to natural systems, and are often dominated by introduced or
invasive species. Waterbirds, when present, typically include a low
diversity and density of herbivorous and piscivorous species. Canals,
ditches and drains may provide pathways for dispersal or colonisation of
native and invasive biota.

\subsection{Key Ecological Drivers}\label{key-ecological-drivers-130}

Engineered levees and channels enable managed water flow for human uses,
including water delivery for irrigation or recreation, water removal
from poorly drained sites or sealed surfaces (e.g.~storm water drains),
or routes for navigation. Deterministic water regimes and often shallow
depths have major influences on the physical, chemical, and biological
properties of the canals, ditches and drains. Flows in some ditches may
be very slow, approaching lentic regimes. Flows in storm water drains
vary with rain or other inputs. Irrigation, transport, or recreation
canals usually have steady perennial flows but may be seasonal for
irrigation or intermittent where the water source is small. Turbidity
varies but oxygen content is usually high. Substrates and banks vary
from earthen material or hard surfaces (e.g.~concrete, bricks),
affecting suitability for macrophytes and niche diversity. The water may
carry high levels of nutrients and pollutants due to inflow and
sedimentation from sealed surfaces, sewerage, other waste sources,
fertilised cropping, or pasture lands.

\subsection{Distribution}\label{distribution-130}

Urban landscapes and irrigation areas mostly in temperate and
subtropical latitudes. Several hundred thousand kilometres of ditches
and canals in Europe.

\section{FM1.1 Deepwater coastal
inlets}\label{fm1.1-deepwater-coastal-inlets-1}

Belongs to biome FM1. Semi-confined transitional waters biome, part of
the Freshwater, Marine realm.

\subsection{Short description}\label{short-description-131}

Ecosystems in these deep, narrow inlets were mostly formed by glaciers
and subsequently flooded (e.g.~fjords). They have some features of open
oceans, but are strongly influenced by freshwater inflows and the
surrounding coast. Productivity by phytoplankton is seasonal and limited
by cold, dark winters. Oxygen may be limited in the deepest parts of
these systems. The diverse biota includes invertebrates and fish, such
as jellyfish and salmon, and predatory marine mammals such as killer
whales.

\subsection{Key Features}\label{key-features-131}

Strong gradients between adjacent terrestrial and freshwater
systems,e.g.~fjords. Seasonaly abundant plankton, jellies, fish and
mammals..

\subsection{Ecological traits}\label{ecological-traits-131}

Deepwater coastal inlets (e.g.~fjords, sea lochs) are semi-confined
aquatic systems with many features of open oceans. Strong influences
from adjacent freshwater and terrestrial systems produce striking
environmental and biotic gradients. Autochthonous energy sources are
dominant, but allochthonous sources (e.g.~glacial ice discharge,
freshwater streams, and seasonal permafrost meltwater) may contribute
10\% or more of particulate organic matter. Phytoplankton, notably
diatoms, contribute most of the primary production, along with biofilms
and macroalgae in the epibenthic layer. Seasonal variation in inflow,
temperatures, ice cover, and insolation drives pulses of in situ and
imported productivity that generate blooms in diatoms, consumed in turn
by jellyfish, micronekton, a hierarchy of fish predators, and marine
mammals. Fish are limited by food, density-dependent predation, and
cannibalism. As well as driving pelagic trophic networks, seasonal
pulses of diatoms shape biogeochemical cycles and the distribution and
biomass of benthic biota when they senesce and sink to the bottom,
escaping herbivores, which are limited by predators. The vertical flux
of diatoms, macrophytes, and terrestrial detritus sustains a diversity
and abundance of benthic filter-feeders (e.g.~maldanids and oweniids).
Environmental and biotic heterogeneity underpins functional and
compositional contrasts between inlets and strong gradients within them.
Zooplankton, fish, and jellies distribute in response to resource
heterogeneity, environmental cues, and interactions with other
organisms. Deep inlets sequester more organic carbon into sediments than
other estuaries (FM1.2, FM1.3) because steep slopes enable efficient
influx of terrestrial carbon and low-oxygen bottom waters abate decay
rates. Inlets with warmer water have more complex trophic webs, stronger
pelagic-benthic coupling, and utilise a greater fraction of organic
carbon, sequestering it in sea-floor sediments at a slower rate than
those with cold water.

\subsection{Key Ecological Drivers}\label{key-ecological-drivers-131}

Deepwater coastal systems may exceed 300 km in length and 2 km in depth.
Almost all have glacial origins and many are fed by active glaciers. The
ocean interface at the mouth of the inlets, strongly influenced by
regional currents, interacts with large seasonal inputs of freshwater to
the inner inlet and wind-driven advection, to produce strong and dynamic
spatial gradients in nutrients, salinity and organic carbon. Advection
is critical to productivity and carrying capacity of the system.
Advection drives water movement in the upper and lower layers of the
water column in different directions, linking inlet waters with coastal
water masses. Coastal currents also mediate upwelling and downwelling
depending on the direction of flow. However, submerged glacial moraines
or sills at the inlet mouth may limit marine mixing, which can be
limited to seasonal episodes in spring and autumn. Depth gradients in
light typically extend beyond the photic zone and are exacerbated at
high latitudes by seasonal variation in insolation and surface ice.
Vertical fluxes also create strong depth gradients in nutrients, oxygen,
dissolved organic carbon, salinity, and temperature.

\subsection{Distribution}\label{distribution-131}

Historically or currently glaciated coastlines at polar and
cool-temperate latitudes.

\section{FM1.2 Permanently open riverine estuaries and
bays}\label{fm1.2-permanently-open-riverine-estuaries-and-bays-1}

Belongs to biome FM1. Semi-confined transitional waters biome, part of
the Freshwater, Marine realm.

\subsection{Short description}\label{short-description-132}

These coastal ecosystems are shifting mosaics of different habitats,
depending on the shape of the local coast, and proportional inflow of
freshwater and seawater. Combined nutrients from marine, freshwater and
land-based sources support very high productivity. Transient large
animals like dugongs, dolphins, turtles and shorebirds feed on abundant
fish, invertebrates and plant life, and they commonly serve as sheltered
nursery areas for fish. Many organisms are adapted to large variations
in salinity.

\subsection{Key Features}\label{key-features-132}

Productive mosaic systems with variable salinity, often nuseries for
fish and supporting abundant seabirds and mammals..

\subsection{Ecological traits}\label{ecological-traits-132}

These coastal water bodies are mosaic systems characterised by high
spatial and temporal variabilities in structure and function, which
depend on coastal geomorphology, ratios of freshwater inflows to marine
waters and tidal volume (hence residence time of saline water), and
seasonality of climate. Fringing shoreline systems may include
intertidal mangroves (MFT1.2), saltmarshes and reedbeds (MFT1.3), rocky
(MT1.1), muddy (MT1.2) or sandy shores (MT1.3), while seagrasses and
macrophytes (M1.1), shellfish beds (M1.4) or subtidal rocky reefs (M1.6)
may occur in shallow intertidal and subtidal areas. Water-column
productivity is typically higher than in nearby marine or freshwater
systems due to substantial allochthonous energy and nutrient subsidies
from shoreline vegetation and riverine and marine sources. This high
productivity supports a complex trophic network with relatively high
mosaic-level diversity and an abundance of aquatic organisms. Planktonic
and benthic invertebrates (e.g.~molluscs and crustaceans) often sustain
large fish populations, with fish nursery grounds being a common
feature. Waterbirds (e.g.~cormorants), seabirds (e.g.~gannets),
top-order predatory fish, mammals (e.g.~dolphins and dugongs), and
reptiles (e.g.~marine turtles and crocodilians) exploit these locally
abundant food sources. Many of these organisms in upper trophic levels
are highly mobile and move among different estuaries through connected
ocean waters or by flying. Others migrate between different ecosystem
types to complete their various life-history phases, although some may
remain resident for long periods. Most biota tolerate a broad range of
salinity or are spatially structured by gradients. The complex spatial
mixes of physical and chemical characteristics, alongside seasonal,
inter-annual, and sporadic variability in aquatic conditions, induce
correspondingly large spatial-temporal variability in food webs.
Low-salinity plumes, usually proportional to river size and discharge,
may extend far from the shore, producing tongues of ecologically
distinct conditions into the marine environment.

\subsection{Key Ecological Drivers}\label{key-ecological-drivers-132}

Characteristics of these coastal systems are governed by the relative
dominance of saline marine waters versus freshwater inflows (groundwater
and riverine), the latter depending on the seasonality of precipitation
and evaporative stress. Geomorphology ranges from wave-dominated
estuaries to drowned river valleys, tiny inlets, and enormous bays.
These forms determine the residence time, proportion, and distribution
of saline waters, which in turn affect salinity and thermal gradients
and stratification, dissolved O2 concentration, nutrients, and
turbidity. The water column is closely linked to mudflats and sandflats,
in which an array of biogeochemical processes occurs, including
denitrification and N-fixation, and nutrient cycling.

\subsection{Distribution}\label{distribution-132}

Coastlines of most landmasses but rarely on arid or polar coasts.

\section{FM1.3 Intermittently closed and open lakes and
lagoons}\label{fm1.3-intermittently-closed-and-open-lakes-and-lagoons-1}

Belongs to biome FM1. Semi-confined transitional waters biome, part of
the Freshwater, Marine realm.

\subsection{Short description}\label{short-description-133}

Opportunistic, short-lived organisms live in these ecosystems, where
conditions change rapidly as lagoon entrances to the open ocean open or
close. Periodic opening and closure influences dynamic gradients in
salinity, nutrients, temperature, and water level. Algae, invertebrates
like shrimps, and small fish rely on nutrients from land and, when open,
the sea. Timing of opening or closing depends on transport of sand and
mud by currents or freshwater inflow, or on anthropogenic processes.

\subsection{Key Features}\label{key-features-133}

Shallow water systems, highly variability depending on opening or
closing of lagoonal entrance. Detritus-based foodwebs with plankton,
invertebrates and small fish..

\subsection{Ecological traits}\label{ecological-traits-133}

These coastal water bodies have high spatial and temporal variability in
structure and function, which depends largely on the status of the
lagoonal entrance (open or closed). Communities have low species
richness compared to those of permanently open estuaries (FM1.2).
Lagoonal entrance closure prevents the entry of marine organisms and
resident biota must tolerate significant variation in salinity,
inundation, dissolved oxygen, and nutrient concentrations. Resident
communities are dominated by opportunists with short lifecycles. Trophic
networks are generally detritus-based, fuelled by substantial inputs of
organic matter from the terrestrial environment and, to a lesser extent,
from the sea. As net sinks of organic matter from the land, productivity
is often high, and lagoons may serve as nursery habitats for fish. High
concentrations of polyphenolic compounds (e.g.~tannins) in the water and
periods of low nutrient input limit phytoplankton populations. Benthic
communities dominate with attached algae, microphytobenthos and micro-
and macro-fauna being the dominant groups. The water column supports
plankton and small-bodied fish. Emergent and fringing vegetation is a
key source of detrital carbon to the food webs, and also provides
important structural habitats. Saltmarsh and reedbeds (MFT1.3) can
adjoin lagoons while seagrasses (M1.1) occupy sandy bottoms of some
lagoons, but mangroves (MFT1.2) are absent unless the entrance opens

\subsection{Key Ecological Drivers}\label{key-ecological-drivers-133}

These are shallow coastal water bodies that are intermittently connected
with the ocean. Some lagoons are mostly open, closing only once every
few decades. Some open and close frequently and some are closed most of
the time. The timing and frequency of entrance opening depend on
trade-offs between sedimentation from fluvial and shoreline processes
(which close the connection) and flushes of catchment inflow or erosive
wave action (which open the entrance). Opening leads to changes in water
level, tidal amplitude, salinity gradients, temperature, nutrients,
dissolved oxygen, and sources of organic carbon. Human-regulated opening
influences many of these processes.

\subsection{Distribution}\label{distribution-133}

Wave-dominated coastlines worldwide, but prevalent along microtidal to
low mesotidal mid-latitude coastlines with high inter-annual variability
in rainfall and wave climate. Intermittent closed open lakes and lagoons
(ICOLLs) are most prevalent in Australia (21\% of global occurrences),
South Africa (16\%), and Mexico (16\%).




\end{document}
